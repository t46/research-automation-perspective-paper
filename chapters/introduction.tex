\section{Introduction}
Research has been the foundation of human progress. Through research, humans have deepened their understanding of the world and created unprecedented innovations, leading to groundbreaking advancements. It would be not an exaggeration to say that the future development of humanity heavily depends on the evolution and progress of research endeavors.

Since the inception of artificial intelligence (AI) research, a key goal has been to develop AI capable of conducting research \cite{zenil2023future}. AI-led research not only accelerates existing research and development but also offers significant potential for improving research practice and methodologies themselves, unencumbered by human cognitive limitations, research conventions, or unnecessary social constraints \cite{zenil2023future,kitano2021nobel}.

Researchers have developed systems that automate scientific decision makings \cite{lindsay1993dendral}, infer natural laws \cite{langley1987scientific}, autonomously cycle through hypothesis and experimentation \cite{king2004functional}, and many more \cite{zenil2023future,zenil2023}. With advancements in machine learning, there have been remarkable successes in using it for scientific discovery \cite{wang2023scientific,xu2021artificial,zhang2023artificial,ai4science2023impact}. These efforts by humanity have significantly advanced the potential of machines as valuable assistants in research.

On the other hand, the journey toward fully autonomous intelligent agents\footnote{
In this paper, the terms AI, machine, and agent are used interchangeably.
} capable of conducting research is still ongoing \cite{zenil2023future,coley2020autonomousII}. ``Autonomous'' here implies functioning without human intervention, prior design, or preparation. An agent is considered more autonomous if it can conduct research with less human involvement. By ``agents that conduct research,'' I refer to a single agent capable of performing research activities in various fields, like history, mathematics, or physics. The wider the range of research a single agent can handle, the more general it is considered. In this paper, when referring to an agent capable of conducting research, it denotes a general and autonomous artificial researcher. The realization of such agents has been a deeply held aspiration in the quest for advancing human research capabilities.

While there are excellent papers presenting perspectives on AI capable of conducting research \cite{zenil2023future,coley2020autonomous,coley2020autonomousII,kitano2021nobel,wang2023scientific,zenil2023,zhang2023artificial,hope2022computational}, there is still much to discuss about the nature of such agents and what it means to create an agent capable of doing research. Therefore, this paper presents a speculative thought around the concept of artificial intelligent agents capable of conducting research. This discussion aims to provide an opportunity to consider what future discussions are needed as we work towards realizing agents capable of conducting research 

First, I will explore conceptually how to characterize the research activities. This preliminary discussion is intended to help us consider what it would mean to create an agent capable of conducting research and what discussions would be necessary for its development. Next, I will discuss elements widely considered essential in research, specifically exploring the nature of question construction, hypothesis generation, and hypothesis verification, along with the potential challenges in autonomously performing these tasks.  Subsequently, I will consider topics that combine these elements, topics common to them, and points that could not be discussed previously. Finally, as a reference, I will share some simple, preliminary ideas for prototyping aimed at identifying challenges in developing an agent capable of conducting research.

The speculative discussion in this paper is still in its early stages and is provisional. Given the breadth of the subject matter and my limited capabilities, each point of discussion may be somewhat superficial or not entirely accurate. I plan to update these discussions continuously. Therefore, if anyone notices any points that should be improved, errors, or topics that would be beneficial to discuss, I would greatly appreciate your feedback.
