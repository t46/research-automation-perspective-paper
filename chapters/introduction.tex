\section{Introduction}
Research has been the foundation of human progress. Humans have deepened their understanding of the world and created things never seen before through research, leading to groundbreaking innovations. It would not be an exaggeration to say that the future development of humanity heavily depends on how the endeavor of research evolves and progresses.

Since the beginning of artificial intelligence (AI) research, a key goal has been to develop AI that conducts research. Researchers have developed systems that automate scientific judgments \cite{lindsay1993dendral}, infer scientific laws \cite{langley1987scientific}, and autonomously cycle through hypothesis and experimentation \cite{king2004functional}. Recently, with advancements in machine learning, we've seen remarkable successes in using it for scientific discovery \cite{wang2023scientific,xu2021artificial,zhang2023artificial,ai4science2023impact}. These efforts by humanity have greatly developed the potential of machines as excellent assistants in research.

On the other hand, humanity's attempt to realize fully autonomous intelligent agents that conduct research is still midway in its journey \cite{zenil2023,coley2020autonomousII}. ``Autonomous'' implies the absence of human intervention, prior design, or preparation. I will refer to an agent as being more autonomous if it can conduct research with less human intervention. By ``agents that conduct research,'' I mean a single system that can conduct any research as humans do. This includes exploring various fields like history, mathematics, or physics. The more fields of research a single agent can handle, the more general I consider that agent to be. Hereinafter, when referring to an agent capable of conducting research in this paper, I mean a general and autonomous artificial researcher. If such agents were to be created, it would be accurate to say that they are agents capable of conducting research.

% I emphasize autonomy and generality because it is believed that only when an agent possesses these qualities can I say that it's not just a tool for research, but that the agent itself is capable of conducting research. Realizing such a general and autonomous artificial researcher is a significant milestone for humanity.

We still have much room for discussion about what such agents would be like, and what it means to create an agent capable of doing research. Therefore, in this paper, I present a speculative thought around the concept of artificial intelligent agents capable of conducting research. By discussing this concept, I hope to provide an opportunity to think about what we need to discuss in the future as we work towards realizing agents capable of conducting research \footnote{
The manuscript is being managed on GitHub \url{https://github.com/t46/research-automation-perspective-paper}. The version you are currently reading is a provisional one, and I plan to continue updating it regularly.
}. 

First, I will conceptually explore how we could characterize the activities called research. I hope this preliminary discussion helps us to consider what would it be to create an agent capable conducting research and what would be required to discuss to create the agent. Next, I will discuss elements that are widely considered essential to research. Specifically, I will explore the nature of question construction, hypothesis generation, and hypothesis verification, as well as the possible challenges for autonomously performing these tasks. Finally, as a reference, I will share some simple, preliminary ideas for prototyping aimed at identifying challenges in developing an agent capable of conducting research.

The speculative discussion developed in this paper is still in its early stages and is provisional. Due to the breadth of the subject matter and the author's capabilities, it is undeniable that each point of discussion may be somewhat superficial or incorrect. I plan to update these discussions, as I continue to conduct further research. Therefore, if anyone notices any points that should be improved, errors, or topics that would be beneficial to discuss, I would greatly appreciate your feedback.
% この論文で展開する思索的な議論はまだ初期的なもので暫定的なものです。議論の対象の広さから、一つ一つの議論が浅くなることも否めません。今後調査を重ねていく中でこれらの議論をアップデートしていきたいと考えていますので、何か改善すべき点や誤っている点、議論した方が良い点についてお気づきの方がいらっしゃいましたら、ぜひフィードバックを頂けますと幸いです。
