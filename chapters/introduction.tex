\section{Introduction}

\textcolor{red}{AI を Agents に変える}

Research has been the foundation of human progress. Humans have deepened their understanding of the world and created things never seen before through research, leading to groundbreaking innovations. Research has been considered highly intellectual human activities.

Since the beginning of artificial intelligence (AI) research, a key goal has been to develop AI that conducts research. There have been advances in systems that automate scientific judgments \cite{lindsay1993dendral}, infer scientific laws \cite{langley1987scientific}, and autonomously cycle through hypothesis and experimentation \cite{king2004functional}. Recently, with advancements in machine learning, we've seen remarkable successes in using it for scientific discovery \cite{wang2023scientific,xu2021artificial,zhang2023artificial}.

On the other hand, the quest to develop a fully autonomous AI that conducts research is still ongoing \cite{zenil2023}. ``Autonomous'' implies the absence of human intervention, prior design, or preparation. I will refer to an AI as being more autonomous if it can conduct research with less human intervention. By ``AI that conducts research,'' I mean a single system that can conduct any research. This includes exploring various fields like history, mathematics, or physics. I label such AI as a ``general artificial researcher.'' The more fields of research a single AI can handle, the more general-purpose I consider that AI to be. Hereinafter, when referring to an AI capable of conducting research, I mean a general and autonomous artificial researcher.

I emphasize autonomy and generality because it is believed that only when AI possesses these qualities can I say that it's not just a tool for research, but that the AI itself is capable of conducting research. Realizing such a general and autonomous artificial researcher is a significant milestone for humanity.

In this paper, I present a speculative discussion on AI capable of conducting research, aiming to set the stage for discussions of limitation and possibility for its realization \footnote{
The manuscript is being managed on GitHub \url{https://github.com/t46/research-automation-perspective-paper}. The version you are currently reading is a provisional one, and I plan to continue updating it regularly.
}. 
First, I will explore what would be activities called research. I hope this discussion helps us to clarify what would it be to create an AI capable conducting research and what would be required to create the AI. Next, I will discuss in more detail the elements that are considered essential to research. Specifically, I will explore the nature and role of question construction, hypothesis generation, and hypothesis verification, as well as the challenges involved in creating AI capable of performing these tasks. Finally, I will briefly discuss a first step to create such an AI.