\chapter{Introduction}

\section{Towards Understanding this Universe}
It has been a long-standing goal of humanity to understand nature, the universe, and the world. Through this intellectual curiosity, mankind has made remarkable intellectual advancements and uncovered astonishing findings about nature. While perfect understanding of this universe may be infeasible, the desire to know as much as possible about as many things as possible in this universe is a shared sentiment among researchers.

However, it seems that there are limitations to achieving this goal solely through current human capabilities. This is because humans have several cognitive constraints when it comes to understanding phenomena. For instance, there is a limit to the amount of information humans can process, and comprehending highly complex, nonlinear, and non-equilibrium systems in their entirety is challenging. Such constraints appear to narrow the scope of nature that can be understood. \textcolor{red}{TODO: update the explanation of cognitive limitation}

Furthermore, there is a possibility that the methodologies of knowledge production employed by humans currently may not be optimal. This is because the research practices we have developed are heavily influenced by historical and societal factors. Many of these practices have emerged in a relatively short span of history, and there is still a vast unexplored territory in the methods of conducting research.

We believe that artificial intelligence capable of autonomously conducting all research holds great potential in overcoming these challenges and getting closer to the goal. Firstly, AI is not constrained by a physical device like the human brain, enabling processing speeds and capacities that are orders of magnitude beyond human capabilities \cite{hope2022computational,kitano2021nobel}. Secondly, by optimizing AI for knowledge production, it becomes possible to generate knowledge more efficiently, without being constrained by extraneous factors commonly encountered in human research. Lastly, and most importantly, AI has the potential to possess a different understanding of nature compared to humans. Moreover, AI might have the capacity to comprehend nature more extensively than humans. If that's the case, it seems essential to develop autonomously researching artificial intelligence if one aims to acquire as much knowledge as possible about this universe. Therefore, we propose the pursuit of autonomous AI researchers to realize these possibilities.

\section{Purpose of this Perspective Paper}
The purpose of this perspective paper is to provide a stepping stone towards achieving a general and autonomous artificial researcher as quickly as possible. It aims to offer information that enables all those pursuing this goal to focus on essential issues, avoid reinventing the wheel, and maximize their own potential. This is because we believe that enabling the full potential of brilliant researchers worldwide is the most crucial for accomplishing this objective.

To achieve the purpose of this perspective paper, we will discuss three key points in chapters 2, 3, and 4, respectively: What is the goal, where are we currently standing, and what to do to achieve the goal. First, in chapter 2, we will delve into the discussion of ``What is the goal'' to make the common objective clearer for everyone involved. By doing so, researchers and engineers all over the world are able to devise methods to achieve this goal even before looking at our proposals in this paper. Furthermore, this discussion aims to clarify the limits and possibilities of a general and autonomous artificial researcher. Specifically, we will thoroughly discuss ``What is research?'' as it is essential to understand what a ``general'' artificial researcher entails. Starting with the definition of research, we will examine the functions that various research activities serve and discuss the implications of achieving this through artificial intelligence.

Next, in chapter 3, we will discuss ``Where are we currently standing'' to work towards a state where researchers pursuing research automation no longer need to gather preliminary information on their own. This is expected to enable new researchers engaged in research automation to concentrate on the actual tasks they should be working on and reduce the risk of reinventing the wheel. To accomplish this, we conduct a literature review on research automation. We will organize studies related to research automation, discuss their relationships and current status. In this survey, we will strive for as much comprehensiveness as possible, including anything relevant to research automation, no matter how superficial, as we believe that once researchers become aware of the research area, they can delve deeper into specific aspects on their own. Additionally, we observed a lack of information sharing among various individual efforts related to research automation, and by addressing these diverse approaches in a unified manner within one paper, we aim to promote information exchange between different fields.

In chapter 4, we will present our proposals on ``What to do to achieve the goal.'' Building upon the clarity of the goal from chapter 2 and the understanding of the current state from chapter 3, we will discuss how to bridge the gaps between them. We will also explore the directions and specific approaches for research and development, suggesting the policies and avenues that may be most effective. Through these discussions, our aim with this paper is to providing useful information for as many individuals striving for research automation.


\section{Pursuing Better Research}
Please emphasize once again that what we are aiming for is to achieve a better way of knowledge production. The realization of general and autonomous artificial intelligence is a means to that end. Therefore, even if certain research practices are widely adopted at present, we will strive for better methods if they are not essential for knowledge production and if there are superior alternatives. In other words, we propose to pursue the realization of artificial intelligence that enables a different and improved way of knowledge production, rather than simply automating the current research practices.

While the practices of research developed by humans are highly sophisticated, robust, and productive, it is not necessarily true that all of them have reached the absolute optimum way of doing research. Firstly, research practices are strongly influenced by history and society. For example, peer review is widely accepted today, but it is said to have gained such dominant status because foundations in Cold War-era U.S. demanded peer review to ensure accountability \cite{baldwin2018scientific}. Moreover, the fact that research outcomes are still represented in the form of papers is clearly a remnant of the era when print was predominant. These practices might have been optimal under the social conditions and technologies of that time, but they may not necessarily be optimal in the present, as society has changed a lot. Additionally, meeting societal demands does not always lead to the optimization of knowledge production. Secondly, it seems that we sometimes lack a widely shared understanding of how to optimally carry out certain aspects of research. For instance, it is important to formulate good research questions and hypotheses, but there are still many aspects about how to do this effectively that remain unclear, and it appears to rely heavily on individuals' tacit knowledge. In such a situation, simply replacing current practices with machines may not guarantee a good knowledge production process. Thirdly, as Nielsen and Qiu say \cite{nielsen}, it is believed that we have only explored in a very limited subspace in the space of possible research practices. The establishment of current research practices is a very recent event in human history, and we may not yet have arrived at the optimum way of research. \textcolor{red}{TODO: Add examples.} Finally, as we mentioned above, current research assumes that humans are the main creators and consumers of knowledge. This naturally imposes human cognitive constraints, which may significantly limit the range of activities that can be conducted for knowledge production. Based on these reasons, we aim to progress the discussion in a way that allows us to identify what is fundamentally essential for knowledge production, while also drawing inspiration from the positive aspects of past practices. We will not be bound by the current approaches but instead strive to clarify what is truly crucial for research.

\section{Call for Cooperation}
I would be delighted if all of you could contribute to writing this research paper. The initial draft of this paper was created by a novice machine learning researcher named OO. We have made efforts to write the text as carefully as possible, but due to his limited research experience, He may not be able to provide profound insights into the topic, and there might be inaccuracies in his understanding of fields that are not his expertise. In particular, the literature review chapter prioritizes comprehensive coverage of the literature, which means that the evaluation of individual references might not be appropriate. Also, there are naturally limitations and biases regarding the scope of papers that can be covered and the research areas targeted. Therefore, if you find any errors, missed literature, or points for improvement in the paper, we would be grateful if you could let us know.

The manuscript is being managed on GitHub\footnote{\url{https://github.com/t46/research-automation-perspective-paper}}, and contributions from everyone to the paper are welcous. The version you are currently reading is a provisional one, and we plan to continue updating it regularly. If you have any suggestions for modifications, it would be incredibly helpful if you could submit a pull request to the repository.

Moreover, this paper is licensed under the OO license, so feel free to make any modifications you see fit. If you wish to follow the main structure, you can fork the repository, or if you have a different approach in mind, you can create an entirely new project.

We hope this paper can contribute in some way to the advancement of automation in research. Thank you.


% Additionally, this paper is intended to be updated on an ongoing basis. If you find points that you think are lacking or inaccuracies in understanding, please send a Pull Request on GitHub. Based on that, we will revise the content of the paper as needed.

% \section{Others}

% The purpose of this perspective paper is to discuss the possibility of automating this activity of research, that is, the realization of intelligent agents capable of conducting research autonomously. After summarizing attempts at research automation so far, we intend to present our personal perspective on promising ideas, necessary elements, and ways to proceed in order to create artificial researchers. In particular, this paper is aimed at machine learning researchers and developers, with the goal of increasing the number of colleagues aiming for research automation by providing concrete possible actions.

% One thing to note here is that our aim is not to automate or substitute the current research tasks, but to create intelligent agents capable of conducting research. In other words, as long as it is research, automated research may at first glance seem to be far removed from the current research activities. This is because the current research activity may not necessarily be the absolute optimal form in light of its purpose. 

% First, the optimal form is always influenced by the context of history. Of course, the optimum in the era when letterpress printing was mainstream is different from the optimum in the era when the Internet became infrastructure. Research itself has changed its form over time, and the optimal form of research in the present or future can naturally change. Moreover, as Nielsen and Qiu say \cite{nielsen}, it is believed that we have only explored a very limited range regarding the practice of research. The establishment of current research practices is a very recent event in human history, and we may not yet have arrived at the optimum way of intellectual production. Furthermore, current research assumes that humans are the main creators of knowledge. This naturally imposes human cognitive constraints, which may significantly limit the range of activities that can be conducted for knowledge production. 

% Therefore, instead of thinking about how to automate and streamline current tasks, we discuss the possibility of realizing intelligent agents that can autonomously perform the fundamental and core elements of research activity in a more optimal form, and what is necessary for this.

\textcolor{red}{TODO: Add lots of high-level illustration}