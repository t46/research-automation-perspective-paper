\chapter{What is Research?}
\label{chapter-what-is-research}

To create an agent capable of conducting research autonomously, it is crucial to understand what research is in the first place. Therefore, in this section, I would like to discuss the characteristics of what is called research.

Our aim here is not to identify a universal, historically independent, and absolutely singular definition of what research is. It seems almost impossible to specify such a thing \cite{chalmers2013thing,sep-scientific-method}. I do not totally believe that I could provide an answer to a question that has yet to be resolved even within the philosophy of science.

Rather, my goal is to endeavor to describe various characteristics of research to the extent that it can provide a starting point for engineers and researchers who aim for realizing AI that can conduct research autonomously.In the following sections, I first discuss a possible working definition of research.

\section{Defining Research}

\subsection{Research as Knowledge Production}

A widely accepted definition of research is that research is the act of generating new knowledge. For instance, in physics, a new explanation for a phenomenon is produced, in mathematics, a new proof for a theorem is presented, and in engineering, a new design blueprint for creating something is generated, all as new knowledge. I will adopt this definition as the working definition in this paper. In particular, I consider that \textbf{research is to produce new knowledge for a society.} I included ``for a society'' because knowledge appears to be relative to society, as I will discuss later.

\begin{figure}[htb]
    \centering
    \includegraphics[width=\linewidth]{figs/definition.jpg}
    \caption{Definition of Research}
    \label{fig:definition}
\end{figure}

% \subsubsection{Some Notes on Research and Science}

The reason why I deliberately use the word ``research'' instead of ``science'' is because I want to include fields like engineering, humanities and arts, which are not typically referred to as science, within the scope of automation in the long run. Science refers to a methodology for generating knowledge, and I believe that new knowledge is not necessarily produced only by scientific methods. I believe that so-called humanities and arts also share some commonality of producing new knowledge. Therefore, the definition of generating new knowledge can be said to encompass these fields as well.

Of course I admit that science is the most rigorous and reliable framework for knowledge production. Because of its power and popularity, most of the existing analysis on research is about science. My discussion would also be centered around the discussion of science, though I try to present the view not limited to science. 

Please note that this definition is provisional. By viewing research as the production of knowledge, I can indeed characterize the wide range of current research activities of humans comprehensively, which is why I adopted this definition. However, as I will see later, difficulties also arise when defining it this way. Perhaps in some cases it might be better to focus on natural science and define research more narrowly as something like ``finding patterns in nature''. Defining what constitutes research is a challenging issue, and I believe that researchers should spend time discussing this definition, as I aim to create AI that can conduct research. The purpose of creating AI capable of research and how I should perceive the act of research should be discussed further in the future.

\subsection{Knowledge Production as Belief Revision}
I have defined research as the generation of new knowledge. Then, what exactly is knowledge, and what does it mean to produce knowledge? I will explore this question in this section. 

Defining knowledge and knowledge production rigorously is a philosophical debate that has not yet been settled \cite{sep-epistemology}, and I won't delve into it deeply here. Instead, I would like to provide some primitive ideas that can serve as a starting point for further discussions on how to realize an artificial researcher.

\subsubsection{Knowledge as Belief}
The question of what is the thing called knowledge has been a subject of debate for a long time in the field of \textit{epistemology}, which is one of the branches of philosophy. In epistemology, knowledge has been traditionally considered to be \textit{justified true belief (JTB)} \cite{sep-epistemology}. The term ``true'' is difficult to define rigorously, but for the purpose of discussion, let's think of it as something being fact. ``Belief'' can be provisionally understood as someone's thought or conviction about something. And ``justified'' means that it is deemed reasonable to hold such a belief. The meaning, necessity, and sufficiency of justification has been discussed in epistemology as a central point of contention. Since I am not well-versed in epistemology, I will refrain from delving into detailed discussions on that topic as it would go beyond the scope of this paper.

Of course, there is much debate about whether JTB truly is knowledge, and many philosopher agree that JTB is not the sufficient condition of knowledge. While these characteristics are not be sufficient conditions, there seems to be generally some agreement that they may be necessary. As epistemological discussions revolve around these notions, they seem to provide a reasonable starting point just for the purpose of my discussion. Therefore, for the sake of further argumentation, let's tentatively understand the generation of knowledge as the process of justifying, validating, or confirming beliefs about a truth.

\subsubsection{Knowledge Production as Belief Revision}
I believe that viewing research as an update of beliefs is somewhat reasonable. For example, let's consider a scenario where a hypothesis is proposed for a certain phenomenon, and then it is subjected to a test. This is a common practice in research. Now, suppose that through this testing, the hypothesis has been validated. This means that as a result of the test, my belief that the hypothesis is true has been strengthened. It could be said that this is an update of beliefs. Therefore, the view that research involves the update of belief seems to be reasonably valid.

% In the nature of research, which aims to generate new knowledge, it is necessary to engage in inference and justification regarding the unknown. Even if a hypothesis is not explicitly stated, some form of implicit hypothesis generation and testing should occurs in research. Therefore, the view that research involves the update of belief seems to be reasonably valid.

Let us explain a bit more about the reason why I deliberately say ``my belief that the hypothesis is true has been strengthened'' instead of just claiming that ``hypothesis turned out to be true.'' This is partly because, in almost all fields that rely on empirical methods, except for deductive disciplines like mathematics and logic, it is often impossible for verification to definitively determine if a hypothesis is true or false. Most research evaluates the validity of a hypothesis based on observations, relying on inductive reasoning \cite{sep-scientific-method}. However, inductive reasoning is unable to determine the truth or falsehood of propositions as deduction does without assuming some assumptions \cite{sep-induction-problem}. Even though we cannot judge if the hypothesis is true or false, we can still argue that the evidence obtained through verification makes us feel more confident that the hypothesis is true or false. This belief update holds for both empirical and deductive research. Thus, I do not say 3that a hypothesis became truth but instead say that my belief is updated.

I don't totally believe that because inductive reasoning doesn't rigorously determine truth or falsehood in the same way deduction does, it renders the reliance on inductive reasoning meaningless. This is because the assumptions that are said to make the inductive reasoning valid would sound natural to most humans and leave little room for doubt. As an example, it is said that we need to assume that ``under the same conditions, the same phenomenon will continue to hold'' (principle of uniformity of nature) \cite{sep-induction-problem}, which is an intuitive and natural assumption without which we could hardly even go about our daily lives.\footnote{
I understand that justifying the validity of inductive reasoning based on our experiences would be circular reasoning. The problem of what assumptions make us inductive reasoning consider rational is a challenging unanswered philosophical issue, which is beyond the scope of this paper. Thus, I will refrain from delving into the details here and leave it for future discussion.
} 

For the same reasons, I do not believe that the concept of belief being subjective makes it unsuitable for characterizing research. What I want to emphasize here is that research can be seen as linking or replacing the weak belief in the plausibility of newly conceived hypotheses with such extremely strong beliefs. For example, believing in the effectiveness of statistical methods is strongly related to believing in the effectiveness of inductive reasoning. Therefore, if the validity of a hypothesis is confirmed using statistical methods, we would believe it as highly reliable. Hence, considering research as the updating of beliefs can be reasonably felt, and I don't intend to argue that research is meaningless just because it is the belief revision.\footnote{
I naively assume that beliefs rooted directly in perception are more robust, and the notion that anchoring a belief on those foundations enhances its reliability. However, the question of how to justify beliefs is a complex problem with extensive discussions \cite{sep-epistemology} (my position seems to align somewhat with what is called \textit{empirical foundationalism}).Due to my limited philosophical knowledge and the ability to delve deeper into philosophical discussions, this paper will not further pursue these arguments.
}


% Therefore, while I don't know what exactly constitutes the basis of inductive reasoning, it would seem to be a deeply rooted and unwavering belief that is grounded in my perception and experiences. 

% \textcolor{red}{TODO: Add excuse for empirical foundationalism}
% The idea that there is ``strength'' in beliefs may be related to a position called \textit{foundationalism} in epistemology, which claims that ``\textit{my justified beliefs are structured like a building: they are divided into a foundation and a superstructure, the latter resting upon the former. Beliefs belonging to the foundation are basic. Beliefs belonging to the superstructure are nonbasic and receive justification from the justified beliefs in the foundation.}'' \cite{sep-epistemology}. In particular, I believe that the more fundamental beliefs to which these beliefs belong are rooted in human perceptual experience and empirical knowledge. This position is commonly referred to as \textit{empirical foundationalism} \cite{sep-epistemology}.

\subsection{Novelty of Knowledge}
No matter how firmly a belief is confirmed, if it is already known, it cannot be called research. Therefore, we have defined research as the process of producing ``new'' knowledge. In this section, I will briefly discuss what would it mean for knowledge to be unknown or novel.

As mentioned later, research can be described as the act of posing questions, proposing hypotheses in response to them, and then verifying those hypotheses to update beliefs on whether the hypotheses are correct. Therefore, it can be said that when knowledge born from research of a question is considered new, it refers to a situation where there has been no hypothesis that everyone believes to be sufficiently plausible for the question \footnote{
While presenting a question that no one has posed ensures that the knowledge is new. Even in such cases, no one also yet knows which hypothesis is correct for that question. This is why I have mentioned only about the unkownness of hypotheses
}. For example, we do not know how to create a general-purpose artificial intelligence, which can be said to mean that we do not have an answer (a hypothesis with high confidence) to the question, ``How do you create a general-purpose artificial intelligence?''.

% I said in the paragraph above that ``research is the process of transforming the unknown into the known.'' However research can also be seen as the updating of beliefs, as I have explained. Therefore, the binary depiction of an object suddenly transitioning from the states of unknown to known does not seem appropriate. Rather, it seems more reasonable to consider that beliefs continuously change and I just call some group of belief states unknown and others known, for convenience.

% I don't know precisely what it means to be unknown. This is a difficult problem, but let's consider it naively. 
% I consider a knowledge to be unknown when for a question a subject lacks any hypothesis which he/she has a JTB that the hypothesis is true. For example, I do not have the answer to the question of ``How to realize an artif icial general intelligence.'' So the knowledge of ``the way to realize an artificial general ingelligence'' is unknown. \textcolor{red}{TODO: Add explanation}

In research, a single verification does not always immediately turn a hypothesis into true or false. Rather, hypotheses that withstand repeated verifications through various experiments by different researchers gradually come to be regarded as more plausible. Therefore, it seems to me appropriate to say that research transforms a hypothesis with low confidence into one with higher confidence, rather than turning the unknown suddenly into the known. In that sense, it seems somewhat justified to describe new knowledge as something for which there wasn't a highly confident hypothesis before.

% The expression ``unknown'' could be more appropriately expressed as ``high degree of unknownness''. Because knowledge production is a continuous concept of belief updating, therefore, unknownness is also a continuous concept. In reality, there are multiple hypotheses with varying degrees of certainty concerning a particular question. The state of knowledge being unknown can be expressed as not having any hypothesis among these hypotheses with a particularly strong level of justified certainty.

% First, research begins with a particular question. The state of not knowing the answer to this question is what I consider the state of being unknown. In other words, it can be thought of as a state where I don't know what the candidate hypotheses, which are the potential answers, are like, or a state where I know the candidate hypotheses but don't know their plausibility. These are states where I have not been able to find hypotheses or sets of hypotheses that can be assigned a particularly high degree of confidence from the set of potential answers to a given question. Therefore, provisionally and casually, it might be said that the state of ``not having highly confident beliefs (or a set of beliefs) for a particular question'' is unknown, and the state of having highly confident justified beliefs is known.

% Of course, there are issues with this clarification. For example, it is unlikely that I can select a single highly confident hypothesis from the entire set of possible hypotheses. Also, rather than feeling equally confident about all possible hypotheses, it seems that I implicitly distinguish between hypotheses that seem relevant and those that do not. Furthermore, it is unclear to what extent I consider a state to be unknown based on the degree of confidence. However, these are highly challenging philosophical discussions, so I will refrain from delving further into them and, for now, would like to conclude with the vague and provisional definition above and move on to the next topic.

\subsection{Publicity of Knowledge}
\subsubsection{Knowledge as Human Knowledge}
The knowledge generated through research seems to be required to be knowledge for humanity. In reality, even if it is a strong belief, one person's belief is not recognized as knowledge unless other researchers gain a similar level of conviction from the research. Beliefs, knowledge, and understanding are indeed subjective concepts by nature, but it seems that we demand that they go beyond subjectivity and become comprehensible to others as well. 

Since knowledge production is belief update, generating knowledge for humanity requires updating the beliefs of all, or at least certain number of, people. \footnote{
Even if the knowledge is not immediately understandable, it should be potentially understandable to a considerable number of people. This means that the generated knowledge is not only for the society at the time of its production but also for the broader humanity, including future human societies. For instance, it might be challenging for children to determine whether the research outcomes in physics qualify as knowledge. However, by studying physics extensively, they may eventually comprehend cutting-edge research in the future. 
} \footnote{
Because it's impossible for different individuals to have exactly the same belief, as long as a justification can update the beliefs of different individuals in a similar direction in some way, I think it sufficient to be regarded as updating shared beliefs.
} 
It seems that researchers achieve this by verifying hypotheses in a manner that any human could find convincing. For instance, while most empirical scientific research verifies hypotheses through hypothesis testing, it seems that this is because results deemed plausible through statistics or inductive reasoning are thought to be plausible by virtually any human.

% Therefore, I believe research is the act of not only changing an individual's belief but also changing multiple individuals' beliefs. 

% Although I don't think it is possible for multiple people to hold exactly the same belief or for their beliefs to change in exactly the same direction, I at least need to use a way to change a collection of human beliefs in a similar direction.

In other words, this can be said to underscore the plausibility of the belief that ``a hypothesis is true'' with the extremely robust belief that ``inductive reasoning is valid.'' These robust beliefs seem to have been shaped through the process of evolution for the long time, being rooted in humans biologically. It seems that because science appeals to such fundamental beliefs inherent in humans as biological entities, its verification manages to convince many people.
% are rooted in our biological structures and perceptions acquired through the processes of evolution and development, as briefly introduced in the preceding section. For instance, beliefs such as the validity of logical reasoning, the uniformity of nature, and the tendency to find something more credible with an increase in observations fall into this category of beliefs. 

% And I believe that I achieve this by reducing hypotheses to strong convictions that everyone, regardless of individual differences, possesses as human beings. For example, assumptions underlying inductive reasoning, as I described earlier, will fall into this category. Another strong conviction humans believe in is that the more I observe an increasing number of results derived from a certain hypothesis, the more reliable and certain that hypothesis feels. Research need to be objective because it's required to generate knowledge for humanity, and I believe it is because I strive to reduce hypotheses to such strong convictions that it is called objective. 

\subsubsection{Knowledge as Non-Human Agents' Knowledge}
I said that research is to generate knowledge for humanity. However, this is merely because humans have been the ones conducting research thus far. I believe that in principle there could be knowledge for beings other than humans and hence research for them as well. A group of agents, each capable of holding certain belief states and having some level of similarity, seem to be able to hold a shared belief. If these agents have a way to ground their belief that a hypothesis is true to their shared belief, then this can be regarded as research within that society. Therefore, I define research as producing new knowledge for a society, which is not necessarily human society but also includes non-human agents'.

\begin{figure}[htb]
    \centering
    \includegraphics[width=\linewidth]{figs/shared_belief_revision.jpg}
    \caption{Belief Revision}
    \label{fig:shared_belief_revision}
\end{figure}

There are two implications here. Firstly, fully autonomous artificial researcher would produce nonsense for humans. This is because, even if a group of agents grounds their justifications in solid shared beliefs, it would be mere gibberish for humans if it is not shared among humans. \footnote{
This is akin to Khun's incommensurability \cite{kuhn1962}. 
} A straightforward way to solve this problem is to align the verification methods with those of humans. If we machines to execute even the verification methods entirely autonomously, it is essential to ensure some commonality with the methods humans use, which seems extremely challenging. This issue is essentially related to the problem of aligning human values with AI, and it is an extremely difficult problem.

% This implication means that if I want non-human agents to produce meaningful knowledge for humans, I need to ground it not in their shared beliefs but in human shared beliefs, or somehow ensure a commonality between these two sets of shared beliefs. If I aim for the former, the non-human agents may no longer be considered as conducting research in the same sense as humans. Additionally, regardless of whether I pursue the former or the latter, the problem remains of how to impart human shared beliefs to these agents, which is a fundamentally challenging issue. This issue is essentially related to the problem of aligning human values with AI, and it is an extremely difficult problem. I shall discuss this point separately.

% Certainly, even if I could share belief systems, if knowledge or the processes involved in acquiring knowledge were expressed using entirely incomprehensible systems for humans, it would still be meaningless to us. Therefore, if I seek to generate knowledge that is practically meaningful to humans, it seems essential for both humans and these systems to utilize, at the very least, a common representation format for knowledge, such as human language.

The second implication is that the artificial researcher would not even produce knowledge meaningful for understanding nature. This is because while human beliefs have been shaped to be consistent with nature through interactions with it, the beliefs of machines are not necessarily so. Through the process of evolution, humans have spent vast amounts of time interacting with and modifying their bodies to survive in nature. Our beliefs are formed in such contexts, and the beliefs we currently possess are likely advantageous for living in nature. Therefore, I believe that our strong belief serves as a reliable foundation for understanding nature. Machines do not possess bodies embedded with such interactions with nature. Therefore, just because they have strong convictions does not guarantee that these will serve as a reliable foundation for explaining nature. This seems to be a more serious issue than the first implication.

% In other words, the essential information for explaining and predicting nature is already internalized in my strong beliefs. It is precisely because I can trace these beliefs back to such interactions that I have been able to produce knowledge that deepens my understanding of nature. Now, suppose an agent has had no interactions with nature whatsoever until now. In that case, it seems highly unlikely that such an agent could acquire beliefs that are in harmony with nature. This goes beyond merely stating the importance of embodiment for AI. It suggests that an immense amount of interaction and learning, enough to form highly strong beliefs and internalize nature, is required and that learning these elements are inevitable for comprehending the natural world. 

\begin{figure}[htb]
    \centering
    \includegraphics[width=\linewidth]{figs/incommensurability.jpg}
    \caption{Incommensurability}
    \label{fig:incommensurability}
\end{figure}

% First, knowledge is based on beliefs, so if a non-human agent holds beliefs, it would be reasonable to consider that they possess knowledge, even if they may not resemble human knowledge on the surface. The nature of beliefs is a complex matter to define concretely. However, even in current machine learning, models have confidence levels and prediction errors, which I believe are not completely unrelated to what I consider beliefs. Furthermore, even if the knowledge is not understandable to humans, if it is understood among a group of agents, meaning it is reduced to a shared belief, then it can be called ``knowledge for that group of agents.'' If a group of agents has means to update a belief to such a strong shared belief, then I consider it as research. From these reasons, I believe that research is conceivable for entities other than humans. And it feels like this is a highly significant conclusion in terms of contemplating the development of intelligent systems capable of conducting research.

% \textcolor{red}{TODO: Add relations with relativisim, maybe added to alignment section?}
% This position may have some similarities with a famous stance in philosophy of science that argues everything is, at its core, subjective, relative or a matter of belief \cite{chalmers2013thing,kuhn1962,howson2006scientific}. However, what I want to emphasize here is that I think the characteristic of research lies in tying beliefs about an object to stronger beliefs. And I believe that it is this strength of conviction, which seems self-evident to many people, that lends objectivity and persuasiveness to the claims of science.

% This is not definition but practically I think that one characteristics of research is its rigor of confirmation method. Firstly, you may wonder if anything unknown would suffice. I believe that research does not choose its subject and anything unknown is acceptable. Rather, what's important is the strictness of the methodology – whether the unknown truly became closer to the known, whether it was concluded into a stronger belief. Generally, the method employed by what I call research is designed with extreme precision, and as a result, it seems to withstand rigorous evaluations. This can be considered a major feature that distinguishes research from other various activities.

\subsection{Alternative Definitions}
I have adopted the definition that research is to produce new knowledge for a society. While the current definition of research was useful for a comprehensive discussion, adopting this definition led to counterintuitive conclusions and seemingly intractable problems. Therefore, there might be debate over whether we truly want an artificial intelligence capable of research in this sense. In this section, I would like to discuss the possibility of alternative definitions of research.

\subsubsection{Limited to Human Knowledge}
One of the biggest issues is that a fully autonomous artificial researcher is likely to produce knowledge that is meaningless to humans, or even more so, not connected to an understanding of nature. Thus, one alternative proposal is to limit what an artificial researcher produces strictly to knowledge that is relevant for humans. Notably, this problem arose entirely from allowing artificial intelligence to autonomously construct even the foundation for validation. Therefore, what this alternative definition demands is to ensure that the foundation for validation remains fundamentally human.

This is somewhat less restrictive than ``producing knowledge that humans can understand''. That's because it is conceivable that there is a procedure where humans might be convinced something is true if that is confirmed by the procedure even if they can't fully grasp the details of the mechanism of the procedure It seems that what most people desire is an artificial intelligence that produces knowledge understandable to humans, so for most people, this definition should be more than sufficient.

One downside of adopting this definition is that it might preclude the production of knowledge that, while certainly explaining nature, cannot be verified within the framework of human validation. If one believes that nature transcends human cognitive limitations, it seems reasonable to assume that such knowledge exists. While this may not be a concern for most people, it seems to be a significant constraint when considering ``how far nature can be understood beyond human limits.''

% This all arises due to the expansion of the definition of research to include non-humans. Consequently, it is possible to aim for a definition of research strictly limited to knowledge production for the human society, and to ensure that artificial researchers exclusively engage in research for the benefit of humans. This appears to be more of a decision-making issue about what kind of future I want to pursue, rather than determining which definition of research is correct. Alternatively, it might be worth reconsidering the definition of knowledge production as the mere updating of beliefs. Engaging in discussions about this problem seems to be of great importance for the future of human society.

\subsubsection{Knowledge Consistent with Nature}
There is also the perspective that if we truly understand nature even if it's beyond human comprehension, then that's not a problem. However, there were concerns that machines, not having interacted with nature like humans have, might produce knowledge based on their beliefs that does not lead to a true understanding of nature. Therefore, the second alternative is to impose the constraint that the belief system must always be consistent with nature. In this case, it would be necessary to first allow the machine to interact with nature and then have it form a belief system that aligns with nature. However, I don't know what that might look like or how it would be constructed.

If it turns out that the knowledge system might not always be comprehensible to humans, then we face the issue of being unable to confidently verify whether the artificial intelligence is truly producing meaningful knowledge in terms of understanding nature. Thus, if we adopt this definition, the alignment issue I mentioned earlier still remains. While there's the advantage that it might enable an understanding of nature beyond human limits, many people might not think it worth adopting this definition.
 
\subsubsection{Research as New Pattern Discovery}
One of the main reasons the definition I adopted lead to counterintuitive results may be because I regard research as the production of ``knowledge'', and knowledge as ``belief''. I interpret research as the updating of beliefs just because it is consistent with human research practices so far. Rather, an essential aspect of research seems to be discovering or creating something ``novel'', and whether or not that constitutes ``knowledge'' seems secondary.
% knowledge is JTB, which implicitly assumes that the generation of knowledge, i.e., research, aims at acquiring a truth. 

Given this, it might be worthwhile to simply define research as the ``discovery of new patterns (in nature)''. This definition seems to at least encompass our current research endeavors. It emphasizes the discovery of new things, is not dependent on the subject, and, (by specifying ``in nature'',) appears to align with the goal of understanding nature.

This might be too abstract, so it would be necessary to re-analyze what this definition truly means. I'd like to consider that discussion as future work.

\subsubsection{Research as Question Answering}
Defining research as merely a question-and-answer process might be one approach. However, a distinguishing feature of research is that the answer is unknown to anyone. Nobody knows the correctness of the answer, but it is indirectly provided by the subject of study, such as nature. While it's unclear to what extent this formulation can alleviate the issues with the definition we've adopted, it seems worth considering for furhter study.

\subsubsection{Pragmatist Epistemology}
Up to now, following the tradition of epistemology, we have assumed knowledge to be JTB. That is, we have been evaluating the justification of a belief based on how close it is to the truth. However, there is also a position that evaluates beliefs not by whether they approach truth but by whether they are useful for human or not. This position is known as \textit{pragmatist epistemology}. 

From this perspective, it could be possible to argue that the outputs of current machine learning models, for example, may be considered knowledge since it shows high predictive performance, leading humans to the desired behaviours. In this sense, this position is highly compatible with statistical machine learning \cite{otsuka2022thinking}.

As such, this pragmatism presents a different view of research from that of majority of researchers, and it could be seen as advocating a redefinition of the act of research. Many researchers may not accept this view. Nevertheless, in recent years, deep neural networks have made rapid advancements and have produced many ``useful'' outcomes in research, which has made the debate on whether to consider their outputs as knowledge much more relevant and realistic than ever before.

In this paper, I take the position that knowledge is JTB, and research is an endeavor to approach a truth. Therefore, I will not address this issue in the following sections that much. However, we must seriously consider the implications presented by deep neural networks and delve deeper into the discussion of how we should define research, what constitutes research, and what purpose research serves.

\subsection{Interim Summary}
In this section, I have discussed a provisional working definition of research. I started with the naive intuition that research is the endeavor to generate new knowledge for a society. I then explained that knowledge is belief, the production of knowledge involves updating beliefs, and the produced knowledge needs to be novel and supported by the common strong beliefs of a community. Lastly, I discussed based on these conclusions the possibility of research conducted by agents other than humans.

% Reflecting on what research entails is highly important as it provides clarity on what we should truly make. The discussion in this chapter suggests us that we need to discuss how to instill these abilities to artificial intelligence so that we realize an AI to conduct research.

The definition discussed here is merely a provisional one based on the naive intuition. By combining insights from philosophers, scientists, and all working researchers, we can engage in a deeper analysis of the definition of research, developing more fruitful and reliable guidelines for our goal.

% The reason I deliberately distinguished between mere knowledge and knowledge for humanity is because I believe that there could be knowledge for beings other than humans. As I have repeated, understanding is subjective, so if machines start conducting research in the future, it is natural to think that there will be unknowns for machines and beliefs for machines. Therefore, it's possible that I could live in a world where humans produce knowledge for humans, machines produce knowledge for machines, I could live in a world where knowledge is produced for all agents including machines and humans, or I could continue living in a world where knowledge is produced solely for humans. In this sense, I believe one of the problems that will be questioned in the future is whose objectivity and whose belief I am talking about. I will discuss this point in more detail later.

%%%%%%%%%%%%%%%%%%%%%%%%%%%%%%%%%%%%%

% When a hypothesis survives a test, it was said that my belief in the likelihood of the hypothesis strengthens. Empirical science implicitly assumes a principle that relies on inductive reasoning as the basis for these tests. For example, I believe that if the number of observations consistent with a certain claim increases, that claim is more likely to be valid (this is called the principle of confirmation). Additionally, I hold the belief that unless other factors change, what has held true so far will continue to hold true (this is called the principle of uniformity of nature). These principles of confirmation and uniformity are the foundations of inductive reasoning, and they are unavoidable in empirical science. However, both the principle of confirmation and the principle of uniformity are merely beliefs, and there is no guarantee anywhere that these are ``correct''. The reason these can serve as the foundation for testing is because these beliefs are far more solid than the belief in the likelihood of a hypothesis that someone has just recently presented. That is, empirical science could be described as an endeavor to update the likelihood of a hypothesis by tying the belief held about a certain hypothesis to a more solid belief. While mathematics and logic are rare examples, considering that all other research endeavors are fundamentally empirical, \textbf{it may be possible to rephrase the endeavor of research, that is, the production of knowledge, as largely an endeavor to update my belief in the correctness of a certain object.} In fact, it is widely accepted in philosophy that knowledge requires belief \cite{sep-epistemology}. More formally, it's believed that "\textit{the three conditions—truth, belief, and justification—are individually necessary}" for knowing a fact \cite{sep-epistemology}.


% \subsection{Disclaimer Regarding the Characteristics of Research}
% I have mentioned that research is an endeavor to transform the unknown into the known, but you may question what distinguishes it from other activities that appear to do the saus. This issue arises from the fact that I have not properly defined knowledge in this paper. In this paper, I won't discuss this in detail but will limit myself to a brief disclaimer.


% \subsection{The Previous Discussions Regarding the Definition of Research (Scientific Discovery).}

\subsection{Question Construction, Hypothesis Generation, and  Verification}

% \textcolor{red}{TODO: more focus on the implication for research automation}

% It is believed that research began with individual and concrete tasks. Among them, common actions were patterned and crystallized as a scientific method. I currently recognize this abstract set of behaviors as research. For example, hypothetico-deductive method and hypothesis testing are abstracted scientific method.

% Also, researchers use a research paper as a medium of knowledge transfer. Therefore, there are patterned activities related to a research paper. Examples of these include conducting surveys, gathering information from papers, and writing a thesis.

% Note that these are necessary tasks just because I use a paper as a medium of knowledge transfer, but they may not necessarily be indispensable for generating new knowledge. There are other such tasks as well. For example, peer review and fund raising are essential to current research practices in society, but they may not necessarily be indispensable for knowledge production.

% In this way, various tasks arise in conjunction with research. When considering the automation and optimization of research, it is desirable to consider streamlining all of these tasks. However, in this article, I focus on the process from determining a research topic to publishing a research paper. I will refer to this process simply as the \textit{research process} from here on.

% \subsection{Overview}

% As mentioned earlier, research is an attempt to turn the unknown into the known. Therefore, the research process can be seen as a function that takes the unknown as input and outputs the known. However, in reality, a single research paper may not be enough to turn the unknown into the known. Therefore, in practice, the research process is considered to be a procedure that takes the unknown as input, and outputs a text that describes the procedures and their results, as well as their interpretation, in order to turn the unknown into the known.

% First, let us structure the common research process. In particular, I will base the structuring of the research process on the method of empirical science, which many researches rely on as a foundation. However, I believe that this framework can be applied to other research activities, such as mathematics, as well. I will explain the reason for this later.

% The research process, especially that of empirical science, is carried out through the following steps: topic decision, hypothesis generation, verification design, verification, and analysis of experimental results. The outputs of these steps are then written into a paper, which undergoes peer review and is eventually published.

% Note that some commonly seen items, such as surveys, are not included here for a reason. First, as mentioned earlier, gathering information from papers is only a means of knowledge transfer through the use of a thesis. Second, information extraction from papers can be done at any stage of the research process. Thus, I believe that processing related to a paper, such as \textit{reading papers} and \textit{writing a paper}, needs to be considered separately from the aforementioned research process.

% \subsection{Overview}

% \textcolor{red}{TODO: reconsider the research process, structure of knowledge production system, and the scope of this paper}

The definition of research stated in this section is somewhat too abstract to serve as a concrete guideline for building something based on it. Moreover, it feels distant from the research practices employed by humans, making it challenging to immediately connect our standing point to with the goal. While the ultimate goal is to achieve a fully aunotomous system for conducting research, it is practical and useful to start by discussing how to realize individual sub-processes within the research process. Breaking down the research process into partial processes will make the functions that need to be achieved much clearer.

Therefore, I view research process as a \textit{knowledge production system} and attempt to structure the process by decomposing it into its constituting modules. During this division, each sub-process will be structured with the level of abstraction required for all types of research. This is because the focus of this paper is a general artificial researcher. 

% Hence, even if an element seems crucial in current research, if it is not necessarily essential to the definition of research seen so far, I will treat those elements separately from the constituents of the research process in this study. By separating these elements, I intend to clarify the indispensable components for realizing an artificial researcher. 

% \subsubsecti on{Outline of the Structure}
% In this section, I will discuss the three things tightly related to the knowledge production. I will first discuss the functionally essential elements for knowledge production system, which is the abstract structure of the research process that has been emphasized so far. These elements correspond to the modules in knowledge production system. This is a main topic of this chapter. 

% For convenience, I will refer to the entire structured research process as the \textit{knowledge production system} or just \textit{research process} throughout the rest of this paper.

% In the previous chapter, I discussed the definition of research. In this chapter, I will focus on the high-level abstract structuring of the research process while paying attention to its functional aspects. By emphasizing the functional aspects, I mean paying attention to the role that each step plays in knowledge production. By higher-level abstract structuring, I intend to focus on the processes that is as universal as possible across research fields, regardless of the specific domain. The purpose of this kind of structuring is to clarify what kind of modules should be created as intermediate steps of research when aiming for research automation. In the following, I will structurize the research process into a chronological sequence for the purpose of clarity. However, it is important to note that the focus lies not on the temporal order nor human convention, but rather on the functionality in relation to knowledge production and the inputs and outputs of each of these processes. Also, as previously mentioned, because humans are currently the primary knowledge generators, there are many constraints that come from human society. Thus I will do my best to distinguish and organize what is dependent on humans and what is not. Before delving into specific discussions, let us first explain the scope of this chapter. After that, I will discuss the outline to be addressed in this section, followed by the main discussions.


% \textcolor{red}{TODO: add the excuse that research is social activity}



% \begin{figure}[htb]
%     \centering
%     \includegraphics[width=\textwidth]{figs/researchprocess.jpg}
%     \caption{Caption}
%     \label{fig:research_process}
% \end{figure}


% Next, I will discuss a high level description of how human beings have been conducting research. I'll structurize the abstract pattern of the process (which I will call\textit{ research process}) from determining the unknown to it turning into known. 


% \subsubsection{Note}

% note, direction
% Though my structuring may seem to represent scientific methods, I believe this pattern cam apply to other research fields, such as mathematics and humanities as well. When describing the structure, I will make a conscious effort to clearly distinguish between essential elements for knowledge production and those that are not. As previously mentioned, because humans are currently the primary knowledge generators, there are many constraints that come from human society. When considering the possibility of machines conducting research in the future, it will be important to distinguish and organize what is dependent on humans and what is not.

% I believe that the conduct of human research activities can be roughly divided into three stages: knowledge production, knowledge evaluation, and knowledge sharing. 

% Although these may not necessarily be distinctly separable from each other, I adopt this classification because it is useful for advancing discussion. The process of knowledge production consists largely of the steps: problem determination, hypothesis generation, and hypothesis verification. And in this process, the ability to read and write documents and analyze data are required as necessary skills. Below, I will examine each of these in more detail. 


% This structuring is tentative and there may be a better way to structure the research process. However, I have created this structure for practical purposes in order to move the discussion forward. I hope the structure of this article be a starting point for conceiving a better structurization. I believe that structuring and deepening understanding of the elements that are essentially important for knowledge production is extremely crucial when aiming for the automation of research.

% Though I explained that research is belied revision, it would be convenient to see as a function that takes the unknown as input and outputs the known.  % TODO: rearrange

\subsubsection{Three Sub-Modules in Knowledge Production System}
In the following sections, I will explain three essential elements that I think are functionally necessary in knowledge production: \textit{question construction}, \textit{hypothesis generation}, and \textit{hypothesis verification}. These are the abstract structure of the research process that has been emphasized so far.

Constructing questions is equal to deciding the unknown to be investigated, which is necessary by definition for research, which is the art of bringing the unknown closer to the known. Since answering this question is the objective, hypothesis generation is necessary. The reason it's not just a simple question answering, but question construction and hypothesis generation, is because no one knows the answer of the question yet. The third process is hypothesis verification, which is a process that shows us how probable it is that a hypothesis is true. This is indispensable, as it is converting our beliefs that a hypothesis is true into knowledge.\footnote{
While the term ``verification'' is used here, it does not imply a definitive determination of the truth or falsehood of propositions. It is used in the sense of just strengthening beliefs as used in everyday language. Thus, ``confirmation'' may be a more accurate term, but verification is more widely recognized, so I will use that.
}

% When summarizing these intermediate steps, it goes as follows: First, upon receiving any input, I formulate a question. Taking this question as input, hypotheses are generated. Finally, these hypotheses are input and validated resulting in the creation of knowledge as a research outcome. These processes are considered a proper decomposition of the research process, as they request each other's outputs as inputs and seamlessly connect the entire research process from input to output without any gaps.
% \textcolor{red}{TODO: However, I can identify discovery with justification because both are belief updates. I'll add this point wherever in this paper.} Fig. \ref{fig:knowledge_production_system}.

% \textcolor{red}{TODO: Add fig like this. add this sentence to the explanation of the fig "we approach the unknown to the known by asking questions, providing tentative answers to them, and then verifying the validity of those answers. " }.
% Fig. \ref{fig:research_process}.

\subsubsection{Human Research Practice and Knowledge Production System}
In actual research, while addressing the initial question posed, another question may arise and the focus may shift to that new question. Also, before reaching the final hypothesis that gets reported, several different hypotheses are tested repeatedly. When compared to this trial-and-error process, the framework I have proposed may seem overly simplified at first glance. However, I believe that the framework I proposed encompasses these human practices as well.

The framework I presented claims that in the process of generating knowledge, there is always a construction of questions about the process, the generation of hypotheses, and the verification of these hypotheses. I have not specified how these are done. Therefore, a question might have arisen while working on a different question, or it might have been directly derived to achieve a specific goal. Similarly, a hypothesis might have been developed after multiple comparisons with real data, or it could be something that came to mind while walking. What I'm arguing is that, regardless of the diverse methods employed, when an activity is termed research, there should always be a stage where questions related to the knowledge produced, the generation of hypotheses, and the verification of these hypotheses are addressed. This may indeed look like a linear process when abstracted, but given that it allows for trial-and-errors as its internal implementation, I believe it encapsulates the complex practices of human research.

Furthermore, setting aside whether a strict methodology like that of research is required, I believe that all endeavors to transform the unknown into the known, to some extent, inevitably involve the construction of questions, the generation of hypotheses, and the verification of these hypotheses. In other words, I think this framework might be an embodiment of the very act of turning the unknown into the known. In this sense, this framework might be better understood as a fundamental unit in research.

\section{Question Construction}

The first module of research process should be question construction. This is because in order to produce new knowledge, one must first be aware of what they don't know and strive to generate that unknown knowledge. Questioning is an act of trying to fill a gap in the  information that the questioner possesses, and in that sense, the act of trying to produce missing knowledge for humanity is the very act of posing and answering a question.

% This is because research is an act of generating new knowledge, and questioning is the act of seeking missing information. 

% To initiate the process of generating new knowledge, one must first know what could potentially be new knowledge. In other words, we must know what I don't know. This is an essential function for research to be the production of new knowledge. And it is the task of question construction to fulfill this function. Therefore, question construction is an indispensable module in the knowledge production system.

The process of questioning involves two steps: 1. Recognizing missing information, and 2. Attempting to fill that gap. For example, the process of asking why the sky is blue starts as follows: when prompted by, for example the vision of sky to your eyes, you suddenly (with complex perceptual and cognitive processes) realize that you don't know why the sky is blue, and then, if you would like to know the answer, utter the question, ``Why is the sky blue?'' \footnote{
Here, I explained the process of questioning as first determining something as unknown and then deciding if you construct question about the unknown. However, the order doesn't matter. For instance, there may be knowledge that you want to know first, and then you confirm that it is indeed unknown. What matter is that, to create a module for question construction, it would be necessary to consider how to achieve these abilities: recognizing missing information and deciding if you will do questioning. 
}

\subsection{What is Questioning?: Recognizing Unknown}
When prompted with a request for a specific type of knowledge and upon referencing your own knowledge base, if you judge that you do not possess that knowledge in your knowledge base, you recognize it as unknown to you. In the example of the blue sky, the knowledge of ``reason'' or the ``cause'' of the blueness of the sky is requested. When you think about the knowledge and cannot find the answer, you would recognize that you do not possess the knowledge. 

% Here, two things are happening. First, you refer your own knowledge, and second, assessing whether there is expected knowledge in your knowledge base. This means that recognizing the unknown can be understood as a combination of these two processes. I would like to delve into this in more detail in the following section.

% \subsubsection{How Human Judge Unknownness}
For an individual, the knowledge base is the memory within the brain. If certain knowledge is not in the memory, or cannot be constructed from other knowledge that is in memory, we would recognize that we don't know that knowledge. 

% In the case of an individual, the process of recognizing such knowledge as unknown seems to involve retracing whether they have the knowledge in your memory. When no relevant memories come to mind, the person may infer that they do not possess the knowledge. Even if the memory does not exist, they might try to use existing knowledge to answer the question. When they cannot construct a valid answer they are confident about, you consider it a recognition of not knowing. As mentioned earlier, to recognize a piece of knowledge as unknown means not having a justified belief that a proposition is true for a particular subject. Thus, the criterion for judging ``being unable to construct a confident answer'' can be seen as not having a justified belief.

% any ``memories'' of knowing, using, or hearing about that knowledge.

% While I have discussed recognizing unknowns for individuals, in research, what is essential is recognizing unknowns for society as a whole. If there is a way for an entity to identify societal unknowns without going through the process of identifying unknowns for individuals, that would be ideal. Therefore, I will discuss how to recognize societal unknowns.

In the case of research, the knowledge base consists of the collective media that records human knowledge, such as all the research papers, books, and articles on the web. Specifically, the most widely used method for determining novelty for humanity is through literature surveys of academic papers. If we examine as many academic papers as possible and find no prior research that has presented a properly validated hypothesis for the same question, we recognize that knowledge as unknown. \footnote{
n other words, in order to determine whether a certain piece of knowledge is unknown, one must also be able to judge whether a certain validation is valid.
}

% Knowledge for humanity is primarily disseminated through research outcomes. Therefore, when examining all the research outcomes that have been generated thus far and finding that none of them provide an answer to a specific question, it seems reasonable to conclude that the question possesses sufficient uncertainty to warrant further investigation as a research endeavor. In particular, humanity has developed the culture to preserve the research outcomes in the form of papers. Therefore, it seems feasible to assess the unknown nature of an inquiry by examining all academic papers. There are various methods to determine whether the desired knowledge already exists. Typically, it is considered that candidates can be narrowed down in two steps. First, since research papers are composed of questions and their corresponding answers, one can search for papers that have similar questions and answers to the desired knowledge. Once these papers are found, the validity of their verification methods is evaluated. If it is determined that sufficient verification has not been conducted, it can be concluded that the knowledge does not exist. 

% It is impossible to review them all due to constraints in terms of time, technology, and cognitive limitations. Therefore, in practice, I consider a question as unknown if it has been sufficiently and comprehensively explored through an extensive examination of these academic papers. I conduct literature reviews to synthesize existing research, identify research gaps in existing studies, and thereby ascertain the unknowness of my own questions or construct question for which the answers are unknown. However, in reality, such rigorous literature research is not always conducted in every case. Instead, researchers judge the unknownness of the answer to the question by referencing only a few related works and explaining that none of them have yet resolved the unknown. This means that a subjective evaluation criterion is being used because of the limitation of the number of papers I can read. 

% philosophy of literature survey \cite{schryen2015theory}

\subsubsection{Implication for Artificial Researcher}
So far, I have provided an overview of what it means to recognize unknowns and how it is currently being done. Now, let's discuss the implications if these processes were autonomously performed by machines.

The first possibility is an alternative to literature surveys. This situation involves the world's knowledge being stored discretely in units such as research papers, which is essentially the same method humans currently use to determine novelty. However, machines far surpass humans in terms of information processing capacity and speed, which could lead to significantly more efficient and statistically accurate determinations of novelty compared to humans. Since unknownness is a fundamental aspect of research, statistically rigorous judgement by machines would be desirable for the future of research.

% This current convention stems from the cognitive constraint that there is a limit to the literature that humans can examine. Since unknownness is a fundamental aspect of research, ideally, it should be evaluated objectively and rigorously. For instance, it would be desirable to quantitatively state which journals, what types of papers, and how many have been examined, and the result indicating their unknownness. Although systematic reviews already employ such approaches, there is, of course, a limit to the number of papers that can be evaluated manually and selection biases cannot be removed.

Considering that knowledge production by machines is likely to occur in the future, it may still be challenging to examine all knowledge exhaustively. Therefore, discussions on efficient search methods and how far one should investigate before determining novelty become crucial. Additionally, as mentioned earlier, in this method, it is essential to determine whether studies that propose hypotheses for the same questions are adequately verified. Therefore, I need to consider how to enable machines to judge valid verifications.

The second possibility involves determining novelty based on knowledge stored as distributed representations within machines. Current large-scale language models that have been pre-trained on massive datasets fall under this category. This process is similar to how humans determine novelty by comparing knowledge in their minds. The difference is that humans can only judge novelty subjectively due to cognitive limitations, while machines have the potential to determine novelty for a society as well, given their ability to compress vast amounts of data. While there may be many challenges, most of the issues that might arise in this scenario are not unique to machines; they can also occur with humans, boiling down to the question of how much to trust subjective reports. Hence, the critical aspect is to ensure that machines can appropriately represent knowledge. From the perspective of determining novelty, understanding should encompass not only the question and hypothesis but also how they were verified and to what extent they are valid. The reliability of the machine's judgment can only be trusted if it has appropriately assessed the effectiveness of its own verification. Therefore, the ability to understand and evaluate verification seems essential in determining the novelty of research. As with humans, it is challenging to evaluate the validity of judgments based on machine outputs. However, at least, I should discuss how to represent as much information as possible and how to create machines that learn to use this information autonomously.

\subsection{What is Questioning?: Seeking for the Answer}
I do not formulate questions for everything I know to be unknown. Instead, I construct questions for things I am curious about and wish to know the answers to. Choosing which questions to pose is a crucial aspect as it determines what kind of knowledge I will generate. In this sense, it holds significant importance. 

Making a choice involves evaluating and assigning some form of superiority or inferiority. This means that I am assessing the ``value'' of a question using some criteria. Based on these criteria, I differentiate between questions about the unknown that are worth pursuing and those that are not. For example, I do not know the name of the fifth tallest person in the world, but I am not interested, so I would not ask, ``What is the name of the fifth tallest person in the world?'' 

The crucial point here is that the criteria for determining value are arbitrary. This is because the essential requirement for research is that the answer to the question is unknown, and in principle, I do not require additional properties question to have. For example, you can ask about anything if it is unknown, or you can choose to ignore it if it's not intellectually intriguing. Alternatively, you can opt for something that seems to lead towards a specific goal you are pursuing. This has a critical implication for aiming to automate research. If the value standards are always given by humans, then there is no issue. However, when considering the possibility of automating even that aspect, it becomes necessary to discuss how I can make it adhere to the desired criteria. While it is natural to want them to ask ``good'' or ``important'' questions, it also becomes challenging when it comes to thinking about this issue. Let us discuss this further in below.

% I claimed that the value judgement is arbitrary. This is because the essential requirement for research is that the answer to the question is unknown, and in principle, any question is valid. In light of the definition of knowledge production, value judgement is not necessarily required for knowledge production to be as it is. From the perspective of knowledge production, as long as the unknown is truly unknown and it be rigorously approached towards becoming known, there should be no problem. The unknown can be anything arbitrary, and knowledge production itself does not demand a specific nature for it. 
\subsubsection{Relativity of Value}

The value of knowledge is inherently dependent on context, so the significance or goodness of a certain knowledge is not determined a priori from the moment it is generated. Goodness becomes an issue only when that knowledge is use by some member of the society in some form. In other words, the demand for importance and goodness is a constraint imposed by society not by knowledge production. For example, certain knowledge may be considered ``important'' in the sense that it addresses the interests of all the researchers working on similar topics, providing solutions to their concerns. Alternatively, that knowledge might be deemed ``important'' to someone simply because they find it intellectually interesting. On the other hand, if it takes a considerable amount of time for this knowledge to translate into practical applications, it may not be considered ``important'' to someone who wants to start a business immediately. Hence, this is the choice of us (or them) on what objective I would like to maximize by knowledge production. \footnote{
The fact that the value is arbitrary and not necessary condition for knowledge production doesn't mean I do not have to discuss about the value. In the first place, it is inherently impossible for all actions to be value-neutral. So it is inevitable for us to conduct some value judgement. Moreover, the realm of possible unknowns is too vast, so without any constraints, only nonsensical questions would arise. Most of us are interested in ``good'' or ``important'' questions. So, what I should consider is how to identify them and how to instill them to machines.
}

This provides important implications when attempting to create an artificial intelligence that autonomously conducts research, at least in two aspects. Firstly, if I am interested in making machine autonomously construct ``good'' question for humans, I should consider what is ``goodness'' for humans and how to instill them to machines. However, it seems that I have not yet fully established a unified common understanding of what constitutes a "good" question. ``What makes a question good for humans'' and ``how to formulate them effectively'' are crucial aspects that require further in-depth discussions. Therefore, I will revisit and explore these important points separately later on.

% It seems that I still lack an understanding of what kind of research questions are important, but in the field called \textit{science of science} \cite{wang2021}, which studies the research itself, scientific studies on research impact are also advancing. The insights gained here may provide important knowledge in building new algorithms and optimization metrics.

% The second implication when attempting to create an artificial researcher is that if I do not intentionally impose any constraints, there is a possibility that the intelligence produced may not align with the knowledge I desire. This is an important point and thus will be discussed in a separate chapter.

Secondly, it suggests the possibility of adopting values that are different from the ones I currently employ can result in better knowledge production. I have explained that I make judgments about certain questions being good or important based on some criteria or standards. However, there may be questions that are not considered ``important'' according to current criteria but actually be extremely significant. As mentioned earlier, the value of knowledge is determined by its usage and context, and it can vary over time and in different environments. Therefore, it is highly challenging to determine the importance of knowledge during the stage of knowledge production. Even in the same environment, it is difficult to assess the significance of knowledge. This is because knowledge results from complex accumulations, leading to new insights, and there is a intricate chain of connections before a particular knowledge becomes recognized as important within a society. In addition, I am bound by various cognitive limitations inherent to being human. Therefore, I can only assess the importance of knowledge within the confines of these limitations. 

Given these circumstances, it is highly possible that the current adopted criteria for value judgments are missing out on the production of potentially important knowledge. The development of knowledge production systems that embrace new value judgment criteria can be expected to increase the potential for generating such knowledge by expanding the scope of exploration. If an artificially intelligent system capable of autonomous research is developed, it can be expected that research based on these new criteria will become more feasible. This could potentially enable the resolution of many previously unsolved problems that were not attainable before.

Kitano referred to the science in which humans adopt their own value judgment criteria to determine questions and hypotheses as \textit{value-driven science} \cite{kitano2021nobel}. He argued that advancing \textit{exploration-driven science}, which focuses on more comprehensive and thorough exploration rather than criteria based on specific human values, is important for societal development. I admit that a completely value-neutral system is impossible. However, I agree with the idea that employing new and diverse criteria would matter for future research. By adopting more diverse and extensive criteria than the value judgment criteria humans have used thus far, I could expand the exploration space of knowledge. The realization of research through artificial intelligence is expected to open up possibilities for such a future.

As a conclusion, let us emphasize again that I first need to be aware that these ``good'' aspects do not occur naturally. To create an intelligence that constructs ``good'' questions, I first need to understand what I consider a ``good'' question. Also, it's important to turn my attention to things that are not currently considered ``good,'' but should be deemed as ``good'' in essence. Only then can I discuss how to align that value with the agent. Therefore, I think I should start by listing the criteria for determining the ``goodness'' of a question. For this, discussions in the philosophy of science and meta-sciences like the Science of Science may be referenced. Alternatively, large-scale surveys of researchers engaged in actual research could also be important. Once the value is clarified, I might be able to think about creating an intelligence equipped with these values using the value alignment techniques that are currently being developed.

% Science based on the importance of questions discussed above is a \textit{value-driven science} \cite{kitano2021nobel}. However, as previously mentioned, these may be due to cognitive constraints imposed by human society, including the inability to handle knowledge that is deemed ``unimportant.'' Therefore, when automating question generation, it may be possible to explore a wide range of questions, including those that were previously considered ``unimportant.'' In doing so, it is possible that knowledge that was not considered ``important'' according to previous criteria could actually be extremely significant. This is referred to as \textit{exploration-driven science} \cite{kitano2021nobel}, and it could become a new form of research liberated from constraints imposed by humans. 

% Indeed, it is impossible to create a completely value-neutral system. All agent systems must have some form of bias. However, what I would like to emphasize is the potential to incorporate biases different from the criteria previously used by humans, and how this can enable us to consider more diverse approaches to research. This highlights the importance of creating agents capable of autonomously conducting research.

\subsection{Autonomy and Trigger of Question Construction}
For creating an autonomous system, the question construction module faces challenges due to being the initial building block of the system. This challenge is determining ``what should be the input to the question construction module.''

Questions do not arise from nowhere. There is always something before reaching a question. In the example I mentioned earlier, for instance, the question may arise as a result of a literature review. Then, why did you conduct that literature review in the first place? It could be because there is a research theme you want to know about in the research field. And then why are you interested in that research theme? It could be because the topic of the first paper you encountered during graduate school was fascinating, or it could be because you have been interested in it since childhood. And there may be causes behind those as well.

In this way, identifying where a question begins is a hard problem. If you think seriously about it, it will lead to an infinite regress. This is a significant problem when I want to realize an autonomous artificial researcher. As infinite regress can occur, the decision of where to terminate lies solely with the designer, and it is not automatically resolved by creating the system. Can I say that question construction is autonomous if the literature to read were given? Can I say that question construction is autonomous if research theme were given? I believe it is necessary to accumulate such discussions and determine how far to consider something as given, in order to define what qualifies as autonomous.

In this paper, I assume that a trigger that requests a specific type of knowledge is given. And from there, it makes decisions about determining the unknown and constructing questions. The reason is that discovering questions with unknown answers is a necessary condition for research, and I believe this is the minimal requirement for it. However, one could also try to automate even the aspect of requesting a specific type of knowledge. The reasons for expecting the existence of certain knowledge can vary and are arbitrary. For example, I might have an objective and first consider what I need to do to achieve it. I then anticipate the necessary knowledge to accomplish those tasks. This corresponds to the demanding for a specific knowledge.

As an another example, let's consider the case of a child asking, ``Why is the sky blue?'' In this case, the child may already have prior knowledge of the concept of ``sky'' and ``blue.'' Additionally, they may possess a naive concept of causality, believing that ``A is B, so there must be a reason for it.'' Thus, they may have expected to have the knowledge that ``the sky is blue because of B.'' However, when they reference their internal knowledge, they find that it does not contain the corresponding knowledge. Therefore, they may have asked the question ``Why is the sky blue?'' to evoke the knowledge they were lacking. In this case, the required knowledge is ``the sky is blue because of B.'' and this is induced just because the result of children combining known concepts.

The question of ``why do I seek information'' has been extensively discussed in the context of curiosity. Indeed, as I proceed with the automation of these components, it becomes essential to delve into research on curiosity. Regarding this matter, I will touch upon the aspects mentioned in Chapter 3, if possible. 

\subsection{Constructing Good Questions}
Gap spotting vs problem solving \cite{alvesson2013constructing}

Up until now, I have been explaining that values like goodness are relative and subjective. However, it is natural for artificial researchers to autonomously construct ``good'' or ``important'' questions. While I admit that adopting diverse value criteria and not being bound by traditional standards matters as Kitano said, even in that case I still have to determine a criteria that lead to ``good'' questions in some sense. Therefore, I would like to discuss what constitutes a ``good'' question for us and how I can construct such questions, drawing upon the discussion of how I have characterized ``good'' questions thus far.

\subsubsection{Goal Oriented Question}
One of the most general, significant, and widely accepted criteria for what I consider as ``goodness'' is that a question is deemed ``good'' if its answer contributes to achieving a desired, yet unrealized, goal. This is because researchers often seek to address long-term problems and engage in knowledge production for that purpose. For instance, physicist who pursue a unified theory would think that a question that furthers the realization of a unified theory is a good question. This can be referred to as a \textit{goal oriented research question}.

In this approach, I first set the ultimate objective. Then, I identify the most critical bottlenecks, or sub-goals, that are essential to achieving that objective. I once again consider sub-goals for these identified bottlenecks. This process is repeated, converging on more specific and feasible sub-goals that are of high importance. Finally, I frame the question to address these sub-goals as the research question. 

For identifying the knowledge required to achieve the ultimate goal, I typically start by listing the necessary elements to accomplish it. For example, to achieve general artificial intelligence, I may think that it requires the ability to handle language, understand the real world, be proficient in systematic reasoning, and align with human values, etc. 

Then, I break down them again into the necessary things to achieve the them. For instance, to understand the real world, for instance, I may need the capability for interacting with the physical world, processing visual information, and so on. These requirements can be further broken down into multiple necessary elements. By repeating this process, I can narrow down the specific tasks that can be directly addressed. Then, the required knowledge to accomplish those tasks is demanded, and that's where it directly connects to the research question. The process is conceptualized in Fig. \ref{fig:unknown_tree}.


\begin{figure}[htb]
    \centering
    \includegraphics[width=\textwidth]{figs/unknown_tree.jpeg}
    \caption{Caption}
    \label{fig:unknown_tree}
\end{figure}

Several things are happening here. Firstly, listing the elements necessary for achieving the goal means generating sub-goals from the main goal. However, it's always a challenging problem to evaluate how a particular sub-goal contributes to the achievement of a given goal. Especially in the case of research, the target might be an too general and ambitious vision that nobody has achieved before, so I need to think about what needs to be done to break it down into appropriate sub-problems. In other words, it is necessary to construct a tree with nodes representing sub-goals. Namely, this is the tree of repetition of constructing a question and generating multiple hypotheses without verification. I will delve into the discussion of hypothesis generation in the next chapter, so I won't go into the details here.

Secondly, it is necessary to identify the most important and feasible sub-goal from the selected candidate sub-goals. This is because only one question can be addressed in the end. However, assessing and comparing sub-goals is a challenging task since I have no experience to realize the ultimate goal and so have no data what sub-goal actually is the most important.

Thirdly, the question to ultimately arrive at must be verifiable. If the question is not specific, meaningful verification cannot be performed. Overly broad or ambiguous questions can result in countless or trivial answers, or they may be too unclear to provide practical answers. Increasing the specificity of the question corresponds to deepening the depth of the sub-goal tree, so it may be important to construct a sufficiently deep tree and find an efficient way to navigate it. The verifiability is constrained by the knowledge, resources, such as funding and technology, that I currently have. Therefore, when conducting verification in reality, it is necessary to consider such feasibility. Whether to tackle a question with high feasibility or to further divide it into more sub-tasks for its realization is a matter of judgment. In any case, it is necessary to appropriately evaluate such feasibility. The scope of feasibility is vast, so it is a challenging problem to determine how to consider it in creating intelligent systems.

In this discussion, I considered the method of outputting questions from the goal through the construction and exploration of a tree structure. However, as mentioned by the predecessors, if an end-to-end approach ultimately becomes a powerful method, it may be more desirable to consider a direction in which questions are directly output from the goal. In particular, even when performing multi-step reasoning, it seems more natural to improve reasoning abilities using the recently developed approaches to multi-step logical reasoning, rather than explicitly considering tree structures. 

To pursue this direction in research automation, specifically, it may be worth considering the construction of higher-quality datasets for goals and research questions. For example, it may be possible to construct a dataset by extracting only the ultimate goal and the research questions actually solved from the introductions of papers. However, an important point to note here is that the research questions created by humans so far are not necessarily optimal for achieving research goals. Firstly, machines may be capable of maximizing the objective better than humans due to cognitive constraints. Secondly, not all human research has been conducted by working backward from a clear goal. Some studies were conducted simply because they seemed interesting while reading papers. In this regard, simply learning from human data may constrain the potential capabilities that machines can possess. Therefore, it becomes important to consider how to formulate the maximization of the probability of achieving research goals as a problem, rather than naive imitation learning of human data.

As evident from the formulation, to construct research questions for contributing to achieve a specific goal, I need to solve long range reasoning problems. This problem is widely studied in machine learning research community to improve the reasoning capabilities of machine learning models and on generate intermediate goals in reinforcement learning. If these research fields produce significant results, they can be directly applied. In this sense, it might be beneficial to seek cooperation from those who are actively conducting research in these areas. One of the unique aspects of long-distance reasoning problems in research automation is that the goal is something that has never been achieved before. This means that you cannot naively learn from data and need to generalize out of distribution. Therefore, it's essential to acquire skills not just to recognize patterns but to properly trace the path of reasoning. Moreover, because the goal has not been realized, sub-goals and the paths that connect them are ultimately based on the accumulation of hypothesis generation. In this sense, it can be said that this is a highly uncertain inference. This implies that the choice of which node to select is far from self-evident compared to other logical reasoning problems. Furthermore, there is the issue of the complexity of the distance between the goal and the question, which is far more intricate than, for example, games or planning everyday trips. For instance, to truly achieve the goal, it may be necessary to build large-scale apparatus like particle accelerators from scratch. This also means that the temporal distance between the goal and the current location is very long. Therefore, it becomes a problem that feedback on how much solving the question contributed to the goal is significantly delayed. While we've only listed a few examples here, there may be other unique challenges and issues that become more serious in research. It will be necessary to work on refining these technical challenges into specific research tasks through discussions with researchers in reasoning and planning.

\subsubsection{Important but Unnoticed Questions}

One example of a ``good'' question seems to be one that, in its construction itself, brings benefits to many researchers. By considering why hypotheses regarding the question are unknown, it becomes somewhat clearer what kind of question this is.

The reasons why the answer to a question remains unknown are diverse. In some cases, the question is simply new, and no one has had the time to come up with an answer yet. For instance, a question about the internet posed on the day it was invented may still remain unanswered because it has only been a day since its birth, and no one may have delved into it yet. In other cases, the question might simply not interest anyone, leading to an absence of answers despite the availability of time. This is an example of a question that remains unanswered because nobody is willing to tackle it, even though time exists.

Furthermore, there are questions that are difficult, and no one has been able to solve them. For instance, the question of how to achieve human immortality has been contemplated by many, but due to its complexity, no answer has been found yet. Lastly, there are questions that are important for a particular purpose but remain unnoticed by everyone. As previously mentioned, realizing challenging objectives that are yet to be achieved poses difficult problems. In such cases, it is often unclear what is not known or what is at the heart of the problem. In such situations, clarifying the question itself holds great significance.

Thus, constructing questions of this kind seems to be important as it can help shed light on the unknown. With this in mind, it appears worthwhile to discuss this topic in detail. \footnote{
Please note that the concept of a question being unnoticed and the answer to the question being unknown are different. The necessary condition for research is that the answer to the question is unknown. 

If the existence of a question is not known to anyone, then naturally, its answer would also be unknown since no one would have answered it. So, if the existence of a question is unknown, then the answer to the question is also unknown.
}

\textcolor{red}{TODO: Add discussion}

\subsubsection{Questions by Observation}

hyp.gen. by experiments etc...

% Multiple Reasons for Unknownness

% new, unimportant, difficult, unnoticed, ... etc.

% This means that I distinguish between questions that are ``good'' and those that are not, based on certain criteria. 

% This means that I distinguish between questions that are ``important'' and those that are not, based on certain criteria. For example, \cite{alon2009choose} claims that a good question is one that solves challenges facing the research community. Likewise, I consider a question to be important if it generates knowledge that greatly contributes to a certain purpose. Valuing the degree of contribution to a purpose also implies viewing research as a form of problem-solving. \textcolor{red}{TODO: Add explanation of what this sentence means}

% Thus, in realizing an agent that autonomously constructs questions, it may become important to consider how to automatically determine ``goodness'' of the questions. To achieve this, it would be important to first understand in more detail what kind of questions I consider ``good''. \textcolor{red}{TODO: Add possible directions}

% \subsubsection{How to Practically Construct a Question}

% There has been much discussion on how to actually generate questions. Off course, these discussions primarily focus on how to formulate good questions. Therefore, please note that the examples mentioned here are proposals for generating such kind of questions.

% One typical approach to formulating a research question is to conduct a literature survey, identify research gaps in existing studies, and propose a question that aims to fill those gaps.

% \textcolor{red}{TODO: Add survey of how to construct questions; gap spotting, problemization, etc}

% Next, let's consider the process of constructing purpose-driven research questions. When aiming to conduct impactful research, I believe that constructing purpose-driven research questions is crucial. In this approach, I first set the ultimate objective. Then, I identify the most critical bottlenecks, or sub-goals, that are essential to achieving that objective. I once again consider sub-goals for these identified bottlenecks. This process is repeated, converging on more specific and feasible sub-goals that are of high importance. Finally, I frame the question to address these sub-goals as the research question. 

% In practice, I seem to determine the questions I should tackle in this way, implicitly and explicitly. For example, let's say that someone try to answer a question of ``How neural networks have reasoning capability?'' in his/her study. This question may come from a thought process of ``we want to create artificial general intelligence, which requires systematic thinking, that needs ...'' In this case, the final purpose is to achieve ``artificial general intelligence'', and the question addressed as a result is ``ow neural networks have reasoning capability?'' In other words, when I want to conduct important research, I follow a process that starts with the goal I want to achieve, considers the tree of important unknowns that should be clarified for its achievement, and sets the end of that tree as the research question. This process is summarized in Fig. \ref{fig:unknown_tree}.


% Of course, the purpose mentioned here may be a sub-goal of a higher-level goal. For example, the goal of ``creating general artificial intelligence'' may be a sub-goal of a more fundamental goal of ``satisfying intellectual curiosity,'' and the goal of ``satisfying intellectual curiosity'' may be biologically demanded for better exploration of the environment. These can lead to an infinite regression when considered strictly, so I won't delve into it any further here, but it could become an important issue when considering how to realize fully autonomous agent to construct questions.

% \subsubsection{Question}

% The construction of a question is the act of seeking information \cite{watson2015ask}. Specifically, in the context of research, I consider information as knowledge. The act of seeking knowledge involves two steps: 1. Recognizing the lack of knowledge and 2. Attempting to fill that knowledge gap. In this discussion, I assume that intelligence is designed to consistently generate questions when given input. Therefore, I temporarily set aside the aspect of "triggering action" related to the second step of attempting to fill the knowledge gap.

% The recognition of a knowledge gap occurs when I expect to have certain knowledge and, upon referencing my accessible knowledge, I find that it is not available. For example, when running a program and encountering an error that I cannot resolve on my own, I recognize that I lack the necessary knowledge.

% The reasons for expecting the existence of certain knowledge can vary and are arbitrary. In this case, I assume that a purpose given by a third party creates an expectation of certain knowledge. For example, in the case of humans, I first consider what I need to do to achieve a certain purpose. I then anticipate the necessary knowledge to accomplish those tasks, and when I find that it is not present within my existing knowledge, I recognize the knowledge gap.

% Lastly, in this discussion, knowledge refers to the collective body of research findings, particularly academic papers. In actual research, a researcher may personally have a question and then investigate previous studies to confirm that it is indeed unknown before formulating it as a research question. However, what is important in the construction of a research question is that it is unknown to other entities. Therefore, for simplicity, I directly refer to the entirety of academic papers without including the step of comparing personal knowledge.

% To summarize, to create an intelligence capable of constructing questions in this setting, I need to design it to expect the necessary knowledge to achieve a given purpose provided by a third party, search for that knowledge in academic papers, assess whether the papers contain sufficient knowledge to achieve the purpose, and express any knowledge gaps as questions.

% In this case, I excluded the discussion of triggering action by design. However, when considering increasing autonomy, it is important to discuss how to incorporate this aspect into learning and acquisition. The question of "why do I seek information" has been extensively discussed in the context of curiosity.

% Furthermore, in this case, I defined the expectation of knowledge as aiming to achieve a given purpose. However, as mentioned earlier, this does not affect the formulation of questions. For example, let's consider the case of a child asking, ``Why is the sky blue?'' In this case, the child may already have prior knowledge of the concept of ``sky'' and ``blue.'' Additionally, they may possess a naive concept of causality, believing that ``A is B, so there must be a reason for it.'' Thus, they may have expected to have the knowledge that ``the sky is blue because of B.'' However, when they reference their internal knowledge, they find that it does not contain the corresponding knowledge. Therefore, they may have asked the question ``Why is the sky blue?'' to evoke the knowledge they were lacking.

% In this way, the reasons for expecting the existence of certain knowledge can vary, and what, why, and how I seek information (knowledge) are not constrained by specific conditions. Therefore, when attempting to create an intelligence capable of constructing questions in the future, it is feasible to develop a more flexible intelligence.

% Additionally, in this case, I assumed that the given purpose and its achievement are predefined goals. However, humans naturally set their own goals. When considering the design of a more autonomous intelligence, it is conceivable to aim for automation in this aspect as well. However, as mentioned earlier, the question of what I seek knowledge about is not specific to research. Therefore

% , I temporarily set it aside for now. If I were to pursue this direction further, it would ultimately lead to an infinite regress, raising the question of how much information to consider as given.


%%%%%%%%%%%%%%%%%% Rearangement %%%%%%%%%%%%%%%%%%%%%%%%%%%%

% \subsection{question construction}
% Research is an endeavor to bring the unknown closer to the known. Therefore, it is necessary to first determine what unknown I aim to make known. And this unknown often takes the form of questions. For example, ``Why do deep neural networks with a large number of parameters generalize well?'', ``How can I prevent the problem of vanishing gradients?'', and like these. These are commonly referred to as \textit{research questions} or \textit{research problems}. Therefore, in this paper, I will refer to the step of determining this unknown as \textit{question construction}.

% \textcolor{red}{TODO: should describe question construction itself first. What is research question or research problem?}

% \subsubsection{Unknownness}
% As I have reiterated, it is a necessary condition for research that the answer to a question is unknown, or in other words, that there is a high degree of uncertainty. Therefore, it is essential in research to have methods that ensure the answer to a posed question is truly unknown, or to formulate questions that truly have unknown answers.

% Knowledge for humanity is primarily disseminated through research outcomes. Therefore, when examining all the research outcomes that have been generated thus far and finding that none of them provide an answer to a specific question, it seems reasonable to conclude that the question possesses sufficient uncertainty to warrant further investigation as a research endeavor. In particular, humanity has developed the culture to preserve the research outcomes in the form of papers. Therefore, it seems feasible to assess the unknown nature of an inquiry by examining all academic papers. However, it is impossible to review them all due to constraints in terms of time, technology, and cognitive limitations. Therefore, it is realistic to consider a question as unknown if it has been sufficiently and comprehensively explored through an extensive examination of these academic papers. In practice, I conduct literature reviews to synthesize existing research, identify research gaps in existing studies, and thereby ascertain the unknowness of my own questions or construct question for which the answers are unknown \cite{schryen2015theory}. \textcolor{red}{TODO: Consider where I will explain about literature review}

% % In the previous statement, it was mentioned that as long as the unknown is truly unknown and it can be approached towards becoming known, there should be no problem. The process of approaching the known will be explained in the next section, and here I will delve a bit more into the determination of unknownness. 

% However, in reality, such rigorous literature research is not always conducted in every case. Currently, researchers often demonstrate the unknownness of the answer to the question by referencing only a few related works and explaining that none of them have yet resolved the unknown. And when the paper is evaluated by reviewers, who are a small group of experts, if it is determined that the question has indeed not been answered so far, the provisional recognition of the unknown nature of the question is granted. This means that a subjective evaluation criterion is being used, where researchers and a small number of reviewers consider a question as unknown when none of their known studies provide an answer.

% % This implies the use of subjective evaluation criteria, where researchers examine several papers considered ``major literature'' in a field and consider them as unknown if none of them have provided an answer. Furthermore, as mentioned later, I evaluate the quality of research outcomes by having them assessed by a small number of experts in the same field. If these researchers determine that the previous studies have been sufficiently comprehensive, the determination of unknownness is considered somewhat valid. In other words, ultimately, the evaluation by a few experts may serve as the basis for establishing the unknownness.

% This current convention stems from the cognitive constraint that there is a limit to the literature that humans can examine. Since unknownness is a fundamental aspect of research, ideally, it should be evaluated objectively and rigorously. For instance, it would be desirable to quantitatively state which journals, what types of papers, and how many have been examined, and the result indicating their unknownness. Although systematic reviews already employ such approaches, there is, of course, a limit to the number of papers that can be evaluated manually and selection biases cannot be removed. \textcolor{red}{TODO: Add the explanation of systematic review, problems of it, and how AI can mitigate them.}

\subsubsection{Diverse Good Questions}
There have been various discussions on the elements that good research possesses. For example, \cite{hulley2007designing} proposed that good research question should satisfies FINER criteria (feasible, interesting, novel, ethical, and relevant) and \cite{alon2009choose} claims that a good problem is one that is most feasible and interesting to oneself.

\textcolor{red}{TODO: Add more discussion on ``good'' questions, examples, discussion, what is good, what is important, specific question is good, etc.}

% However, I have not found a unified consensus on the definition of good question. but 

% \subsubsection{Questioning as Information Seeking Behavior}
% \textcolor{red}{TODO: Reconstruct}

% The construction of a question is the act of seeking information \cite{watson2015ask}. Specifically, in the context of research, I consider information as knowledge. The act of seeking knowledge involves two steps: 1. Recognizing the lack of knowledge and 2. Attempting to fill that knowledge gap. In this discussion, I assume that intelligence is designed to consistently generate questions when given input. Therefore, I temporarily set aside the aspect of "triggering action" related to the second step of attempting to fill the knowledge gap.

% The recognition of a knowledge gap occurs when I expect to have certain knowledge and, upon referencing my accessible knowledge, I find that it is not available. For example, when running a program and encountering an error that I cannot resolve on my own, I recognize that I lack the necessary knowledge.

% The reasons for expecting the existence of certain knowledge can vary and are arbitrary. In this case, I assume that a purpose given by a third party creates an expectation of certain knowledge. For example, in the case of humans, I first consider what I need to do to achieve a certain purpose. I then anticipate the necessary knowledge to accomplish those tasks, and when I find that it is not present within my existing knowledge, I recognize the knowledge gap.

% Lastly, in this discussion, knowledge refers to the collective body of research findings, particularly academic papers. In actual research, a researcher may personally have a question and then investigate previous studies to confirm that it is indeed unknown before formulating it as a research question. However, what is important in the construction of a research question is that it is unknown to other entities. Therefore, for simplicity, I directly refer to the entirety of academic papers without including the step of comparing personal knowledge.

% To summarize, to create an intelligence capable of constructing questions in this setting, I need to design it to expect the necessary knowledge to achieve a given purpose provided by a third party, search for that knowledge in academic papers, assess whether the papers contain sufficient knowledge to achieve the purpose, and express any knowledge gaps as questions.

% \begin{figure}[htb]
%     \centering
%     \includegraphics[width=\textwidth]{figs/question_formulation.jpg}
%     \caption{question construction}
%     \label{fig:enter-label}
% \end{figure}

% \textcolor{red}{TODO: Is question construction information retrieval??}


% Asking questions is an act of seeking information \cite{watson2015ask}. The act of seeking information (or knowledge) arises from realizing the lack of one's own knowledge and the desire to fill that gap. Therefore, to create intelligence that can autonomously ask questions, it is necessary to incorporate mechanisms that induce these behaviors.

% Determining what triggers this behavior in humans is a challenging problem. However, when designing a system, it is sufficient if it can induce such behavior, regardless of what it actually is. The principle that ``behavior is triggered when it is somehow desirable for the agent'' represents this idea. You are probably familiar with this concept in reinforcement learning, where rewards (desirability of actions) are provided, and maximizing the expected value of these rewards is formulated as an appropriate way to induce behavior.

% The notion of triggering behavior by realizing the lack of one's own knowledge and attempting to fill it has been extensively studied in the context of curiosity in the field of reinforcement learning. In research, agents that pose questions can generally be formulated within this framework.

% \subsection{question construction}
% The construction of a question is the act of seeking information \cite{watson2015ask}. Specifically, in the context of research, I consider information as knowledge. The act of seeking knowledge involves two steps: 1. Recognizing the lack of knowledge and 2. Attempting to fill that knowledge gap. In this discussion, I assume that intelligence is designed to consistently generate questions when given input. Therefore, I temporarily set aside the aspect of "triggering action" related to the second step of attempting to fill the knowledge gap.

% The recognition of a knowledge gap occurs when I expect to have certain knowledge and, upon referencing my accessible knowledge, I find that it is not available. For example, when running a program and encountering an error that I cannot resolve on my own, I recognize that I lack the necessary knowledge.

% The reasons for expecting the existence of certain knowledge can vary and are arbitrary. In this case, I assume that a purpose given by a third party creates an expectation of certain knowledge. For example, in the case of humans, I first consider what I need to do to achieve a certain purpose. I then anticipate the necessary knowledge to accomplish those tasks, and when I find that it is not present within my existing knowledge, I recognize the knowledge gap.

% Lastly, in this discussion, knowledge refers to the collective body of research findings, particularly academic papers. In actual research, a researcher may personally have a question and then investigate previous studies to confirm that it is indeed unknown before formulating it as a research question. However, what is important in the construction of a research question is that it is unknown to other entities. Therefore, for simplicity, I directly refer to the entirety of academic papers without including the step of comparing personal knowledge.

% To summarize, to create an intelligence capable of constructing questions in this setting, I need to design it to expect the necessary knowledge to achieve a given purpose provided by a third party, search for that knowledge in academic papers, assess whether the papers contain sufficient knowledge to achieve the purpose, and express any knowledge gaps as questions.

% In this case, I excluded the discussion of triggering action by design. However, when considering increasing autonomy, it is important to discuss how to incorporate this aspect into learning and acquisition. The question of "why do I seek information" has been extensively discussed in the context of curiosity.

% Furthermore, in this case, I defined the expectation of knowledge as aiming to achieve a given purpose. However, as mentioned earlier, this does not affect the formulation of questions. For example, let's consider the case of a child asking, ``Why is the sky blue?'' In this case, the child may already have prior knowledge of the concept of ``sky'' and ``blue.'' Additionally, they may possess a naive concept of causality, believing that ``A is B, so there must be a reason for it.'' Thus, they may have expected to have the knowledge that ``the sky is blue because of B.'' However, when they reference their internal knowledge, they find that it does not contain the corresponding knowledge. Therefore, they may have asked the question ``Why is the sky blue?'' to evoke the knowledge they were lacking.

% In this way, the reasons for expecting the existence of certain knowledge can vary, and what, why, and how I seek information (knowledge) are not constrained by specific conditions. Therefore, when attempting to create an intelligence capable of constructing questions in the future, it is feasible to develop a more flexible intelligence.

% Additionally, in this case, I assumed that the given purpose and its achievement are predefined goals. However, humans naturally set their own goals. When considering the design of a more autonomous intelligence, it is conceivable to aim for automation in this aspect as well. However, as mentioned earlier, the question of what I seek knowledge about is not specific to research. Therefore

% , I temporarily set it aside for now. If I were to pursue this direction further, it would ultimately lead to an infinite regress, raising the question of how much information to consider as given.

% \begin{figure}[htb]
%     \centering
%     \includegraphics[width=\textwidth]{figs/question_formulation.jpg}
%     \caption{question construction}
%     \label{fig:enter-label}
% \end{figure}


% \textcolor{red}{TODO: Is question construction information retrieval??}


% \subsection{Conclusion}

% \begin{enumerate}
%     \item Determing the existence of expected knowledge:
%     \begin{enumerate}
%         \item Searching for knowledge directly related to the expected knowledge.
%         \item Determining whether the knowledge has been properly validated.
%     \end{enumerate}
% \end{enumerate}

% 1.b is specific to the automation of research. 





% I have listed what I believe are important elements in the construction of questions. However, these are considered important under the assumptions mentioned earlier. For instance, if the goal is not to acquire knowledge necessary for achieving an objective, but to generate knowledge that an individual finds interesting, the necessary elements in question construction (particularly in parts 1.a and 1.b) would change. As previously mentioned, the value of knowledge is determined in relation to context and there's a high degree of uncertainty about how the value of knowledge will evolve in the future. This makes it fundamentally important to have a diverse range of ways to generate questions. The object achievement is highly prevalent and is expected to produce ``important'' knowledge, which is why it is discussed here. However, it is important to discuss what other ways of formulating questions could exist and how they can be implemented.

\section{Hypothesis Generation}
I began by explaining that I first formulate the unknown I am addressing in the form of a question. The process of finding answers to this question is research. Here, because the answer to this question is, of course, unknown, it requires inference as to what the answer could be. As a result of this inference, a plausible answer is formulated. This process corresponds to the \textit{hypothesis generation}. Hypothesis generation is the inference about the unknown and the definition of research is to transform unknown to known. Thus, every research including deductive research must entail hypothesis generation implicitly or explicitly. In this sense, hypothesis generation should be the second module of knowledge production system.

The belief that a hypothesis is true is the very object that can become knowledge in response to a question. If a hypothesis withstands proper testing, the belief in its plausibility strengthens. Conversely, if a hypothesis does not withstand testing, that belief is weakened. Therefore, the former generates knowledge that ``the answer to question A is hypothesis B,'' while the latter generates knowledge that ``the answer to question A is not hypothesis B.'' 

Hypothesis generation is the act of creating potential answers to a question, so naturally, it is essential to generate plausible hypotheses that are close to the actual answer. Therefore, let's start by referring to how humans generate hypotheses and then discuss how I can generate reliable hypotheses, drawing inspiration from human methods.

\subsection{Generating Plausible Hypothesis}
% \subsection{Hypothesis Generation: Proposal}
% As mentioned earlier, from the perspective of knowledge production, I believe that hypotheses are sufficient in principle if they fulfill the function of providing provisional answers to questions. However, in order to avoid unnecessary testing, it is important to derive ``plausible'' hypotheses. There are various approaches to this, but to provide some food for thought, I would like to share my personal view on how humans generate hypotheses.

Hypothesis generation is sometimes considered an unanalyzable, emergent process or a result of genius.\footnote{
Since 19th centuries, the act of producing knowledge, in particular hypothesis, and that of verification of it have been distinguished as the context of discovery and context of justification \cite{sep-scientific-discovery}. And during the most of the 20th centuries, the discovery caught much less attention by philosophers of science.

In engineering discussions, there is often explicit formulation of sets or candidates of hypotheses, and the discovery of hypotheses in such situations is often discussed \cite{simon1973does,kitano2021nobel,bengio2022ml4sci}. However, when automating hypothesis generation it is also important to consider how such candidates arise in the first place. There have been attempts to understand this process, such as highlighting the importance of analogy \cite{thagard1984conceptual} or mental model \cite{nersessian1999model}, but the question of how to generate ``good'' hypotheses still remains unanswered. The generation of initial hypothesis candidates has been discussed in the context of creativity in science. As you are well aware, however, current machine learning models are already capable of executing creative tasks effectively \cite{sep-creativity}. Thus, there is a debate about how much emphasis should be placed on creativity when considering the development of artificial intelligence capable of generating new hypotheses.
} However, I believe that human-generated hypotheses are produced through a process of trial and error in rational inference. In the following discussion, for the sake of simplicity, I will focus on hypothesis generation for why questions. \footnote{
Please note that not all research questions are why questions (e.g., ``Does life exist beyond Earth?'' is not a why question, but it is a scientifically investigable question).
}

% \subsection{Plausible Hypothesis and Unknownness}

% While predictions that are highly unlikely to be confirmed, such as random guesses, can still qualify as research, they do not contribute much knowledge because they are expected to be easily rejected without undergoing rigorous testing. It becomes a waste of resources to invest in validating such predictions. Therefore, it seems to be crucial for ``meaningful'' research to propose hypotheses that are somewhat plausible.

% It is immediately evident that this is a non-trivial problem. This is because research, despite being an endeavor to answer unanswered questions, requires considering plausible candidate answers for those questions. Here, if the unknown under investigation is entirely unrelated to existing knowledge, it is impossible to make meaningful predictions about it. This is because predictions are based on experiences and data. In other words, high novelty and high uncertainty indicate a complex structure that cannot be immediately predicted from past knowledge and experiences. It implies that constructing plausible hypotheses necessitates the ability to discern these complex structures and patterns from past experiences. While it is difficult to determine what is unknown and what constitutes a complex structure, these can be crucial points to consider when advancing the automation of research.

\subsubsection{Analyzing Question}
Researchers start with a question, the composition of which has been discussed in the previous section. Given the question, I break down the content of the question and analyze each part in detail. For example, let's say I have a question, ``Why are apples red?'' The first thing I would consider is what an apple is and what red means. I would also think about what it means for something to be red. Then, I focus on the properties of apples and the color red and abstract them. I may also think about other red things besides apples. If I already know the reason why tomatoes are red before knowing why apples are red, I might consider that the reason for tomatoes being red could apply to apples as well. By conducting this kind of analysis, I can connect my understanding to existing similar knowledge and attempt to explain using those existing reasons. These ways of focusing on abstract structural similarities between specific concepts and inferring that what can be said about A, which has the same structure, can also be said about B is called \textit{analogical reasoning}. This has been considered to be important method in hypothesis generation \cite{hesse1965models,thagard_1984,gentner1993shift,holyoak1996mental,dunbar1997scientists}.

For those simple examples given above, one can easily find analogical examples. However, many of the questions researchers actually grapple with are much more complex, and it's not immediately clear how they relate to existing knowledge. Even in such situations, researchers have managed to generate plausible hypotheses by analyzing research question thoroughly.

Such thorough analysis of questions seems particularly important when it comes to making significant discoveries that are remembered in history. Let's consider Charles Darwin as an example, who proposed the concept of natural selection. Darwin appears to have gone through a process of trial and error before arriving at the idea of natural selection \cite{gribbin2022origin}. After returning from his voyage on the HMS Beagle, he began to question how evolution occurs. It seems that he read the works of Lyell and Linnaeus, and particularly from Lyell's writings, he realized the importance of selection in evolution. The question then shifted to what could serve as a natural equivalent of artificial selection. Later, after reading Malthus' book on population, Darwin understood that in nature, competition leads to the preservation of advantageous species and the extinction of disadvantageous ones, which is the process of natural selection.

In essence, Darwin initially had the question of ``how does evolution occur,'' but through analyzing this question, referring to previous research, and conducting experimentations and observations, the question transformed into ``what is the equivalent of artificial selection in nature?.'' And it was through this transformation of the question that he was able to recognize the similarities between Malthus' discussion on human society and the mechanism he was seeking. Although the process is complex, this is the same as the hypothesis generation process I explained above in that he analyzed and transformed the question, finding analogies with existing knowledge and reaching a plausible hypothesis.

In the case of Darwin, it involved a more observational approach within the field of natural history, which prompted the transformation of his question. However, in fields such as theoretical research that utilize mathematics, different methods may be employed. For example, consider the scenario where there are two theories, theory A and theory B, and the goal is to achieve a unified description between them. In this case, one might repeatedly perform mathematical transformations on the objects described by each theory, reducing the significant problem of ``incompatibility between Theory A and Theory B'' to inconsistencies between specific properties of each theory. By introducing axioms that resolve these inconsistencies, it may be possible to construct a unified theory. It is important to reiterate that in cases where the subject matter is mathematically described, formal operations can be applied to objects of interest, such as object A and object B, to transform them into different forms. This process can lead to the discovery of unexpected similarities. It is through this approach that one can discuss the similarities between objects even when they may defy intuition or cannot be imagined through empirical means. 

Therefore, I believe that thorough analysis of question plays a powerful role in identifying similarities between objects, especially in cases where it is necessary to discuss the similarities between objects that may not be intuitively evident or imagined through experience. I believe that this is the core of the human hypothesis generation. The art of discovering structural similarities between two different objectives is called \textit{analogical reasoning}. This has been considered to be crucial for hypothesis generation \cite{hesse1965models,thagard_1984,gentner1993shift,holyoak1996mental,dunbar1997scientists,gentner2002analogy}. 

\subsubsection{Confirming Plausibility in Hypothesis Generation}

In the previous chapter, I explained how I transform questions by analyzing them and connecting them with existing knowledge. However, there seem to be countless ways to bring about changes in the questions. So, how exactly do I go about transforming the questions? Let's now describe the process of question transformation that I have in mind.

% In simple cases like ``apples are red,'' it may be sufficient to apply existing knowledge. However, in research, especially when tackling challenging questions, I believe that the following steps are involved. 

First, from the knowledge at hand, I select several hypotheses that are strongly related to the question and have a high level of confidence, and then proceed with accepting these hypotheses as premises. For example, the existing knowledge that tomatoes are red for reason A, strawberries are red for reason B  are strongly linked with the proposition that apples are red for reason C by the presence of the word ``red.'' Let's assume that I have a high level of confidence in the proposition that tomatoes are red for reason A, but only a low level of confidence in the proposition that strawberries are red for reason B. In this case, the reason A for tomatoes being red would be selected as the premise. \footnote{
In the case of humans, it may not always be the case that I prioritize knowledge with a high level of confidence. For example, let's consider a situation where I have recently read a book and acquired some highly impactful knowledge. In this scenario, even during the process of question analysis, I might be inclined to consider using this newly acquired knowledge because it has left a strong impression on us. This inclination is not solely based on its level of confidence, but rather because the knowledge has made a lasting impression, leading us to take it as a premise while analyzing the question. Like this, determining what knowledge humans choose to adopt as premises is a complex issue, and I refrain from delving into this issue in this paper.
}

Next, I will not focus on the parts of the question that pertain to the meaning of the premises or the deduced results that answer them. For example, going back to Darwin's example, when Darwin read Lyell's book, he took the proposition that ``selection matters in evolution'' as a premise. As a result, he left aside the question of ``what is important for evolution.'' Instead, he shifted his focus to solving the question of ``how does nature make selections?''

Once a certain premise is accepted, the next step involves examining the consistency between the consequences brought about by that assumption and one's existing knowledge. If introducing that premise does not lead to contradictions with strongly held beliefs, then the introduction of the premise is deemed acceptable. On the other hand, if introducing the premise results in contradictions with one's existing knowledge, it becomes a matter of choosing between the two. 

% This involves comparing the consequences brought about by the introduction of that premise with retaining one's existing knowledge and selecting the one that seems more favorable in some sense. For instance, Kepler compared the consequences of introducing the assumption of circular motion for planets with observational data and the results of his proofs, and eventually decided to discard the assumption of circular motion.

% or generating hypotheses that maintain consistency with this knowledge. Specifically, I generate hypotheses that seem relevant, and then verify that they do not contradict existing knowledge.

By considering the relationships between existing knowledge and new knowledge, and reevaluating which ones are deemed more plausible, one can undergo a process of transforming the question into its more essential form and generating plausible hypotheses. I seem to generate plausible hypothesis by transforming the question in like these steps.

\subsubsection{Continuity between Hypothesis Generation and Hypothesis Verification}

From the explanation so far, it becomes evident that in order to generate ``plausible'' hypotheses, I am conducting operations that involve changing my beliefs about the truth of those hypotheses. This seems natural since generating ``plausible'' hypotheses inherently involves such belief updates. However, this observation may offer significant insights into my current discussion. I define knowledge production as the updating of beliefs, which includes the construction of questions, the generation of hypotheses, and the operation I refer to as hypothesis testing, where I update my belief about whether a hypothesis is true. The implication of this discussion is that the process of generating ``plausible'' hypotheses and hypothesis testing share similarities. In reality, there are cases where hypothesis generation and justification are tightly connected in research practice \cite{arabatzis2006inextricability}. This leads to the question of whether I need to separate the processes of hypothesis generation and hypothesis testing.

However, I do not think that the fact that a single process play roles for both hypothesis generation and verification means that they are also play the same role for knowledge production. I believe that hypothesis generation and verification play two distinct role for knowledge production, and that's why I discuss separate them as two distinct modules.

You may say that it is not entirely implausible to interpret both hypothesis generation and verification as being ``belief updates.'' Certainly, this similarity becomes more evident, particularly when contemplating the generation of ``probable'' hypotheses. For instance, let's consider a scenario where belief values are continuous. In this case, generating a hypothesis involves changing the belief value of a certain hypothesis from its original value of, say, 0 to something like 0.1. Subsequently, if multiple hypothesis candidates are considered, and one is chosen as probable based on previous research, the belief in this hypothesis might increase to around 0.3. Through the final verification of this hypothesis, the belief might reach approximately 0.9, resulting in the formation of knowledge. From this perspective, hypothesis generation and verification can be seen as the process of modifying beliefs.

However, when aiming for knowledge production within a specific society, there is a distinction. During the verification phase, the beliefs that are updated are shared beliefs. In contrast, during the generation of probable hypotheses, the updating of shared beliefs is not required. Consequently, in the context of knowledge production, hypothesis generation generates beliefs that can be updated, while hypothesis verification serves as the means to update those beliefs, thus playing distinct roles.

% \subsection{Continuity between Hypothesis Generation and Verification}
% So far, I have emphasized that hypothesis generation and hypothesis testing are functionally distinct, and that seems somewhat reasonable. However, when considering the generation of "plausible" hypotheses, these boundaries become somewhat ambiguous.

% Since testing incurs costs, in reality, humans carefully choose hypotheses worthy of testing. Even after hypothesis testing is made more efficient by machines, narrowing down hypotheses to a certain extent remains practically essential. Therefore, this is a problem that persists even after knowledge generation is achieved through machines.

% When considering the generation of ``plausible'' hypotheses, it simply means strengthening the certainty about the truth or falsehood of a hypothesis. In other words, in terms of changing the belief about the truth or falsehood of a hypothesis, this can be seen as similar to testing. For example, in experimental research, I believe there is often a pilot study conducted before starting full-scale experiments. This is nothing more than evaluating whether the hypothesis is worth pursuing for rigorous testing.

% Therefore, it may be possible to identify hypothesis generation and hypothesis testing as being synonymous in the sense of belief updating. However, I still believe that there are differences between hypothesis generation and hypothesis testing. The confidence in hypothesis generation is ultimately a matter of individual researchers or those involved in the research, and it is not necessary for all members of society to share that confidence. On the other hand, testing requires methods that update the common beliefs of members of society. Therefore, while hypothesis generation and hypothesis testing can be regarded as equivalent in terms of belief updating, they may differ in the strength of that belief.

% Obviously it is not possible to consider the consistency for all possible knowledge, so in practice, researchers seem to be checking the consistency of the hypotheses with several studies or knowledge that they have in mind. The knowledge  may be the recent interesting papers they have read, knowledge they strongly believe in, and so on. The knowledge that comes to the researcher's mind at that moment and is prioritized in their thinking is likely to become the premises for these hypotheses. 

% In highly mathematicalized disciplines like physics, for example, one might judge the plausibility of a hypothesis by comparing the deduced consequences from that hypothesis with existing knowledge. In this way, hypotheses that are judged to be somewhat consistent with the assumed premises may be recognized as ``plausible'' and worthy of rigorous testing. Of course, there are factors other than the plausibility of a hypothesis that can affect its ``value,'' but I believe that plausibility is the most important value in terms of knowledge production.

\subsection{Hypothesis Generation from Verification Results}
So far, I have been discussing the process of generating hypotheses by analyzing a given question. In this case, the hypotheses are intended to answer the question from start to finish. However, in reality, I may end up generating hypotheses that are for completely different questions than the one I initially set out to solve. Surprisingly, some of these serendipitously generated hypotheses can become profoundly important and leave a mark in history. Let's take a moment to consider how such hypotheses are generated through this process.

Such hypotheses are born while attempting to identify the causes of verification results. Once a hypothesis is generated, the validity of it is assessed through hypothesis testing. In actual research practice, even if hypothesis testing yields negative results, it does not necessarily mean that the proposed hypothesis is completely discarded. Instead, supplementary premises may be introduced into the initially proposed hypothesis, and the modified hypothesis may be retested. On the other hand, it is also possible that the assumptions of the hypothesis are reconsidered, leading to the generation of new hypotheses. These are all practices of researchers involved in hypothesis generation, so let us explain them in more detail.

First, when humans think about something, they always need some assumptions. These assumptions refer to the knowledge or beliefs that individuals provisionally consider as ``correct.'' For example, when interpreting data at hand, one may assume that the measuring instrument used to generate the data is reliable, or that the principles of classical mechanics used for data collection are valid, or that the insights from several previous studies are valid. Additionally, researchers often introduce auxiliary hypotheses implicitly or explicitly during their investigations. For example, tentatively decided hyperparameters or auxiliary hypotheses introduced during calculations also become premises. Furthermore, there are implicit or explicit guiding principles and beliefs, such as the belief that ``hypotheses should be simple'' or that ``theories should be beautiful,'' which serve as premises for hypothesis derivation as well.

When verification yields negative results, all these assumptions, including the ones that are implicit, can be the cause for the result. The proposed main hypothesis itself may be incorrect, the auxiliary hypotheses may be incorrect, the observations may be incorrect, or all of them may be incorrect. Researchers must identify which of these possibilities is the cause. This is known as the problem of Duhem-Quine's thesis \cite{sep-scientific-underdetermination}. This is like trying to generate hypothesis on the question of ``What is the cause of the error?''. As you can understand, this is an extremely challenging problem since this is like debugging a system that is unstructured, where the entire code is not visible, and there is no systematic approach, and you have to grope in the dark. 

In practice, it seems that humans adopt a strategy of revising beliefs from the weakest ones first to tackle this proble. This strategy, in my opinion, is somewhat reasonable and rational. For example, let's say the parameters introduced in this study were chosen arbitrarily. This could be one of the first things to be revised because there is no reason for it to be that way. However, just because something was introduced in this study does not mean it will always be revised. For example, if results from the hypothesis under investigation are remarkably consistent with the background assumptions, I should keep the hypothesis. Alternatively, let's say the verification results are not completely off track but slightly deviated. In such cases, the fact that the verification results were negative may not make it reasonable to completely discard the hypothesis. The researcher may add terms or ad-hoc assumptions only to resolve those small errors or inconsistencies.

On the other hand, if, for example, the hypothesis of this study is based on classical mechanics, it would not be reconsidered unless there is a substantial reason to do so. Classical mechanics has been shown to be highly consistent with an immense amount of knowledge based on it, so revisiting it unconditionally would require an explanation sufficient to negate the entire system. In this sense, researchers will have a fairly strong belief in the validity of classical mechanics. The important point here is that, regardless of the reasons, when a researcher has a strong belief or takes certain assumptions for granted to the extent that they no longer doubt them, the priority of revisiting those assumptions is diminished. For example, during the time of Johannes Kepler, the belief that planetary orbits were perfect circles was widely shared as an unquestionable assumption. Therefore, it took considerable trial and error before Kepler started to question that assumption. Believing that planetary orbits are perfect circles and believing in the validity of classical mechanics are vastly different in terms of the reasons for belief, but they share the common aspect that researchers at the time strongly believe the principle, revisiting those assumptions would not occur unless there is a substantial reason to doubt them.

In this way, researchers start by revising less firmly held beliefs first and repeat this process until contradictions are resolved. In my opinion, this is a somewhat reasonable strategy. Of course, as with the belief in the circular orbits of planets in Kepler's time, there are beliefs that are implicitly assumed but not verified, and researchers still hold these beliefs. However, many strong beliefs are directed toward knowledge that has withstood numerous tests, and in that sense, it seems natural to first attribute the cause to weaker beliefs than to these strong beliefs.

Thinking in this way, I can understand why mathematics has played a crucial role in science. First and foremost, in mathematics, it is necessary to state the assumptions explicitly. Therefore, among all the potential influencing assumptions, only the assumptions under consideration are represented as mathematical objects. Furthermore, mathematics is a deductive system, so if any contradictions arise in the results, I can conclude that one of the assumed premises must be incorrect as long as the inference rules are correctly applied. Consequently, I can confine the process, which could have an infinite number of causes, to a finite and debuggable system for discussion. Additionally, when verifying the consistency of a hypothesis with the underlying assumptions used to generate it, the relationship of how the hypothesis is deduced from those assumptions can also be mathematically examined. This allows us to proceed with confidence, understanding the reasons why and to what extent I need to question each assumption. Finally, mathematics is abstract so is help us to find the analogical relationships. I believe these factors have hugely contributed for humans to generate plausible question.



% In research, a hypothesis is a prediction of the answer to a question that no one knows the answer to. Therefore, a ``good'' hypothesis is primarily one that represents the true answer to the question. To generate such a ``good'' hypothesis, I need to consider several factors. Here, I would like to discuss the reasons for the unknown that I previously discussed. Easy questions naturally lead to the generation of good hypotheses, and there is not much meaning in discussing the quality of hypotheses in such situations. Therefore, in this context, I will focus on how to generate ``good'' hypotheses for difficult questions that many people are challenging but have not yet been answered.

% I do not precisely know what it means for a question to be difficult. However, if I consider an analogy with machine learning, inference about patterns that rarely occur in the training data is challenging. Therefore, difficult questions may correspond to the tail part of the distribution during the training phase in machine learning or cases where there is a distribution shift (specifically, when the true distribution is mistakenly inferred during training). If that is the case, the ability to appropriately infer hidden patterns without being misled by spurious correlations in past experiences could be crucial in generating ``good'' hypotheses for difficult questions.

% Humans have employed various means to solve this difficulty. One approach is to incorporate new perspectives by borrowing knowledge from other fields or old papers. As I accumulate knowledge for research, I implicitly acquire the dominant ways of looking at things in that era. While it may not be false correlation, it undoubtedly makes certain patterns less visible. Bringing in insights from unrelated fields to the current domain might relativize these perspectives and provide a trigger to notice hidden patterns. Another approach is to leverage the power of mathematics. For example, in the field of physics, hypotheses for a certain question are built as theories with the help of mathematical tools. This involves deductive operations at various points, allowing for leaps of inference that go beyond human past experiences. I have also utilized various techniques such as analogies and the use of computers. However, all of these methods have been human approaches to overcoming broad out-of-distribution generalization.

% \textcolor{red}{TODO}

\subsection{Hypothesis Generation by Machines}

% Within a single process of knowledge production, they serve the role of being subject to verification and are mere beliefs if not tested or refuted. However, when considering the entire ecosystem of knowledge production, hypotheses play an extremely important role. This is because research is the act of generating new knowledge based on past knowledge and hypotheses are potential knowledge, justifying them indirectly influences future knowledge production.

% \subsubsection{What is Necessary for Hypothesis Generating Machine}
Hypothesis generation is just a prediction based on experience. As you all know, statistical machine learning is to do predictions about unknown from data. Moreover, if both question and hypothesis are expressed in text, hypothesis generation is simply question-answering task in machine learning. In this sense, the generation of hypotheses is exactly what statistical machine learning does, and in terms of formulation, there is nothing fundamentally new as a machine learning problem.\footnote{
I stated that hypotheses are simply predictions. However, even if their validity is verified in a manner that other members find acceptable, if the content of those predictions cannot be interpreted to other society member, it seems that the predictions generated there cannot be called knowledge of that society. This is because common beliefs are not formed. Thus, it seems necessary for hypotheses to adopt a form of representation that all members can interpret the content.
} In fact, numerous machine learning techniques have already been applied in scientific research, with many of them using machine learning models as generators of hypotheses. If that's the case, are there any specific technical challenges that need to be addressed when creating artificial intelligence capable of generating hypotheses in research? 

% \subsection{Machine Prediction and Hypothesis}
% The issue that becomes important here is the problem of representing knowledge and hypotheses. As mentioned earlier, research is the process of generating knowledge for a society constructed by certain agents. Therefore, it is necessary for the produced knowledge to be interpretable, or at least usable, by at least some members of the society, even if not by all of them. In human society, it seems that knowledge is made possible by expressing it in a form understandable to the members of the society, such as natural language or mathematical language.

% \textcolor{red}{This (commented out sentences) is about hypothesis generation automation so will be moved to survey section of perspective section}

% Particularly, is it merely a result of human cognitive constraints that I explicitly transform a question statement to find analogies, or is this an important aspect in hypothesis generation not limited to humans?

\subsubsection{In the Case of Hypothesis Generation Through Question Analysis}

In this chapter, I mentioned the possibility that humans formulate plausible hypotheses by analyzing questions and finding analogies with existing knowledge. However, the discovery of analogies, while complex, is also essentially part of the formulation of machine learning since it is merely pattern recognition. I already know that machines can find intricate patterns that humans might not discover without heuristic feature engineering. Then, is explicitly conducting such analysis and transformations of questions an essential aspect of hypothesis generation, even when machine conduct research? Or is it simply a result of cognitive limitations in humans?

The answer to this seems to vary depending on who the question's answer is unknown to. Firstly, let's discuss the scenario where the answer to the question is unknown to humanity. This scenario corresponds to utilizing artificial intelligence just for a hypothesis generator for humans. In this case, it may not necessarily be required for machines to undergo the intermediate step of explicitly converting the question, as it is possible for them to directly generate an answer to the question. This is because what is unknown to humans may already be known or obvious to machines. Currently, artificial intelligence demonstrates capabilities that surpass human abilities, and as their capabilities continue to advance, this trend is likely to accelerate further.

Next, let's discuss the scenario where the answer to the question is unknown to the machine itself. This corresponds to the situation where artificial intelligence itself engages in research as a researcher to explore the unknown from its own perspective. Here, artificial intelligence aims to generate knowledge rather than being a tool for human research. This is what I would like to realize in the future. In this case, I hypothesize that explicitly performing question transformation is important in hypothesis generation for machines, even if it doesn't need to be done in the same way as humans. If the most plausible answer for a question is always correct for the machine, it can be said that the question was not unknown to the machine in the first place. Therefore, if a question is unknown to the machine, it means that the machine needs to employ some means to extract patterns or structures that it is not capturing from the question. This is similar to situations where a machine is overfitting, being misled by spurious correlations, or failing to extract the patterns it should extract due to lack of out-of-distribution generalization ability. These are issues that will inevitably arise as long as machine learning continues to focus on minimizing errors as its central objective. For a machine to autonomously engage in research, it needs to be aware of being in such a situation and take actions to grasp the currently unapprehended patterns. Analyzing the question and converting it into a different form to facilitate the discovery of unknown patterns is precisely one of the most fundamental efforts in this regard. Therefore, it appears crucial for a machine to undergo such transformations in order to autonomously engage in research.

\subsubsection{In the Case of Hypothesis Generation from Verification Results}
In the sense of seeking causes from outcomes, this type of hypothesis generation can simply be described as causal inference. However, the difficulty in this type of hypothesis generation lies in the fact that there are countless possible factors that could be the causes behind the outcomes. Furthermore, it can be said that the more strongly I believe in a premise, the more it may not even be recognized as a premise to begin with.

One of the measures to address this is obviously to carefully design a verification plan. However, since this topic overlaps with the discussion in the section of hypothesis verification, I won't delve into it here.

One naive approach to tackle this issue in machine learning is to prepare data comprising a verification result and all possible premises, then train the model to predict the causes of verification. In fact, humans have also become capable of consciously considering premises that might influence experimental results based on factors that affected past experiments. For instance, if one had previous experience where room temperature unexpectedly influenced experimental outcomes, they would likely be more attentive to examining the impact of room temperature in subsequent experiments.

Additionally, instead of doubting all assumptions equally from the beginning, limiting the questioning to newly introduced assumptions for the current study can help narrow down the premises to be considered. Moreover, explicitly distinguishing the task of questioning assumptions with strong beliefs that might not be recognized as premises could be beneficial. In practice, even in human research, questioning these aspects based solely on the result of a single verification is not a common approach.

\textcolor{red}{TODO: FIX}

\subsubsection{How to Free from Commonsense}

One possible issue is that forcing machines learn the current value society has from vast amount of data might impede finding a innovative hypotheses. This is because such agent would be strongly influenced and biased by the current prevailing beliefs of the society. For example, during the time of Kepler, most people believed that planets move in circular orbits. If a machine trained in such an era, reflecting the biases of those people, were to generate hypotheses, it would likely exhibit a bias towards supporting circular orbits. The way I instill values into machines is precisely akin to this. For instance, RLHF (Reinforcement Learning from Human Feedback) essentially make agent do what the labelers have deemed correct. Teaching current human societal common sense to machines, not limited to RLHF, always leads to such issues.
 
% The problem that arises here is whether such learning processes can lead to the proposal of innovative hypotheses. For example, during the time of Kepler, most people believed that planets move in circular orbits. If a machine trained in such an era, reflecting the biases of those people, were to generate hypotheses, it would likely exhibit a bias towards supporting circular orbits. Therefore, it seems that more than just accepting what many people say as correct, some criteria, goals, or frameworks beyond that are necessary.

% Another issue is that a fundamental principle that what others consider plausible knowledge is deemed plausible as well. For example, widely used techniques like instruction tuning involve supervised learning where the training data consists of human or machine responses. In essence, the optimal solution is to mimic the behavior of these agents. Similarly, RLHF (Reinforcement Learning from Human Feedback) essentially learns to consider as correct what the labelers have deemed correct. In both cases, the common aspect is that the reward or training data tends to mimic the behavior of other agents in society.

As mentioned earlier, mathematics has been a powerful tool for freeing humans from such biases. Regardless of one's strong personal beliefs, the strength of deduction lies in the necessity of accepting the results that emerge from appropriate formal operations. Thus, it seems crucial in hypothesis generation to possess the ability to manipulate some form of deductive or system, even if it is not the same mathematical system as humans.

Another aspect that can be learned from human examples is that humans determine their level of confidence in knowledge through their own verification or tracking the verification conducted by others. This is closely related to the topic of verification, which will be discussed in the next section. Humans read books, papers, and other sources, carefully examining the claims and the verification processes presented in them. When they judge these sources to be trustworthy, there seem to be a significant number of individuals who increase their confidence in the validity of that knowledge, regardless of how it may differ from common sense. This attitude contrasts with the attitude of ``it's correct because many people say so'' and is referred to as a scientific and critical attitude. And, as I mentioned earlier, I believe this attitude can be rephrased as an approach that evaluates the reliability of knowledge through verification. Without such an attitude, it would be difficult to propose innovative hypotheses that challenge established theories. Therefore, the ability to verify hypotheses, as explained in the next section, also seems important in creating machines that generate hypotheses.

% \subsubsection{Conclusion}
% In this section, I have presented a hypothesis that the tasks involved in hypothesis generation may vary depending on who the answer to the question is unknown to. If the answer to a hypothesis is unknown to humans, it may be sufficient for the machine to generate questions directly based on the hypothesis. However, if the answer to a hypothesis is unknown to the machine, I have suggested that the machine needs to analyze and appropriately transform the hypothesis.

% Nevertheless, it is worth noting that the aspects discussed above can also be useful in the context of hypothesis generation for humans.

\subsection{Understanding and Explanation of Why a Hypothesis is True}
In research, an explanation often refers to making the process of hypothesis generation understandable to anyone.\footnote{
The need for explanations arises during hypothesis generation because hypothesis verification should be transparent to everyone, and the questions can be arbitrary. As mentioned earlier, it is possible to abandon hypothesis verification, but this would lead to serious issues in the entire knowledge production process. Since this article assumes hypothesis verification, I will not touch on this aspect.
} For example, when proposing a theory, others can understand why the theory holds by examining the proof procedure that led to the theory's derivation. Additionally, if I simplify phenomena into a simple model while generating hypotheses, I can gain a clearer understanding of what the hypothesis implies. Moreover, as previously mentioned, even without using mathematics, I generate plausible hypotheses through deductive reasoning based on assumptions. By examining this reasoning, I can understand why I chose to investigate that particular hypothesis.

When automating hypothesis generation, how to handle explanations becomes an important issue. This is because hypothesis generation does not necessarily require an explanation of the generation process. The purpose of hypothesis generation is to provide answers (candidates) to questions, and it is not always necessary to explain how those answers were arrived at. In fact, current mainstream machine learning methods such as deep neural networks trade interpretability for high hypothesis generation capability, making it difficult for humans to understand why certain inferences were made or why specific results were produced. The interpretability issue arises when using deep neural networks for hypothesis generation in scientific research. Increasing explainability for humans might come at the cost of reducing the hypothesis generation capability of the neural network, creating a trade-off relationship.

It is true that humans have historically provided explanations for hypotheses. However, this might be due to certain cognitive constraints. As mentioned earlier, humans could only generate plausible hypotheses by analyzing questions and engaging in logical reasoning. On the other hand, as I already know, machine learning models can directly extract complex patterns from data and answer questions that humans might not be able to.

So, is explanation merely an unnecessary byproduct of hypothesis generation, or does it serve some useful purpose? Eliminating explanations could lead to the following problems: 1. Reduction in the amount of information obtainable from knowledge production, and 2. Decline in future knowledge productivity. The reduction in the amount of information obtainable from knowledge production is a clear loss if explanations are deemed unnecessary.

The potential decline in future knowledge productivity arises from the fact that explanations increase the reusability of knowledge. Throughout history, humans have constructed hypotheses in the form of models with a small number of variables to explain various phenomena. 
\textcolor{red}{TODO: Explanation}

Regarding approaches to deal with the issue of interpretability, I believe there are at least three stances. First, one may completely abandon the explanation of hypothesis generation. As repeatedly mentioned, whether or not an explanation exists for how hypotheses are generated does not affect the reliability of knowledge as long as the verification process is transparent to us. Concerning the reusability of knowledge, it is not an issue if machines can reuse knowledge among themselves, even if it is not interpretable to humans. Therefore, this approach aims to separate interpretability and the reusability of knowledge and seeks ways for machines to reuse knowledge among themselves.

Second, one may adopt a stance where machines are required to produce explainable outputs. In this case, I should separate the process of hypothesis generation from the explanation of that process. Demanding that hypotheses be generated using interpretable "methods" can place significant constraints on the machine's capability for hypothesis generation. Moreover, since machines themselves do not produce knowledge in a way interpretable to humans, there are limits to understanding by examining the machine's internals. Instead, it might be better to have the machine itself translate the process of its output generation into a form understandable to humans. This is because the machine itself knows how it generated the hypothesis and is likely to perform better than humans in translating between different agents' tasks in the future.

Third, one may not require explanations from machines but instead seek to enhance my own human abilities to understand. This is because, under human cognitive constraints, understanding what machines are doing might seem limited. Therefore, the goal is to remove cognitive constraints and expand human capabilities. This is a long-term objective that cannot be achieved immediately, so I will not delve into it here. However, this is one possibility that should be seriously considered when thinking about the alignment between humans and AI, not limited to explaining hypotheses.
% \subsection{Prediction, Explanation, and Understanding}
% Humans, in order to derive hypotheses in response to questions, often require thinking and deliberation on why those hypotheses are considered plausible. Without such cognitive processes, it is difficult to generate hypotheses in many cases. This is not merely a cognitive constraint; it has played a significant role in the advancement of research. The explanation of how a hypothesis is derived and the reasoning behind its plausibility have provided crucial additional information for understanding the subject of study. This has assisted people in gaining a better understanding of the phenomenon.

% As mentioned earlier, predictions do not necessarily require such processes. Even without explanations for how such predictions arise, one can make inferences, and if they are rigorously tested, they can be considered knowledge. However, knowledge lacking such important additional information is expected to contribute less to the understanding of other knowledge and the comprehension of the entities targeted by the knowledge system. This research and understanding problem becomes significant even when artificial intelligence engages in research.

% \textcolor{red}{TODO: Add more}

% If it were not necessary to make inferences about the unknown, it would probably be because the question was not unknown in the first place. Thus that is not research by definition. It may not be explicitly stated, but as long as it is research, it implicitly involves some form of hypothesis generation. Therefore, the generation of a hypothesis, or the inference about the unknown, is essential for research in nature. Let's say, mathematics, a pure deductive field of research. In searching for lemmas to prove a theorem, I sometimes make predictions such as ``this lemma might be useful'' and examine it with few specific examples to check its usefulness. This may be considered as implicitly establishing a hypothesis and roughly verifying it. 

% In particular, science, by explicitly dividing the proposal of this plausible temporary answer and the corroboration of its tentative certainty into steps known as hypothesis generation and hypothesis verification, has enabled these processes to be carried out more systematically and rigorously. This is the well-known hypothesis-testing method, or scientific method. In this approach, a prediction about the unknown is explicitly stated as a hypothesis, and a procedure called verification is established to evaluate the validity of this hypothesis. Through this verification process, the evaluation of the hypothesis is conducted and the uncertainty towards the target unknown is reduced. This is the knowledge production based on the hypothesis testing method. 

% As a practice of human science, there are cases where discovery and justification are not strictly separate in actual scientific endeavors \cite{arabatzis2006inextricability}. However, functionally, discovery and justification can be classified separately, and it seems convenient to consider them as distinct in knowledge production that is not dependent on human conventions. Therefore, I will discuss them separately here.

% Although I said that hypothesis generation is the important part of scientific method, the hypothesis generation is not necessarily unique to the empirical science. It is a task that inevitably arises when dealing with the unknown. 

% \subsubsection{Note on Unknown}
% Let us make a note here about the term unknown again. I wrote that if an answer can be immediately derived by inference, it was not unknown in the first place. However, research is an activity to transform the unknown into the known, so in some form, unknown should eventually approach the known (reducing uncertainty). Moreover, if there is something that seems to be completely unrelated to the known or experiences (though it is hard to imagine), it seems inherently impossible to bring it closer to the known. In that sense, it is reasonable to think that the unknown has some relationship with the known. Therefore, the above description might be more accurate if it refers to cases where the confidence in the answer is lower, or the path to the answer is complex, and uncertainty is high. This is similar to the discussion on the definition of unknown in the previous section.

% \subsection{\textcolor{red}{Reference}}

%%%%%%

% \subsubsection{Good Hypothesis}

% As emphasized above, what I aim for is not merely imitating human knowledge production, but rather the autonomous practice of knowledge production that is not bound by human conventions. And in terms of its functionality, hypothesis generation is identical to inference towards the unknown. In that sense, while understanding how humans generate hypotheses can be very useful in considering how to achieve effective hypothesis generation, I do not consider it is always necessary.

% Of course, hypothesis may have additional constraints on top of the prediction. For instance, there might be an implicit assumption that a hypothesis is something described in natural language. However, I think this is only due to the conventions of current human society. What is functionally important for the purpose of knowledge production is to provide a tentative answer to the unknown, regardless of its form or property. In this sense, in this paper, I take the position that hypothesis generation is the prediction.

% Indeed, much of the current discourse on hypothesis generation focuses on specific research domains or discusses the exploration of hypotheses based on hypotheses that humans have explicitly defined. I understand this situation because hypothesis is an answer to a tentative question and hence it varies depending on the question. For example, if a study aims to enhance my understanding of a particular phenomenon, then the description or mechanism of that phenomenon would become the hypothesis. Similarly, if the objective is to solve a particular problem, then the solution to that problem would be considered the hypothesis. However, it is essential to consider how I can realize agents that construct candidates rather than just search for them, regardless of the specific research domain. This is a crucial question to address when contemplating artificial researchers.

% \subsubsection{Others}

% In engineering research, as part of new knowledge, it is often required to propose actual design plans or algorithms. This can be considered as having a similar function to a hypothesis in the sense that it is a proposal for addressing a problem and is evaluated in some way.

% In mathematical research, proving a theorem that was previously unknown is the production of knowledge. However, since mathematics is a deductive system, if the proof is correctly executed, it can be said to be ``correct'' in that sense. In other words, the proof itself is both the proposal and the verification. Therefore, mathematics is not a type of work that separates hypothesis and verification.

\section{Hypothesis Verification}
The third sub-module is hypothesis verification. Knowledge production in this paper is defined as the process of justifying beliefs in a way that convinces all members of the society. Hypothesis verification corresponds to this justification. As emphasized repeatedly, hypothesis verification in research aims to reduce my belief about the truth or falsity of a hypothesis to a strong conviction. In particular, science has been able to convince many people by reducing their belief about it to such a strong conviction that I can no longer live without assuming it.

I consider verification to be a particularly crucial process in the generation of knowledge. Questions and hypotheses can be generated in arbitrary ways, and they can take any form. However, verification must not be taken like this. Verification must be done in a manner that can convincingly update the shared beliefs of other members of society. Therefore, it is the most demanding process in the research journey in terms of rigor and caution. I believe that the rigor of this process is what truly sets research apart from other activities. The rigor and systematicity of the verification process is considered the very reason why science or research is referred to as rigorous \cite{sep-scientific-method,hoyningen2008systematicity,haack2003defending}.

\subsection{Statistical Hypothesis Verification}
Analyzing how statistical methods can validate hypotheses is crucial for automating hypothesis testing. Many research endeavors heavily rely on inductive inference to validate hypotheses, and statistics provide specific techniques for inductive reasoning. Therefore, I examine how I perform inductive inference using statistical methods, taking inspiration from Ohtsuka's arguments \cite{otsuka2022thinking}.

\subsubsection{Inductive Inference and Inference Statistics}

Inductive inference is a method of constructing general principles from data. I assume that data is sampled from a probability model and formulate the problem of inferring this probability model from the data as inferential statistics. The assumption of the existence of a probability model corresponds to the assumption of the uniformity of nature, which posits that similar situations will hold even in unobserved circumstances. As the uniformity of nature is a prerequisite for inductive inference, these formulations correspond to setting the foundation for inductive inference.

This probability model is assumed to be true as a prerequisite for inductive inference. However, in actual statistical inference, I perform inductive inference by establishing statistical models that approximate this probability model. In many cases, I assume specific families of distributions for these statistical models and reduce the problem to estimating parameters of these distribution families from the data. This is the formulation of inductive inference using statistical terms.

Within these formulations, the question arises of ``what kind of'' inductive inference I am performing. This precisely pertains to ``what is considered validated'' or ``how beliefs are justified.'' While multiple perspectives are possible, I will introduce discussions on classical hypothesis testing (referred to as just hypothesis testing below) and the approach using Bayesian statistics (referred to as just Bayesian statistics below).\footnote{
Note that this is not a discussion about what each position think about probability is, but rather a discussion on what they think is justification means.
}

\subsubsection{Bayesian Statistics and Epistemological Internalism}
In Bayesian statistics, I perform inductive inference by updating the degree of belief in a hypothesis based on the evidence provided by the data. The validity in inductive inference is adjusting the degree of belief in the conclusion in a consistent manner with the degree of belief in the premise. Representing the prior belief with prior probability and the likelihood with evidence, Bayesian statistics calculate the posterior probability following probability rules like Bayes' theorem. In this sense, Bayesian statistics, including Bayes' theorem, provides the logic for inductive inference. 

Ohtsuka argues that Bayesian statistics justifies beliefs based on a valid inference rule from other beliefs and compares this position to epistemological internalism \cite{otsuka2022thinking}.

\subsubsection{Classical Statistics and Epistemological Externalism}
In classical hypothesis testing, I first formulate some hypothesis regarding the probability model and then compare it with the data to either reject or retain the hypothesis. In other words, testing is a function that maps data to two options: rejecting or retaining the null hypothesis. Based on this result, I perform inductive inference (or behaviour). 

Ohtsuka interprets testing as a kind of examining tool that makes a judgment based on certain data and sees testing theory as a theory that measures the reliability of this examination tool. Making judgments about hypotheses based on data can be seen as a belief formation process, and testing theory educates the reliability of this process through measures like the size and power of the test. From this, classical statistical testing takes a stance that the justification of a belief is determined by the process through which it was formed, similar to epistemological externalism \cite{otsuka2022thinking}.

\subsubsection{For Automating Statistical Hypothesis Verification}

As such, when it comes to statistical methods for hypothesis verification, there are multiple perspectives on what constitutes justification. It is a challenging issue to known how various statistical techniques, each with their own approach, can provide evidence to substantiate a hypothesis based on the data \cite{otsuka2022thinking,sober2008evidence,sep-statistics}. Therefore, when seeking to allow machines to autonomously conduct hypothesis verification, determining what they should be capable of is a challenging issue.

\subsection{Challenges for Automating Hypothesis Verification}

\subsubsection{The Inherent Difficulty of the Act of Verification}
In the first place, the act of verification is a highly challenging process. Firstly, as mentioned earlier, inductive approaches cannot verify hypotheses in the same way as deductive reasoning. Also, I notice that hypothetico-deductive method, which involves verifying claims derived from a hypothesis to confirm its validity, is still widely used today. However, verifying a deduced claim does not support the hypothesis because there may be many hypothesis that can result in the same deduced claim. In response to these, Karl Popper proposed that while hypotheses cannot be confirmed, they can be falsified \cite{sep-scientific-method}. However, in practice, the verification of hypotheses involves implicitly relying on numerous auxiliary hypotheses. When using experimental apparatus, it requires many assumptions to trust them. Even when an experiment fails, determining whether the hypothesis was incorrect or the experiment itself was flawed is not as straightforward as one might think. Thus, there is inherent uncertainty in attributing the results of verification to a specific cause \cite{chalmers2013thing,sep-physics-experiment,sep-scientific-underdetermination} as I have discussed in the section of hypothesis generation. Furthermore, all reasoning and observational evidence are inevitably influenced by some form of theories, individuals, or societal factors \cite{sep-science-theory-observation}. Therefore, it is necessary to carefully examine them to ensure that they are not distorted by unintended influences. 

\subsubsection{The Difficulty of General Hypothesis Verification}
I recognize that the the way to verify a hypothesis is strikingly diverse, as it can significantly differ depending on the subject of research. For instance, if one wishes to study the behavior of rat, they may need to train the rat. On the other hand, if you want to test a theory of particle physics, you may have to construct and run a huge particle accelerator. In the field of history, the existence of historical records might serve as evidence, while in mathematics, the verification process revolves around the proofs themselves. Due to this high degree of flexibility, hypothesis verification can be the most challenging aspect to automate for realizing a ``general'' artificial researcher.

% \subsection{How to Justify Beliefs}
% In many empirical sciences, statistical methods occupy a privileged position as the primary means of verifying hypotheses. Therefore, it seems important to consider in what sense I can say that a statistical method can validate hypotheses, while I admit that the methods of verification vary greatly depending on the type of question or hypothesis and it cannot be said that there is any universal method for verification. The use of inferential statistics methods is widely accepted, but it is a challenging issue to known how various statistical techniques, each with their own approach, can provide evidence to substantiate a hypothesis based on the data in the real world \cite{otsuka2022thinking,sober2008evidence,sep-statistics}. Thus, what constitutes the verification of a hypothesis, or in other words, the justification of a belief, is a very difficult debate in which it is likely challenging to arrive at a unified answer. Please note that the discussions mentioned earlier, such as the uniformity of nature, pertain to the validity of induction itself. In statistical methods, however, induction is assumed and the discussion begins with formulating this uniformity by assuming a probability model. 

Furthermore, as explained in the section on statistical hypothesis testing, it was also described how even with inductive inference using the same statistical theory, there are multiple perspectives on how it justifies beliefs. Even empirical sciences, which employ highly universal verification methods based on statistical approaches, face such difficulties. Therefore, aiming for autonomous acquisition of verification methods in broader fields of research would likely pose even greater challenges. These issues will require further in-depth discussions in the future.

% This discussion of statistical method provides important insights for the pursuit of developing AI capable of conducting autonomous research. As can be seen from the assumption of being rooted in machine learning, the validity of inductive inference itself is generally accepted and not a practical concern in creating such AI. However, how to make agents acquire the methods of justification is a important issue. If the criterion for justifying beliefs is to be acquired completely autonomously from scratch, it may very well lead to criteria that are meaningless for humans. Additionally, it seems that the criteria for justification employed by humans are diverse. When properly considering the reasons that these criteria are believed to provide justification, I can recognize that there are highly intricate structures involved. In light of such considerations, the extent to which one can understand and acquire criteria for justification from the criteria humans already employ is a nontrivial problem. 


% I do not intend to claim that these uncertainties undermine the reliability of research. Such uncertainties exist in almost all human endeavors. Research, among these activities, strives to confront these uncertainties with great care and rigor. Above all, the fact that the results generated through research have effectively supported my lives demonstrates their efficacy.

% I mentioned the difficulties inherent in verification to highlight their significance, particularly when considering autonomous artificial intelligence capable of self-verification. Ideally, autonomous agents are expected to establish their own methods and criteria for verification. However, if verification inherently carries such difficulties and uncertainties, the more rigorously I consider it, the more paralyzed the agent becomes, as it seemingly cannot accomplish anything. \textcolor{red}{TODO: Add explanation}

% A characteristics of research can be found in the systematicity, rigor, and objectivity of research practice \cite{sep-scientific-method,hoyningen2008systematicity,haack2003defending}. 

% In particular, I believe that a characteristic of research lies not in the way of determining questions or generating hypotheses, but in the fact that the verification of hypotheses is done in an extremely rigorous and careful manner. 

% It could be said that I'm taking a view similar to the new experimentalism, placing emphasis on verification in research, or experiments \cite{chalmers2013thing,mayo1996error}.

% Of course, generating a hypothesis is not a simple task. What I want to say is that, as long as any method of hypothesis generation is properly verified, it is considered research, and no matter how properly a hypothesis is proposed, if the verification is sloppy, it is not considered research. This means that verification may be at the heart of knowledge production. In other words, in order to create artificial intelligence that produces knowledge, it is important to consider how to create an intelligence that can perform verification.

\subsubsection{The Difficulty of Autonomous Acquisition of Hypothesis Verification}
In the case of hypothesis testing, just like in hypothesis generation, the tasks for automating hypothesis testing vary depending on who the hypothesis testing is for. If artificial intelligence is used as a tool for human research, it is sufficient for artificial intelligence to faithfully reproduce what humans do as verification. For example, it would be great if it could conduct, for instance, hypothesis testing. If the machine is capable of automatically preparing for statistical hypothesis testing, calling the appropriate statistical hypothesis testing libraries at the right timing, and using them correctly for the relevant subjects, then I would have no complaints about their performance.

In this case, it is not necessary for the machine to strictly know why it constitutes verification, as long as it can learn from numerous examples of human verification and use it appropriately. In other words, in this case, the required understanding can be described as indirect and practical understanding through examples of human usage. Furthermore, if it can be confirmed that the machine not only mimics humans by using libraries but also understands what statistical hypothesis testing is and its principles, this would be a significant advancement in the automation of hypothesis testing. 

I used statistical hypothesis testing as an example, but the same applies to other verification methods as well. The point is that in this scenario, I seek to enable artificial intelligence to appropriately utilize the verification techniques employed by humans. However, it is essential to note that the verification itself remains relevant and meaningful to humans.

On the other hand, aiming to make machines understand and acquire the concept of verification autonomously from their own perspective becomes an immensely challenging task. This is because, as repeatedly emphasized, hypothesis verification involves the updating of beliefs, and the belief system of machines can significantly differ from that of humans. It doesn't seem that machines can acquire the concept of their own verification just by observing examples of human verification. Furthermore, as explained earlier, machines have not evolved their belief systems through interaction with the natural world. Therefore, verification methods entirely composed and developed by machines may no longer serve as effective tools for understanding nature. How to enable machines to autonomously acquire verification principles from the ground up, address the alignment issues between machines and humans, and ensure that these efforts lead to a better understanding of nature are challenges that the entire community should discuss and explore in the future.
% \subsubsection{Verification and Alignment}

% As I have explained repeatedly, research is the act of updating beliefs by reducing them to a common conviction that satisfies most members of society. Verification is precisely the act that carries out this belief update. Therefore, when considering knowledge production by agents that are not limited to humans, it is an important issue whether to completely rely on the criteria autonomously generated by the agent's society, including the act of verification. This is because what may lead to belief updates for a certain artificial intelligence group may not be the same for humans. In such a case, the knowledge produced there would be completely detached from humans. Therefore, when I want non-human intelligence to produce knowledge that is meaningful to humans, at least the criteria of verification should include elements that humans can agree with.

% Furthermore, as mentioned earlier, the reason why verification, even as the update of collective beliefs, has brought about such progress and benefits to the human society may be because the subject of research is (in a broad sense) nature and because human beliefs themselves are inherently constrained by nature. When I say that human beliefs are constrained by nature, as explained in the section on induction, it means that what I strongly feel to be certain is a reaction to what I have been exposed to through the process of evolution and development as organisms, interacting with nature.

% Considering this, when a fully autonomous artificial researcher is created, I think it is far from obvious whether they would autonomously conduct meaningful research on nature or be able to share knowledge with us humans without possessing at least these naturally constrained intuitions, sensory organs, and basis of beliefs. Therefore, when creating autonomous artificial researchers, it may be necessary to provide them with such nature-rooted beliefs in the form of some inductive bias.

\subsection{Towards Autonomous Verification}

As mentioned above, I believe that the process of verification is highly diverse. This diversity makes achieving an end-to-end approach immediately difficult. Therefore, it would be better to start by further breaking down the process of hypothesis verfication into more detailed sub-processes. For a practical first step for the verification process automation, I propose tentatively dividing the verification process into three stages: \textit{verification design}, \textit{verification instantiation}, and \textit{verification execution}. At each stage, you will create a verification plan, prepare for the verification, and carry out the verification.

\begin{figure}[htb]
    \centering
    \includegraphics[width=\textwidth]{figs/verification.jpg}
    \caption{Verification}
    \label{fig:verification}
\end{figure}

\subsubsection{Verification Design} 
In verification design, the agent will create a plan to execute the verification, which consists of hypotheses, verification criteria (criteria to determine if the output confirms or refutes the hypotheses), and the procedures for conducting the verification. The verification plan should be specific and detailed enough so that anyone faithfully executing it can reproduce the same verification process and result. For example, suppose I am considering an experiment to compare a proposed method with a baseline and validate the hypothesis that the proposed method performs better. In this case, the plan should describe in detail which model to use, what dataset to employ, and how to measure the performance. Thanks to this specification, I can separates the two functionally distinct stages of verification: the phase of considering what needs to be validated and the phase of actually executing the verification.

It is during the verification design stage that specific skills for hypothesis verification are required. This is because automating the preparation and execution of a plan involves the automation of more general intellectual tasks, not limited to research automation, while creating a verification plan demands an explicit understanding of what verification entails and what criteria constitute a successful verification. In aiming to automate this verification plan, the key lies in teaching machines how to understand the concept of verification. \footnote{
Even humans may sometimes do verification without proper understanding of the method. For example, someone may conduct hypothesis testing just because ``everyone does it.'' While this may be an extreme example, it is not common for researchers to understand the epistemological implications of inductive reasoning and statistical inference and in what sense I can say I verify a hypothesis. In that sense, the requirement for machines to understand verification may be somewhat challenging. However, if the aim is to truly enable AI to autonomously conduct research, it appears crucial for the AI to have a proper understanding of what constitutes verification.
}

% Multiple abilities are needed to create a verification plan, but based on the discussion so far, it is evident that at least two abilities are required.


% At this stage, I contemplate how to validate the hypotheses and proceed to formulate a verification plan. In a verification plan, the agent writes about the hypotheses, verification criteria (what will be considered as evidence for verification), and the procedures for conducting the verification. These plans should be as much detailed as possible. The Fig. provides an example in the context of machine learning. The idea is to create a blueprint for a pipeline where verification is automatically executed by faithfully following the plan.

% Furthermore, for artificial intelligence to perform autonomous verification, it seems essential that the AI not only adopts certain verification criteria but also be able to explain the meaning behind them and how they contribute to the verification process. Even humans may sometimes engage in this without conscious awareness, such as conducting hypothesis testing because ``everyone does it.'' In that sense, this requirement may be somewhat challenging. However, if the aim is to truly enable AI to autonomously conduct research, it appears crucial for the AI to have a proper understanding of what constitutes verification and be able to design it itself or, at the very least, explain it adequately.

There are several abilities that a machine must possess in the context of verification planning, but based on the discussion so far, it is evident that at least two abilities are required: understanding what verification is and planning to achieve objectives.

Let's first discuss the ability to understand verification. As mentioned in the sections on question construction and hypothesis generation, this is an extremely crucial ability in the overall automation of the research process. As before, the extent to which ``understanding'' is required depends on how much autonomy is expected in the automation of verification. If that is used as a verification tool for humans, what machines to do is just using human verification methods properly. On the other hand, if they were required to understand the concept of verification from scratch, I would face a lot of challenges as I just described in this chapter. 

If you aim to achieve the former, one of the initial steps could be creating a dataset specifically tailored for verification. By gathering research papers that utilize widely used verification methods like statistical hypothesis testing, you can train a model to construct verification methods from these hypotheses. One naive approach could be to start by generating method descriptions in research papers from the given hypotheses. 

If you were to achieve the latter, you might want to begin by contemplating what beliefs mean for machines. Alternatively, placing agents in situations where verification becomes necessary could implicitly help them acquire the concept of verification. In this case, addressing the issue of generalization across environments, such as enabling machines to explicitly reuse the concept of verification, will be crucial. Eventually, as repeatedly emphasized, discussing how to resolve the alignment problem will be necessary.

% If artificial intelligence is used as a tool for human research, it is sufficient for artificial intelligence to faithfully reproduce what humans do as verification. For example, it would be great if it could use basic concepts such as hypothesis testing, controlled experiments, and interventions and automatically create experimental plans based on them. In this case, it is not necessary for the machine to strictly know why it constitutes verification, as long as it can learn from numerous examples of human verification and use it appropriately. In other words, in this case, the required understanding can be described as indirect and practical understanding through examples of human usage. Furthermore, if it can understand the concepts of hypothesis testing and controlled experiments from first principles, it would be a significant achievement in terms of automatic verification.

% On the other hand, if artificial intelligence itself is allowed to conduct research for its own knowledge generation, it seems that artificial intelligence needs to understand what verification is, whether explicitly or implicitly. And similar to the unknown nature of the answer to a problem, it seems to be an ability that cannot be acquired just by observing examples of human verification. This is because verification involves updating the beliefs of the members of a society, which depends on the nature of the beliefs of the machine group itself. Whether the knowledge generated by such agents is understandable by humans or can say something about nature is not obvious, but this will be discussed in detail in later chapters.

Let us move on the second ability: planning. Understanding verification is a necessary condition for setting verification criteria by considering what can be tested against a given hypothesis. When conducting actual verification, under the assumption of a hypothesis and verification criteria, one must devise procedures for carrying out the verification. For example, in experimental research, let's say a hypothesis A is formulated, and a verification criterion is established that considers the hypothesis valid if it meets certain criteria through statistical hypothesis testing. To actually perform this verification, it is necessary to generate data to be used for verification, and if there is no apparatus to generate the data, one may need to create it. Thus, in a verification plan, one must develop a plan to fulfill the purpose of executing the verification.

Creating a plan is known to be a challenging task, not limited to verification plans. To achieve a goal, one must understand what is necessary and comprehend the appropriate sequence of steps to achieve the goal. As mentioned earlier in the section on constructing questions, when creating a plan, it is essential to consider feasibility, taking into account the complex external factors such as current financial capabilities and accessible resources. Understanding and incorporating these constraints appropriately can be a highly challenging task, as they are intricately linked to various external factors.

While there are already a difficult task, the particular challenge in creating verification plans for research lies in the fact that one may need to create things necessary for verification if they do not already exist. This is an extremely high-level difficulty problem. To carry this out, one must first accurately identify what is currently lacking. After recognizing the deficiencies, one must consider how to create what does not exist. Once the method is known, materials for creating it must be prepared, and it must be actually built. And even if it is created, one must investigate whether they function properly. This is an unbelievably complex task, and it seems highly unlikely to simultaneously expect the creation of such intermediate products by directly generating a verification plan from the verification itself. Therefore, in aiming for true automation of verification, it is crucial to seriously consider how to solve this problem.


Finally, once the necessary elements for verification are understood, they need to be represented as something that can be understood by other researchers. This is not a requirement of a verification plan but a necessary condition for knowledge to become knowledge for society. In order to generate knowledge for society, it is necessary not only to verify hypotheses but also for the verification procedures to be understandable to other members of society and judged as valid. \footnote{
While it is desirable for the question generation process and hypothesis generation process to be publicly available, it is of utmost importance that the verification process is disclosed. 
} Also, as discussed in the section on hypothesis generation, verification must ensure that the causes behind the verification results are traceable as much as possible. While this is not an inherent requirement of the act of verification a hypothesis, it is a crucial characteristic needed for proper interpretation of verification results and for better knowledge production through hypothesis testing. As mentioned earlier, it is impossible to enumerate all potential assumptions, so it becomes necessary to explicitly state assumptions as comprehensively and in as much detail as possible to make the process realistic. Recognizing the underlying assumptions, including those behind the scenes, and selecting which assumptions to prioritize as potential causes of the results, is indeed a challenging task.



\subsubsection{Verification Instantiation} 
At this stage, the research plan that has been developed is translated into an executable instance. For example, if the research plan states, ``Train the model B with the dataset A...'' the necessary steps would involve acquiring dataset A from the appropriate source, formatting it to be compatible with the model's input requirements, and preparing the data for training, etc. Similarly, if the plan states, ``When a rat presses the switch B, food is dispensed...'' the agent has to prepare rats, food, and create a machine that dispenses food upon pressing the switch, and so on. This process involves translating the plan into physical or computational instance for execution.

\begin{figure}[htb]
    \centering
    \includegraphics[width=0.5\textwidth]{figs/verification_instantiation.jpg}
    \caption{Verification Instantiation}
    \label{fig:verification_instantiation}
\end{figure}

As evident upon reading, this process is highly challenging to automate. Even research confined to the realm of computers, such as research of computer science, requires accomplishing a vast number of complex tasks. For research that necessitates physical realization in the real world, the development of robotics and embodied agents are necessary. Regarding the question of where and how to tackle these problems, I will provide my perspective in the later chapter. However, it is important to note that unless the research is constrained by questions and hypotheses, aiming for true automation will inevitably require overcoming these challenges

To reiterate, this process can be accomplished through the automation of actions, and it does not necessarily require unique technological developments for research automation. For instance, in studies that are entirely computer-based, automating all actions within the computer system would suffice. In recent times, attempts have been made to have language models operate computers, and the progress in such research corresponds to the advancement of automation in this process. If the research involves activities in the physical world, achieving full automation of human movements (or their equivalents) would make this process feasible. Therefore, research focused on developing humanoid robots that can move like humans will contribute to the automation of this process.

% Once a hypothesis has been established, a verification plan is created to determine how to verify it. The specific method of verification depends on the subject being investigated, making this aspect of research difficult to structurize and automate in a unified way.

% However, in many empirical sciences, the likelihood of a hypothesis is evaluated based on statistical significance. This is done by \textit{hypothesis testing} in practice. As this is a hypothesis test, it can only reject the null hypothesis, rather than directly determining the correctness of the hypothesis. Therefore, it can only be said that the hypothesis has survived for the time being. The belief that the surviving hypothesis is more likely to be valid is the basis for decision-making.

% In any case, humans seem to use statistics or probability as the basis for assessing the validity of a hypothesis. In other words, I seem to concede to consider a hypothesis as plausible if something that cannot happen by chance, such as observing the same number repeatedly. This is based on the assumption of the ``principle of confirmation,`` which assumes that if the number of observations increases, it can be considered more reliable, and the ``principle of uniformity,`` which assumes that things will continue to proceed as they have been if the conditions remain the saus. These beliefs ultimately serve as the basis for verification and scientific knowledge production. 

% I will not delve into the validity of these beliefs here. What matters is that my research activity follow a practice that ``when a hypothesis is present, and a certain criterion and procedure are prepared, and the hypothesis is considered valid according to that procedure, I consider it valid.''

% In theoretical research, sometimes there is no verification plan. Theory is a hypothesis, and its validity is determined separately through verification (not in the sense of whether it is mathematically valid, but for example whether it explains physical phenomena or not). However, in complex modern science, theorists propose a theory, and experimentalists verify it.

% Therefore, it is understood that in current research practice, shared knowledge in the form of papers may not necessarily provide a complete answer to a given question. This is similar to research on negative data. Negative data cannot solve the unknown initially declared, but it can reduce a certain degree of uncertainty towards it. This is because the validity of the presented hypothesis may have decreased somewhat. If this is the case, each research shared in the form of a paper may be more appropriate to describe as "reducing uncertainty towards the unknown," rather than "making the unknown known." This can become complicated when scrutinized strictly, so let's put this aside for now and continue to discuss how "producing new knowledge" is research.

% In reality, conducting research is expected to be done with limited resources (time, funding, computing resources, people, etc.). Therefore, it is necessary to consider these resources when determining the verification approach. After a research design is determined at an abstract level, the feasibility of the research plan is roughly evaluated through a simple problem setting. This is known as a pilot study.



\subsubsection{Verification Execution}
Finally, the instantiated research plan is executed according to the prescribed procedures. In verification design, I am crafting the essence of the verification, and during verification instantiation, I am preparing it to be executed in a concrete and feasible manner. Therefore, there are almost no tasks to be done in this process. The preparation for verification, being distinct from the actual execution of verification, is not a verification in itself. With this understanding, I have adopted this formal classification for the current context.
\footnote{
Typically,``experiments'' refers to the data generation process and subsequent analysis and interpretation are conducted separately. However, since I am discussing a verification plan in this context, all of them are in the same single plan. Therefore, please note that what emerges from executing the verification plan is not data but the verification results.

In many empirical studies, the data generated from experiments is often used not only for the verification of hypotheses but also for generating new hypotheses or giving some insights. However, as hypothesis generation and hypothesis verification play different roles in knowledge production, I do not assume any uses of the generated data beyond verification in this context. Of course, please note that this does not imply that such actions are prohibited in practice. Data analysis will be discussed in a separate section.
}

\subsubsection{Starting from Verification Design}

As mentioned above, I believe that the verification process can be divided into three stages: formulating a verification plan, preparing for the execution of the verification plan, and executing the verification plan. Among these stages, the execution of the verification plan and the preparation for it require interaction with the physical world in many fields. For example, in certain fields, you may need to purchase and raise rats for training, while in others, you may need to observe physical objects directly. On the other hand, formulating a verification plan is a process that is purely confined to the human mind in a wide range of fields. Therefore, it can be a good idea starting from automating verification design process.

% As mentioned earlier, in the case of empirical sciences, testing is often performed. Therefore, data is first generated, processed, and finally verified using the processed data. If I summarize the process of generating and processing raw data as data generation, this process can be broadly divided into data generation and judgment based on verification criteria. It may be rather said that the act of research itself is a process of repeatedly generating data and performing some kind of processing on it.

% I separated the verification plan from the verification because I want to separate the description and execution of the process. The verification plan is analogous to coding, while the verification is more similar to executing the code.

% The output of this process is usually wrtitten in the result section in the paper.

\section{Conclusion}
In this chapter, I discussed what research is. Research is the act of producing new knowledge for a community of constituents capable of forming shared beliefs. I characterized the production of new knowledge as the process of posing unanswered questions, generating hypotheses in response to those questions, and verifying them. Additionally, starting from the standpoint that knowledge is belief, I also presented the depiction of research as belief updating.
\textcolor{red}{TODO}