\chapter{Conclusion}

In this paper, we explored what needs to be done to create intelligence capable of autonomously conducting any research. Firstly, we examined the definition of what constitutes research. We then proposed the idea that research might be the act of producing new knowledge for a society and that producing knowledge might be updating the collective beliefs of a society. As a result, we discussed that the core of research lies in formulating questions, generating hypotheses, and verifying those hypotheses. We also discussed the implications provided by the relativity of the research subject to society. Next, we briefly introduced examples of initiatives trying to automate research. While there has been significant progress, we explained that there are still barriers to realizing a general-purpose and autonomous artificial researcher. Lastly, based on these discussions, we debated the challenges we believe are crucial in realizing a general-purpose and autonomous artificial researcher. As a first step, we proposed building a prototype of an autonomous research pipeline driven solely by general instructions.