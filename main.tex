\documentclass{article}
% \documentclass{book}

\let\cleardoublepage\clearpage

% \setlength{\parindent}{0pt}
% \setlength{\parskip}{5.5pt}

\usepackage[english]{babel}

% Set page size and margins
% Replace `letterpaper' with `a4paper' for UK/EU standard size
\usepackage[letterpaper,top=2cm,bottom=2cm,left=3cm,right=3cm,marginparwidth=1.75cm]{geometry}
\usepackage{graphicx}
\usepackage{xcolor}
\usepackage[colorlinks=true,linkcolor=blue,citecolor=red]{hyperref}

% \setlength{\parindent}{0pt}

% \usepackage{natbib}
% \newenvironment{abstract}{}{}
\usepackage{abstract}
\usepackage{here}
\usepackage{tabularx}

\usepackage[edges]{forest}
\usepackage{tikz}
\usepackage{tikz-qtree}
\usepackage{longtable}

\hypersetup{
    colorlinks=true,
    linkcolor=[RGB]{255,34,255},
    citecolor=[RGB]{255,34,255},
}

\usepackage[whole]{bxcjkjatype}  % Japanese

\title{Speculative Exploration of Artificial Intelligence that Can Conduct Research}
\author{Shiro Takagi \footnote{The LaTeX file of this perspective paper is stored on GitHub. The paper is still under active development, and because I have not read all cited literature in detail, it may contain error. If you have any proposals to improve this paper or find any mistakes and misunderstandings, please feel free to submit a pull request to the repository. Let's work together to create a better paper! \\ GitHub: TODO. }}

\begin{document}
\sloppy
\maketitle
\tableofcontents

\begin{abstract}
This paper presents a speculative discussion on artificial intelligence capable of conducting research. First, it examines what research entails, aiming to lay the groundwork for a discussion on what it would mean to create AI capable of conducting research. Next, it focuses on the construction of questions, generation of hypotheses, and verification of hypotheses, which are considered essential elements in research, and discusses the possibilities and challenges of AI capable of these tasks. Finally, the paper discusses what could be the first steps towards realizing AI capable of conducting research.
\end{abstract}


\chapter{Introduction}

\section{Towards Understanding this Universe}
It has been a long-standing goal of humanity to understand nature, the universe, and the world. Through this intellectual curiosity, mankind has made remarkable intellectual advancements and uncovered astonishing findings about nature. While perfect understanding of this universe may be infeasible, the desire to know as much as possible about as many things as possible in this universe is a shared sentiment among researchers.

However, it seems that there are limitations to achieving this goal solely through current human capabilities. This is because humans have several cognitive constraints when it comes to understanding phenomena. For instance, there is a limit to the amount of information humans can process, and comprehending highly complex, nonlinear, and non-equilibrium systems in their entirety is challenging. Such constraints appear to narrow the scope of nature that can be understood. \textcolor{red}{TODO: update the explanation of cognitive limitation}

Furthermore, there is a possibility that the methodologies of knowledge production employed by humans currently may not be optimal. This is because the research practices we have developed are heavily influenced by historical and societal factors. Many of these practices have emerged in a relatively short span of history, and there is still a vast unexplored territory in the methods of conducting research.

We believe that artificial intelligence capable of autonomously conducting all research holds great potential in overcoming these challenges and getting closer to the goal. Firstly, AI is not constrained by a physical device like the human brain, enabling processing speeds and capacities that are orders of magnitude beyond human capabilities. Secondly, by optimizing AI for knowledge production, it becomes possible to generate knowledge more efficiently, without being constrained by extraneous factors commonly encountered in human research. Lastly, and most importantly, AI has the potential to possess a different understanding of nature compared to humans. Moreover, AI might have the capacity to comprehend nature more extensively than humans. If that's the case, it seems essential to develop autonomously researching artificial intelligence if one aims to acquire as much knowledge as possible about this universe. Therefore, we propose the pursuit of autonomous AI researchers to realize these possibilities.

\section{Purpose of this Perspective Paper}
The purpose of this perspective paper is to provide a stepping stone towards achieving a general and autonomous artificial researcher as quickly as possible. It aims to offer information that enables all those pursuing this goal to focus on essential issues, avoid reinventing the wheel, and maximize their own potential. This is because we believe that enabling the full potential of brilliant researchers worldwide is the most crucial for accomplishing this objective.

To achieve the purpose of this perspective paper, we will discuss three key points in chapters 2, 3, and 4, respectively: What is the goal, where are we currently standing, and what to do to achieve the goal. First, in chapter 2, we will delve into the discussion of ``What is the goal'' to make the common objective clearer for everyone involved. By doing so, researchers and engineers all over the world are able to devise methods to achieve this goal even before looking at our proposals in this paper. Furthermore, this discussion aims to clarify the limits and possibilities of a general and autonomous artificial researcher. Specifically, we will thoroughly discuss ``What is research?'' as it is essential to understand what a ``general'' artificial researcher entails. Starting with the definition of research, we will examine the functions that various research activities serve and discuss the implications of achieving this through artificial intelligence.

Next, in chapter 3, we will discuss ``Where are we currently standing'' to work towards a state where researchers pursuing research automation no longer need to gather preliminary information on their own. This is expected to enable new researchers engaged in research automation to concentrate on the actual tasks they should be working on and reduce the risk of reinventing the wheel. To accomplish this, we conduct a literature review on research automation. We will organize studies related to research automation, discuss their relationships and current status. In this survey, we will strive for as much comprehensiveness as possible, including anything relevant to research automation, no matter how superficial, as we believe that once researchers become aware of the research area, they can delve deeper into specific aspects on their own. Additionally, we observed a lack of information sharing among various individual efforts related to research automation, and by addressing these diverse approaches in a unified manner within one paper, we aim to promote information exchange between different fields.

In chapter 4, we will present our proposals on ``What to do to achieve the goal.'' Building upon the clarity of the goal from chapter 2 and the understanding of the current state from chapter 3, we will discuss how to bridge the gaps between them. We will also explore the directions and specific approaches for research and development, suggesting the policies and avenues that may be most effective. Through these discussions, our aim with this paper is to providing useful information for as many individuals striving for research automation.

Chapter 2, 3, and 4 focus on creating a general and autonomous artificial researcher, with only minimal suggestions about its impact on society once completed. Chapter 5 will address issues such as the impact on society from the perspective of ``alignment.'' Furthermore, we will discuss how alignment can play an essential role not only in mitigating the risks of AI but also in automating research.

\section{Pursuing Better Research}
Please emphasize once again that what we are aiming for is to achieve a better way of knowledge production. The realization of general and autonomous artificial intelligence is a means to that end. Therefore, even if certain research practices are widely adopted at present, we will strive for better methods if they are not essential for knowledge production and if there are superior alternatives. In other words, we propose to pursue the realization of artificial intelligence that enables a different and improved way of knowledge production, rather than simply automating the current research practices.

While the practices of research developed by humans are highly sophisticated, robust, and productive, it is not necessarily true that all of them have reached the absolute optimum way of doing research. Firstly, research practices are strongly influenced by history and society. For example, peer review is widely accepted today, but it is said to have gained such dominant status because foundations in Cold War-era U.S. demanded peer review to ensure accountability \cite{baldwin2018scientific}. Moreover, the fact that research outcomes are still represented in the form of papers is clearly a remnant of the era when print was predominant. These practices might have been optimal under the social conditions and technologies of that time, but they may not necessarily be optimal in the present, as society has changed a lot. Additionally, meeting societal demands does not always lead to the optimization of knowledge production. Secondly, it seems that we sometimes lack a widely shared understanding of how to optimally carry out certain aspects of research. For instance, it is important to formulate good research questions and hypotheses, but there are still many aspects about how to do this effectively that remain unclear, and it appears to rely heavily on individuals' tacit knowledge. In such a situation, simply replacing current practices with machines may not guarantee a good knowledge production process. Thirdly, as Nielsen and Qiu say \cite{nielsen}, it is believed that we have only explored in a very limited subspace in the space of possible research practices. The establishment of current research practices is a very recent event in human history, and we may not yet have arrived at the optimum way of research. \textcolor{red}{TODO: Add examples.} Finally, as we mentioned above, current research assumes that humans are the main creators and consumers of knowledge. This naturally imposes human cognitive constraints, which may significantly limit the range of activities that can be conducted for knowledge production. Based on these reasons, we aim to progress the discussion in a way that allows us to identify what is fundamentally essential for knowledge production, while also drawing inspiration from the positive aspects of past practices. We will not be bound by the current approaches but instead strive to clarify what is truly crucial for research.

\section{Call for Cooperation}
I would be delighted if all of you could contribute to writing this research paper. The initial draft of this paper was created by a novice machine learning researcher named Shiro Takagi. We have made efforts to write the text as carefully as possible, but due to his limited research experience, He may not be able to provide profound insights into the topic, and there might be inaccuracies in his understanding of fields that are not his expertise. In particular, the literature review chapter prioritizes comprehensive coverage of the literature, which means that the evaluation of individual references might not be appropriate. Also, there are naturally limitations and biases regarding the scope of papers that can be covered and the research areas targeted. Therefore, if you find any errors, missed literature, or points for improvement in the paper, we would be grateful if you could let us know.

The manuscript is being managed on GitHub\footnote{\url{https://github.com/t46/research-automation-perspective-paper}}, and contributions from everyone to the paper are welcous. The version you are currently reading is a provisional one, and we plan to continue updating it regularly. If you have any suggestions for modifications, it would be incredibly helpful if you could submit a pull request to the repository.

Moreover, this paper is licensed under the OO license, so feel free to make any modifications you see fit. If you wish to follow the main structure, you can fork the repository, or if you have a different approach in mind, you can create an entirely new project.

We hope this paper can contribute in some way to the advancement of automation in research. Thank you.


% Additionally, this paper is intended to be updated on an ongoing basis. If you find points that you think are lacking or inaccuracies in understanding, please send a Pull Request on GitHub. Based on that, we will revise the content of the paper as needed.

% \section{Others}

% The purpose of this perspective paper is to discuss the possibility of automating this activity of research, that is, the realization of intelligent agents capable of conducting research autonomously. After summarizing attempts at research automation so far, we intend to present our personal perspective on promising ideas, necessary elements, and ways to proceed in order to create artificial researchers. In particular, this paper is aimed at machine learning researchers and developers, with the goal of increasing the number of colleagues aiming for research automation by providing concrete possible actions.

% One thing to note here is that our aim is not to automate or substitute the current research tasks, but to create intelligent agents capable of conducting research. In other words, as long as it is research, automated research may at first glance seem to be far removed from the current research activities. This is because the current research activity may not necessarily be the absolute optimal form in light of its purpose. 

% First, the optimal form is always influenced by the context of history. Of course, the optimum in the era when letterpress printing was mainstream is different from the optimum in the era when the Internet became infrastructure. Research itself has changed its form over time, and the optimal form of research in the present or future can naturally change. Moreover, as Nielsen and Qiu say \cite{nielsen}, it is believed that we have only explored a very limited range regarding the practice of research. The establishment of current research practices is a very recent event in human history, and we may not yet have arrived at the optimum way of intellectual production. Furthermore, current research assumes that humans are the main creators of knowledge. This naturally imposes human cognitive constraints, which may significantly limit the range of activities that can be conducted for knowledge production. 

% Therefore, instead of thinking about how to automate and streamline current tasks, we discuss the possibility of realizing intelligent agents that can autonomously perform the fundamental and core elements of research activity in a more optimal form, and what is necessary for this.

\textcolor{red}{TODO: Add lots of high-level illustration}
\section{What Would Activities Called Research Be?}
\label{section-what-is-research}

To create an agent capable of conducting research autonomously, it is crucial to understand what research is in the first place. Therefore, in this section, I would like to discuss the characteristics of what is called research. Particularly, I will strive to discuss characteristics that can be found activities called research, regardless of differences in fields such as humanities, natural sciences, and mathematics.

However, please note that my aim here is not to identify the universally acceptable definition of research, which is beyond my ability. Rather, I will endeavor to describe various characteristics of research to provide a starting point for engineers and researchers who aim for realizing research-capable AI. Therefore, the definition of research that I'm going to discuss in the following sections is merely a tentative and operational one. I hope that further discussions among researchers will deepen towards a better characterization of research.

\subsection{Research as Knowledge Production}

% 正当化の目的が真理促進的であることを書いたら、for society はいらない気がする?認識である以上主体は必要だというのは間違いない。全部書いてみて後で検討。

% 研究による成果には、偽であるが真であるように見える命題がある。なので、知識の生産というよりも、知識の候補を生み出しその確からしさについての確証を強めていくという方が適切かもしれない。偽であった時にも有用な情報。なので、知識を生み出すというよりも、知識を生み出すことを試みるとした方がいいかも。

% \textcolor{red}{
% 知識を生み出すではなく、知識を生み出すことを目指す・試みる、とする。なぜなら命題は必ずしも真ではないし、偽であると適切に判定されることは有用な情報をもたらすから。そう考えると、「この世界の未知の真理を明らかにしようとすることが研究」という方が適切????←知識の中に「真である」ということが含まれているのと、正当化に真理促進性を要求してるので、どちらでも変わらない気がする?知識を生み出そうとするだと生み出せなかった時が単に失敗という感じになってしまう?違うな。検証が命題Aを肯定すると、Aという知識が生まれるが、命題Aを否定すると、Aの否定という知識が生まれるというだけの話。なので問題なし。なので、「知識を生み出すことを試みる行為」でいい気がする。
% }

\textcolor{red}{
既存の定義に触れる
}

It seems infeasible to find a unified, all-encompassing definition of research or science \cite{chalmers2013thing,sep-scientific-method}, but there are various thoughts on what research is. For example, there is a view that ``we do research whenever we gather information to answer a question that solves a problem'' \cite{booth2003craft}, and it is also said that research (and development) ``comprise creative and systematic work undertaken in order to increase the stock of knowledge'' \cite{manual2015guidelines}. These characterizations each capture important aspects of research, and none of them are necessarily wrong, nor is any one of them absolutely right.
% 研究や科学の統一的な全てを包括する定義というものはありえなさそうですが \cite{chalmers2013thing,sep-scientific-method}、研究とはなんであるかについてはさまざまな考えがあります。 例えば、``we do research whenever we gather information to answer a question that solves a problem'' \cite{booth2003craft} という考えもありますとし、research (and development) は ``comprise creative and
% systematic work undertaken in order to increase the stock of knowledge'' 言われることもあります \cite{manual2015guidelines}。

Among such characterizations, a widely accepted definition of research would be that research is the attempt of generating new knowledge. It seems that research, regardless of the field, requires some novelty, and creating new knowledge based on past knowledge. Therefore, this characterization seems to be a good starting point for considering what research is, at least for the time being. In this paper, I will tentatively adopt this characterization as a working definition of research. In particular, I will consider that research is the attempts to produce new knowledge for certain society. I included ``for society'' because knowledge appears to be relative to society. I will explain this point in later sections. 

% \textcolor{red}{
% In particular, I will consider that research is the production of new knowledge for certain society. I included ``for society'' because knowledge appears to be relative to society, as I will discuss later. 
% }

% For instance, in physics, a new explanation for a phenomenon is produced, in mathematics, a new proof for a theorem is presented, and in engineering, a new design blueprint for creating something is generated, all as new knowledge.

% \begin{figure}[htb]
%     \centering
%     \includegraphics[width=0.8\linewidth]{figs/reseach_definition.pdf}
%     \caption{Definition of Research}
%     \label{fig:definition}
% \end{figure}

% \subsubsection{Some Notes on Research and Science}

% The reason why I deliberately use the word ``research'' instead of ``science'' is because I want to include fields like engineering, humanities and arts, which are not typically referred to as science, within the scope of automation in the long run. Science refers to a methodology for generating knowledge, and I believe that new knowledge is not necessarily produced only by scientific methods. I believe that so-called humanities and arts also share some commonality of producing new knowledge. Therefore, the definition of generating new knowledge can be said to encompass these fields as well.

% Of course I admit that science is the most rigorous and reliable framework for knowledge production. Because of its power and popularity, most of the existing analysis on research is about science. My discussion would also be centered around the discussion of science, though I try to present the view not limited to science. 

Please note again that this definition is provisional. What is important is to consider how research can be characterized, and the discussion here is just one example of that. I believe that finding a better characterization of research through further deepening discussions is crucial for realizing agents capable of conducting research. I hope that the discussion here can aid in such endeavors.

% \textcolor{red}{
% 論文の目的をもっと先に明確に書くことで、こういうただし書き的なのを減らす。
% Please note again that this definition is provisional. By viewing research as the production of knowledge, we can indeed characterize the wide range of current human research activities comprehensively, which is why I adopted this definition. However, as we will see later, in attempting to create a comprehensive definition, it can result in aspects that seem counterintuitive to characterize certain research. It's possible that aiming for a definition that is not field-specific is not desirable (i.e., striving for the realization of an intelligence capable of such a task is not desirable), and it might be better to seek a definition with a more narrowed scope of application. I will revisit and discuss how research should be defined at the end of this section.
% }

\subsection{Knowledge Production as Belief Revision}
\label{section-knowledge-production-as-belief-revision}
I have defined research as the generation of new knowledge. Then, what exactly is knowledge, and what does it mean to produce knowledge? I will explore this question in this section. 

Defining knowledge and knowledge production rigorously is a philosophical debate that has not yet been settled \cite{sep-epistemology}, and I won't delve into it deeply here. Instead, I would like to provide some primitive ideas that can serve as a starting point for further discussions on how to realize an artificial researcher.

\subsubsection{Knowledge as Belief}
The question of what is the thing called knowledge has been a subject of debate for a long time in the field of \textit{epistemology}, which is one of the branches of philosophy. So, for now, I would like to refer to the discussion in this field as an example and see how research can be characterized if I do so.

In epistemology, knowledge had been classically considered to be \textit{justified true belief (JTB)} \cite{sep-epistemology}. The term ``true'' is difficult to define rigorously, but for the purpose of discussion, let's think of it as something being fact. ``Belief'' can be provisionally understood as someone's thought or conviction about something. And ``justified'' means that it is deemed reasonable to hold such a belief. What justification means has been discussed in epistemology as a central point of contention. Since the criticism that JTB is not appropriate as the definition of knowledge \cite{gettier1963justified}, how to modify or expand JTB to make it a suitable definition of knowledge has seem to be a major discussion in epistemology \cite{sep-epistemology}. 

While most philosophers don't seem to believe that these properties are sufficient to define knowledge, there seems to be some agreement that they may be necessary \cite{sep-epistemology}. Even in the epistemology discussion, it seems that rather than abandoning the JTB altogether, the discussion seemed to center on how to expand upon it, using the JTB as a base. Therefore, let me tentatively assume that knowledge is JTB in this paper as a starting point for the discussion. In the following section, I will examine how research is perceived as an activity when adopting this definition. Examining the discrepancy and alignments between the consequences of this examination and what we expect from a research-capable AI will provide seeds of thought for a better definition of research and hence what that AI should be able to do.

\subsubsection{Knowledge Production as Belief Update}
The purpose of research can be said to be revealing the unknown truths of this world. Therefore, the justification in research is demanded to be such that it determines propositions as true if they are true, and false if they are false. Such justification is called truth-conducive. There are various discussions about what justification is, but it can be said that justification in research should be truth-conducive.

Therefore, producing knowledge could be said to be constructing a new proposition that refers to this world and determining its truth or falsity through such justification. If knowledge is regarded as belief, this can be rephrased as holding the belief that a proposition is true and updating this belief through truth-conducive justification.

% Furthermore, as I will discuss later, we justify the belief that a certain hypothesis is true or false in an extremely robust and sound way that would convinces everyone. Given such justification, it seems not so problematic to depict research through the subjective concept of belief in a substantial sense.


% 知るということは、私たちに対してこの世界とその真実があり、私たちがその真実を認識するということです。すなわち、知るという概念はあくまで私たちと世界との両方の存在を前提としている概念であり、その意味で知る主体に依存する概念です。知識の定義に信念という主観的な概念が出てくるのはこのためです。したがって、知識が信念であるとは直感に反するように思われるかもしれませんが、この意味で妥当であるように思われます。

% 知識が信念であるというのは、知る/知っているという概念がそれを知る主体に依存する概念だからです。私たちの外側に世界があり、その中のある真実を私たちが認識するというのがこの定義の意味での知るという行為であり、それはあくまでも世界とそれを認識する主体との関係性を前提として成り立つ概念です。


% It may feel counterintuitive to hear that knowledge is belief. However, there is a reason for this. It is because the concept of ``knowing'' depends on the subject.

% It may feel counterintuitive to hear that knowledge is belief and research is the updating of belief. However, I think that perceiving the production of knowledge as the updating of beliefs is not so unreasonable. This is because, for example, the validation of a hypothesis by an experiment can be interpreted as a strengthening of the belief that the hypothesis is true. Specially in science, we become more convinced of its validity as a hypothesis survives repeated various verifications. This research practice appears to be well aligned with the view of research as a renewal of beliefs.

\subsection{Knowing Depends on Knowing Subjects}

% \subsubsection{知るということと主体の存在の関係}
To know means we recognizing the truths of this world. In other words, the concept of knowing fundamentally presupposes the existence not only the known object but also the knowing subject. This is why the definition of knowledge involves the subjective concept of belief. Therefore, while the idea that knowledge is belief may seem counterintuitive, it appears valid in this sense. Particularly, since justification in research should be truth-conducive, the knowledge produced would be objective. In this sense, using belief, which is a subjective concept, to define research does not seem to be much of a problem.

Moreover, viewing research as the updating of beliefs does not seem too far removed from the practice of research itself. For example, the validation of a hypothesis by an experiment can be interpreted as a strengthening of the belief that the hypothesis is true. Specially in science, we become more convinced of its validity as a hypothesis survives repeated various verifications. This research practice appears to be well aligned with the view of research as a renewal of beliefs.


\subsubsection{Knowledge for Humans}
As mentioned earlier, research is an endeavor aimed at revealing the unknown truths of this world, and the knowledge generated there must be novel. Then, what does it mean for knowledge to be new or unknown?

Knowledge was a justified belief about whether a certain hypothesis is true or false. Therefore, unknown knowledge could be a state where such a belief is not held at all, or even if it is held, it is not justified. I believe that this is what ``unknown'' means.

Since the concept of knowing depends on the knowing subject, naturally, the concept of the unknown is also subject-dependent. In research, we do not consider knowledge unknown just because a single individual is unaware of it. It is only when none of us know it that we consider it truly unknown. That is, it seems that the knowledge we create through research is required to be knowledge for all of humanity. This is why the definition adopted this time includes ``for society''.


% 前述したように、研究はこの世界の未知の真理を明らかにすることを目指す営みであり、そこで生み出される知識は新規である必要があります。それではある知識が新しい、あるいは未知であるとはどういう状態でしょうか。

% 知識がある仮説が真である/偽であるという正当化された信念でした。であるならば、未知の知識とは、そのような信念がそもそも抱かれていない、あるいは信念が抱かれていてもそれが正当化されていない、状態だということができるでしょう。

% 知るという概念は知る主体に依存した概念でしたので、当然未知という概念も主体に依存した概念です。研究においては誰か一人の個人が知らないだけではその知識が未知であるとみなしません。我々の誰一人としてそれを知らない時に初めて未知であると考えます。すなわち、私たちが研究で生み出す知識は、人類全体の知識であることを要求されているように思われます。これが今回採用する定義に for society を加えた理由です。


% No matter how rigorously you produce knowledge, if it is already known, it cannot be called research. Therefore, we have defined research as the process of producing ``new'' knowledge. In this section, I will briefly discuss what would it mean for knowledge to be unknown or novel.

% It seems that something being unknown to someone can be described as a state where the person has attempted to know something, but the answer is not present in their mind. Therefore, it can be said that when knowledge born from research of a question is considered new, it refers to a situation where there has been no hypothesis that everyone believes to be sufficiently plausible for the question.\footnote{
% While presenting a question that no one has posed ensures that the knowledge is new. Even in such cases, no one also yet knows which hypothesis is correct for that question. This is why I have mentioned only about the unkownness of hypotheses
% }
% In other words, once knowledge is considered as belief, being unknown seems to be interchangeable with having a low level of confidence, which sounds a bit counterintuitive. In this sense, it seems appropriate to say that research transforms a hypothesis with low confidence into one with higher confidence, rather than turning the unknown suddenly into the known.\footnote{
% Here, saying ``there has been no hypothesis that everyone believes to be sufficiently plausible'' is merely to imply nuances that knowledge is belief and that a hypothesis is never fully proven to be true. You can just read it as ``there is no answer'' for simplicity.
% }



% As mentioned later, research can be described as the act of posing questions, proposing hypotheses in response to them, and then verifying those hypotheses to update beliefs on whether the hypotheses are correct. Therefore, it can be said that when knowledge born from research of a question is considered new, it refers to a situation where there has been no hypothesis that everyone believes to be sufficiently plausible for the question\footnote{
% While presenting a question that no one has posed ensures that the knowledge is new. Even in such cases, no one also yet knows which hypothesis is correct for that question. This is why I have mentioned only about the unkownness of hypotheses
% }. For example, we do not know how to create a general-purpose artificial intelligence, which can be said to mean that we do not have an answer (a hypothesis with high confidence) to the question, ``How do you create a general-purpose artificial intelligence?''.

% I said in the paragraph above that ``research is the process of transforming the unknown into the known.'' However research can also be seen as the updating of beliefs, as I have explained. Therefore, the binary depiction of an object suddenly transitioning from the states of unknown to known does not seem appropriate. Rather, it seems more reasonable to consider that beliefs continuously change and I just call some group of belief states unknown and others known, for convenience.

% I don't know precisely what it means to be unknown. This is a difficult problem, but let's consider it naively. 
% I consider a knowledge to be unknown when for a question a subject lacks any hypothesis which he/she has a JTB that the hypothesis is true. For example, I do not have the answer to the question of ``How to realize an artif icial general intelligence.'' So the knowledge of ``the way to realize an artificial general ingelligence'' is unknown. \textcolor{red}{TODO: Add explanation}

% In research, a single verification does not always immediately turn a hypothesis into true or false. Rather, hypotheses that withstand repeated verifications through various experiments by different researchers gradually come to be regarded as more plausible. Therefore, it seems to me appropriate to say that research transforms a hypothesis with low confidence into one with higher confidence, rather than turning the unknown suddenly into the known. In that sense, it seems somewhat justified to describe new knowledge as something for which there wasn't a highly confident hypothesis before.

% The expression ``unknown'' could be more appropriately expressed as ``high degree of unknownness''. Because knowledge production is a continuous concept of belief updating, therefore, unknownness is also a continuous concept. In reality, there are multiple hypotheses with varying degrees of certainty concerning a particular question. The state of knowledge being unknown can be expressed as not having any hypothesis among these hypotheses with a particularly strong level of justified certainty.

% First, research begins with a particular question. The state of not knowing the answer to this question is what I consider the state of being unknown. In other words, it can be thought of as a state where I don't know what the candidate hypotheses, which are the potential answers, are like, or a state where I know the candidate hypotheses but don't know their plausibility. These are states where I have not been able to find hypotheses or sets of hypotheses that can be assigned a particularly high degree of confidence from the set of potential answers to a given question. Therefore, provisionally and casually, it might be said that the state of ``not having highly confident beliefs (or a set of beliefs) for a particular question'' is unknown, and the state of having highly confident justified beliefs is known.

% Of course, there are issues with this clarification. For example, it is unlikely that I can select a single highly confident hypothesis from the entire set of possible hypotheses. Also, rather than feeling equally confident about all possible hypotheses, it seems that I implicitly distinguish between hypotheses that seem relevant and those that do not. Furthermore, it is unclear to what extent I consider a state to be unknown based on the degree of confidence. However, these are highly challenging philosophical discussions, so I will refrain from delving further into them and, for now, would like to conclude with the vague and provisional definition above and move on to the next topic.

% \subsection{Publicity of Knowledge}

%%%%%%%%%%%%%%%%%%%%%%%%%%%%%%%%%%%%%%%%%%%%%%%%%

% What I want to emphasize here is that research can be seen as linking or replacing the weak belief in the plausibility of newly conceived hypotheses with such extremely strong beliefs. For example, believing in the effectiveness of statistical methods is strongly related to believing in the effectiveness of inductive reasoning. Therefore, if the validity of a hypothesis is confirmed using statistical methods, we would believe it as highly reliable. Hence, considering research as the updating of beliefs can be reasonably felt, and I don't intend to argue that research is meaningless just because it is the belief revision.\footnote{
% I naively assume that beliefs rooted directly in perception are more robust, and the notion that anchoring a belief on those foundations enhances its reliability. However, the question of how to justify beliefs is a complex problem with extensive discussions \cite{sep-epistemology} (my position seems to align somewhat with what is called \textit{empirical foundationalism}).Due to my limited philosophical knowledge and the ability to delve deeper into philosophical discussions, this paper will not further pursue these arguments.
% }

% It is said that we need to assume that ``under the same conditions, the same phenomenon will continue to hold'' (principle of uniformity of nature) \cite{sep-induction-problem}, which is an intuitive and natural assumption without which we could hardly even go about our daily lives.\footnote{
% I understand that justifying the validity of inductive reasoning based on our experiences would be circular reasoning. The problem of what assumptions make us inductive reasoning consider rational is a challenging unanswered philosophical issue, which is beyond the scope of this paper. Thus, I will refrain from delving into the details here and leave it for future discussion.
% } 

%%%%%%%%%%%%%%%%%%%%%%%%%%%%%%%%%%%%%%%%%%%%%%%%%%


% For instance, while most empirical scientific research verifies hypotheses through hypothesis testing, it seems that this is because results deemed plausible through statistics or inductive reasoning are thought to be plausible by virtually any human.

% Since knowledge production is belief update, generating knowledge for humanity requires updating the beliefs of all, or at least certain number of, people.\footnote{
% Even if the knowledge is not immediately understandable, it should be potentially understandable to a considerable number of people. This means that the generated knowledge is not only for the society at the time of its production but also for the broader humanity, including future human societies. For instance, it might be challenging for children to determine whether the research outcomes in physics qualify as knowledge. However, by studying physics extensively, they may eventually comprehend cutting-edge research in the future. 
% }\footnote{
% Because it's impossible for different individuals to have exactly the same belief, as long as a justification can update the beliefs of different individuals in a similar direction in some way, I think it sufficient to be regarded as updating shared beliefs.
% } 

% Therefore, I believe research is the act of not only changing an individual's belief but also changing multiple individuals' beliefs. 

% Although I don't think it is possible for multiple people to hold exactly the same belief or for their beliefs to change in exactly the same direction, I at least need to use a way to change a collection of human beliefs in a similar direction.

% In other words, this can be said to underscore the plausibility of the belief that ``a hypothesis is true'' with the extremely robust belief that ``inductive reasoning is valid.'' These robust beliefs seem to have been shaped through the process of evolution for the long time, being rooted in humans biologically. It seems that because science appeals to such fundamental beliefs inherent in humans as biological entities, its verification manages to convince many people.
% are rooted in our biological structures and perceptions acquired through the processes of evolution and development, as briefly introduced in the preceding section. For instance, beliefs such as the validity of logical reasoning, the uniformity of nature, and the tendency to find something more credible with an increase in observations fall into this category of beliefs. 

% And I believe that I achieve this by reducing hypotheses to strong convictions that everyone, regardless of individual differences, possesses as human beings. For example, assumptions underlying inductive reasoning, as I described earlier, will fall into this category. Another strong conviction humans believe in is that the more I observe an increasing number of results derived from a certain hypothesis, the more reliable and certain that hypothesis feels. Research need to be objective because it's required to generate knowledge for humanity, and I believe it is because I strive to reduce hypotheses to such strong convictions that it is called objective. 

\subsubsection{Knowledge for Non-Humans}
If the act of knowing is dependent on the subject, then it's theoretically possible to consider non-human knowledge and non-human research by assuming knowing subjects to be non-human. 

Although it might seem insignificant since most people desire AI that produces knowledge for humans, I believe the observation that what is unknown to a machine may differ from what is unknown to humans provides implications for what we should do to realize an AI capable of conducting research.

When what is unknown differs between humans and machines, creating a machine that can research in the same way as humans may not necessarily produce new knowledge for humans. This is because some of the methods developed by humans are aimed at revealing truths unknown to themselves, and even if a machine can mimic these methods, it would only reveal truths that are unknown to the machine, which is not necessarily unknown to human.

Considering this, if the goal is to create AI that produces knowledge for humans, then AI should either be used as just a tool to assist human research, develop new methodologies that discover the unknown for humans through machines rather than merely mimicking human methods, or, if the AI is allowed to research autonomously, there might be a need to teach it what is unknown to humans. These are merely a list of possibilities, but it is hoped that there will be further discussion in the future about whether these could truly become issues.

% %%%%%%%%%%%%%%%%%%%%%%%%%%%%%%%%%%%%%%%%%%%%

\subsubsection{Truth-Conducive Justification Developed by AI?}

While beliefs are indeed subjective, in research, justification is expected to be conducive to truth, thus making justified knowledge objective. This is because whether a proposition is true or not determines whether that belief is justified, regardless of what we think.

Since our research involves inductive reasoning, we cannot ``prove'' that our justification methods are strictly truth-conducive, but considering the numerous discoveries we have made through them, I believe these are valid justifications. If AI can fully master and appropriately utilize these justifications developed by humans, it would be enough for them to uncover numerous unknown truths.

So, is it possible for an AI to construct new truth-conducive justification methods on its own instead of just mastering human developed methods? Human beings, for example, have developed statistical hypothesis testing as a method for evaluating hypotheses, so can AI similarly develop its own unique methods for evaluating new hypotheses?

As I mentioned that we cannot ``prove'' our justification to be truth-conducive, our justification is based on several premises. This means that we make an interpretation of ``how a justification is truth-conducive'' in some sense, and depending on that interpretation, there actually exist multiple methods of justification even in the case of humans \cite{otsuka2022thinking}.
% \footnote{
% For example, validating a hypothesis through statistical hypothesis testing essentially means believing that the inductive reasoning, which is a premise of statistical hypothesis testing, is valid. Using inductive reasoning implies believing in its premises, such as the principle of uniformity of nature \cite{sep-induction-problem}. Of course, there is no guarantee that this principle always holds, but it is such a natural assumption that has been shaped by nature through evolution and development and doubting it would make living practically impossible.
% }
% \footnote{
% I naively assume that beliefs rooted directly in perception are more robust, and the notion that anchoring a belief on those foundations enhances its reliability. However, the question of how to justify beliefs is a complex problem with extensive discussions \cite{sep-epistemology} (my position seems to align somewhat with what is called \textit{empirical foundationalism}).Due to my limited philosophical knowledge and the ability to delve deeper into philosophical discussions, this paper will not further pursue these arguments.
% }

\textcolor{red}{may remove the part of nature}
Therefore, when allowing AI to autonomously construct up to the method of justification, the AI is expected to make value judgments about ``how a justification is truth-conducive'' and to select the optimal method of justification. In the case of humans, who have bodies formed through interaction with nature in the process of evolution and development and a belief system based on these bodies, it seems that they have already acquired a value system useful for understanding nature. Whether machines, which do not possess such a value system, can construct such methods of justification is an open problem.

This is not merely a philosophical inquiry. This is because, as long as we do not understand what constitutes the most truth-conducive justification, there is a possibility that better methods of justification may exist. Moreover, machines, not bound by cognitive constraints like humans, theoretically have the potential to discover such justifications. Particularly, the quality of a truth-conducive justification can be determined regardless of whether it involves humans or machines. Therefore, in the sense that its quality can be determined without human judgment, there is a possibility that machines could autonomously construct it. If this were to happen, it might enable the production of knowledge that humans alone could not have achieved.

\subsection{Conclusion}
In this section, I have discussed a provisional working definition of research. I started with the naive intuition that research is the endeavor to generate new knowledge for a society. I then explained that knowledge is belief, the production of knowledge involves updating beliefs, and the produced knowledge needs to be novel and supported by the common strong beliefs of a community. Lastly, I discussed based on these conclusions the possibility of research conducted by agents other than humans.

The definition discussed here is merely a provisional one based on the naive intuition. By combining insights from philosophers, scientists, and all working researchers, we can engage in a deeper analysis of the definition of research, developing more fruitful and reliable guidelines for our goal.

\section{Question, Hypothesis, and  Verification}

\label{section-question-hypothesis-verification}

\textcolor{red}{question, hypothesis, verification を並べて、
context of discovery と context of verification だとどこがまとめられるか
question answering だとどこがまとめられるか
みたいなのをまとめた図はあってもいいかもしれない}

In the previous section, I examined the conceptual definition of research and its implications. In this section, I will discuss the widely recognized essential elements in research: question construction, hypothesis generation, and hypothesis verification. Throughout this section, I aim to advance the discussion more concretely than in the previous section on what research is and what it could mean for machines to be capable of it.

It goes without saying that the construction of questions, the generation of hypotheses, and the verification of hypotheses are widely accepted as indispensable elements in research. What I will discuss is how these can be characterized as activities, what roles they play in the production of knowledge, and what it could mean for artificial intelligence to perform these tasks. Additionally, at the end of this section, we aim to provide a foundation for discussion by briefly exploring how we combine these elements in conducting our research and what an AI capable of research must be able to do

% 問いの構築、仮説の生成、仮説の検証が大事なのはパッと聞くと当たり前なので、なぜわざわざ言ってるのかを書く。

% \textcolor{red}{TODO: more focus on the implication for research automation}

% It is believed that research began with individual and concrete tasks. Among them, common actions were patterned and crystallized as a scientific method. I currently recognize this abstract set of behaviors as research. For example, hypothetico-deductive method and hypothesis testing are abstracted scientific method.

% Also, researchers use a research paper as a medium of knowledge transfer. Therefore, there are patterned activities related to a research paper. Examples of these include conducting surveys, gathering information from papers, and writing a thesis.

% Note that these are necessary tasks just because I use a paper as a medium of knowledge transfer, but they may not necessarily be indispensable for generating new knowledge. There are other such tasks as well. For example, peer review and fund raising are essential to current research practices in society, but they may not necessarily be indispensable for knowledge production.

% In this way, various tasks arise in conjunction with research. When considering the automation and optimization of research, it is desirable to consider streamlining all of these tasks. However, in this article, I focus on the process from determining a research topic to publishing a research paper. I will refer to this process simply as the \textit{research process} from here on.

% \subsection{Overview}

% As mentioned earlier, research is an attempt to turn the unknown into the known. Therefore, the research process can be seen as a function that takes the unknown as input and outputs the known. However, in reality, a single research paper may not be enough to turn the unknown into the known. Therefore, in practice, the research process is considered to be a procedure that takes the unknown as input, and outputs a text that describes the procedures and their results, as well as their interpretation, in order to turn the unknown into the known.

% First, let us structure the common research process. In particular, I will base the structuring of the research process on the method of empirical science, which many researches rely on as a foundation. However, I believe that this framework can be applied to other research activities, such as mathematics, as well. I will explain the reason for this later.

% The research process, especially that of empirical science, is carried out through the following steps: topic decision, hypothesis generation, verification design, verification, and analysis of experimental results. The outputs of these steps are then written into a paper, which undergoes peer review and is eventually published.

% Note that some commonly seen items, such as surveys, are not included here for a reason. First, as mentioned earlier, gathering information from papers is only a means of knowledge transfer through the use of a thesis. Second, information extraction from papers can be done at any stage of the research process. Thus, I believe that processing related to a paper, such as \textit{reading papers} and \textit{writing a paper}, needs to be considered separately from the aforementioned research process.

% \subsection{Overview}

% \textcolor{red}{TODO: reconsider the research process, structure of knowledge production system, and the scope of this paper}

% The definition of research stated in this section is somewhat too abstract to serve as a concrete guideline for building something based on it. Moreover, it feels distant from the research practices employed by humans, making it challenging to immediately connect our standing point to with the goal. While the ultimate goal is to achieve a fully aunotomous system for conducting research, it is practical and useful to start by discussing how to realize individual sub-processes within the research process. Breaking down the research process into partial processes will make the functions that need to be achieved much clearer.

% Therefore, I view research process as a \textit{knowledge production system} and attempt to structure the process by decomposing it into its constituting modules. During this division, each sub-process will be structured with the level of abstraction required for all types of research. This is because the focus of this paper is a general artificial researcher. 

% Hence, even if an element seems crucial in current research, if it is not necessarily essential to the definition of research seen so far, I will treat those elements separately from the constituents of the research process in this study. By separating these elements, I intend to clarify the indispensable components for realizing an artificial researcher. 

% \subsubsecti on{Outline of the Structure}
% In this section, I will discuss the three things tightly related to the knowledge production. I will first discuss the functionally essential elements for knowledge production system, which is the abstract structure of the research process that has been emphasized so far. These elements correspond to the modules in knowledge production system. This is a main topic of this chapter. 

% For convenience, I will refer to the entire structured research process as the \textit{knowledge production system} or just \textit{research process} throughout the rest of this paper.

% In the previous chapter, I discussed the definition of research. In this chapter, I will focus on the high-level abstract structuring of the research process while paying attention to its functional aspects. By emphasizing the functional aspects, I mean paying attention to the role that each step plays in knowledge production. By higher-level abstract structuring, I intend to focus on the processes that is as universal as possible across research fields, regardless of the specific domain. The purpose of this kind of structuring is to clarify what kind of modules should be created as intermediate steps of research when aiming for research automation. In the following, I will structurize the research process into a chronological sequence for the purpose of clarity. However, it is important to note that the focus lies not on the temporal order nor human convention, but rather on the functionality in relation to knowledge production and the inputs and outputs of each of these processes. Also, as previously mentioned, because humans are currently the primary knowledge generators, there are many constraints that come from human society. Thus I will do my best to distinguish and organize what is dependent on humans and what is not. Before delving into specific discussions, let us first explain the scope of this chapter. After that, I will discuss the outline to be addressed in this section, followed by the main discussions.


% \textcolor{red}{TODO: add the excuse that research is social activity}



% \begin{figure}[htb]
%     \centering
%     \includegraphics[width=\textwidth]{figs/researchprocess.jpg}
%     \caption{Caption}
%     \label{fig:research_process}
% \end{figure}


% Next, I will discuss a high level description of how human beings have been conducting research. I'll structurize the abstract pattern of the process (which I will call\textit{ research process}) from determining the unknown to it turning into known. 


% \subsubsection{Note}

% note, direction
% Though my structuring may seem to represent scientific methods, I believe this pattern cam apply to other research fields, such as mathematics and humanities as well. When describing the structure, I will make a conscious effort to clearly distinguish between essential elements for knowledge production and those that are not. As previously mentioned, because humans are currently the primary knowledge generators, there are many constraints that come from human society. When considering the possibility of machines conducting research in the future, it will be important to distinguish and organize what is dependent on humans and what is not.

% I believe that the conduct of human research activities can be roughly divided into three stages: knowledge production, knowledge evaluation, and knowledge sharing. 

% Although these may not necessarily be distinctly separable from each other, I adopt this classification because it is useful for advancing discussion. The process of knowledge production consists largely of the steps: problem determination, hypothesis generation, and hypothesis verification. And in this process, the ability to read and write documents and analyze data are required as necessary skills. Below, I will examine each of these in more detail. 


% This structuring is tentative and there may be a better way to structure the research process. However, I have created this structure for practical purposes in order to move the discussion forward. I hope the structure of this article be a starting point for conceiving a better structurization. I believe that structuring and deepening understanding of the elements that are essentially important for knowledge production is extremely crucial when aiming for the automation of research.

% Though I explained that research is belied revision, it would be convenient to see as a function that takes the unknown as input and outputs the known.  % TODO: rearrange

% \subsubsection{Three Sub-Modules in Knowledge Production System}
% In the following sections, I will explain three essential elements that I think are functionally necessary in knowledge production: \textit{question construction}, \textit{hypothesis generation}, and \textit{hypothesis verification}. These are the abstract structure of the research process that has been emphasized so far.


% \footnote{
% While the term ``verification'' is used here, it does not imply a definitive determination of the truth or falsehood of propositions. It is used in the sense of just strengthening beliefs as used in everyday language. Thus, ``confirmation'' may be a more accurate term, but verification is more widely recognized, so I will use that.
% }

% When summarizing these intermediate steps, it goes as follows: First, upon receiving any input, I formulate a question. Taking this question as input, hypotheses are generated. Finally, these hypotheses are input and validated resulting in the creation of knowledge as a research outcome. These processes are considered a proper decomposition of the research process, as they request each other's outputs as inputs and seamlessly connect the entire research process from input to output without any gaps.
% \textcolor{red}{TODO: However, I can identify discovery with justification because both are belief updates. I'll add this point wherever in this paper.} Fig. \ref{fig:knowledge_production_system}.

% \textcolor{red}{TODO: Add fig like this. add this sentence to the explanation of the fig "we approach the unknown to the known by asking questions, providing tentative answers to them, and then verifying the validity of those answers. " }.
% Fig. \ref{fig:research_process}.

\subsection{Question Construction}
\label{section-question-construction}

← 問いの生成のところで関係するかも(Distributed Science - The Scientific Process as Multi-Scale Active Inference)

The first element is \textit{question construction}. In order to produce new knowledge, one must be aware of what they don't know and strive to generate that unknown knowledge. This process of deciding the unknown to be investigated can be regarded as questioning. And generating the candidates of the answer for that question is hypothesis generation. That is, research may be rephrased as the act of posing questions and answering them. Moreover, as long as there is an attempt to reduce uncertainty, questions are necessary, and we generate multiple questions in the process of research other than the research question. Thus, in the act of research, which confronts the unknown, question construction is inevitable.

In relation to efforts concerning the generation of questions by machines, there are studies aimed at discovering research questions and challenges from academic literature \cite{lahav2022search,liu2023creative,oppenlaender2023mapping,surita2020can}, and others focused on identifying research trends \cite{krenn2022scientific,krenn2022predicting}. However, efforts to let machines autonomously generate questions on their own are not as prevalent. In the field of question answering, there is a task for generating questions \cite{pan2019recent,zhang2021review}, but these studies have different motivations than generating research questions. There has been research on creating artificial curiosity to generate non-textual questions \cite{schmidhuber1991possibility}, but there is yet nothing that can generate research questions like a human. With the advent of large-scale language models in recent years, there have been attempts to generate research questions \cite{liu2023creative,lahat2023evaluating}, but this field is still in the early stage.

While the attempts for automation exist, compared to the those for generation and verification of hypotheses, these efforts are limited. Automating the construction of questions, or setting the goal behind, is recognized as a challenge that needs to be addressed in research automation research \cite{coley2020autonomousII,zenil2023,kitano2021nobel}. In this section, I would like to start by reviewing what questions are and what it means to pose a question in the first place. Them I would extend the discussions to open problems to realize an AI to do research.

% As for attempts to formalize the discovery of problems in the field of machine learning, there is work by Zhang \cite{zhang2021problem}. On the other hand, in this section, we aim to examine the characteristics by qualitatively emphasizing aspects related to the generation of research questions.

\subsubsection{What is Questioning?}
Asking questions seems to be defined as an information seeking behavior \cite{watson_2021,taylor1962process}. The act of seeking information is considered to consist of two steps: recognizing an \textit{information need} and then taking action to acquire that information \cite{wilson1997information,case2016looking}. Not all information-seeking behaviors involve linguistic expressions \cite{watson_2021}, but in research, we always form a query expressed in text between the steps of the information need recognition and the onset of information seeking behavior. Especially, in research, the process of question construction generally seems to refer to the process leading up to this query formulation. Therefore, in this paper, we regard question construction as the process leading up to the query formulation. The subsequent process of seeking information will be treated as part of hypothesis generation and verification.

The process of recognizing an information need appears to involve at least two sub-processes: recognizing the missing knowledge and deciding to fill it (judging that missing information is ``need'').\footnote{
Here, I explained the process of recognizing information need as first determining something as unknown and then deciding if you construct question about the unknown. However, the order doesn't matter. For instance, there may be knowledge that you want to know first, and then you confirm subsequently that it is indeed unknown. What matters is that the process involves these two elements.
} Therefore, to have an AI to construct questions, it would be necessary to consider how to instill these abilities to the AI. In the following sections, I will discuss these steps. However, please note that the question in research is not a personal one, but a question for society. This difference can bring about slight variations in the process of constructing questions. I will touch upon this point again later.

% 情報ニードを認識するという過程は、少なくとも、未知を認識することと、それを埋めことを判断する(欠落した情報がニードであると判断する)という二つの過程を伴うように思われる。したがって、これらの過程について検討する。

% Questioning seems to be an act of trying to fill a gap in the information that the questioner possesses. For example, a philosopher Watson defines a question as an information seeking act \cite{watson_2021} and curiosity, a concept related to questioning, is defined as intrinsically motivated information seeking \cite{kidd2015psychology}. In that sense, the act of trying to produce missing knowledge for humanity is the very act of posing and answering a question. In this paper, we provisionally consider that constructing questions in research is an act of seeking missing knowledge.

% The process of questioning seems to involve two steps: 1. Recognizing missing information, and 2. Attempting to fill that gap. For example, the process of asking why the sky is blue starts as follows: when prompted by, for example the vision of sky to your eyes, you suddenly (with complex perceptual and cognitive processes) realize that you don't know why the sky is blue, and then, if you would like to know the answer, utter the question, ``Why is the sky blue?'' Therefore, to have an AI to construct questions, it would be necessary to consider how to instill these abilities to the AI. In the following sections, I will discuss these steps.\footnote{
% Here, I explained the process of questioning as first determining something as unknown and then deciding if you construct question about the unknown. However, the order doesn't matter. For instance, there may be knowledge that you want to know first, and then you confirm subsequently that it is indeed unknown.
% }

\subsubsection{Recognizing Unknown}
To recognize that certain knowledge is unknown to you, it seems necessary to first attempt to refer to that knowledge. It appears that when we refer to our own knowledge base and do not find the knowledge, we judge it to be unknown. For an individual, the knowledge base is the memory within the brain. On the other hand, the unknown researchers wish to clarify is unknown to a certain society. That is, a research-capable AI does not need to judge whether it is unknown to itself, but rather it can directly determine whether it is unknown to certain society. Therefore, the knowledge base can be not only its own memory but also the knowledge base of the society, e.g. a collection of research papers.\footnote{
As previously mentioned, being unknown means that a proposition either does not exist or, even if it does, it is not accompanied by a justified belief. Therefore, it seems that in order to precisely determine unknowingness from academic papers, it seems to be necessary to judge whether each paper has been justified, that is, whether verification has been appropriately conducted. 
}

\textcolor{red}{ここに論文からの情報取得の自動化の話入れる?可能ならここではない論文のどこかで IR の枠組みの話してその流れで述べたい。結局機械にとって未知のものじゃないと仮説生成がそのまま答えの生成になるのでは?}

% It seems that in order to recognize that certain knowledge is unknown to you, you need to attempts to reference that knowledge and finds it unavailable to you. When prompted with a request for a specific type of knowledge and upon referencing your own knowledge base, if you judge that you do not possess that knowledge in your knowledge base, you recognize it as unknown to you. For example, ``why question'' is driven by the request of the knowledge of ``reason'' or the ``cause''.\footnote{
% Such demands for knowledge would likely be triggered by various factors (for example, attention may be drawn to the causal relationship between the sky and its blueness because one tried to explain why the sky is blue but couldn't). Identifying the mechanism of what and how questions are prompted remains an open problem with no answer yet.
% }

% For an individual, the knowledge base is the memory within the brain. If certain knowledge is not in the memory, or cannot be constructed from other knowledge that is in memory, we would recognize that we don't know that knowledge. 

% In research, what interests us is not whether something is unknown to an individual, but whether it is unknown to humanity. Thus, the knowledge base consists of the collective media that records human knowledge, such as the research papers. If we examine as many academic papers as possible and find no prior research that has presented a properly validated hypothesis for the same question, we recognize that knowledge as unknown.\footnote{
% As previously mentioned, being unknown means that a proposition either does not exist or, even if it does, it is not accompanied by a justified belief. Therefore, it seems that in order to precisely determine unknowingness, it seems to be necessary to judge whether each paper has been justified, that is, whether verification has been appropriately conducted. 
% } Even in a world where machines conduct research, this method of determining unknowns will likely persist.

If machines would determine the novelty in a research survey like humans do, then it might not be a problem. However, there is a question as to whether we should consider something as unknown to us if a machine determines it to be unknown based solely on its own memory. Such judgment may sound reliable if the AI was pre-trained with all scholarly papers. However, as mentioned before, just because AI deems something unknown, it doesn't guarantee that it's unknown to humans. This issue is an open problem.

% philosophy of literature survey \cite{schryen2015theory}
% GPT-3 learn to quantify the uncertinty of its answer \cite{lin2022teaching}

% One challenge in creating an intelligence capable of hypothesis generation, not just as a tool for humans, is the need to empower the machine itself to form plausible hypotheses for questions to which even the machine doesn't know the answer. Current machine learning models have been criticized for potentially not knowing what they don't know \footnote{
% In our discussion with Wataru Kumagai, we were reminded once again of the importance of self-awareness in creating an AI capable of conducting research.
% }. 
% Moreover, they are known to confidently provide answers or fabricate falsehoods about topics they are ignorant of. Therefore, it seems essential to first accurately recognize what is unknown, either for oneself or the world at large, as told in sections of question construction. Upon facing an unknown subject, there's a need to reduce uncertainty and approach understanding. As mentioned in Chapter 2, humans attempt to understand uncertain subjects by gathering information from papers, experiments, or by reframing questions. While it may not be necessary to adopt the exact same approach, it seems essential to enable machines to autonomously adopt strategies to reduce uncertainties.

\subsubsection{Deciding What Knowledge to Seek}
\label{section-deciding-what-knowledge-to-seek}
We do not always formulate questions for everything we don't know since not all unknowns are equally ``important'' or ``interesting'. Instead, we construct questions for things we wish to know the answers to. This means that we assess some ``value'' of questions using some criteria to determine if it is worth pursuing. For individual, this can be an unconscious internal process, but for research, this doesn't need to be so as long as it is some value judgment process.

Knowledge in itself is value-neutral. The ``value'', ``significance'', or ``goodness'' of knowledge is determined by those who using it. The crucial point here is that the criteria for determining ``value'' are arbitrary. Therefore, if we want AI to pose questions that are meaningful to humanity, we should recognize what is ``good'' or ``significant'' questions and instill and make AI adhere to such values. 

However, we should also be mindful that there might be questions that we would judge as ``unimportant'' but are actually ``important'' under some criteria. For instance, there is a myriad of knowledge born from fundamental research that, at first glance, might seem ``useless'' but actually leads to later innovations. Due to human cognitive limitations, there might also be instances where we cannot fully assess the utility of that knowledge. Additionally, in human society, sometimes social factors unrelated to the original purpose of the knowledge production influence value judgments. That is, our value judgments are not necessarily always optimal.

Since AI is not bound by such constraints, there's a possibility it can make better value judgments. Therefore, while it might be essential to provide some minimal guidance to ensure AI generates questions meaningful to humans, it would be good to aim for the development of AI that can autonomously construct good value criteria by themselves. How to create such an AI seems to be one of the important open problem in building an AI capable of research.\footnote{
Kitano referred to the science in which humans adopt their own value judgment criteria to determine questions and hypotheses as \textit{value-driven science} \cite{kitano2021nobel}. He argued that advancing \textit{exploration-driven science}, which focuses on more comprehensive and thorough exploration rather than criteria based on specific human values, is important for societal development. Although a completely value-neutral system would be impossible, I agree with the idea that employing new and diverse criteria would matter for future research. By adopting more diverse and extensive criteria, we could expand the exploration space of knowledge.
} 

\subsubsection{Origin of Information Need}
I explained that the question begins with the recognition of an information need. So, what causes the recognition of an information need (or the sub-processes of the recognition of unknowns and the judgment of value) in the first place?

Various factors can be considered as triggers. Some people may generate research questions as they logically think about how to achieve a goal. Others might come up with a question upon noticing some anomaly while observing experimental data or realizing that there might be something wrong with the underlying assumptions based on a discrepancy between results deduced from some assumptions and actual observational data. Furthermore, in reality, humans do not evaluate based on a single criterion of value judgment. It seems that they combine these criteria in a complex manner, weighting them according to the situation, before arriving at a final value judgment.
If one wants to create an AI that can autonomously construct research questions like humans, it seems necessary to develop an AI with a general methodology that can construct questions in any of these situations.

How these various factors are connected to information need in humans, and how we can create an AI capable of this, remains one of the most important questions and is still an open problem. Researchers studying curiosity have been tackling this difficult problem. Curiosity, while it is difficult to precisely define, is viewed as \textit{a drive state for information} \cite{kidd2015psychology}. In this sense, curiosity can be considered as something that gives rise to an information need. Indeed, especially in the process of information-seeking, it is characterized as one of the precursors of information need \cite{case2016looking}. Efforts to instill curiosity \cite{schmidhuber1991possibility} or knowledge-based intrinsic motivation \cite{oudeyer2007intrinsic} in artificial intelligence have been researched in the field of reinforcement learning. There, curiosity is formulated as novelty, information gain, or prediction error, and is perceived as something that encourages exploration \cite{aubret2019survey}.

These efforts have provided guidelines on how to implement mechanisms that drive intelligence towards questions. However, we are still halfway to realizing a system that autonomously construct research questions under complex value judgments in all situations, as humans do. Especially, while a question becomes the minimum input in hypothesis generation, and a hypothesis in hypothesis validation, in question generation, it's unclear what the minimum input should be. Need to design a complex internal driving force adaptive in various situations seems to be a major difficulty for creating an autonomous research questioner. Identifying what is necessary to realize such AI is an important open problem.
% これらの取り組みは、知能を問いに突き動かす機序をどのように実装するかについての指針を与えてくれました。しかし、依然として人間のようにあらゆる状況で複雑な価値判断のもと研究対象となるような問いを自律的に生成するシステムの実現にはまだ道半ばです。特に、仮説生成であれば問いが、仮説検証であれば仮説が最低限の入力となるのに対して、問いの生成ではそのような最低限の入力が何か分からず、自律的に問いを生成するには内的報酬のようなうまい機序を作らなければいけない点が大きな困難であるように思われます。このような AI を実現するためには何が必要なのかを明らかにしていくことは重要なオープンプロブレムの一つです。

% One of the challenges in trying to create an autonomous research question constructor is determining how much input humans should provide to the AI and to what extent humans should design the mechanism for constructing questions. For example, the minimum input for an AI constructing questions for a goal is the goal itself. However, a goal might not be the minimum necessary input for other types of question generation methods. Also, as is often discussed in the context of research automation, if one aims to make the AI autonomously generate even goal itself, we would likely necessitate assuming even more primitive inputs for it, e.g. higher-level goal.

% For creating an autonomous artificial researcher, it seems important that these are driven by intrinsic motivations. However, it doesn't necessarily have to be intrinsic motivation in the commonly understood sense. In the context of research, intrinsic motivation often seems to be associated with feelings like ``interest.'' But, as we'll discuss later, the motivations for posing questions can be diverse. For instance, questions could be framed for the purpose of optimizing some externally set metric. Therefore, if we narrowly define intrinsic motivation in this way, it might be argued that an AI capable of research doesn't necessarily need to have curiosity.

% Of course, if we take a broader view of intrinsic motivation, considering it akin to the evaluation function an agent possesses, then an autonomously inquiring AI is essentially the same as a curious AI.


% Research do not always have to be intrinsically motivated.

% curiosity is intrinsically motivated information seeking \cite{kidd2015psychology}

% theory that people decide what information to seek by considering the positive and negative effect of it to their action, affection, and cognition \cite{sharot2020people}

% curiosity -> exploration \cite{oudeyer2018computational}

% book of intrinsic motivation \cite{baldassarre2013intrinsically}



% This is because the essential requirement for research is that the answer to the question is unknown, and in principle, I do not require additional properties question to have. For example, you can ask about anything if it is unknown, or you can choose to ignore it if it's not intellectually intriguing. Alternatively, you can opt for something that seems to lead towards a specific goal you are pursuing. This has a critical implication for aiming to automate research. If the value standards are always given by humans, then there is no issue. However, when considering the possibility of automating even that aspect, it becomes necessary to discuss how I can make it adhere to the desired criteria. While it is natural to want them to ask ``good'' or ``important'' questions, it also becomes challenging when it comes to thinking about this issue. Let us discuss this further in below.

% I claimed that the value judgement is arbitrary. This is because the essential requirement for research is that the answer to the question is unknown, and in principle, any question is valid. In light of the definition of knowledge production, value judgement is not necessarily required for knowledge production to be as it is. From the perspective of knowledge production, as long as the unknown is truly unknown and it be rigorously approached towards becoming known, there should be no problem. The unknown can be anything arbitrary, and knowledge production itself does not demand a specific nature for it. 

% % \subsection{Relativity of Value}
% Knowledge in itself is value-neutral. The value, significance, or goodness of knowledge is determined by those who using it. 


% The value of knowledge is inherently dependent on context, so the significance or goodness of a certain knowledge is not determined a priori from the moment it is generated. Goodness becomes an issue only when that knowledge is used by some member of the society in some form. In other words, the demand for importance and goodness is a constraint imposed by society not by knowledge production. For example, certain knowledge may be considered ``important'' in the sense that it addresses the interests of all the researchers working on similar topics, providing solutions to their concerns. Alternatively, that knowledge might be deemed ``important'' to someone simply because they find it intellectually interesting. On the other hand, if it takes a considerable amount of time for this knowledge to translate into practical applications, it may not be considered ``important'' to someone who wants to start a business immediately. Hence, this is the choice of us (or them) on what objective I would like to maximize by knowledge production.\footnote{
% The fact that the value is arbitrary and not necessary condition for knowledge production doesn't mean I do not have to discuss about the value. In the first place, it is inherently impossible for all actions to be value-neutral. So it is inevitable for us to conduct some value judgement. Moreover, the realm of possible unknowns is too vast, so without any constraints, only nonsensical questions would arise. Most of us are interested in ``good'' or ``important'' questions. So, what I should consider is how to identify them and how to instill them to machines.
% }

% This provides important implications when attempting to create an artificial intelligence that autonomously conducts research, at least in two aspects. Firstly, if I am interested in making machine autonomously construct ``good'' question for humans, I should consider what is ``goodness'' for humans and how to instill them to machines. However, it seems that I have not yet fully established a unified common understanding of what constitutes a "good" question. ``What makes a question good for humans'' and ``how to formulate them effectively'' are crucial aspects that require further in-depth discussions. Therefore, I will revisit and explore these important points separately later on.

% It seems that I still lack an understanding of what kind of research questions are important, but in the field called \textit{science of science} \cite{wang2021}, which studies the research itself, scientific studies on research impact are also advancing. The insights gained here may provide important knowledge in building new algorithms and optimization metrics.

% The second implication when attempting to create an artificial researcher is that if I do not intentionally impose any constraints, there is a possibility that the intelligence produced may not align with the knowledge I desire. This is an important point and thus will be discussed in a separate chapter.

% Secondly, it suggests the possibility of adopting values that are different from the ones I currently employ can result in better knowledge production. I have explained that I make judgments about certain questions being good or important based on some criteria or standards. However, there may be questions that are not considered ``important'' according to current criteria but actually be extremely significant. As mentioned earlier, the value of knowledge is determined by its usage and context, and it can vary over time and in different environments. Therefore, it is highly challenging to determine the importance of knowledge during the stage of knowledge production. Even in the same environment, it is difficult to assess the significance of knowledge. This is because knowledge results from complex accumulations, leading to new insights, and there is a intricate chain of connections before a particular knowledge becomes recognized as important within a society. In addition, I am bound by various cognitive limitations inherent to being human. Therefore, I can only assess the importance of knowledge within the confines of these limitations. 

% Given these circumstances, it is highly possible that the current adopted criteria for value judgments are missing out on the production of potentially important knowledge. The development of knowledge production systems that embrace new value judgment criteria can be expected to increase the potential for generating such knowledge by expanding the scope of exploration. If an artificially intelligent system capable of autonomous research is developed, it can be expected that research based on these new criteria will become more feasible. This could potentially enable the resolution of many previously unsolved problems that were not attainable before.

% As a conclusion, let us emphasize again that I first need to be aware that these ``good'' aspects do not occur naturally. To create an intelligence that constructs ``good'' questions, I first need to understand what I consider a ``good'' question. Also, it's important to turn my attention to things that are not currently considered ``good,'' but should be deemed as ``good'' in essence. Only then can I discuss how to align that value with the agent. Therefore, I think I should start by listing the criteria for determining the ``goodness'' of a question. 

% For this, discussions in the philosophy of science and meta-sciences like the Science of Science may be referenced. Alternatively, large-scale surveys of researchers engaged in actual research could also be important. Once the value is clarified, I might be able to think about creating an intelligence equipped with these values using the value alignment techniques that are currently being developed.

% Science based on the importance of questions discussed above is a \textit{value-driven science} \cite{kitano2021nobel}. However, as previously mentioned, these may be due to cognitive constraints imposed by human society, including the inability to handle knowledge that is deemed ``unimportant.'' Therefore, when automating question generation, it may be possible to explore a wide range of questions, including those that were previously considered ``unimportant.'' In doing so, it is possible that knowledge that was not considered ``important'' according to previous criteria could actually be extremely significant. This is referred to as \textit{exploration-driven science} \cite{kitano2021nobel}, and it could become a new form of research liberated from constraints imposed by humans. 

% Indeed, it is impossible to create a completely value-neutral system. All agent systems must have some form of bias. However, what I would like to emphasize is the potential to incorporate biases different from the criteria previously used by humans, and how this can enable us to consider more diverse approaches to research. This highlights the importance of creating agents capable of autonomously conducting research.

\subsubsection{What Are the Criteria for the Value of a Research Question?}

% \subsubsection{Diverse Good Research Questions}
So far, I have discussed somewhat abstract topics related to questions in general. Now, I would like to discuss the characteristics related to research questions specifically, and how humans judge the value of these questions. 

The ``quality'' of a research question can be judged based on various criteria. I will introduce some examples from among them to make it clearer how humans have determined the value of a research question. Please note that the following examples are just a few of the many criteria humans use to judge the value of a question and are far from comprehensive. I hope this will aid in further deepening the discussion on how humans determine the value of a question.

The idea that questions which would bring about new perspective or understanding, or \textit{conceptual advance}, especially which would overturn our common sense or underlying assumptions are important is widely accepted within the research community. As an example, Alvesson and Sandberg point out the importance of these kind of questions and discuss the strategies to construct them \cite{alvesson2013constructing}. 
This criterion is based on the premise that a good question is one that produces knowledge which has a significant impact on our current body of knowledge.

No matter how significant a question is, if it's nearly impossible to address with current technology, producing meaningful research outcomes from that question may be infeasible. Therefore, some argue that the feasibility of answering a question should be considered when evaluating its quality, with questions that are not overly implausible being deemed good ones \cite{hulley2007designing,alon2009choose,huntington2021effect}. To determine feasibility, it seems that complex decision-making is necessary, as it requires consideration of various factors such as the resources and funding currently accessible, the capabilities of the researchers, deadlines, and even the limitations of the technology currently available to humanity. In reality, an AI capable of conducting research would likely need to be able to make such complex decisions.

The notion that research questions should be based on an individual's intellectual curiosity also seems to be widely accepted. Curiosity is the driver of exploration \cite{oudeyer2018computational}, so curiosity-driven research might have promoted exploration in the research theme space. Research can be seen as an exploration of the space of truths in this world, hence it sounds reasonable that value standards that promote exploration are important.
To discover things that are so unknown they are not even known to be unknown, exploration is essential as an entire endeavor of science."
Conversely, the criteria should not necessarily be curiosity as long as it promotes the exploration in the space of knowledge. The exploration driven by curiosity is merely a byproduct, and there may exist better heuristics for exploration. If AI acquires such value standards, it might be able to unravel truths more efficiently than humans have done.

% To construct such questions, we generate hypotheses by repeatedly engaging in question and hypothesis generation.\footnote{
% For instance, let's say we have a goal of ``creating AGI''. Initially, we pose the question, ``How can we create AGI?''. In response to that, we consider elements necessary to create AGI as our hypotheses. For example, we might think, ``Realizing an AI capable of logical reasoning, an AI with embodiment, etc., might be necessary.'' Then, we ask, ``How can we create an AI capable of logical reasoning?'' and consider hypotheses for that question, and this process is repeated.
% }

Contrary to research driven by bottom-up curiosity, the notion that questions contributing to the achievement of specific goals set in a top-down manner is valuable is equally common. For instance, for those aiming to realize AGI, questions about how to create elements deemed necessary for AGI would be significant for them, even if it is not interesting or incremental. Especially in corporate research or government-led research, there are likely many studies aimed at achieving goals set in a top-down manner. It is assumed that many people expect AI capable of conducting research autonomously to contribute to goals set by humans. Therefore, the ability of AI to make this kind of value judgment is an important requirement.

Lastly, there is also a perspective that emphasizes the value for individual. Alon expresses the view that a good research problem is one that is interesting to the individual and has an appropriate level of difficulty for them \cite{alon2009choose}.

In reality, instead of adopting just one of these criteria, we determine the value of the research question by comprehensively considering multiple criteria. For example, Hulley et al. suggest that questions that is feasible, interesting, novel, ethical, and relevant (FINER) should be considered good ones \cite{hulley2007designing}. Huntington-Klein presents the view that good research question is answerable and the answer to the question will improve our understanding of this world \cite{huntington2021effect}.

As already mentioned, these are just a part of the value judgments that humans make in determining questions. I hope that future discussions will delve deeper into what kind of value judgments humans make, what function they serve in scientific discovery, and how we can realize these functions in AI.

% Gap spotting vs problem solving \cite{alvesson2013constructing}

% Up until now, I have been explaining that values like goodness are relative and subjective. However, it is natural for artificial researchers to autonomously construct ``good'' or ``important'' questions. While I admit that adopting diverse value criteria and not being bound by traditional standards matters as Kitano said, even in that case I still have to determine a criteria that lead to ``good'' questions in some sense. Therefore, I would like to discuss what constitutes a ``good'' question for us and how I can construct such questions, drawing upon the discussion of how I have characterized ``good'' questions thus far.

% \subsubsection{Goal Oriented Question}
% One of the most general, significant, and widely accepted criteria for what I consider as ``goodness'' is that a question is deemed ``good'' if its answer contributes to achieving a desired, yet unrealized, goal. This is because researchers often seek to address long-term problems and engage in knowledge production for that purpose. For instance, physicist who pursue a unified theory would think that a question that furthers the realization of a unified theory is a good question. This can be referred to as a \textit{goal oriented research question}.

% In this approach, we first set the ultimate objective. Then, we identify the most critical bottlenecks, or sub-goals, that are essential to achieving that objective. We once again consider sub-goals for these identified bottlenecks. This process is repeated, converging on more specific and feasible sub-goals that are of high importance. Finally, I frame the question to address these sub-goals as the research question. 

% For identifying the knowledge required to achieve the ultimate goal, we typically start by listing the necessary elements to accomplish it. For example, to achieve general artificial intelligence, we may think that it requires the ability to handle language, understand the real world, be proficient in systematic reasoning, and align with human values, etc. 

% Then, we break down them again into the necessary things to achieve the them. For instance, to understand the real world, for instance, we may need the capability for interacting with the physical world, processing visual information, and so on. These requirements can be further broken down into multiple necessary elements. By repeating this process, we can narrow down the specific tasks that can be directly addressed. Then, the required knowledge to accomplish those tasks is demanded, and that's where it directly connects to the research question. The process is conceptualized in Fig. \ref{fig:unknown_tree}.


% \begin{figure}[htb]
%     \centering
%     \includegraphics[width=\textwidth]{figs/unknown_tree.jpeg}
%     \caption{Caption}
%     \label{fig:unknown_tree}
% \end{figure}

% Several things are happening here. Firstly, listing the elements necessary for achieving the goal means generating sub-goals from the main goal. However, it's always a challenging problem to evaluate how a particular sub-goal contributes to the achievement of a given goal. Especially in the case of research, the target might be an too general and ambitious vision that nobody has achieved before, so I need to think about what needs to be done to break it down into appropriate sub-problems. In other words, it is necessary to construct a tree with nodes representing sub-goals. Namely, this is the tree of repetition of constructing a question and generating multiple hypotheses without verification. I will delve into the discussion of hypothesis generation in the next chapter, so I won't go into the details here.

% Secondly, it is necessary to identify the most important and feasible sub-goal from the selected candidate sub-goals. This is because only one question can be addressed in the end. However, assessing and comparing sub-goals is a challenging task since I have no experience to realize the ultimate goal and so have no data what sub-goal actually is the most important.

% Thirdly, the question to ultimately arrive at must be verifiable. If the question is not specific, meaningful verification cannot be performed. Overly broad or ambiguous questions can result in countless or trivial answers, or they may be too unclear to provide practical answers. Increasing the specificity of the question corresponds to deepening the depth of the sub-goal tree, so it may be important to construct a sufficiently deep tree and find an efficient way to navigate it. The verifiability is constrained by the knowledge, resources, such as funding and technology, that I currently have. Therefore, when conducting verification in reality, it is necessary to consider such feasibility. Whether to tackle a question with high feasibility or to further divide it into more sub-tasks for its realization is a matter of judgment. In any case, it is necessary to appropriately evaluate such feasibility. The scope of feasibility is vast, so it is a challenging problem to determine how to consider it in creating intelligent systems.

% In this discussion, I considered the method of outputting questions from the goal through the construction and exploration of a tree structure. However, as mentioned by the predecessors, if an end-to-end approach ultimately becomes a powerful method, it may be more desirable to consider a direction in which questions are directly output from the goal. In particular, even when performing multi-step reasoning, it seems more natural to improve reasoning abilities using the recently developed approaches to multi-step logical reasoning, rather than explicitly considering tree structures. 

% To pursue this direction in research automation, specifically, it may be worth considering the construction of higher-quality datasets for goals and research questions. For example, it may be possible to construct a dataset by extracting only the ultimate goal and the research questions actually solved from the introductions of papers. However, an important point to note here is that the research questions created by humans so far are not necessarily optimal for achieving research goals. Firstly, machines may be capable of maximizing the objective better than humans due to cognitive constraints. Secondly, not all human research has been conducted by working backward from a clear goal. Some studies were conducted simply because they seemed interesting while reading papers. In this regard, simply learning from human data may constrain the potential capabilities that machines can possess. Therefore, it becomes important to consider how to formulate the maximization of the probability of achieving research goals as a problem, rather than naive imitation learning of human data.

% As evident from the formulation, to construct research questions for contributing to achieve a specific goal, we need to solve long range reasoning problems. This problem is widely studied in machine learning research community to improve the reasoning capabilities of machine learning models and on generate intermediate goals in reinforcement learning. If these research fields produce significant results, they can be directly applied. In this sense, it might be beneficial to seek cooperation from those who are actively conducting research in these areas. One of the unique aspects of long-distance reasoning problems in research automation is that the goal is something that has never been achieved before. This means that you cannot naively learn from data and need to generalize out of distribution. Therefore, it's essential to acquire skills not just to recognize patterns but to properly trace the path of reasoning. Moreover, because the goal has not been realized, sub-goals and the paths that connect them are ultimately based on the accumulation of hypothesis generation. In this sense, it can be said that this is a highly uncertain inference. This implies that the choice of which node to select is far from self-evident compared to other logical reasoning problems. Furthermore, there is the issue of the complexity of the distance between the goal and the question, which is far more intricate than, for example, games or planning everyday trips. For instance, to truly achieve the goal, it may be necessary to build large-scale apparatus like particle accelerators from scratch. This also means that the temporal distance between the goal and the current location is very long. Therefore, it becomes a problem that feedback on how much solving the question contributed to the goal is significantly delayed. While we've only listed a few examples here, there may be other unique challenges and issues that become more serious in research. It will be necessary to work on refining these technical challenges into specific research tasks through discussions with researchers in reasoning and planning.

% \subsubsection{Important but Unnoticed Questions}

% One example of a ``good'' question seems to be one that, in its construction itself, brings benefits to many researchers. By considering why hypotheses regarding the question are unknown, it becomes somewhat clearer what kind of question this is.

% The reasons why the answer to a question remains unknown are diverse. In some cases, the question is simply new, and no one has had the time to come up with an answer yet. For instance, a question about the internet posed on the day it was invented may still remain unanswered because it has only been a day since its birth, and no one may have delved into it yet. In other cases, the question might simply not interest anyone, leading to an absence of answers despite the availability of time. This is an example of a question that remains unanswered because nobody is willing to tackle it, even though time exists.

% Furthermore, there are questions that are difficult, and no one has been able to solve them. For instance, the question of how to achieve human immortality has been contemplated by many, but due to its complexity, no answer has been found yet. Lastly, there are questions that are important for a particular purpose but remain unnoticed by everyone. As previously mentioned, realizing challenging objectives that are yet to be achieved poses difficult problems. In such cases, it is often unclear what is not known or what is at the heart of the problem. In such situations, clarifying the question itself holds great significance.

% Thus, constructing questions of this kind seems to be important as it can help shed light on the unknown. With this in mind, it appears worthwhile to discuss this topic in detail.\footnote{
% Please note that the concept of a question being unnoticed and the answer to the question being unknown are different. The necessary condition for research is that the answer to the question is unknown. 

% If the existence of a question is not known to anyone, then naturally, its answer would also be unknown since no one would have answered it. So, if the existence of a question is unknown, then the answer to the question is also unknown.
% }

% \textcolor{red}{TODO: Add discussion}

% \subsubsection{Diverse Good Questions}
% There have been various discussions on the elements that good research possesses. For example, \cite{hulley2007designing} proposed that good research question should satisfies FINER criteria (feasible, interesting, novel, ethical, and relevant) and \cite{alon2009choose} claims that a good problem is one that is most feasible and interesting to oneself.

% \textcolor{red}{TODO: Add more discussion on ``good'' questions, examples, discussion, what is good, what is important, specific question is good, etc.}
 

% \subsubsection{Where Does a Question Come from?}
% Research questions can arise in various situations. Some people may generate research questions as they logically think about how to achieve a goal. Others might come up with a question upon noticing some anomaly while observing experimental data. Furthermore, as with Kepler, one might realize that there might be something wrong with the underlying assumptions based on a discrepancy between results deduced from some assumptions and actual observational data. If one wants to create an AI that can autonomously construct research questions like humans, it seems necessary to develop an AI with a general methodology that can construct questions in any of these situations.

% One of the challenges in trying to create such a general and autonomous research question constructor is determining how much input humans should provide to the AI and to what extent humans should design the mechanism for constructing questions. For example, the minimum input for an AI constructing questions for a goal is the goal itself. However, a goal might not be the minimum necessary input for other types of question generation methods. Also, as is often discussed in the context of research automation, if one aims to make the AI autonomously generate even goal itself, we would likely necessitate assuming even more primitive inputs for it, e.g. higher-level goal.

% For creating an autonomous system, the question construction module faces challenges due to being the initial building block of the system. This challenge is determining ``what should be the input to the question construction module.''

% Questions do not arise from nowhere. There is always something before reaching a question. In the example I mentioned earlier, for instance, the question may arise as a result of a literature review. Then, why did you conduct that literature review in the first place? It could be because there is a research theme you want to know about in the research field. And then why are you interested in that research theme? It could be because the topic of the first paper you encountered during graduate school was fascinating, or it could be because you have been interested in it since childhood. And there may be causes behind those as well.

% In this way, identifying where a question begins is a hard problem. If you think seriously about it, it will lead to an infinite regress. This is a significant problem when I want to realize an autonomous artificial researcher. As infinite regress can occur, the decision of where to terminate lies solely with the designer, and it is not automatically resolved by creating the system. Can I say that question construction is autonomous if the literature to read were given? Can I say that question construction is autonomous if research theme were given? I believe it is necessary to accumulate such discussions and determine how far to consider something as given, in order to define what qualifies as autonomous.

% In this paper, I assume that a trigger that requests a specific type of knowledge is given. And from there, it makes decisions about determining the unknown and constructing questions. The reason is that discovering questions with unknown answers is a necessary condition for research, and I believe this is the minimal requirement for it. However, one could also try to automate even the aspect of requesting a specific type of knowledge. The reasons for expecting the existence of certain knowledge can vary and are arbitrary. For example, I might have an objective and first consider what I need to do to achieve it. I then anticipate the necessary knowledge to accomplish those tasks. This corresponds to the demanding for a specific knowledge.

% As an another example, let's consider the case of a child asking, ``Why is the sky blue?'' In this case, the child may already have prior knowledge of the concept of ``sky'' and ``blue.'' Additionally, they may possess a naive concept of causality, believing that ``A is B, so there must be a reason for it.'' Thus, they may have expected to have the knowledge that ``the sky is blue because of B.'' However, when they reference their internal knowledge, they find that it does not contain the corresponding knowledge. Therefore, they may have asked the question ``Why is the sky blue?'' to evoke the knowledge they were lacking. In this case, the required knowledge is ``the sky is blue because of B.'' and this is induced just because the result of children combining known concepts.

% The question of ``why do I seek information'' has been extensively discussed in the context of curiosity. Indeed, as I proceed with the automation of these components, it becomes essential to delve into research on curiosity. Regarding this matter, I will touch upon the aspects mentioned in Chapter 3, if possible. 

% Multiple Reasons for Unknownness

% new, unimportant, difficult, unnoticed, ... etc.

% This means that I distinguish between questions that are ``good'' and those that are not, based on certain criteria. 

% This means that I distinguish between questions that are ``important'' and those that are not, based on certain criteria. For example, \cite{alon2009choose} claims that a good question is one that solves challenges facing the research community. Likewise, I consider a question to be important if it generates knowledge that greatly contributes to a certain purpose. Valuing the degree of contribution to a purpose also implies viewing research as a form of problem-solving. \textcolor{red}{TODO: Add explanation of what this sentence means}

% Thus, in realizing an agent that autonomously constructs questions, it may become important to consider how to automatically determine ``goodness'' of the questions. To achieve this, it would be important to first understand in more detail what kind of questions I consider ``good''. \textcolor{red}{TODO: Add possible directions}

% \subsubsection{How to Practically Construct a Question}

% There has been much discussion on how to actually generate questions. Off course, these discussions primarily focus on how to formulate good questions. Therefore, please note that the examples mentioned here are proposals for generating such kind of questions.

% One typical approach to formulating a research question is to conduct a literature survey, identify research gaps in existing studies, and propose a question that aims to fill those gaps.

% \textcolor{red}{TODO: Add survey of how to construct questions; gap spotting, problemization, etc}

% Next, let's consider the process of constructing purpose-driven research questions. When aiming to conduct impactful research, I believe that constructing purpose-driven research questions is crucial. In this approach, I first set the ultimate objective. Then, I identify the most critical bottlenecks, or sub-goals, that are essential to achieving that objective. I once again consider sub-goals for these identified bottlenecks. This process is repeated, converging on more specific and feasible sub-goals that are of high importance. Finally, I frame the question to address these sub-goals as the research question. 

% In practice, I seem to determine the questions I should tackle in this way, implicitly and explicitly. For example, let's say that someone try to answer a question of ``How neural networks have reasoning capability?'' in his/her study. This question may come from a thought process of ``we want to create artificial general intelligence, which requires systematic thinking, that needs ...'' In this case, the final purpose is to achieve ``artificial general intelligence'', and the question addressed as a result is ``ow neural networks have reasoning capability?'' In other words, when I want to conduct important research, I follow a process that starts with the goal I want to achieve, considers the tree of important unknowns that should be clarified for its achievement, and sets the end of that tree as the research question. This process is summarized in Fig. \ref{fig:unknown_tree}.


% Of course, the purpose mentioned here may be a sub-goal of a higher-level goal. For example, the goal of ``creating general artificial intelligence'' may be a sub-goal of a more fundamental goal of ``satisfying intellectual curiosity,'' and the goal of ``satisfying intellectual curiosity'' may be biologically demanded for better exploration of the environment. These can lead to an infinite regression when considered strictly, so I won't delve into it any further here, but it could become an important issue when considering how to realize fully autonomous agent to construct questions.

% \subsubsection{Question}

% The construction of a question is the act of seeking information \cite{watson2015ask}. Specifically, in the context of research, I consider information as knowledge. The act of seeking knowledge involves two steps: 1. Recognizing the lack of knowledge and 2. Attempting to fill that knowledge gap. In this discussion, I assume that intelligence is designed to consistently generate questions when given input. Therefore, I temporarily set aside the aspect of "triggering action" related to the second step of attempting to fill the knowledge gap.

% The recognition of a knowledge gap occurs when I expect to have certain knowledge and, upon referencing my accessible knowledge, I find that it is not available. For example, when running a program and encountering an error that I cannot resolve on my own, I recognize that I lack the necessary knowledge.

% The reasons for expecting the existence of certain knowledge can vary and are arbitrary. In this case, I assume that a purpose given by a third party creates an expectation of certain knowledge. For example, in the case of humans, I first consider what I need to do to achieve a certain purpose. I then anticipate the necessary knowledge to accomplish those tasks, and when I find that it is not present within my existing knowledge, I recognize the knowledge gap.

% Lastly, in this discussion, knowledge refers to the collective body of research findings, particularly academic papers. In actual research, a researcher may personally have a question and then investigate previous studies to confirm that it is indeed unknown before formulating it as a research question. However, what is important in the construction of a research question is that it is unknown to other entities. Therefore, for simplicity, I directly refer to the entirety of academic papers without including the step of comparing personal knowledge.

% To summarize, to create an intelligence capable of constructing questions in this setting, I need to design it to expect the necessary knowledge to achieve a given purpose provided by a third party, search for that knowledge in academic papers, assess whether the papers contain sufficient knowledge to achieve the purpose, and express any knowledge gaps as questions.

% In this case, I excluded the discussion of triggering action by design. However, when considering increasing autonomy, it is important to discuss how to incorporate this aspect into learning and acquisition. The question of "why do I seek information" has been extensively discussed in the context of curiosity.

% Furthermore, in this case, I defined the expectation of knowledge as aiming to achieve a given purpose. However, as mentioned earlier, this does not affect the formulation of questions. For example, let's consider the case of a child asking, ``Why is the sky blue?'' In this case, the child may already have prior knowledge of the concept of ``sky'' and ``blue.'' Additionally, they may possess a naive concept of causality, believing that ``A is B, so there must be a reason for it.'' Thus, they may have expected to have the knowledge that ``the sky is blue because of B.'' However, when they reference their internal knowledge, they find that it does not contain the corresponding knowledge. Therefore, they may have asked the question ``Why is the sky blue?'' to evoke the knowledge they were lacking.

% In this way, the reasons for expecting the existence of certain knowledge can vary, and what, why, and how I seek information (knowledge) are not constrained by specific conditions. Therefore, when attempting to create an intelligence capable of constructing questions in the future, it is feasible to develop a more flexible intelligence.

% Additionally, in this case, I assumed that the given purpose and its achievement are predefined goals. However, humans naturally set their own goals. When considering the design of a more autonomous intelligence, it is conceivable to aim for automation in this aspect as well. However, as mentioned earlier, the question of what I seek knowledge about is not specific to research. Therefore

% , I temporarily set it aside for now. If I were to pursue this direction further, it would ultimately lead to an infinite regress, raising the question of how much information to consider as given.


%%%%%%%%%%%%%%%%%% Rearangement %%%%%%%%%%%%%%%%%%%%%%%%%%%%

% \subsection{question construction}
% Research is an endeavor to bring the unknown closer to the known. Therefore, it is necessary to first determine what unknown I aim to make known. And this unknown often takes the form of questions. For example, ``Why do deep neural networks with a large number of parameters generalize well?'', ``How can I prevent the problem of vanishing gradients?'', and like these. These are commonly referred to as \textit{research questions} or \textit{research problems}. Therefore, in this paper, I will refer to the step of determining this unknown as \textit{question construction}.

% \textcolor{red}{TODO: should describe question construction itself first. What is research question or research problem?}

% \subsubsection{Unknownness}
% As I have reiterated, it is a necessary condition for research that the answer to a question is unknown, or in other words, that there is a high degree of uncertainty. Therefore, it is essential in research to have methods that ensure the answer to a posed question is truly unknown, or to formulate questions that truly have unknown answers.

% Knowledge for humanity is primarily disseminated through research outcomes. Therefore, when examining all the research outcomes that have been generated thus far and finding that none of them provide an answer to a specific question, it seems reasonable to conclude that the question possesses sufficient uncertainty to warrant further investigation as a research endeavor. In particular, humanity has developed the culture to preserve the research outcomes in the form of papers. Therefore, it seems feasible to assess the unknown nature of an inquiry by examining all academic papers. However, it is impossible to review them all due to constraints in terms of time, technology, and cognitive limitations. Therefore, it is realistic to consider a question as unknown if it has been sufficiently and comprehensively explored through an extensive examination of these academic papers. In practice, I conduct literature reviews to synthesize existing research, identify research gaps in existing studies, and thereby ascertain the unknowness of my own questions or construct question for which the answers are unknown \cite{schryen2015theory}. \textcolor{red}{TODO: Consider where I will explain about literature review}

% % In the previous statement, it was mentioned that as long as the unknown is truly unknown and it can be approached towards becoming known, there should be no problem. The process of approaching the known will be explained in the next section, and here I will delve a bit more into the determination of unknownness. 

% However, in reality, such rigorous literature research is not always conducted in every case. Currently, researchers often demonstrate the unknownness of the answer to the question by referencing only a few related works and explaining that none of them have yet resolved the unknown. And when the paper is evaluated by reviewers, who are a small group of experts, if it is determined that the question has indeed not been answered so far, the provisional recognition of the unknown nature of the question is granted. This means that a subjective evaluation criterion is being used, where researchers and a small number of reviewers consider a question as unknown when none of their known studies provide an answer.

% % This implies the use of subjective evaluation criteria, where researchers examine several papers considered ``major literature'' in a field and consider them as unknown if none of them have provided an answer. Furthermore, as mentioned later, I evaluate the quality of research outcomes by having them assessed by a small number of experts in the same field. If these researchers determine that the previous studies have been sufficiently comprehensive, the determination of unknownness is considered somewhat valid. In other words, ultimately, the evaluation by a few experts may serve as the basis for establishing the unknownness.

% This current convention stems from the cognitive constraint that there is a limit to the literature that humans can examine. Since unknownness is a fundamental aspect of research, ideally, it should be evaluated objectively and rigorously. For instance, it would be desirable to quantitatively state which journals, what types of papers, and how many have been examined, and the result indicating their unknownness. Although systematic reviews already employ such approaches, there is, of course, a limit to the number of papers that can be evaluated manually and selection biases cannot be removed. \textcolor{red}{TODO: Add the explanation of systematic review, problems of it, and how AI can mitigate them.}


% However, I have not found a unified consensus on the definition of good question. but 

% \subsubsection{Questioning as Information Seeking Behavior}
% \textcolor{red}{TODO: Reconstruct}

% The construction of a question is the act of seeking information \cite{watson2015ask}. Specifically, in the context of research, I consider information as knowledge. The act of seeking knowledge involves two steps: 1. Recognizing the lack of knowledge and 2. Attempting to fill that knowledge gap. In this discussion, I assume that intelligence is designed to consistently generate questions when given input. Therefore, I temporarily set aside the aspect of "triggering action" related to the second step of attempting to fill the knowledge gap.

% The recognition of a knowledge gap occurs when I expect to have certain knowledge and, upon referencing my accessible knowledge, I find that it is not available. For example, when running a program and encountering an error that I cannot resolve on my own, I recognize that I lack the necessary knowledge.

% The reasons for expecting the existence of certain knowledge can vary and are arbitrary. In this case, I assume that a purpose given by a third party creates an expectation of certain knowledge. For example, in the case of humans, I first consider what I need to do to achieve a certain purpose. I then anticipate the necessary knowledge to accomplish those tasks, and when I find that it is not present within my existing knowledge, I recognize the knowledge gap.

% Lastly, in this discussion, knowledge refers to the collective body of research findings, particularly academic papers. In actual research, a researcher may personally have a question and then investigate previous studies to confirm that it is indeed unknown before formulating it as a research question. However, what is important in the construction of a research question is that it is unknown to other entities. Therefore, for simplicity, I directly refer to the entirety of academic papers without including the step of comparing personal knowledge.

% To summarize, to create an intelligence capable of constructing questions in this setting, I need to design it to expect the necessary knowledge to achieve a given purpose provided by a third party, search for that knowledge in academic papers, assess whether the papers contain sufficient knowledge to achieve the purpose, and express any knowledge gaps as questions.

% \begin{figure}[htb]
%     \centering
%     \includegraphics[width=\textwidth]{figs/question_formulation.jpg}
%     \caption{question construction}
%     \label{fig:enter-label}
% \end{figure}

% \textcolor{red}{TODO: Is question construction information retrieval??}


% Asking questions is an act of seeking information \cite{watson2015ask}. The act of seeking information (or knowledge) arises from realizing the lack of one's own knowledge and the desire to fill that gap. Therefore, to create intelligence that can autonomously ask questions, it is necessary to incorporate mechanisms that induce these behaviors.

% Determining what triggers this behavior in humans is a challenging problem. However, when designing a system, it is sufficient if it can induce such behavior, regardless of what it actually is. The principle that ``behavior is triggered when it is somehow desirable for the agent'' represents this idea. You are probably familiar with this concept in reinforcement learning, where rewards (desirability of actions) are provided, and maximizing the expected value of these rewards is formulated as an appropriate way to induce behavior.

% The notion of triggering behavior by realizing the lack of one's own knowledge and attempting to fill it has been extensively studied in the context of curiosity in the field of reinforcement learning. In research, agents that pose questions can generally be formulated within this framework.

% \subsection{question construction}
% The construction of a question is the act of seeking information \cite{watson2015ask}. Specifically, in the context of research, I consider information as knowledge. The act of seeking knowledge involves two steps: 1. Recognizing the lack of knowledge and 2. Attempting to fill that knowledge gap. In this discussion, I assume that intelligence is designed to consistently generate questions when given input. Therefore, I temporarily set aside the aspect of "triggering action" related to the second step of attempting to fill the knowledge gap.

% The recognition of a knowledge gap occurs when I expect to have certain knowledge and, upon referencing my accessible knowledge, I find that it is not available. For example, when running a program and encountering an error that I cannot resolve on my own, I recognize that I lack the necessary knowledge.

% The reasons for expecting the existence of certain knowledge can vary and are arbitrary. In this case, I assume that a purpose given by a third party creates an expectation of certain knowledge. For example, in the case of humans, I first consider what I need to do to achieve a certain purpose. I then anticipate the necessary knowledge to accomplish those tasks, and when I find that it is not present within my existing knowledge, I recognize the knowledge gap.

% Lastly, in this discussion, knowledge refers to the collective body of research findings, particularly academic papers. In actual research, a researcher may personally have a question and then investigate previous studies to confirm that it is indeed unknown before formulating it as a research question. However, what is important in the construction of a research question is that it is unknown to other entities. Therefore, for simplicity, I directly refer to the entirety of academic papers without including the step of comparing personal knowledge.

% To summarize, to create an intelligence capable of constructing questions in this setting, I need to design it to expect the necessary knowledge to achieve a given purpose provided by a third party, search for that knowledge in academic papers, assess whether the papers contain sufficient knowledge to achieve the purpose, and express any knowledge gaps as questions.

% In this case, I excluded the discussion of triggering action by design. However, when considering increasing autonomy, it is important to discuss how to incorporate this aspect into learning and acquisition. The question of "why do I seek information" has been extensively discussed in the context of curiosity.

% Furthermore, in this case, I defined the expectation of knowledge as aiming to achieve a given purpose. However, as mentioned earlier, this does not affect the formulation of questions. For example, let's consider the case of a child asking, ``Why is the sky blue?'' In this case, the child may already have prior knowledge of the concept of ``sky'' and ``blue.'' Additionally, they may possess a naive concept of causality, believing that ``A is B, so there must be a reason for it.'' Thus, they may have expected to have the knowledge that ``the sky is blue because of B.'' However, when they reference their internal knowledge, they find that it does not contain the corresponding knowledge. Therefore, they may have asked the question ``Why is the sky blue?'' to evoke the knowledge they were lacking.

% In this way, the reasons for expecting the existence of certain knowledge can vary, and what, why, and how I seek information (knowledge) are not constrained by specific conditions. Therefore, when attempting to create an intelligence capable of constructing questions in the future, it is feasible to develop a more flexible intelligence.

% Additionally, in this case, I assumed that the given purpose and its achievement are predefined goals. However, humans naturally set their own goals. When considering the design of a more autonomous intelligence, it is conceivable to aim for automation in this aspect as well. However, as mentioned earlier, the question of what I seek knowledge about is not specific to research. Therefore

% , I temporarily set it aside for now. If I were to pursue this direction further, it would ultimately lead to an infinite regress, raising the question of how much information to consider as given.

% \begin{figure}[htb]
%     \centering
%     \includegraphics[width=\textwidth]{figs/question_formulation.jpg}
%     \caption{question construction}
%     \label{fig:enter-label}
% \end{figure}


% \textcolor{red}{TODO: Is question construction information retrieval??}


% \subsection{Conclusion}

% \begin{enumerate}
%     \item Determing the existence of expected knowledge:
%     \begin{enumerate}
%         \item Searching for knowledge directly related to the expected knowledge.
%         \item Determining whether the knowledge has been properly validated.
%     \end{enumerate}
% \end{enumerate}

% 1.b is specific to the automation of research. 





% I have listed what I believe are important elements in the construction of questions. However, these are considered important under the assumptions mentioned earlier. For instance, if the goal is not to acquire knowledge necessary for achieving an objective, but to generate knowledge that an individual finds interesting, the necessary elements in question construction (particularly in parts 1.a and 1.b) would change. As previously mentioned, the value of knowledge is determined in relation to context and there's a high degree of uncertainty about how the value of knowledge will evolve in the future. This makes it fundamentally important to have a diverse range of ways to generate questions. The object achievement is highly prevalent and is expected to produce ``important'' knowledge, which is why it is discussed here. However, it is important to discuss what other ways of formulating questions could exist and how they can be implemented.

% \subsubsection{Challenges}
% The first issue discussed in Chapter 2 is the problem of determining the unknown nature of the answer to a question. Given that research is an endeavor to produce new knowledge, it is necessary for the answer to the question to be unknown. Therefore, there is a need to generate such questions or later verify that the answer to the question is indeed unknown.

% The second challenge is the issue of how to make a machine generate a ``good'' question. Firstly, what researchers consider as a ``good'' question is not always consistently agreed upon among them. Furthermore, we pointed out that the ``goodness'' of a question is inherently a concept relative to the individual or society. Hence, there seems to be a need to clearly define what constitutes a good question and think about how to effectively integrate these definitions.

% The third challenge is that as we demand more autonomy from AI, the automation of question formulation becomes more difficult. Inherently, questions are constructed based on various motivations, such as pure intellectual curiosity or for specific objectives. It is very challenging to automate the construction of questions without defining which of these motivations should be the source. Moreover, as mentioned in Chapter 2, the construction of questions encounters the problem of infinite regression when pursued with strict autonomy. Even if not taken to that extreme, setting higher-order objectives behind questions for AI is an exceptionally difficult task.

% It seems that the automation of question formulation has received relatively less attention compared to other processes. Since the formulation of questions is an essential element in conducting research, it would be desirable for more focus to be directed towards the research on automating this process.

\subsection{Hypothesis Generation}
二つ目の要素は仮説の生成です。研究とは問いを立てて、その問いに対する答えようとする営みです。研究者はこの問いに答える過程を仮説の生成と仮説の検証の二つに分離して行います。仮説の生成は質問に対する答えの予想であり、仮説の検証はそれが実際に確からしいかを調べる操作です。問いに答える過程をこのように二つの段階に分けるのは、リサーチクエスチョンは人類が誰もその答えを知らないようなものであり、そのため一度に答えを言い当てることは事実上困難であること、そして一個人の思考は必ずしも真理促進的ではないことなどが理由として考えられます。言い換えるなら、仮説生成と検証への分離は人間が極めて不確実性の高い対象の真実を明らかにするために開発した方法論であると言えます。

仮説の生成は科学における人間の創造性が発揮される場所として認識されてきました。人間の創造性は分析できないとの考えから、問いの構築と合わせて仮説の生成は長らく分析が困難なものであると認識されてきました \cite{sep-scientific-discovery}。しかし、20世紀中頃からこの創造的な営みを特徴づけようという試みが生まれてきました。例えば、why question に対する仮説生成におけるアブダクションの重要性の指摘 \cite{hanson1965patterns,magnani2011abduction}、仮説生成における類推的推論の重要性の指摘 \cite{hesse1965models,gentner2002analogy}、科学的発見の探索としての解釈 \cite{langley1987scientific}、確率的サンプリングとしての定式化 \cite{dasgupta2017hypotheses} などがなされました。機械による仮説の生成も人工知能研究の登場以来の最大の関心事の一つであり、早くからその実現が試みられました。初期の例としては1900年代にはすでに仮説を生成する機械の開発が試みられ \cite{langley1987scientific,lindsay1993dendral}、2000年代半ばには自律的に科学的発見を行う機械が開発されました \cite{king2004functional}。

\subsubsection{Hypothesis Generation and Machine Learning}

仮説の生成は既存の知識から未知の問いの答えを予測することです。これはすでに見たデータからまだ見ぬデータに対して予測をするという意味で、機械学習の定式化に還元されるということができます。特に、問いに対する答えであるという点から、機械学習における質問応答タスクそのままであるということもできます。また、誰も答えを知らないような答えの予想であるという点から、大きな分布シフトがあるような状況での機械学習の予測問題だと捉えることもできます。

% 機械学習モデルは予測です。検証を伴わない予測という意味で仮説生成と同じ。広義の意味ではこの定義に吸収される。フレームとしては質問応答だし、ood 予測。

実際に、機械学習モデルは科学の仮説生成に広く使われています \cite{xu2021artificial,zhang2023artificial,wang2023scientific}。たとえば、新しいタンパク質や新薬候補や新物性の予測などの、機械学習モデルによる検証を伴わないあらゆる科学的発見は、機械の科学的仮説生成への応用例だと解釈することができます。各々の応用例における具体的な仮説生成の方法は通常全然異なるものですが、それらは知識生成の過程において仮説生成の機能を担っているという点では共通しています。

何を元に仮説を生成するのか、仮説の空間がどのようなものであるか、仮説がどのように表現されるかは研究分野によってさまざまです。例えば仮説は論文から生成されるかもしれませんし、科学的データから生成されるかもしれませんし、事前に定められた探索空間から見つけられるかもしれません。また、仮説は複数の要素の組み合わせとして表現されるかもしれませんし、数式で表現される数理モデルかもしれませんし、テキストで表現されるかもしれませんし。たとえば化学では主に仮説を組み合わせ的に表現し、この chemical design space の中を探索することで機械による仮説の発見が試みられてきました \cite{coley2020autonomous}。仮説を方程式で表現するものとしては、科学的データから背後にある法則を記述するシンボリックな方程式を発見する研究が行われてきました \cite{kramer2023automated}。また、論文群から仮説をテキストとして生成・抽出する研究も多くあります \cite{kang2022augmenting,chan2018solvent,wang2023learning,xu2023exploring,yang2023large}。
近年では、大規模言語モデルの台頭によって、論文の情報を直接的に与えずに、言語モデル自身の内部の知識のみから仮説生成をさせることを試みる取り組みも現れています \cite{park2023can}。

このように、個々の仮説生成の具体的な内実は異なりますが、これらをある観点で同一視しようとするものもあります。まず、仮説空間を人間が与えて所与とすれば、there is a perspective that many scientific discoveries can be seen as search problems \cite{coley2020autonomous}.特に、科学における仮説空間は組み合わせ的に広大であることが往々にしてあります。したがって、仮説生成する AI にはこのような広大な探索空間を効率的に探索する能力が重要であると考えられており、active learning などによって効率的な探索を実現することが試みられています。また、Wang らは機械学習を仮説生成に応用した研究群を別の観点で整理しました。彼らは、have categorized and organized how AI is being used for scientific hypothesis generation from the perspective of using it for black box prediction, aiding in the exploration of hypothesis space, and finding solutions in a differentiable hypothesis space \cite{wang2023scientific}.

このように、機械学習の仮説生成への応用は問いの構築や仮説の検証に比べて大きく進んでいます。ここで挙げた例はほんの一部であり、機械学習を仮説生成へ用いた研究はこの他にも膨大にあります。本論文では個々の研究について詳細に検討することはできませんが、ご関心がある方はそれぞれの分野のサーベイ論文を参照してください。

このように機械学習を仮説生成へ応用した事例は膨大にあります。しかし、問いに応じて仮説生成ができる AI をどのようにして実現するかはオープンクエスチョンです。このような AI を実現するためには、上述したような仮説の表現や設計法を問いに合わせて適切に切り替えたり、仮説空間を自らゼロから設計したり \cite{coley2020autonomousII} などする必要もあるでしょう。このようなものをいかに実現していくかについて今後議論が深まっていくことが望まれます。

\subsubsection{Things Seemingly Important in Hypothesis Generation?}
問いに応じて仮説を自律的に生成できる AI の実現についての課題は、問いの生成や仮説の検証と比べると、すでに多くの注目を集めていると言っても良いでしょう。その結果、すでに多くの議論が蓄積されていると思われます。

なぜならば、すでに述べたように仮説の生成はある問いに対する答えの予測であって、これはすでに機械学習で多くの研究者が取り組んでいる問題だからです。特に、問いに応じて仮説を生成できるAIとは汎用的な予測ができるAIのある特殊な場合であるため、このようなAIを実現する上での課題は、汎用人工知能を実現するための課題と大部分が重複するように思われます。例えば、より強い人工知能を作るためには、演繹などの系統的思考、分布外汎化、因果推論、効率的な探索、問題の部分問題への分割などが重要であろうと考えられていると思いますが、これらはより良い仮説生成をする上でも重要な要素であると考えられます。

このセクションでは、問いに応じて自律的に仮説生成ができる AI の実現にとって重要だと思われる要素について少し検討しようと思いますが、上述の理由から、ここで議論する内容はすでに強いAIを実現する課題として議論されているものと大部分が重複しています。その意味で、ここで展開する議論は新規性に乏しいとは思いますので、すでになされている議論の繰り返しになるかもしれません。以下では、新規性があまりないであろうことを承知の上で、仮説の生成において重要であろうと思われる観点を2つ取り上げて、それがなぜ重要だと思われるのかについて、簡単に議論したいと思います。ただし、これらは重要だと考えられている要素のほんの一部であり、その全てを網羅できていないであろう点についてはご了承ください。

まず、仮説生成の一つの重要な要素として、その問いの答えがまだ誰にも知られていないという点があります。人間にとって答えが未知な仮説を生成する機械にとっては、生成する仮説が必ずしもその機械にとって未知である必要はありません。しかし、その答えが人間だけではなく機械にとっても未知な場合は強いAIにとっても仮説生成はチャレンジングなタスクになるでしょう。このような場合、これまで人間がやってきたのと同様に、機械は自らが答えを知らない対象に接近する方法を自律的に構築して活用できなければいけないように思われます。これは定式化の上では単に分布外汎化になりますが、その答えが人類にも機械にも誰一人わかっている人がいないようなものであるという点で、特に難しい分布外汎化の問題であると考えられます。このような問題を解くためには、機械は人間がやってきたように何らかの方法で段階的に不確実性を削減することで答えに接近していく必要があるように思われます。このような不確実性の高い状況下での推論ができる AI をどのように実現できるかはオープンクエスチョンです。

二つ目に、仮説生成における演繹の重要性について再度強調したいと思います。人間の仮説生成においては演繹、特に数学が非常に強力であったことは疑いがありません。なぜなら、演繹においては、前提が真である限り演繹で得られた結果はどれだけ直観に反していても真であることが保証されるからです。これによって、経験からは自然には出てこないような予測もすることができるようになります。経験からの推論を基本とするAIにとってこれは非常に強力なアドバンテージとなります。また、人間が仮説演繹法として行なってきたように、演繹によって直接は検証され得ない仮説の確からしさについて議論することができるようになります。演繹結果が棄却されれば仮説は偽であるとわかりますし、採択されてもその仮説が確からしいという信念を強めることにつながります。これは研究について経験的に知りうる可能範囲を押し広げる上で重要な要素でしょう。演繹/仮説演繹法の重要性についてはすでにに様々に議論されていますが、仮説生成においてはこのような利点があると考えられます。

以上、2つの仮説の生成に関連しうる要素について議論しました。ここではこれらを二つの別々の要素として議論しましたが、これらのいずれにとっても、系統的思考(系統的汎化や構成的汎化)の獲得が重要であるように思われます。そのため、私はこのような AI の実現にとっては、多くの人が思っているのと同様に系統的思考の獲得が重要だと考えています。しかし、系統的思考の重要性についてはすでに多くの人が同意するところだと思われますし、すでに多くの優れた議論がなされているので \cite{goyal2022inductive}、ここでこれ以上は立ち入りません。

% 多くの仮説生成が仮説空間の探索として定式化できるとしても、仮説空間は問いに依存します。したがって、AIに問いの構築から自律的に実行させる場合は、AIはそれに応じて自ら適切な仮説空間を自ら設計しなければなりません。そもそも仮説空間がどのようになっているか明示的にわからない場合も多く、人間はそこを明らかにするところから進める場合も少なくありません。さらには、すべての仮説生成が組み合わせ的に必ずしも表現できるわけでもなく、原理的にできたとしてもそのように定式化することが効率が悪いものも存在します。仮説が生成できる AI はこれらのさまざまな場合に対応できることが期待されるように思われます。

% このように、機械に仮説生成を手伝わせる研究はすでに数多く存在します。しかしそれらは必ずしも自律的に仮説の生成の全過程を行わせる場合ではありません。特に、汎用的に仮説生成を行わせる取り組みはそこまで多くありません。

% 自律的かつ汎用的に仮説生成を統一的に扱う枠組みとしては、科学的発見を探索とみなすものがあります。←これを否定するのではなく、これはそうで、その上でこの探索のこの要素も含めて自律的にやらせると予測の問題になるよねみたいな書き方する。

% 前述したように、仮説の生成は答えの予測と同一視することができると思います。この意味で、より良い仮説生成ができる AI の実現は問いの生成や仮説の検証に比べてすでに多くの注目を集めています。したがって、それを実現するための課題についてもすでに多くの議論がなされています。(例えば分布外汎化や系統的推論や因果推論が重要であるなど指摘されています。)したがって、このセクションではこの論点についてこれ以上深掘りすることを避けます。この点については改めて議論できればと思います。人間のように汎用的に人間のような仮説を生成する AI を作ることまでまだ多くの課題があるのはそうだが、それはより強力な汎用モデルを作成する課題と重複するのではないかというのが懸念。そうであれば、特にここで強調する必要もなさそう。

% (課題がないということではない。例えば、研究ではかなり多くの問いが ``Why-question'' であり因果の推論が重要であるが、機械学習のほとんどの枠組みは相関の学習であり、適切に因果を推論できる AI が必要である。とか、前述したように科学的発見の多くは探索問題として記述できるので、仮説空間が構成できた後にはどのようにして効率的な探索を行うかという問題がある、とか。あとは、人間は数学を活用することができたために経験を超えた仮説の生成が可能になったため、数学ないし形式的推論を獲得できることが必要とか、数理モデルとして表現することもできるようになったため、数学が活用ができることが大事とか。じゃあ、研究に特有の予測の性質とはなんだろうか?自分にとって未知なものに対する推論←OODの程度がすごい、whyquestionに特化してる←因果推論。より良い推論ができるようになるには、という話で、結局汎用人工知能を作りたいというかなり広い論点と被ってしまう、という注意を入れる
% より良い人工知能を作りたい多くの人が関心がある論点。科学研究にユニークな点としては、探索空間が広大であることなどが知られてる。仮説空間を陽に定めない場合は、自分自身も知らないという状況がユニーク。これを解決するためにつような要素は結局みんなが必要だと思ってる要素だけど、みたいな話。仮説からの演繹の話はどこかで書く。おそらくこの仮説の生え成のところで書く?←仮説の確からしさを見る?その意味だと検証??理論研究の位置付けにもよるけど、演繹はやっぱり仮説のところで扱うべきでは。数式でモデリングしてそこから演繹で予測を出す→仮説演繹法を高度に推し進めた。As one of the promising approaches for tacking unknowns, it seems crucial for AI to acquire systematic thinking to realize this, as humans developed systems like language and mathematics, allowing them to infer about subjects beyond their experience. Systematic thinking is not only important for out-of-distribution generalization but is also valued for interpretability, causal inference, logical reasoning, mathematical processing, and planning.)

% Theoretical virtues \cite{schindler2022theoretical}

% 人間にとっても機械にとっても未知なものについては少し違うかもしれない?

% (人間が発展させてきた仮説生成の方法論は、自分も含めた人類全体にとって答えが未知であるような複雑な問いの答えをいかに探していくかという点が他の単なる予測とは異なる特徴だった。したがって、AI が AI 自身にとって未知なものを予測させることを目指すのであれば、それは単なる予測問題とは少し異なるかもしれない。しかし、機械に人間にとっての仮説を生成させることのみ要求するのであれば、より賢い機械が生まれた後では分布外汎化にすらならないかもしれません。)

% しかし、そのような汎用的な仮説生成AIを作ることが、単により強力な汎用機械学習モデルを作成することとどれほど異なるのか私にはまだわかっていません。前述したように、仮説の生成は質問応答であり、機械学習の予測問題に還元されるように思われるからです。仮説生成という行為がそのような予測のうち具体的にどのような特徴を有しているものなのか私にはわかっていないといった方が正確かもしれません。

% ここで述べたいのは、人間のように仮説を生成する人工知能を作る上で課題がないということではなく、現在あまり議論されていないがそれを実現する上で重要なその人工知能の実現にユニークな課題を私がまだうまく理解し定式化できていないということです。例えば問いの生成や仮説の検証は機械学習の分野の中ではあまり注目されていないタスクですが、研究をする人工知能を実現する上では重要な要素であるため、この論文で議論する価値があると考えています。一方より良い仮説を生成する機械はより良い予測ができる AI の実現という意味で、すでに多くの人が注目し多くの重要と思われる要素が議論されています。



% ランダムソートを並べるけど、すでに何度も語られている点がある

% 課題がないということではない。例えば、研究ではかなり多くの問いが ``Why-question'' であり因果の推論が重要であるが、機械学習のほとんどの枠組みは相関の学習であり、適切に因果を推論できる AI が必要である。とか、前述したように科学的発見の多くは探索問題として記述できるので、仮説空間が構成できた後にはどのようにして効率的な探索を行うかという問題がある、とか。あとは、人間は数学を活用することができたために経験を超えた仮説の生成が可能になったため、数学ないし形式的推論を獲得できることが必要とか、数理モデルとして表現することもできるようになったため、数学が活用ができることが大事とか、

% ただ、これらはより強い人工知能を作るために必要なものとしてすでに広く議論されているので、ここでは改めて取り上げない。


% 例えば、人間は仮説を生成するために類推的な推論を発展させてきましたが、類推的な推論もパターン認識の一種で、現在の機械学習の枠組みとその意味で大きく異なりません。発見するパターンの構造が簡単に見つけられないものであるという程度のものです。

% 個別の課題に最適なものを作るならもちろん別

% じゃあ、研究に特有の予測の性質とはなんだろうか?自分にとって未知なものに対する推論←OODの程度がすごい、whyquestionに特化してる←因果推論、探索空間が広域?\textcolor{red}{自分の、仮説生成と予測の違いがわからないという率直な気持ちを書いてFBを待つのがいい気がする。アナロジーの話とかもみんなわかってるけどそれでもみたいな感じ。仮説を数式で表現するとかは科学にユニーク?モデリング?みんな教えて!って感じのスタンスでいこう。}

% 汎用的な仮説生成機械を作ろうとすると、より良い予測ができる汎用モデルを構築するという話に結局収斂してしまう。

% 仮説の生成と仮説の検証の間に、演繹的推論がある・これが科学ではとても大きな役割を果たしていた。

% より良い推論ができるようになるには、という話で、結局汎用人工知能を作りたいというかなり広い論点と被ってしまう、という注意を入れる
% より良い人工知能を作りたい多くの人が関心がある論点。科学研究にユニークな点としては、探索空間が広大であることなどが知られてる。仮説空間を陽に定めない場合は、自分自身も知らないという状況がユニーク。これを解決するためにつような要素は結局みんなが必要だと思ってる要素だけど、みたいな話。

% 仮説からの演繹の話はどこかで書く。おそらくこの仮説の生え成のところで書く?←仮説の確からしさを見る?その意味だと検証??理論研究の位置付けにもよるけど、演繹はやっぱり仮説のところで扱うべきでは。数式d

% The second component is \textit{hypothesis generation}. Research is the endeavor to find answers to research questions. Here, as mentioned before, these questions are ones for which the answers are not known. Therefore, we generate hypotheses as candidate answers to these questions and find answers by validating them. Given that research is an endeavor to turn the unknown into the known, the generation of hypotheses, whether implicit or explicit, is an inevitable element in research.

% \subsection{What Would be the Challenges for an AI to Generate Hypotheses?}
% \subsubsection{Challenges: Hypothesis Generation as Question Answering}
% In research, hypothesis generation is question-answering based on existing knowledge. That is to say, generating hypotheses is equal to predicting answers to questions from one's own knowledge base. I believe that, by design, machine learning models generate hypotheses (this is different from the case of question construction and hypothesis verification). This is because machine learning models make predictions based on data. Therefore, I consider the generation of hypotheses in itself not to be the challenge for creating an artificial researcher.

% However, the capacity for scientific hypothesis generation by machine learning models seems to remain limited. For exaxmple, there are still no examples where machines have generated hypotheses or theories that fundamentally change science as Newton or Einstein did\footnote{
% If humans define the hypothesis space in advance, machines can generate hypotheses that can solve longstanding problems in a field (e.g. AlphaFold \cite{jumper2021highly}).
% }. Therefore, it seems that there are still challenges to be solved in order to create AI that can generate hypotheses like humans.

% \subsubsection{Challenges: Generating Hypotheses for AI}
% The characteristic of hypothesis generation in research is that it involves question-answering that seeks to find answers that nobody in the world yet knows. Bearing this in mind, I believe that one of the significant challenges is to endow AI with the ability to generate reliable answers to questions that even it, including everyone else, does not yet know the answer to.

% Even now, machine learning models are being used to generate scientific hypotheses, but it seems that they are referred to as hypotheses in the sense that their predictions are uncertain to humans, which may be unknown to us. This might be trivial for the machine. In other words, what is a hypothesis to humans may not be a hypothesis to the machine. 

% Even if hypotheses for humans are successfully generated, the machine may not be able to generate hypotheses effectively for questions whose answers are unknown even to itself. Indeed, it is known that machine learning models can fail to predict accurately under naive distribution shifts \cite{shen2021towards}, and they are also known to fabricate facts about knowledge they do not possess \cite{maynez2020faithfulness}.

% Furthermore, the essence of human hypothesis generation seems to lie in how to generate plausible answers in situations where the answer is not known. On the other hand, machine learning models are known to just fabricate falsehoods when they do not know the answer \cite{maynez2020faithfulness}. In other words, they do not seem to have sufficiently acquired the methodology to reach an answer under circumstances where the answer is not known. Therefore, it seems important to determine how to generate good answers to questions that are unknown even to the machine itself.

% In summary, I believe the challenges are: 1. How to enable the generation of complex hypotheses as humans do, and 2. How to enable the generation of good answers when the answer is unknown to the machine itself. To find leads for solving these challenges, it seems useful to examine how humans generate hypotheses. In the following sections, I will express my personal view on how it is believed that humans generate hypotheses.

% 研究における仮説生成は既存の知識に基づく質問応答です。つまり、自分が持っている知識から、質問に対する答えを予測するのが仮説生成です。

% 機械学習モデルはデータに基づく予測をします。その意味で、機械学習モデルは設計上すでに仮説生成をしていると私は考えています(これは問いの生成や仮説の検証とは大きな違いです)。つまり、仮説を生成させること自体は課題ではないと考えています。

% 一方で、依然として機械学習モデルの科学的な仮説生成能力は限定的であるように思われます。

% 従って、いかにして人間のような複雑な仮説生成ができる AI を作成するか、というのが問題の一つであるように思われます。

% また、人間における仮説生成の本質は、答えがわからない状況でいかに確からしい答えを生成するかという点にあると思われます。他方で、機械学習モデルは自分が知らないことについてはそれを認識できていなかったり、嘘をでっちあげてしまうことが知られています。つまり、答えを知らない状況下で答えに辿り着く方法論を十分に獲得できていなように思われます。したがって、「機械自身にとっても」その答えが未知であるような問いに対して、いかにして良い答えを生成させられるか、というのが重要であるように思われます。

% まとめると、    1. 人間のように複雑な仮説の生成をいかに可能にするか、2. 機械自身にとっても答えが未知であるときにいかに良い答えの生成を可能にするか、が課題であると私は思っています。これらの課題解決への糸口を掴むためには、人間がどのように仮説生成をしているかを見ていくのは有用であるように思われます。そこで、以下のセクションでは人間がどのように仮説生成をしているかを見ていきます。

% \subsubsection{人間の仮説生成において重要であるように思われるもの(上のセクションとまとめてもいい)}

% 数式でモデリングしてそこから演繹で予測を出す→仮説演繹法を高度に推し進めた

% Theoretical virtues \cite{schindler2022theoretical}

% \subsubsection{Generating Hypotheses for Which the Answers are Unknown to Both Humans and Machines}

% \subsubsection{How Do Humans Generate Hypotheses?}

% In order to create an AI that can generate reliable answers to difficult questions whose answers are unknown to anyone, it seems prudent to look at how humans generate hypotheses. Therefore, I would like to briefly present my thoughts on how humans appear to generate hypotheses. 
% Note that the methods humans use to generate hypotheses are diverse, and the discussion here can by no means cover them all. I hope that what I present can be taken as a starting point for further exploration.

% \subsubsection{The Complex Step-by-Step Hypothesis Generation Process}


% hesse1965models,thagard_1984,gentner1993shift,holyoak1996mental,dunbar1997scientists

% I began by explaining that I first formulate the unknown I am addressing in the form of a question. The process of finding answers to this question is research. Here, because the answer to this question is, of course, unknown, it requires inference as to what the answer could be. As a result of this inference, a plausible answer is formulated. This process corresponds to the \textit{hypothesis generation}. Hypothesis generation is the inference about the unknown and the definition of research is to transform unknown to known. Thus, every research including deductive research must entail hypothesis generation implicitly or explicitly. In this sense, hypothesis generation should be the second module of knowledge production system.

% The belief that a hypothesis is true is the very object that can become knowledge in response to a question. If a hypothesis withstands proper testing, the belief in its plausibility strengthens. Conversely, if a hypothesis does not withstand testing, that belief is weakened. Therefore, the former generates knowledge that ``the answer to question A is hypothesis B,'' while the latter generates knowledge that ``the answer to question A is not hypothesis B.'' 

% Hypothesis generation is the act of creating potential answers to a question, so naturally, it is essential to generate plausible hypotheses that are close to the actual answer. Therefore, let's start by referring to how humans generate hypotheses and then discuss how I can generate reliable hypotheses, drawing inspiration from human methods.

% \subsubsection{Types of Questions}
% There are various types of questions. For instance, a ``How'' question is posed when one wants to know a specific way to do something, while a ``Why'' question is asked when one is curious about the cause or mechanism of something. Among them, it would not be an exaggeration to say that the ``why'' question is of the highest interest to researchers. Researchers have developed methodologies for expressing queries about causality and for inductively inferring answers to those queries based on them, i.e.\textit{ causal inference} \cite{pearl2018book}.

% An AI tasked with formulating questions autonomously is expected to be capable of posing questions, regardless of their question type, and to develop methodologies for answering them. The broader the class of questions that need to be addressed, the more general the methodology required becomes. How to realize such an AI remains an open question.

% \subsubsection{Challenges}
% One challenge in creating an intelligence capable of hypothesis generation, not just as a tool for humans, is the need to empower the machine itself to form plausible hypotheses for questions to which even the machine doesn't know the answer. Current machine learning models have been criticized for potentially not knowing what they don't know \footnote{
% In our discussion with Wataru Kumagai, we were reminded once again of the importance of self-awareness in creating an AI capable of conducting research.
% }. 
% Moreover, they are known to confidently provide answers or fabricate falsehoods about topics they are ignorant of. Therefore, it seems essential to first accurately recognize what is unknown, either for oneself or the world at large, as told in sections of question construction. Upon facing an unknown subject, there's a need to reduce uncertainty and approach understanding. As mentioned in Chapter 2, humans attempt to understand uncertain subjects by gathering information from papers, experiments, or by reframing questions. While it may not be necessary to adopt the exact same approach, it seems essential to enable machines to autonomously adopt strategies to reduce uncertainties.

% Outputs from machine learning models are essentially inferences tinged with uncertainty. From this perspective, one could posit that these models are already inherently generating hypotheses. Indeed, they are already employed for hypothesis generation in numerous scientific investigations.

% However, while these hypotheses might be hypotheses in the sense that the answers are unknown to humans, they might be self-evident to the machine learning model. When we talk about AI generating hypotheses in the context of AI conducting research, the ultimate expectation is for the AI to provide plausible answers to what is unknown to AI itself. This remains an unresolved issue.



% \textcolor{red}{TODO}

\subsection{Hypothesis Verification}
最後の要素は仮説の検証です。私たちは問いに対する予想の確からしさを検証によって確かめることで、ある仮説が真である・偽であるという信念を正当化します。したがって、知識を生み出すためには検証は不可欠です。

検証がどのようなものになるかは、当然ですが問いと仮説が何であるかに依存します。例えば、問いが ``Why-question'' であれば、検証は因果関係を明らかにできる方法ではないといけません。問いが物理的世界についての言及であれば検証は当然物理的世界との相互作用を必要とします。仮説が数学的証明で証明され得るものであれば、数学的証明が検証となるでしょう。検証ができる AI は、検証とは何かというハイレベルな理解を持ち、これらのいずれの状況でも、問いと仮説に基づいて、適切な検証法を自ら選択・構築し実行できることが期待されます。

仮説の生成に比べると AI 自身に検証をさせるという話は多くはありません。科学における実験計画やシミュレーションなど検証の一部分のタスクを代替させる研究は以前から多くありますが、AI 自身に検証とは何かを理解させ、仮説を検証するためには何をすべきかゼロから考えさせるような取り組みは、まだ多くないと言って良いでしょう。確かに機械学習の研究では、科学的主張の妥当性を判断する研究や、予測が事実であるかを確認する研究は、検証に関連する研究ですが、いずれも人間が行う科学研究のような検証を構築して実行するものではありません。査読の自動化も論文で提案されている検証の妥当性の判断を求められるという意味で、検証の理解に関連する研究ですが、検証を生成するものではありません。

本セクションでは、自律的に研究ができる AI を作るために必要そうなことを考える思考の種を提供できるよう、検証について分析していきます。検証とは何か、すなわち正当化とは何かについてはすでにセクション 2 で話をしましたので、ここではそれに関する議論については割愛します。代わりに、このセクションでは人間のような検証をする際に重要であると思われる要素について議論します。特に、科学における検証である実験とその重要な要素である統計的分析について議論します。これによって、検証ができる AI は何ができなければならないのかについて少しだけ解像度を上げる手助けとなれば幸いです。

\textcolor{red}{因果とかの話する?}

% The third sub-module is hypothesis verification. Knowledge production in this paper is defined as the process of justifying beliefs in a way that convinces members of the society. Hypothesis verification corresponds to this justification. As explained in Section \ref{section-knowledge-production-as-belief-revision}, I believe that science is a reliable source of knowledge production precisely because it is grounded in methods based on strong beliefs that are held by virtually every human being and are so fundamental that even everyday life would be impossible without them.

% In the previous section, I pointed out the possibility that allowing verification methods to be entirely autonomously constructed could result in outputs that are meaningless to humans. So, what exactly needs to be created, and to what extent, in order to develop an AI capable of performing verification? I would like to consider this point in this section.

% I consider the notion of verification as an update of shared beliefs to be useful in comprehensively describing the process of verification across a wide range of research, from theoretical studies to humanities. However, it might be somewhat too abstract when aiming for the current purpose. To provide a foundation for further discussion, I would like to focus on verification in science, particularly on its core elements: experimentation and statistical analysis.

\subsubsection{Experimentation}
\label{section-experimentation}
研究者で実験の重要性を否定する人はいないでしょう。実験とはある仮説をテストするための一連の手続きのことで、経験科学における検証そのものです。したがって、検証ができる AI は当然実験ができなければなりません。

実験では観察が困難な現象や様々な条件の影響を正確に調べるために、統制された方法で人工的に自然現象を生成し、これ対して積極的に介入をすることで調べたい変数の関係や因果関係などの違いを人為的に作り出します \cite{radder2009philosophy}。これらを観測して実験データとして保存し、このデータを解析することで仮説が真であるか否かを判定します。
(footnote) 必ずしも検証の時だけではなく、仮説を生成する際にも実験は行われます。また、実験で出てきた結果を受けて新しい問いや仮説を生成することが行われます。この時、データを生成するまでの過程を、あるいは必ずしも検証のためだけではなくデータ生成となんらかのデータ解析をすることを指して実験と呼んでいるように思われます。本論文では計画を立てて、準備をして、データを生成してそれを解析して検証結果を判断することを指して実験と呼びますが、それが必ずしも実際の実践を反映していない可能性がある点には注意してください。

実験を実施するためには、まず実験のデザインをし、実験の手続きを書き下し、実験の計画を立てなければなりません。このためには、何をどうしたら仮説を検証したことになるのかという検証についての理解、そして既存の技術を駆使して具体的にそれを実現する方法を考える能力が必要になります。そして、これらを実際に実行可能なものとするため、実験の準備をしなければなりません。例えば薬品を購入したりフラスコを準備したり動物を訓練したり必要な装置, 時には加速器のような巨大な装置, をゼロから作ったり、倫理委員会に申請を出したり、クリーンルームを作ったり、しなければなりません。さらには、研究はまだ誰も知らないことを明らかにすることを目指すので、実験のために使う装置をゼロから作ることは研究においては珍しいことではありません。これらの準備を機械がゼロから自律的に実行するのは相当な困難です。

実験の準備ができた後は、実験プロトコルに従って実験を実行します。これも機械が自律的に実施するのはとても難しいことです。なぜなら、一つの実験をするのにも、掴む、切る、運ぶ、混ぜる、移動する、注ぐ、分注する、洗う、蓋を開ける、などなどの無数の低レベルの操作を自在に組み合わせて実験を実行する必要があるからです。自律的に実験ができる機械には与えられた実験プロトコルに応じてこれらの操作を自在に生成できる能力が必要となります。

% I recognize that the the way to verify a hypothesis is strikingly diverse, as it can significantly differ depending on the subject of research. For instance, if one wishes to study the behavior of rat, they may need to train the rat. On the other hand, if you want to test a theory of particle physics, you may have to construct and run a huge particle accelerator. In the field of history, the existence of historical records might serve as evidence, while in mathematics, the verification process revolves around the proofs themselves. Due to this high degree of flexibility, hypothesis verification can be the most challenging aspect to automate for realizing a ``general'' artificial researcher.

\subsubsection{Automating Experimentation}

このように、実験の計画だけでなく準備や実行まで機械に完全に自律的に実行させるとなると、相当困難であることがわかります。特に、どのような実験をすべきかは問いや仮説を立ててからでないとわかりませんから、問いの構築から自律的に研究をさせるとなると、あらゆる実験の可能性に対応できるような能力が要求されます。私はこれは研究が自律的に実行できる機械を作る上での最大の障壁の一つであると考えています。

このように実験の自動化は非常に難しいタスクですが、人類はこれまで着実にこの難しいタスクを前に進めてきました。実験の計画の段階に関連するものとしては、例えば実験条件の探索の効率化と自動化は長い歴史を持っています。Wang et al. have summarized these studies that utilize AI to assist experiments by research planning, research guidance, and generating observational data through numerical simulations \cite{wang2023scientific}. 

また、laboratory automation と呼ばれる実験の実行も含めた自動化の試みもあります。
 
A notable example is pioneering research in genetics by Ross King, who fully automated the cycle of hypothesis generation, verification, and discovery of new hypotheses with Adam \cite{king2004functional}. Another example is A.I. Cooper, which enabled the use of the same experimental equipment as humans through autonomous robots \cite{burger2020mobile}. These examples of initiatives aim to autonomously drive the research cycle, including hypothesis generation, planning and execution of experiments, and generation of hypotheses based on experimental results. Such initiatives are referred to as the closed-loop automation of scientific discovery     \cite{burger2020mobile,king2004functional}. This represents an example of achieving extremely high autonomy in the quest for research automation. Furthermore, there are efforts to develop humanoid robots for experiments, capable of conducting multiple different experiments with a single robot \cite{yachie2017robotic}. This is deemed to be a foundational step towards potentially general research automation. 

In recent years, there have been studies attempting to automate the experimentation using language models \cite{boiko2023emergent,charness2023generation,qin2023gpt}.

また、実験によって生成されたデータを解析して解釈するためには、それを読み取るだけの事前知識が必要です。なぜなら、何かを観測することは常に何らかの理論を背後に伴うからです(観測の理論負荷性)。したがって、機械学習モデルに科学データを理解するために必要な対称性や微分方程式や直観物理といった物理学的な事前知識を入れ込むことで、機械が科学データを解釈できるようにする研究があります(必ずしも検証のためだけではない)。

このように、さまざまな研究が実験の自動化を着実に推し進めてきました。一方で自律的に実験ができる機械の実現にはまだ多くの課題があるのも事実です。Coley et al. discuss in detail these challenges of automating experimental and computational validations, as well as the selection of experiments, while referencing studies on automated verification \cite{coley2020autonomousII}.特に、さまざまな研究課題に適応可能な汎用的な自動実験機械の実現には、人間のように低レベルの行動を自在に操れるロボットが必要であり、非常に実現が難しいです。そこまでいかなくても、実験の自動化を進めていくためには、ハードウェアなどの制約を取り払って自動化可能な研究を増やしていくことや、研究の自動化のためのコストを下げていくことなどの努力も重要です \cite{coley2020autonomousII}。


% Experiments are generally considered a series of procedures to test a hypothesis, particularly referring to the process of data generation. I believe that the main elements in an experiment are the interventions to test a hypothesis and the generation of data to artificially create observations of phenomena. Verification is then done through the analysis of data obtained from these experiments.

% The experiment can largely be divided into three phases: designing the verification plan, preparing to execute the verification plan, and actually carrying it out. Therefore, it seems that in order to create an AI that can conduct experiments, we must create an AI capable of performing all of these tasks.

% \subsubsection{Automating Experimentation}
% 実験の自動化のレビューの話を書く、LAとか実験計画とか

% \subsubsection{Understanding Verification}
% To become capable of designing verification plans, it seems necessary to have the ability to understand what verification is and the ability to engage in long-term planning. Additionally, preparing for verification, possessing a human-like body, and the capability to manipulate it freely will be required. Among these, the acquisition of intricate planning and flexible action has been the subject of research by many artificial intelligence researchers.

% On the other hand, the acquisition of the ability to understand verification seems to receive relatively little attention. There are indeed studies that focus on the importance of verification for reducing hallucinations and verifying the reliability of scientific claims, but these do not aim to determine what constitutes verification in scientific research. Therefore, I believe that considering how to make machines comprehend what can qualify as verification remains one of the challenges ahead.

\subsubsection{Statistical Inference and Verification}
Science can be seen as the activities to find the law of this world by inductive reasoning based on the observations. Scientists have formalized inductive reasoning with terms from probability theory and constructed methods to perform inductive reasoning as statistical inference. Specifically, we model the law of nature as probability distributions, and by inferring them from a finite sample sampled from that distribution, we attempt to say something about it.


% In science, the validity of hypotheses is judged by inductive reasoning based on data generated from experiments.

% Inductive reasoning is a premise of science, and implementing it via statistical inference is a widely practiced approach. Therefore, when creating an AI capable of performing meaningful verification for humans, it would be sensible to take these as a given.

\textcolor{red}{後で書く。そもそもここで何を言いたいか?何を持って検証とするかは複数あると上で話したけどここでそれについて少しだけ詳しくみようみたいな話し方をする?じゃあそれを話して出てくる帰結は?一緒?主に統計的仮設検定の話を書こう。ネイマン・フィッシャー・ベイズ。もう少し自分の中でちゃんと整理してから書く。あんまり統計的推論とか確率解釈とかそういう可燃性の高い話題は避けたい}

% On the other hand, there can be multiple interpretations of what it means for a hypothesis to be justified by statistical analysis \cite{otsuka2022thinking}. This implies that there are conflicting views on what methodology should be adopted to justify beliefs. As a result, there are multiple methodologies for statistical inference. Ideally, a machine capable of verification should understand ``why a certain inference is valid in justifying a hypothesis''  and be able to select or construct an appropriate methodology from these options. Otherwise, it might not truly understand what constitutes verification.

% To create an AI that can verify hypotheses, I believe it is necessary for the AI to understand the basis for verification. However, to create an AI that can perform human-like science, it may be sufficient if it can appropriately use statistical inference methods, which humans employ. This is because even if it doesn't understand why that constitutes verification, as long as it uses statistical inference correctly without violating its premises or rules, there should be no issue in using it just as a tool. In reality, even humans do not always seem to grasp these differences when conducting verification. For instance, it is not uncommon for people to use hypothesis testing without understanding why it is a valid means of justifying beliefs about a hypothesis. 

% In the case of hypothesis testing, just like in hypothesis generation, the tasks for automating hypothesis testing vary depending on who the hypothesis testing is for. If artificial intelligence is used as a tool for human research, it is sufficient for artificial intelligence to faithfully reproduce what humans do as verification. For example, it would be great if it could conduct, for instance, hypothesis testing. If the machine is capable of automatically preparing for statistical hypothesis testing, calling the appropriate statistical hypothesis testing libraries at the right timing, and using them correctly for the relevant subjects, then I would have no complaints about their performance.

% In this case, it is not necessary for the machine to strictly know why it constitutes verification, as long as it can learn from numerous examples of human verification and use it appropriately. In other words, in this case, the required understanding can be described as indirect and practical understanding through examples of human usage. Furthermore, if it can be confirmed that the machine not only mimics humans by using libraries but also understands what statistical hypothesis testing is and its principles, this would be a significant advancement in the automation of hypothesis testing. 

% I used statistical hypothesis testing as an example, but the same applies to other verification methods as well. The point is that in this scenario, I seek to enable artificial intelligence to appropriately utilize the verification techniques employed by humans. However, it is essential to note that the verification itself remains relevant and meaningful to humans.

% On the other hand, aiming to make machines understand and acquire the concept of verification autonomously from their own perspective becomes an immensely challenging task. This is because, as repeatedly emphasized, hypothesis verification involves the updating of beliefs, and the belief system of machines can significantly differ from that of humans. It doesn't seem that machines can acquire the concept of their own verification just by observing examples of human verification. Furthermore, as explained earlier, machines have not evolved their belief systems through interaction with the natural world. Therefore, verification methods entirely composed and developed by machines may no longer serve as effective tools for understanding nature. How to enable machines to autonomously acquire verification principles from the ground up, address the alignment issues between machines and humans, and ensure that these efforts lead to a better understanding of nature are challenges that the entire community should discuss and explore in the future.

% \subsubsection{Challenges}
% In Chapter 2, we highlighted several challenges in realizing an AI capable of verification. First and foremost, the AI itself needs to understand what verification is, and by what criteria a sequence of actions qualifies as verification. Ideally, it would be preferable for the AI to contemplate and understand from scratch what verification is. However, many humans don't do this either, and as discussed in Chapter 2, the philosophical debate on precisely defining verification is still unresolved. Therefore, it's harsh to demand this of a machine. At the very least, the machine needs to thoroughly understand and proficiently use verification concepts that humans employ, such as statistical hypothesis testing, from first principles.


% Second, the AI must be able to formulate detailed and complex plans to verify a hypothesis. With the advancements in language models in recent years, we are now much more capable of formulating superior plans than before. However, devising detailed plans remains a challenging issue.

% Third, it has to be prepared to carry out these plans and execute the plan with the combination of human-like complex actions.As mentioned in Chapter 2, to achieve this, the AI must be able to search for, create, purchase, and manipulate equipment with almost the same degree of freedom as humans, requiring it to exhibit extremely sophisticated and complex behaviors. This is an immensely challenging issue, and it might even be fair to say it's one of the biggest bottlenecks in realizing an intelligence capable of generic and autonomous research. Laboratory automation have attempted to address this challenge in real world by developing robots. We will discuss the case within the computer below.

% For AI to execute research on a computer, it must perform any operation within the computer. For instance, machine learning research entails, setting an environment, preparing datasets and models, and writing and executing codes. To allow AI to prepare these without human intervention, the AI itself must be able to autonomously search the web, select data, download it, and so on. Furthermore, once the AI generates code for verification, it must operate the shell to execute it.

% There are ongoing initiatives to enable language models to operate browsers \cite{nakano2021webgpt,act1}. While full browser operations might seem ambitious, there are already endeavors to allow language models to conduct searches \cite{mialon2023augmented}. If we achieve browser automation, it will greatly advance research automation involving web operations. Moreover, efforts like the open interpreter \cite{openinterpreter} aim to automate any computer action. This direction holds promise for automating all research confined within a computer. Although these studies are gaining traction in the machine learning domain, they're not always linked to research automation. We advocate recognizing this as a pivotal challenge in the realm of research automation.

% In the field of machine learning, it seems that the discussion on automated validation has not garnered much attention until now. However, recently, the need for verification is recognized in the machine learning community beyond outside of the context of research automation. Studies like \textit{scientific claim verification}, which received much attention during the COVID-20 pandemic \cite{wadden2020fact}, or attempts to minimize hallucination \cite{dhuliawala2023chain} are examples of them. These are not attempts to automate validation in research. Therefore, these findings cannot be directly applied to the automation of research validation. However, we expect that these studies will provide useful insights for the future development of artificial intelligence capable of understanding validation.

\subsection{Combining Questions, Hypotheses, and Verification}

研究の構成要素である問いの構築、仮説の生成、仮説の検証について説明しました。すでに終了した研究を振り返ると、実際にその研究の問いがあり、問い対する仮説があり、仮説に対する検証が見つけられます。この観点からすれば、研究とは、古典的に語られてきたように、問いの構築、仮説の生成、仮説の検証の連続からなる営みであるということもできます。

一方で、実際の研究は極めて複雑な試行錯誤的なプロセスであり、問いの構築、仮説の生成、仮説の検証が一度の研究で最初に想定していた通りそれぞれ一回ずつしか行われないということは稀です。実際にはたとえば一つの仮説を立てるのにも無数の問いや仮説を生成していますし、そのすべてが必ずしも最終的な研究成果につながるわけではありません。問いの構築、仮説の生成、仮説の検証はむしろ不確実性を削減するための基本的な操作と見る方が適切であり、それらを複雑に組み合わせてより大きな研究課題の不確実性を削減していると見る方が実際の研究の営みにも沿っているでしょう。特に、問いの構築や仮説の生成といった発見的な作業の中ではこのような試行錯誤的な要素がより重要になってくると思われています \cite{yanai2020hypothesis}。

I believe that all endeavors to transform the unknown into the known, to some extent, inevitably involve the construction of questions, the generation of hypotheses, and the verification of these hypotheses.

研究ができる AI を作るためには、問いの構築、仮説の生成、仮説の検証を自在に組み合わせて、このような複雑な研究実践ができなければならないように思われます。そこで、本セクションでは、これまであげてこなかったような、研究の特徴に注目しながら、そのような AI を作るために考慮すべきであろう事項について検討していきます。

% \subsubsection{Human Research Practice and Knowledge Production System}
% In actual research, while addressing the initial question posed, another question may arise and the focus may shift to that new question. Also, before reaching the final hypothesis that gets reported, several different hypotheses are tested repeatedly. Like this, the actual practice of research is complex.\footnote{
% For example, the concept of \textit{night science} as proposed by Yanai and Lercher symbolizes the such complex reality of research practice \cite{yanai2019night}.
% } When compared to this trial-and-error process, the framework I have proposed may seem overly simplified at first glance. However, I believe that the framework I proposed encompasses these human practices as well.


\subsubsection{Countless Question, Hypothesis, and Verification in Single Research Process}


\textcolor{red}{問いや仮説や検証が組み合わされる小さな単位であり、それらは単線的なプロセスではないということを強調するポンチ絵を置く}

We generate countless questions and hypotheses, including implicit ones, for the purpose of creating a single question or hypothesis, or for planning and preparing for single verification.

For example, when questioning how an unresolved issue can be solved, we might ask why this problem hasn't been solved so far, or if there are any studies of similar challenges being addressed. In response, we might consult literature or recall our own memory, hypothesizing that ``This could be the reason it hasn't been solved,'' or that ``This could be useful in solving the current challenge.'' We repeat this process innumerably to eventually construct an answer to the original question. Similarly, when planning verification, we implicitly pose many questions and formulate numerous auxiliary hypotheses.

To generate a plausible hypothesis, there must be sufficient grounds to believe it is valid. These grounds could be knowledge from our own memory, descriptions from newly researched literature, opinions from other researchers, or some belief, such as natural law should be simple. In addition, to examine the validity of the hypothesis, we might try simple tests. For example, we might create a toy model to represent the hypothesis and examine its behavior, or conduct preliminary experiments. In other words, each time a plausible hypothesis is generated, it undergoes a sort of verification, whether implicit or explicit, and to varying degrees of simplicity.

In this way, to conduct research, we pose countless questions and hypotheses and conduct several simple  experiments to verify the plausibility of the hypothesis if necessary. In other words, we could say that research is a hierarchical composition of question construction, hypothesis generation, and hypothesis verification. This is because research is an endeavor facing the unknown, and this process contains a lot of uncertainties. By gradually reducing these uncertainties through trial and error, research progresses. This gradual reduction of uncertainty seems essential, especially when tackling difficult questions that no one in the world knows the answers to. Therefore, AI capable of conducting research should autonomously be able to generate numerous questions and hypotheses as needed and choose more plausible hypotheses through simple verification during the knowledge production process.

% 私たちは、一つの問いや仮説を生成するために、あるいは検証の計画や準備をするために、暗黙的なものも含めると無数の問いや仮説を生成しています。

% 例えば、ある解決されてない課題をどのように解くことができるか問うた場合、ではそもそもなぜこの問題はこれまで解かれてこなかったのだろうか、似たような課題に取り組んでいる事例はないだろうか、と問うかもしれません。これに対して文献を調べたり過去の知見を思い出したりして、これが原因で解けなかったのではないか、これは今回の課題の解決に使えるのではないか、といった仮説を立てるでしょう。これを無数に繰り返して、最終的に元々答えたかった問いに対する答えを構成します。また、検証の計画を立てる際にも、私たちは暗黙的に多くの問いを立てて無数の補助仮説を立てています。

% 確からしい仮説を生成するためには、その仮説が妥当であると信ずるだけの根拠がなければなりません。その根拠となるのはは自分の記憶にある知識かもしれませんし、新しく調べた文献の記述かもしれませんし、他の研究者の意見かもしれません。これに加えて、確からしい検証をするために、私たちは試してみる、すなわち簡単な検証をすることもあります。例えば、仮説を表現するトイモデルを作って挙動を調べてみるかもしれませんし、プレ実験をしてみるかもしれません。

% このように、私たちは一つの問いを作り、それに答える一つの仮説を作り、それを一度検証するために、無数の問いと仮説を立て、必要とあれば仮説の確からしさを簡単に検証するいくつもの実験を行います。これは、研究が未知に立ち向かう営みであり、その過程にはいくつもの不確定な要素が存在するからです。それらの不確実性を一歩一歩試行錯誤的に削減していくことで、一つの研究が進んでいくように思われます。このような漸進的な不確実性の削減は、特に誰も答えを知らないような難しい問いに挑戦する際には不可欠であるように思われます。従って、研究ができる AI もこのように、必要に際して問いや仮説を生成し、簡易的な検証によってより確からしい仮説を選択するということを自律的に行えるようになる必要があるでしょう。

\subsubsection{Countless Seemingly Unrelated Operations to Knowledge Production}
\label{section-countless-seemingly-unrelated-operations-to-knowledge-production}

研究は、一見すると知識生産に関係がないような無数の操作によって成り立っています。たとえば、セクション \label{section-experimentation} で述べたように、実験を実施するためには私たちはモノを購入したり何かを作ったりします。 Bruno Latour は一つの研究室で行われる作業を人類学的に観察することで、こうした日常的実践が一見するとその意味がわからないような操作たちによって形作られていることを描き出していますが \cite{latour1987science}、このような文献を参照するまでもなく日々の学術活動が多くの非知的な操作によって形作られていることは多くの研究者にとって疑いのないことであると思います。

問いの構築や仮説の生成や仮説の検証は達成したい目的であり、知識生産過程における機能であって、その実装ではありません。これらの機能を実装するためには上述したような操作ができること、そしてそれらを適切に組み合わせて目標を達成するための作業をすることが不可欠となります。このような複数の異なる操作を適切に組み合わせることは、特定の研究課題に特化したとしてもなお難しい課題です \cite{coley2020autonomousII}。問いの生成から始めて研究の過程の全てを自律的に実行できるようなエージェントを仮に目指す場合、人間のように自在に行為を生成できるような能力を獲得することが不可欠です。

\subsubsection{Discovering New Questions}
\textcolor{red}{手法から問題を考えることがある}

Researchers often set out to solve one question, only to find themselves discovering an entirely unrelated question in the process. This new question, unrelated to the original question and even its underlying purpose, may lead them to pivot their research focus and potentially make significant scientific discoveries. Since research is full of uncertainty, it's not uncommon to be unable to foresee everything from the start. Thus, this discovery and shift of question is not rare.

If an AI capable of conducting research were tasked with achieving a single objective, such serendipitous discoveries might not occur. This is because any new questions it finds, no matter how intriguing or scientifically important, may not be directly relevant to achieving its predefined goal. Therefore, to encourage an AI to uncover such unrelated questions, it might be necessary to set multiple objectives or a broader high-level goal that encompasses both original and newly found questions. 

However, it is not sufficient to just have a common high-level goal.
To switch from the current question to a new one, the AI would need to evaluate which of the two questions holds more value.したがって、セクション \ref{section-deciding-what-knowledge-to-seek} で述べたような問いの価値に関する意思決定は一つのある問いを問うべきかだけではなくこのような高次の目的を共通するものも含めた複数の問いの間の比較もできるものでなければならないように思われます。

% 研究者は、ある問いを解くために別の問いを立てるだけではなく、その問いに答えようとする中で、最初の問いやその背後の目的とは全く関係ない別の問いを見つけることがあります。そして研究の途中でそちらの問いに取り組むことに切り替え、その結果大きな科学的発見を成し遂げることもあります。不確実性が高い研究という取り組みにおいては、最初から全てを見通せることは多くありません。したがって、このようなことが起こることは少なくありません。

% もし 研究ができる AI にある一つの目的を達成することを目指させるとしたら、このようなことは起き得ないでしょう。なぜなら、そこで見つかった問いはどれだけ面白くてもどれだけ科学的に重要でも、その目的を達成するために必ずしも重要とは限らないからです。したがって、AI にこのような問いの発見をさせるためには、複数の目標を持たせる、あるいは両方の問いを包括するようなより大きな(例えば「自然を理解する」のような)大きな目標を持たせる必要があるように思われます。加えて、現在やっている問いに取り組むのをやめて新しい問いに取り組む意思決定をするためには、AI はこれらの異なる二つの問いのどちらが価値があるかを比較しなければならないように思われます。

\subsubsection{Feedback from Verification Result}
% 化学自動化論文の内容もここに追加する
% how should new data acquired through experimental/computational validation be used to
% update models pretrained on literature data?

In research, it's not always common for the initial hypothesis to directly answer the question posed. Rather, the process typically involves revising the hypothesis based on the results of verification, followed by additional rounds of testing. This cycle of hypothesis revision and retesting is crucial in the journey toward finding answers to the posed questions, especially when seeking unknown answers, as mentioned earlier. Consequently, it seems necessary for an AI capable of conducting research to have the ability to revise its hypotheses based on the outcomes of these verifications.

こうした検証結果からのフィードバックも考慮した closed-loop な科学的発見の自動化を目指す試みとしては、laboratory automation \cite{king2004functional} や scientific workflow \cite{gil2022will} に基づく研究などがあります。このような研究は検証結果から仮説を修正するという科学の根幹の営みを自動化していくという上で極めて大きな貢献をもたらしたと言って良いでしょう。

このような先進的な事例はあるものの、検証結果からのフィードバックを自律的に人間のようにできる機械の実現にはまだ多くの課題があるように思われます。例えば、検証結果から機械にフィードバックを出力させる場合、その検証結果を何に反映させるか、それを受けて何を修正すべきかは人間が事前に与えることがほとんどです。 In reality, when the result for verification were negative, identifying the cause of the result is not as straightforward as one might think. This is because the verification relies on numerous implicit and explicit hypotheses and all of them can be the cause of the result \cite{sep-scientific-underdetermination}. The cause may be the proposed main hypothesis, a premise behind the hypothesis, an auxiliary hypothesis, an observation, an experimental instrument, or all of them. An AI that can conduct research must identify which of these possibilities is the cause. Once the candidates for the cause are determined, the AI has to generate a more plausible hypothesis based on the results of verification, as we have discussed so far.

また、In science, to identify the cause, it would be necessary to appropriately interpret the data generated by experiments. However, what one reads from the data can change depending on the individual's beliefs, prior knowledge, theory, and what they expect to find \cite{hanson1965patterns}. Therefore, these verification results may be reinterpreted multiple times, and with each reinterpretation, the hypothesis that needs to be revised may change. さらには、前のセクションで述べたように、人間は検証結果から全く異なる問いを生み出すような、斬新な気づきを得ることもあります。検証結果から自律的にフィードバックを生成する既存の機械も確かに検証結果を解釈はしていますが、このような複雑な検証結果の解釈は依然として実現できていません。Ideally, AI capable of conducting research should be able to interpret verification results with these possibilities in mind.
% 科学であれば、原因を特定するためには、実験によって生成されたデータを適切に解釈する必要があるでしょう。しかし、データから何を読み取るかは、その人の持つ信念や事前知識やそこから何を期待するかによって変わり得ます。したがって、この検証結果は何度も再解釈される可能性があり、その度に修正すべきだと思われる仮説が違ってくるかもしれません。研究ができる AI は理想的にはこのような可能性も視野に入れて検証結果を解釈していくことができるようになることが望まれるでしょう。


\subsection{Common Topics}
これまで、研究において重要であると思われる要素である、問いの構築、仮説生成、仮説検証について議論し、その後でそれらを組み合わせた場合に関連する話題について議論をしました。このセクションでは、これらすべての機能に関連しうるような話題や、これまでのセクションでは触れられなかった話題について議論していきたいと思います。

\subsubsection{Language Models}
近年の言語モデルの急速な発展は数々の革新的な成果を生み出してきました \cite{zhao2023survey}。将来研究ができるエージェントが言語モデルの知見を活用しないことは考えられないでしょう。このセクションでは、言語モデルを使った科学的発見の試みについて議論します。

言語モデルは 2017 年の Transformer の登場以降、大規模コーパスによって事前学習をする試みが一般的となりました。特に、多くの下流タスクに使えるモデルという基盤モデルという概念が導入されると \cite{bommasani2021opportunities}、このような基盤モデルの構築を目指す動きが加速しました。科学の分野においても、科学的テキストで事前学習をした科学的言語モデルを作成する試みや科学の基盤モデルを作ることを目指す動きが出てきました \cite{beltagy2019scibert,singh2022scirepeval,nadkarni2021scientific,cohan2020specter,gupta2022matscibert,taylor2022galactica,azerbayev2023llemma,xie2023darwin,luo2022biogpt,li2023llava}。 また、科学的データはテキストデータだけではなく複数のモダリティのデータが必要となるので、そのようなマルチモーダルなデータで事前学習した科学的汎用モデルを作る取り組みも現れました \cite{singhal2023towards,takeda2023foundation,nguyen2023climax}。

2020 年の GPT-3, 2022 年の ChatGPT (GPT-3.5), 2023 年の GPT-4 が登場すると、これらのモデルがかなりの範囲の知的タスクかなりの精度でこなすことができることで大きな注目を集めました。それを受けて、GPTs の科学的理解を調べる研究や GPTs をそのまま用いて科学的タスクの遂行を試みる研究が増加しました \cite{bordt2023chatgpt,white2022large}。特に、これらの GPTs をつなぎ合わせてパイプラインや自律的エージェントを構成する取り組みが始まると、科学的タスクをこのようなエージェントに実施させる研究も現れました \cite{wang2023survey}。

% それが科学的な基盤モデルを作る試みへとつながりました \cite{taylor2022galactica}。

さまざまな分野の知識発見のあらゆる過程の自動化に大規模言語モデルを用いる様々な試みがあります。研究分野で言えば、化学 \cite{bran2023chemcrow,jablonka202314,white2022large,hatakeyama2023prompt,jablonka202314,guo2023can} や材料科学 \cite{jablonka202314,jablonka202314,xie2023large,kang2023chatmof,merchant2023scaling} などの自然科学への適用 \cite{ai4science2023impact,bran2023chemcrow,white2022large,hatakeyama2023prompt,jablonka202314,guo2023can,boiko2023emergent,charness2023generation,qin2023gpt,zheng2023large,qian2023can,wysocka2023large,lee2023benefits,nori2023capabilities,wang2023large,singhal2023large,yang2022gatortron,deng2023learning,merchant2023scaling}、数学や工学への応用 \cite{wu2023empirical,pursnani2023performance,zheng2023can,zhang2023automl}
社会科学への応用 \cite{wang2023survey,bail2023can,ziems2023can,park2023generative,horton2023large} などがあります。また、全ての研究分野に共通するものとして、学術文書の処理を GPT に行わせる取り組みもあります \cite{alzaabi2023chatgpt}。 Some of them include paper processing \cite{elicit,scispace,van2023chatgpt}, paper search \cite{elicit,scispace}, paper writing \cite{transformer2022can}, abstract generation \cite{gao2023comparing}, literature review generation \cite{aydin2022openai}, and peer review \cite{wexin2023can,liu2023reviewergpt,robertson2023gpt4,hosseini2023fighting}. 

特に、自動化する研究の過程としては、仮説の生成過程はもちろんこと、研究課題やリサーチクエスションの発見や生成 \cite{oppenlaender2023mapping,lahat2023evaluating} をさせるものから、実験も含めた研究の全過程に使用するような例まで登場しました \cite{boiko2023emergent,charness2023generation,qin2023gpt}。

このような言語モデルの革新的な性能と急激な科学研究への広まりは、科学コミュニティに対して多くの議論を呼びました \cite{birhane2023science}。

% 参考
% 自律的エージェントを科学へ応用 \cite{wang2023survey}.

% LLMs are used for scholarly document processing \cite{alzaabi2023chatgpt}, some of which include paper processing \cite{elicit,scispace,van2023chatgpt}, paper search \cite{elicit,scispace}, paper writing \cite{transformer2022can}, abstract generation \cite{gao2023comparing}, literature review generation \cite{aydin2022openai}, and peer review \cite{wexin2023can,liu2023reviewergpt,robertson2023gpt4,hosseini2023fighting}. 

% 研究課題の発見 \cite{oppenlaender2023mapping}、リサーチクエスチョンの生成 \cite{lahat2023evaluating}

% LLM の科学的知識の理解を調べる研究があります。コンピューター科学
% \cite{bordt2023chatgpt} 化学 \cite{white2022large}

% 社会科学への応用があります \cite{wang2023survey,bail2023can,ziems2023can,park2023generative,horton2023large,williams2023algorithmic}.

% 科学の基盤モデルを作る試み \cite{taylor2022galactica}

% molecular data,  protein language models,  DNA and RNA などの事前学習済みモデルを紹介してる \cite{ai4science2023impact}

% generalist model \cite{tu2023towards}

% 科学に 特化した言語モデル\cite{beltagy2019scibert,singh2022scirepeval,nadkarni2021scientific,cohan2020specter,gupta2022matscibert,taylor2022galactica,azerbayev2023llemma}. 

% マルチモーダルな基盤モデル \cite{singhal2023towards,takeda2023foundation,nguyen2023climax}.

% 科学的言語モデル(事前学習済みモデル) \cite{xie2023darwin,luo2022biogpt}

% 科学実験 \cite{boiko2023emergent,charness2023generation,qin2023gpt}.

% 化学 \cite{bran2023chemcrow,jablonka202314,white2022large,hatakeyama2023prompt,jablonka202314,guo2023can}

% 材料科学 \cite{jablonka202314,jablonka202314,xie2023large,kang2023chatmof}

% サーベイ \cite{wang2023survey}

% molecular property prediction \cite{zheng2023large,qian2023can} (Physiology, Biophysics, Physical Chemistry, and Quantum Mechanics)

% buisiness and management \cite{williams2023algorithmic}

% biomedicine, medical science \cite{wysocka2023large,lee2023benefits,nori2023capabilities,wang2023large,singhal2023large,yang2022gatortron} (not factual base)

% 地球科学 \cite{deng2023learning}

% 数学 \cite{wu2023empirical}

% アシスタント \cite{lubiana2023ten}

% エンジニアリング \cite{pursnani2023performance}

% 網羅的 by microsoft \cite{ai4science2023impact}

% 機械学習 \cite{zheng2023can,zhang2023automl}

% Instrut Tuning for Science \cite{horawalavithana2023scitune}

% Science in the age of large language models \cite{birhane2023science}

\subsubsection{Incorporating Scientific Knowledge}

これまで自律的に研究する AI について議論してきましたが、人間においても完全なゼロから研究をするのはほぼ不可能です。人間も事前に研究したい分野の勉強をしてその分野の基本的な知識を獲得した上で、研究をします。したがって、AI も研究をする前にすでにこの世界に生み出されている基礎的な知識に関しては事前に獲得していることを想定するのが妥当でしょう。

The group of studies that incorporate biases or scientific knowledge to handle scientific data on AI is called \textit{physics-informed machine learning} \cite{karniadakis2021physics}. The incorporated biases include the ability to handle partial differential equations (PDE), symmetry, and intuitive physics \cite{hao2022physics}. Karniadakis et al. \cite{karniadakis2021physics} and Hao et al. \cite{hao2022physics} systematically organize existing research in this field. 

Efforts to impart scientific knowledge through pre-training are also common. In recent years, with the advancement of large-scale neural networks, there also have been efforts to inject scientific knowledge to models by pre-training with large corpus of scientific texts \cite{taylor2022galactica,beltagy2019scibert}, and with multimodal data \cite{singhal2023towards}.

また、私たち人間は一度基礎を学習した後も研究と並行して勉強をすることで、自分の科学的知識をアップデートしています。事前学習や機能バイアスによる知識の埋め込みや、推論時の検索だけでなく、機械も同様に研究と並行して勉強をすることで継続的に自らの知識を更新していくことが重要でしょう。

仮説のアップデートの話

\subsubsection{Scientific Understanding}

\subsubsection{Autonomy, Generality, and Open-Endedness}
繰り返し述べてきたように、問いの生成や仮説の生成や仮説の検証を機械に行わせる試みはすでに存在しています。難しいのは、可能な限り人間が介入せずに自律的にこれらを機械に実行させることです。closed-loop research automation は非常に自律性の高い研究の自動化の試みですが、このような場合においても,  it is said that full automation of all processes of science is yet to be realized \cite{zenil2023,coley2020autonomous,coley2020autonomousII}. 

One of the biggest issues is to let machine generate the objectives, problems, and questions of research by themselves \cite{coley2020autonomousII}. なぜなら、繰り返し述べてきたように、目的や問いまでも機械に自律的に生成させる場合、それに対応して機械は自律的に適切な仮説を生成したり検証したりしなければならないからです。したがって、機械はさまざまな場合に対応できる汎用的な方法で仮説生成や検証を実行なければなりません。

このような状況では人間が事前に具体的な仮説生成の方法や仮説の候補や必要な情報を与えることができません。したがって、機械は人間が置かれているのと同じようなオープンエンドな環境から適切に情報を引き出して仮説生成や検証を実行しなければなりません。研究は outer world
of scientific information から情報検索をしてエージェントの inner cognitive world でこれらを処理する過程だと理解できますが \cite{hope2022computational}、この outer world の範囲を可能な限り人間が制約せず、その outer world から何を取得するかは可能な限りエージェント自身に決定させるイメージです。

このような自律性の問題を鑑みて、機械にどの程度の自律性を求めるのか、どの程度の制約であれば機械の潜在能力を過剰に抑えつけることなく自律的な仮説生成や検証を行わせられるのかは、オープンクエスチョンです。

% \subsubsection{Open-Endedness}
% One of the major challenges that can be a common issue for any process, as pointed out in prior research as well  \cite{coley2020autonomousII}, is how to execute these tasks in open-ended situations. For instance, in the automation of experiments, robots use experimental equipment selected, prepared, and set up by humans. However, humans do these tasks from scratch with their own hands. Humans do not use a given corpus of papers; instead, they search for and use them on their own. Candidates for hypotheses are not explicitly provided; humans begin by identifying potential hypotheses. Even when formulating questions that serve a particular goal, humans set that goal themselves. 

% In many cases in research automation, these elements are pre-determined by humans. How to let machines autonomously perform these tasks starting only with the same initial information given to humans is crucial in realizing an autonomous artificial researcher. Moreover, having this kind of freedom is essential to achieving a general artificial researcher as well. This is because if we impose constraints on AI to research only within specific research questions or hypothesis spaces defined by humans, it cannot become an AI capable of conducting arbitrary research. Therefore, a significant challenge is how to make AI acquire complex foundational skills and fundamental reasoning abilities to realize these capacities.

% \subsubsection{Generality}
% 汎用性がなぜいるかなどの説明?上の定義はそもそもそういうモチベーションから始まってるので、ここでわざわざ書かなくても良い。

% 特定の研究課題に特化すれば、問いの構築、仮説の生成、仮説の検証を実行できるような機械はすでに開発されています。これまでのセクションでこれらの概念の定義を改めて検討したのは、研究ができる AI とは特定の研究課題に特化した方法でこれらを実行するのではなく、どのような研究対象に対してもこれらを自律的に遂行できる AI であることが望まれるからです。このような汎用的な研究ができる AI をどのように実現するかというのが課題です?

% As we have introduced so far, there are several attempts to automate research through versatile approaches. For example, the automatic generation and discovery of hypotheses \cite{kang2022augmenting,chan2018solvent,wang2023learning,xu2023exploring} or questions \cite{lahat2023evaluating,liu2023creative,oppenlaender2023mapping,surita2020can} from research papers is an approach that can automate research across a wide range of academic fields, from the humanities to natural sciences. Additionally, efforts to create general-purpose robots  \cite{yachie2017robotic}, foundation or generalist models \cite{singhal2023towards,taylor2022galactica}, or to incorporate the inductive bias for understanding physics can be considered initiatives towards general-purpose system.

% However, most automation studies target specific challenges in particular research fields. It seems that the goal of creating an AI capable of conducting any research is not receiving much attention. As explained in Chapter 2, in order to create such intelligence, it would be necessary to understand the high level concept of research, question construction, hypothesis formulation, and hypothesis verification, and to be able to execute them appropriately depending on the subject at hand.

% \subsubsection{Autonomy}
% As mentioned in the previous section, there are several attempts to automate the entire research process. A seminal early works are Adam \cite{king2004functional}, and Eve \cite{williams2015cheaper}. These are closed-loop scientific discovery systems that autonomously execute everything from hypothesis generation to research planning, based on logic AI and robotics. Furthermore, there is closed-loop automation in some research on automation using scientific workflows \cite{gil2017towards}. These are highly autonomous research automation. Recently, there has also been a movement to create autonomous agents based on language models to tackle scientific problems \cite{wang2023survey}.

% However, most of studies on research automation have targeted  only specific tasks within a sub-process of the entire research process. For example, symbolic regression focuses on automating hypothesis generation, while experiment automation pertains to data generation for hypothesis creation and testing.

% Furthermore, even in the case of closed-loop research automation,  full automation of all processes of science is yet to be realized \cite{zenil2023,coley2020autonomous,coley2020autonomousII}. One of the biggest issues is that the objectives, problems, and questions of research are given by humans \cite{coley2020autonomousII}. We discussed in Chapter 3 that there are not many studies that have attempted to automate the construction of questions. And while we pointed out in Chapter 2 that automating the construction of questions can lead to an infinite regress from the perspective of autonomy, specifying the high-level goals underlying the questions remains a significant challenge \cite{coley2020autonomousII}. Also, even after the construction of the question, there is the issue that humans are providing machines with a search space that is far more restricted than what humans themselves are given. For example, generating hypotheses from an open-ended hypothesis space has not yet been realized \cite{zenil2023,coley2020autonomousII}, and to put it in extreme terms, while humans might prepare experimental equipment from scratch, machines are given those as a given. 

% To begin with, the realization of a fully autonomous artificial intelligence is still one of the major goals yet to be achieved, not just in research. Therefore, a lot more foundational research will likely be needed to accomplish this.

% Many automation studies primarily target specific tasks within a research process. For example, symbolic regression focuses on automating hypothesis generation, while experiment automation pertains to data generation for hypothesis creation and testing. However, as mentioned earlier, some studies, such as automated research workflows and self-driving labs, aim to automate the entire research process. 

% Coley et al. discuss the advancements in automating scientific discoveries in chemistry \cite{coley2020autonomous}. Their discussion is not limited to automating chemistry but extends to the broader context of automation of science. The paper delves into the insightful topics of automated discovery, including defining scientific discoveries and criteria for assessing their autonomy. In the subsequent paper, Coley et al. organized the existing automation studies based on which processes of scientific workflow are automated \cite{coley2020autonomousII}. In that paper, they point out several challenges for science automation, ranging from dataset handling to physical and computational autonomous validation. 

% Autonomous agent for scientific problem \cite{wang2023survey}

% , and in recent years, there are attempts to automate the entire process using language models.

% \subsubsection{Generality}


% \subsubsection{Scientific Workflow}
% The workflow whose process is a research process, tasks are research tasks that receive and output scientific data is called \textit{scientific workflow} \cite{ludascher2009scientific}. By running scientific workflow systems, research process is automatically executed.  Amidst the flourishing of computational and data-driven sciences with the advancement of computers, these efforts seem to have been conceived to better manage in silico experiments \cite{liew2016scientific}. 

% Among such initiatives, just like in the case of laboratory automation, there are efforts that have achieved closed-loop automation, which completely autonomously carries out the cycle of hypothesis generation and verification \cite{gil2017towards}. Gil has presented how machine learning can be incorporated into this scientific workflow \cite{gil2022will}. She also presents a perspective on what kind of AI should be developed in order for it to become a good partner for researchers.

% Additionally, there are attempts to design workflows that are reusable beyond individual workflows, in other words, workflows with high generality \cite{hardisty2020canonical}.

% どのような研究対象に対しても問いを立て、仮説を生成し、仮説を検証できるような人工知能を作るためには、これらを汎用的な方法で実行する必要があります。このように

% \subsection{Challenges}

 % \subsection{General and Autonomous Question Construction, Hypothesis Generation, and Hypothesis Verification}

 % In Chapter 2, we explained that in order to create an AI capable of conducting any research, it is deemed necessary to realize the formulation of questions, generation of hypotheses, and validation of these hypotheses as a combination of universal skills applicable to all research. In this section, we will revisit and organize the potential challenges in achieving an AI that can conduct each of these processes.


% \subsubsection{The Difficulty of General Hypothesis Verification}
% I recognize that the the way to verify a hypothesis is strikingly diverse, as it can significantly differ depending on the subject of research. For instance, if one wishes to study the behavior of rat, they may need to train the rat. On the other hand, if you want to test a theory of particle physics, you may have to construct and run a huge particle accelerator. In the field of history, the existence of historical records might serve as evidence, while in mathematics, the verification process revolves around the proofs themselves. Due to this high degree of flexibility, hypothesis verification can be the most challenging aspect to automate for realizing a ``general'' artificial researcher.

% \subsubsection{Feedback from Verification}
% In the first place, the act of verification is a highly challenging process. Firstly, as mentioned earlier, inductive approaches cannot verify hypotheses in the same way as deductive reasoning. Also, I notice that hypothetico-deductive method, which involves verifying claims derived from a hypothesis to confirm its validity, is still widely used today. However, verifying a deduced claim does not support the hypothesis because there may be many hypothesis that can result in the same deduced claim. In response to these, Karl Popper proposed that while hypotheses cannot be confirmed, they can be falsified \cite{sep-scientific-method}. However, in practice, the verification of hypotheses involves implicitly relying on numerous auxiliary hypotheses. When using experimental apparatus, it requires many assumptions to trust them. Even when an experiment fails, determining whether the hypothesis was incorrect or the experiment itself was flawed is not as straightforward as one might think. Thus, there is inherent uncertainty in attributing the results of verification to a specific cause \cite{chalmers2013thing,sep-physics-experiment,sep-scientific-underdetermination} as I have discussed in the section of hypothesis generation. Furthermore, all reasoning and observational evidence are inevitably influenced by some form of theories, individuals, or societal factors \cite{sep-science-theory-observation}. Therefore, it is necessary to carefully examine them to ensure that they are not distorted by unintended influences.


% \subsubsection{Question Construction}



% Realizing AI that construct a ``good'' question in a generic way is challenging. As discussed in Section \ref{section:the-relativity-of-knowledge-production-to-society}, research is relative to society and different criteria can be considered for what makes a ``good'' question. Thus, some human perspective on the ``goodness'' of a question must be incorporated. We need to discuss what we consider good, what we should prioritize, and how to incorporate the value to AI.

% \textcolor{red}{TODO}

% Moreover, determining inputs to the question construction module is not trivial. In hypothesis generation, the question is the primary input, whereas in verification, it's the hypothesis. However, question formation take any input. Once you seriously try to identify the origin of question, you will encounter infinite regress. This is a unique problem that arises when aiming for a general-purpose and autonomous artificial researcher. This is because the issue revolves around how much input can be assumed while still being considered autonomous, given that it can potentially take any input.

% \cite{wang2023skillqg}

% neural question generation \cite{pan2019recent}


% \subsubsection{Hypothesis Generation}



% \subsubsection{Hypothesis Verification}



% Creating AI that autonomously verifies hypotheses is challenging. While current models can mimic human verification, truly understanding the verification strategy demands more work. Sometimes, they even need to devise the verification measure themselves.

% The biggest challenge for autonomous verification is the need to freely move around in the real world or within a computer, and to manipulate objects within that world at will. We believe this to be one of the greatest barriers to full research automation. Laboratory automation have attempted to address this challenge in real world by developing robots. We will discuss the case within the computer in Section \ref{section:behaviour-inside-the-computer}.

% \subsubsection{AI Capable of Peer Review}

% Given the difficulty of these challenges, automating peer review could be a strategic starting point. This is because peer review is a universal process across diverse research fields and it assesses the validity and quality of problems, hypotheses, and verification methods, which is easier than generating them. Despite some progress, full automation is still elusive \cite{yuan2022can,schulz2022future}.


% \subsection{Behaviour inside the Computer}
% \label{section:behaviour-inside-the-computer}


% Lab Notebook?
% \subsection{Dataset of Research Process}
% It is important to establish the necessary infrastructure for research automation. The two pillars of research automation are the development of basic models that incorporate academic knowledge and the construction of data sets. The development of an infrastructure model that incorporates scientific knowledge has already been proposed in many places and is actually under development, so I will not emphasize its necessity here again. Also, regarding data sets, the construction of data sets for the acquisition of scientific knowledge has been done in various places as well, so I will not emphasize that here either.

% Instead, we propose here to construct a research process dataset. A research process dataset is behavioral log data that incorporates all possible tasks throughout the entire process of the study, from start to finish. Ideally, individual tasks should be labeled as to whether they correspond to question construction, hypothesis generation, or hypothesis testing. We believe that building such a data set is important because, as explained in the Literacy section of Chapter 2, the current paper is not a data log of the entire research process. This makes it difficult to be data-driven and end-to-end learning how to do research itself. I believe that building a research process will help solve these problems and increase the likelihood of more flexible intelligent agents.

% However, building a dataset of the research process seems daunting. This is because researchers who do not currently keep research logs would have to go to the trouble of recording their research process.\footnote{
% In an experimental laboratory in the natural sciences, it is common to take research notes, so it may not be that difficult to record more detailed processes as an extension of this practice. However, in the machine learning field, the culture of taking research notes does not seem to be that common. (We think it is common to keep logs of experiments, but it seems to be rare to describe the details, for example, where and how the data was obtained.) 
% } Therefore, it seems necessary to devise a way to make it easier for researchers to keep logs. It may be to manage the research process on GitHub, or to take research notes as in natural science research, but it is important to discuss how to achieve these things.

% Being able to construct a dataset of the research process would be ideal, but may not be immediately feasible. As an alternative, it seems important to create a dataset designed to automate question construction, hypothesis generation, and hypothesis testing. At its simplest, one might start by building a dataset of papers labeled with the parts that correspond to the question, hypothesis, and test, respectively. This would be a relatively simple but important step in achieving a generic artificial researcher.

% Alternatively, instruction tuning could be done by viewing question construction, hypothesis generation, and hypothesis testing as tasks, respectively. This would produce a language model that can execute question construction, hypothesis generation, and hypothesis testing with greater fidelity. This could be the foundation for a general-purpose, autonomous artificial researcher.





% In the medium to long term, it's essential to devise ways to ensure that AI doesn't engage in research that could be dangerous to humans. 

% In the long term, we must contemplate how to construct a knowledge system that are mutually translatable between human society and AI society. While these issues may not arise in the short term, it's crucial to engage in discussions now, looking towards the long-term future.

% \subsubsection{Understanding}

% Extensive discourse transpires concerning scientific discoveries. Yet, discussions pertaining to scientific comprehension remain relatively unexplored. Krenn et al. delve into the conundrum of what it entails for a machine learning agent to not only unearth scientific knowledge but also to comprehend it \cite{krenn2022scientific}. They adopt a human-centric stance, positing that an agent's ability to offer explanations comprehensible to human scientists signifies the existence of its scientific understanding.

% sun-rise \cite{leslie2023does}

\section{Ideas for Prototyping}
Realizing an intelligent agent that can conduct research is an exceptionally challenging goal that will likely take a long time to achieve. The challenges discussed so far are just the tip of the iceberg found in speculative discussions, and there are undoubtedly many more yet to be identified critical issues. Therefore, it seems crucial to start by identifying and addressing unknown challenges in the first place. A good starting point might be to develop a simplified prototype of an AI capable of research to explore the challenges for our goal in the process of prototyping. In this section, I would like to discuss speculatively and briefly what could be considered as such prototyping.

\subsection{Prototyping Agents that Conduct Research}
% まず初めに、研究をするエージェントのプロトタイプとしてどのようなものが考えられるか検討するところから始めましょう。
Let me start by considering what might be a prototype of an agent for conducting research.

\subsubsection{Requirements for Prototype}
As discussed in Section \ref{section-question-hypothesis-verification}, research, I believe, consists of constructing questions, generating hypotheses, and verifying these hypotheses. Thus, it seems appropriate for this prototype to consist of these functions as modules. The question construction module takes any input and produces a question. The hypothesis generation module takes this question as input and produces a hypothesis. The hypothesis verification module takes the hypothesis as input and provides verification results. The prototype would conduct research by flexibly combining these modules at various levels.

For the prototype agent to be autonomous, human design, implementation, and intervention should be minimized. Consequently, each module, aside from receiving minimal inputs, should autonomously gather information from the open-ended world, which humans interact with to get information for research. That is, the agent should interact with the physical world or the digital realm to get information necessary for research, as humans do.

% For instance, aside from the question input from the question construction module, the hypothesis generation module shouldn't have predetermined inputs.

Furthermore, for the system to be general, the internal workings of each module mustn't depend too much on specific research topics. For example, if the verification method is an experimentation for a specific physics research, it can't be used for psychological research. The human designed inner workings of each module should be as minimal as possible, limited to only what is necessary for the function of each module. This is akin to an abstract class of research.

Creating a system that meets both autonomy and generality requirements while properly constructing questions, generating hypotheses, and verifying them only with this abstract class is infeasible, even in simpler scenarios. Hence, it might be necessary to impose some constraints on this abstract framework. Discussing the extent of these constraints, why they're needed, and how they can be eliminated will help elucidate the challenges in realizing a autonomous research agent. By identifying these challenges and turning them into research themes, and then advancing foundational research, I believe we can more efficiently approach the goal. In the following, I will list up some candidates for potential constraints to provide a first step.

\subsubsection{Candidate Constraints in Prototyping}

In Section \ref{section-what-is-research}, I discussed the view that research can be considered as updating beliefs. I also discussed the possibility of autonomously constructing verification from its foundational concepts and autonomously contemplating the value of questions when conducting autonomous research. However, these ideas are too visionary and challenging to expect immediate, meaningful results for humans by prototyping. Therefore, it seems desirable as a prototype to aim for agents that can master the values system and verification methods humans have built so far.

% \subsubsection{Grand Goal is Given}
As said in Section \ref{section-question-construction}, constructing questions from open-ended situations is a too challenging task where even where to start from is not evident. Therefore, it seems prudent to start by determining in advance what the input for constructing the question should be, rather than assuming the unrestricted information sources. A candidate for the input is a high-level goal since it is 
assumed in many studies. Particularly, it would be desirable as a first step to provide high-level goals that are recognized as objectives in specific research fields, such as machine learning.

% As stated in Section \ref{section-question-construction}, not all research questions are necessarily constructed to achieve some grand goal. However, many studies build their questions by breaking down such goal. Therefore, if research can be conducted to construct questions from overarching goals, then a vast majority of human-conducted research, and notably ``meaningful'' research, potentially becomes feasible.

% \subsubsection{Research is Completed Entirely within a Computer}
One of the biggest bottlenecks in realizing a fully autonomous research agent is the necessity for excellence at complex low-level actions, as discussed in Section \ref{section-countless-seemingly-unrelated-operations-to-knowledge-production}. Especially, developing a robot capable of acting freely in the physical world like humans is an extremely challenging task. Therefore, for prototyping purposes, it seems reasonable to first consider research that does not require interaction with physical world but is confined within a computer. Of course, realizing an agent that can freely operate within a computer is also a very challenging issue, but it seems more feasible than an agent freely operating in the physical world. The purpose of prototyping is to materialize the concept, even if it's rudimentary, and identify challenges. Therefore, it seems desirable for prototyping to first limit the target environment to within a computer and wait for the advancement of foundational research for the realization of free activity in the physical world.

The examples mentioned here are merely a few ideas and are neither absolute nor comprehensive. Instead, it seems important in prototyping to discuss to what extent and what kind of constraints should be applied. It is expected that more appropriate constraints will become apparent as such discussions deepen in the future.

% \subsubsection{Verification is Limited to Humans'}
% The issue that allowing machines to conduct research entirely autonomously could result in constructing knowledge systems meaningless to humans arises when we try to let AI construct even verification from scratch. This is necessary if we want to truly say an AI can verify on its own. On the other hand, many people probably have no interest in generating knowledge that is meaningless to humanity. In the first place, even if such a thing were realized, humans might not be able to evaluate whether the AI has truly constructed a meaningful knowledge system for them. Also, I don't think human researchers always understand or construct verification from first principles. Therefore, it might seem harsh to demand these of AI. So, it seems preferable to first ensure that AI understands and can always use the verification methods that humans use. Specifically, we aim to make sure AI can always proficiently use experiments, statistical hypothesis testing, proofs, etc. By doing so, I believe there's a possibility to realize an AI that conducts highly generalized, autonomous research that is meaningful to humans.

% \textcolor{red}{参考}
% There are ongoing initiatives to enable language models to operate browsers \cite{nakano2021webgpt,act1}. While full browser operations might seem ambitious, there are already endeavors to allow language models to conduct searches \cite{mialon2023augmented}. If we achieve browser automation, it will greatly advance research automation involving web operations. Moreover, efforts like the open interpreter \cite{openinterpreter} aim to automate any computer action. This direction holds promise for automating all research confined within a computer. Although these studies are gaining traction in the machine learning domain, they're not always linked to research automation. We advocate recognizing this as a pivotal challenge in the realm of research automation.

\subsubsection{Implementing Each Module with Large Language Models}

Considering the remarkable performance of large language models and the necessity for generality, it would be inevitable to instantiate each module as a large language model. In reality, as stated in previous sections, studies to construct research pipeline by LLM pipeline have emerged. I believe we should start from prototyping agents capable of conducting research as such LLM pipelines. Particularly, I believe that we should create an autonomous research agent, in line with attempts to realize autonomous agents using language models \cite{wang2023survey,xi2023rise}.

Here is one idea modeled after a typical autonomous agent. The research agent start from formulating a question given a high-level goal input by human. Once the question is posed, the agent then autonomously generates hypotheses that could answer this question and ultimately verifies them. Once the final verification results are produced, they are interpreted in light of the original objective and research question, leading to the generation of the next question. However, as described below, the agent will iteratively and hierarchically repeat the processes of question construction, hypothesis generation, and hypothesis verification to execute each of these subprocesses.
% まず初めに、人間が目標を入力して、それを元に問いを立てるところから出発します。エージェントが問いを立てたら、次にエージェントは自動的にその問いの答えの候補となる仮説を生成し、最終的にこれを検証します。最終的な検証結果が出力されたらそれを当初の目的とリサーチクエスチョンに照らして解釈し、次の問いの生成などに向かいます。

The agent is assumed to perform essentially three actions: 1. formulating questions, 2. determining whether the task is completed, 3. verifying hypotheses, and 4. executing any low-level action on a computer. The processes of formulating questions, generating hypotheses, and verifying them are primarily realized by performing actions on a computer.

If the agent chooses to generate a question, it temporarily suspends the current task, such as hypothesis verification, and always starts generating a hypothesis for that question. Once the generation of hypotheses for that question is deemed complete, the agent chooses whether to verify them or not, and after verification, it updates the hypotheses based on the results. Regardless of whether the hypotheses were verified, the agent then resumes the higher-level process that was previously interrupted, using the outcome of the low-level process. In this manner, the agent repeats the lower-level question construction, hypothesis generation, and verification until the highest-level hypothesis is generated. However, when the highest-level hypothesis is generated, the agent always starts verifying that hypothesis.
% エージェントは大きく分けて、1. 問いを立てる、2. タスクが終了したかを判定する、3. コンピュータ上の任意のアクションを実行する、の基本的には3つの行動を取ると想定します。そして、この問いを立てる過程や仮説を生成するや検証をする過程は基本的にはコンピュータ上の任意のアクションを取ることで実現します。ただし、エージェントが問いを生成することを選択した場合、現在取り組んでいる仮説生成などのタスクを一時中断し、新しくその問いに対する仮説を生成することを開始します。その問いに対する仮説生成が終了したと判断されたら、それを検証するかしないかを選択し、検証を終えたら検証結果を受けて仮説を更新します。検証をしたにせよしなかったにせよ、これらによって生成された仮説の出力結果を受けて、先ほど中断していた上位の仮説生成の過程を再開します。このようにして、最上位の仮説が生成されるまで下位の問いの構築と仮説生成及び検証を繰り返します。ただし、最上位の仮説が生成された場合だけは、必ずその仮説の検証を開始します。

To ensure this system is so general that it can can adapt to many types of research questions, prompts given to these language models should consist only of general instructions. For instance, an instruction like ``generate a hypothesis for the following question'' is general because it can be used for any research questions. Naturally, merely providing such instructions won't automatically yield research outcome from scratch, so there may be a need to provide additional as general as possible auxiliary instruction. 

For open-ended operations within a computer space, ideally, the LLMs should only be given access to nearly all operations on the computer, akin to what an open interpreter \cite{openinterpreter} does. Minimal access to web browsers, search engines, or shells might be acceptable, but provision of custom corpora or predefined hypothesis spaces should be avoided. If research can be autonomously conducted under such conditions, it would indeed signify that the system is capable of independent research.

% I worked with a research group attempting to automate research, and together, we created a simple mock-up expressing such a concept \textcolor{red}{CITATION}. We assumed a given question and examined how much GPT-4 could generate and validate hypotheses using only the most general prompts possible. While this initiative is still in its early stages and is limited to very basic problem settings, based on our findings here, I hope to eventually develop a system that more closely resembles human research capabilities.

% 言語モデルすごいし、言語モデルでやってくの考えたいよね。汎用的なのを言語モデルでやってみるならどんなになるだろう?→できるだけ汎用的な指示ということになるよね!みたいな。言語モデルが研究ができるというのはどう調べるか←汎用的な指示というのが一つのテストとして使う?

\subsubsection{AI that Conducts Machine Learning Research}
Given a high-level goal in a specific research field, there is room for choice regarding which research field's what type of objectives to provide. It seems desirable for this research field to be suitable for prototyping and to easily meet the aforementioned constraints.

I believe that machine learning research is good for such prototyping. Firstly, some machine learning research can be fully completed on a computer, meeting the aforementioned constraints. Secondly, machine learning research has a shorter research cycle compared to other fields, allowing for faster feedback cycles. Thirdly, machine learning technology is essential for the realization of research-capable and currently serves as a foundational technology in many research fields. Thus, automation of it will advance our original goal itself, while contributing the automation across many research fields at the same time. Finally, there already have been efforts for automation, such as AutoML \cite{hutter2019automated,bischl2023hyperparameter,lindauer2020best,white2023neural} and MLOps \cite{kreuzberger2023machine}. Especially in recent years, there have been attempts to perform these tasks using language models \cite{vijay2023prompt,zheng2023can}. These accumulated achievements will likely further assist in creating agents that can conduct machine learning research.


% We believe that automating machine learning research may be suitable to start with in terms of these decision axes. First, let's discuss bootstrapping. Many of the challenges to automating research will be how to get machines to acquire from experience what they are currently hardcoding and doing in the real world. Learning from experience is exactly what machine learning does, and in this sense, many of the challenges in realizing autonomous artificial researchers can be formulated as machine learning research challenges.

% Next, let us discuss feasibility. In the first place, many machine learning studies are conducted entirely on computers. As mentioned earlier, the greatest difficulty in achieving general automation lies in the interaction with the real world. Technologies related to real-world interaction are used for hypothesis verification rather than the verification itself. This requires advancements in robotics research. Therefore, to pursue the automation of the entire research process, it may be best to set aside fields that require interaction with the real world and initially focus on automating research that can be done solely on PCs. 

% Also, many attempts to automate machine learning processes have already been made. For example, in MLOps, various pipelines for automating tasks such as experiment management and training in machine learning have been proposed and put into practical use. AutoML, which is a field of machine learning research, has also produced numerous innovations in automating many of the tasks involved in machine learning. Moreover, the culture of machine learning and related engineering fields already has a wealth of knowledge and insights regarding automation. This means that we do not have to devote many resources to automating research domain-specific tasks. This allows us to focus on more essential questions in our quest to become general-purpose artificial researchers, such as ``How do we allow people to test hypotheses?'' Furthermore, many studies in machine learning and related research areas are open-source. Consequently, it is considered easier to retrieve information from papers compared to other fields. 

% Finally, we would like to discuss the impact on other sectors. As you are already aware, many research fields are currently using machine learning technologies. The AI for Science initiative, which aims to automate the scientific field, also uses machine learning technology. Therefore, the automation of machine learning research and better knowledge production will accelerate all of these efforts. For these reasons, we believe it is a good approach to start by automating a specific research project in the machine learning domain.

% \subsubsection{What Type of Research in Machine Learning Should We Start with and How?}
% There are many different types of machine learning research, but where should we start with automation? Even though we aim to automate the construction of questions, what types of research should we guide them to do?

% It seems that a example of the research appropriate as one a starting point is that on the zero-shot prompt proposal. First, the hypothesis (or proposal) in this study is the specific text of the prompt. This is much less expensive to implement, whereas many empirical machine learning proposals require composing an algorithm or architecture. Validation requires the automation of the task of preparing existing data and models, and this is certainly a difficult task. However, this is an extremely common task in machine learning research and is not unique to this research project. The ability to automate this task would benefit a significant amount of machine learning research. The validation criterion is also generic, as it is a typical validation criterion that compares the proposed group with the control group. We think we will first build a prototype by adjusting the LLM prompts, and the fact that there is no strict mathematical or logical manipulation, which language models are not good at, is another aspect that makes this research easy to do.

% With this in mind, one research project in which the author of this paper is participating is in the process of building a prototype of the pipeline of research for the prompt proposal \footnote{
% Link to the pipeline: \href{https://github.com/t46/mock-pipeline}{https://github.com/t46/mock-pipeline}
% }. It is currently still in the pilot stage and some parts are hard-coded, but will be updated as needed.

% What we have described here is just one example and a suggestion. We hope that more similar initiatives will emerge in other studies.


% We propose to start by building a research pipeline, connecting the modules of the knowledge production system. The research pipeline is a software system that takes input and generates knowledge as output, encompassing the sequence of processes involved in research. Since this process does not require human intervention, it can be considered as an autonomous research system. We propose this system to be composed of the sub-processes of ``question construction,'' ``hypothesis generation,'' and ``hypothesis verification.'' This creates a general system that is potentially applicable to any research. These sub-processes can be likened to abstract classes in programming. Each process automatically formulates appropriate questions, generates hypotheses, and performs verification based on the input. 

% Initially, we will assume a specific research problem, and this research problem can be a simple one. And the inner workings of detailed hypothesis testing and hypothesis generation can be guided (but not hard-coded) to achieve the desired results. Anyway, the high-level concept is to create the minimum necessary to automatically execute each process of question construction, hypothesis generation, and hypothesis testing. Then, by gradually making the contents of each module autonomous and gradually loosening the restrictions on the research problem, we will lead to a general-purpose and autonomous artificial researcher.

% \subsubsection{Guided but Not Hard-Coded}

% It is important to note, however, that even in the prototype stage, the internal implementation of question construction, hypothesis generation, hypothesis testing, etc., should be ``guided'' and not ``hard-coded'' as much as possible. In machine learning, induction is the process of adding words to the prompt that make it easier to output the expected answer, and hardcoding is the process of actually inserting the desired processing into the algorithm. For example, if a statistical hypothesis test is expected to be used as a means of testing a hypothesis, rather than having a human write a program that contains a process for performing a statistical hypothesis test, we would instead instruct to the machine learning model with prompting, ``Statistical hypothesis testing is one of the leading methods in verification. The hypothesis is A. Verify this.'' This is a very important point to emphasize.

% This is a very important point, so let me emphasize it. The reason this is important is that we do not want to automate a particular hypothesis testing process, but rather we want the machine to test the hypothesis itself. Only when you make sure that the machine decides on its own the appropriate verification method according to the hypothesis, will you be able to provide collateral evidence that the machine itself is able to verify the hypothesis. If this can be done not only in hypothesis testing, but also in all aspects of question construction and hypothesis generation, we can call it a prototype of a general-purpose, autonomous artificial researcher.

% \subsubsection{Why Pipeline?}

% There are two reasons why we think it is a good idea to start by building such a research process pipeline. The first is that this is one simplified representation of a generic and autonomous research system. Research is a very complex task, so when we try to automate validation, we inevitably focus on automating individual tasks. In addition, many research automation efforts are aimed at making things better, which often leads to a strong dependence on the domain, for example, in automating hypothesis generation. However, as emphasized above, what we want to achieve is not specific hypothesis generation or verification, but the ability to generate and verify hypotheses themselves. This system emphasizes that point, and once realized, it will be an example of what a general-purpose, autonomous artificial researcher could look like. The creation of such an example will serve as a guidepost for more people to become versatile and autonomous artificial researchers.

% Second, building on this would further clarify the challenges in achieving a general-purpose and autonomous artificial researcher. In this paper, we have discussed the challenges that would be necessary to realize a general-purpose, autonomous artificial researcher. However, we believe that this is a very difficult task and that there are many areas where we do not even know what the problems really are. Therefore, it is important to first identify what the problems are in the first place and where the uncertainties lie. When we move toward such a complex problem, we start with a simple example to understand the structure of the problem. For example, we build toy models in physics, concrete examples in mathematics, and prototypes in programming. The research process pipeline falls under such simple examples in autonomous artificial researchers. In the process of trying to achieve this, we will discover what are the bottlenecks and what are the essentials. In this way, I think it is important to build a research process pipeline in order to first increase the resolution of the problem and clarify the issues.

% \subsubsection{Where to Start?}
% It is advisable to start by representing a specific research as a pipeline. Initially, creating a concrete system helps clarify the actions involved in actual research and makes the specific challenges to be addressed more tangible. When dealing with projects with high uncertainty, it is crucial to concretize the problems to be solved. Specifically, the goal is to programmatically represent the actions that researchers perform as comprehensively as possible. It is acceptable to consider certain aspects as constants if their execution is too challenging to represent as a program. Then, running the system should reproduce the original research. The next step is to progressively automate the processes and constants provided by humans to enhance autonomy. Naturally, automating a specific research pipeline alone does not guarantee the development of an autonomous pipeline. However, this approach allows for the identification of research automation challenges and paves the way for their resolution through research and development. Importantly, it is essential to express individual tasks as components or sub-processes of question construction, hypothesis generation, or hypothesis verification. This is similar to inheriting an abstract class, ensuring that the automation of these processes is achieved as individual tasks are automated.

% In practice, it becomes evident that fully automating an entire research is highly challenging. Therefore, before automating specific research, it may be advantageous to start by creating simplified toy models and aiming to build systems that can execute them automatically. For example, certain parts that require obtaining and using a real dataset can be replaced with appropriately created sample datasets. The approach is similar to that of specific research pipelines, addressing research challenges while aiming to increase autonomy and generality.

% \textcolor{red}{Remarks about Autores PJ}

% \subsection{Which Field of Research to Start with?}
% We suggested that we might start by automating specific research areas and research tasks. So what research areas should we start with? As it turns out, we think it might be a good idea to start by automating machine learning research. There is, of course, a bias due to the fact that the authors of this paper are machine learning researchers and that we are writing this paper primarily for machine learning researchers, but aiming to automate machine learning research makes a lot more sense than that. To illustrate this, let me first introduce some of the perspectives involved in decision making in the area of research to be automated.

% \subsubsection{How to Choose Research Area}
% The first perspective is how much automation of that research area will help achieve a versatile and autonomous artificial researcher. We believe that it would be a good idea to automate research areas that would accelerate the automation of research. For example, if there is a problem to be solved in order to generate a hypothesis, the problem itself could be set as a research problem and automation of this research could be realized. If we can automate such a task, we have not only achieved our goal of automating the entire research process, but we have also solved the problem of research automation. Such bootstrapping will accelerate the automation of the research process and allow it to reach its goals more efficiently. \footnote{
% Inspired by the feedback from Hiroshi Yamakawa
% }

% The second aspect is feasibility. Since the objective of this project is to create a prototype, it is an important policy to start with the least difficult to realize. There are various levels of difficulty, but the most important is whether the level of difficulty is high, especially in areas other than those essential to the automation of a general-purpose research process. For example, it would be more feasible to generate hypotheses from papers now that language models have been developed than to actually construct and conduct a experiment and generate hypotheses from the observations, in the sense that it would be fully automated. Also, if we were to start with automation using a language model, it would be better not to include tasks that the language model is not good at. As emphasized above, it is better to start where it is as easy as possible here, because focusing on automation of the parts that depend on individual research tasks is not important for the goal of acquiring generalizable knowledge to achieve a general-purpose artificial researcher.

% The third aspect is whether the field has an impact on many studies. First of all, an area that has an impact on many studies is one that is used as an elemental technology in those studies. This would be appropriate as a research area to automate for general-purpose artificial researchers, in the sense that it is a general-purpose technology. And if areas that affect many studies can be automated, it will also accelerate the knowledge production of those studies. The efficiency of knowledge production for humanity as a whole will increase, and research automation projects will also benefit from the knowledge produced by them. It would also increase the population of people involved, which may lead more people to pay attention to research automation. This will spawn new flows of people, money, and knowledge, and as a result, projects are expected to move forward more quickly.


\subsection{Prototyping Agents that Conduct Peer Review}
In order to identify challenges for realizing agents capable of conducting research, aiming to automate the peer review process of academic papers might also be a good initial step. There are several reasons why it may be suitable. I'd like to explain them in the following section.

\subsubsection{Why Agents that Conduct Peer Review?}

First and foremost, the elements necessary for peer review are closely related to those required by a research-capable agent. This is because peer review involves judging essential aspects of research, such as the novelty of the question and the soundness of the verification.

Secondly, automating the peer review may be relatively less challenging than realizing an agent capable of conducting research. The reason is that while peer review only requires judging whether the necessary elements for research are present, to realize a research-capable agent, it's not just about judgment but also about being able to compose those elements. It seems desirable to start by tackling simpler problems first as a prototype to highlight challenges.

% To create an AI that can research, one must generate studies from scratch. In contrast, automating peer reviews involves automating the evaluation of already completed research. Typically, it is expected that generative tasks are easier to automate than classification or judgment tasks. As previously mentioned, understanding the essence of research is necessary for automating peer reviews. Hence, the challenges that arise during this automation process are expected to be beneficial for creating an AI capable of conducting research.

Thirdly, peer review is a widely practiced convention regardless of the research field. Therefore, the insight found during peer review automation can be beneficial to realize a general research agent. 
% Third, peer review is a discipline-agnostic practice. Therefore, automating it is expected to be important in gaining insights to realize a general-purpose artificial researcher. 

% Fourthly, compared to the general automation of other processes such as constructing questions, there is already a substantial accumulation of prior research on the automation of peer reviews.

% Fourth, there is more prior research in automating peer review than in automating generic hypothesis testing or hypothesis generation. Therefore, it is an area where the hurdles for starting a new research project are relatively low. 

Fourthly, peer review is mostly completed through text manipulation alone. There is no need for interactions with the physical world or the computer realm that are necessary for autonomously conducting research. While searches to investigate prior research might be necessary, tasks like executing experiments are, at the very least, not performed. Thanks to the advancement of large language models, we are now capable of handling text at a significant level. Therefore, we can purely focus on challenges related to the evaluation of research. This is advantageous when identifying challenges to achieve our objective.

% Fifth, peer review is almost always completed by text manipulation. With the development of language models, the cost of doing this has come down considerably. 

Finally, peer review requires a judgment of value. As mentioned above, alignment is one of the biggest barriers to ultimately achieving autonomous artificial researchers. To solve this problem, it is important to first understand how humans make value judgments in research. And peer review is a rare example where such values are explicitly assessed. For example, peer review evaluates aspects such as the ``importance''of a question. Automating an evaluation of question ``importance'' similar to what humans do is vital in creating an agents capable of producing research beneficial to humans. In this sense, starting with the automation of peer reviews seems to be a good start point.

% One major barrier to automating peer review is the perceived lack of sufficient data for peer review. First, in many research fields, peer review is done in a closed manner and there is no access to peer review data. This makes it difficult to automate peer review in a data-driven way in those fields. However, at least in the field of machine learning, there are many peer-reviewed comments that are open to the public, so this is not so much of a problem in the machine learning field.

% Second, the quality of peer review comments varies. Especially in the machine learning field, the number of reviewers is insufficient for the number of conference submissions. As a result, reviewers sometimes have to review papers in fields in which they do not specialize. This undermines our credibility as the gold standard for peer review comments and evaluations. But even if there were no such circumstances, peer review would still vary from person to person in the first place. This is because there is no clear-cut correct answer to the non-epistemic value judgment of what constitutes ``importance'' and the epistemic value judgment of what constitutes ``validity'' of verification. Currently, each researcher merely makes subjective decisions according to his or her own axis of judgment. In fact, it is known that machine learning research has shown that peer review results vary.
% \textcolor{red}{(citation needed)}

% However, this second point is a difficulty that occurs inherently in the process of peer review, and it is a difficulty that must be resolved in order to realize automation of research. Rather, the main issue is how to make machines acquire the value judgments that are currently tacit knowledge. Therefore, it would be useful to first analyze and discuss these value judgments, at least among humans, and then proceed to form some kind of consensus. \textcolor{red}{TODO:(move to challenge)}

\subsubsection{Peer Review Automation}

There is a bunch of studies that have tried to automate the peer review process. Researchers have tried to automate review generation \cite{yuan2022can,yuan2022kid,wang2020reviewrobot}, paper screening \cite{schulz2022future}, research paper assessment \cite{kousha2022artificial}, reviewer assignment \cite{zhao2022reviewer}, and more. As in other fields, recent years have seen research on the automation of peer review using large language models such as GPTs \cite{wexin2023can,liu2023reviewergpt,robertson2023gpt4,hosseini2023fighting}. For traditional research on the automation of peer review,  Kousha et al. \cite{kousha2022artificial} and \cite{lin2021automated1} Lin et al. have conducted comprehensive literature reviews. 

\section{Conclusion}
In this paper, I conducted a speculative examination of the concept of an intelligent agent that can conduct research. I began by discussing what it means to conduct research, arguing that research could be seen as the act of updating beliefs in hypotheses. I then discussed the construction of questions, generation of hypotheses, and verification of hypotheses, which are seen as essential elements in research. Finally, I pointed out the importance of highlighting challenges in realizing such agents and shared some simple ideas for prototyping.

The discussions in this paper are all speculative. The definition of research discussed is provisional, the challenges and implications identified are just a fraction of the vast possibilities, and the prototyping ideas are akin to simple toys. Also, the literature cited is far from exhaustive, with many important works not covered. My capacity to adequately evaluate each reference might have been insufficient, leading to one-sided assessments or errors. The paper is planned to be updated in the future, and these shortcomings will be addressed in these updates. Any feedback or corrections are highly appreciated.

The reason for publishing this paper in its current, idea-stage form, despite its many insufficiencies, is to provide a starting point for thinking about the concept of a research-capable intelligent agent. I hope this paper will be of some help to researchers aiming to realize such agents and will contribute to more vibrant discussions in the future.

% この論文では、研究ができる知的エージェントという概念について思索的な検討を行いました。まずはじめに、研究をするということはどのようなことと言えそうかを議論しました。認識論の議論を手がかりに、研究とはある答えがわからない仮説が真であるという信念を更新する行為とみなすことができるのではないかという議論をしました。次に、研究において不可欠な要素であると思われる、問いの構築、仮説の生成、仮説の検証について議論しました。これらの特徴とそれを機械が実行することについての示唆について議論したのち、これらを組み合わせた場合やこれらに共通して言える課題やトピックについて話しました。最後に、このようなエージェントの実現を目指すために課題をあぶり出すことの重要性を指摘し、そのためのプロトタイピングの簡単なアイデアを共有しました。

% この論文で展開した議論はいずれも speculative な議論です。この論文で議論した研究の定義は暫定的なものであり、得られた課題や示唆も膨大に考えられるもののうちのほんの一部の側面のみであり、プロトタイピングのアイデアも簡単なおもちゃのようなものです。また、本論文で紹介した文献は私が触れられなかった重要な文献は山ほどあり到底網羅的ではありません。今回の論文が言及した範囲が広すぎるため各文献を適切に評価するには私の能力が不足しており、私の各文献に対する評価も一面的であったり誤りが含まれていたかもしれません。また、今回は触れられなかった重要な論点はいくつもあります。この論文は今後もアップデートをしていく予定で、これらの不足している点についてはアップデートしていきたいと思っています。もし何かお気づきの点がありましたら、ぜひご指摘いただけますと幸いです。

% このようにまだ多くの不十分な点を抱えたアイデア段階でこの論文を公開することにしたのは、研究ができる知的エージェントという概念について是非考えていく一つのきっかけを提供できればとの思いからです。研究ができる知的エージェントの実現を目指す研究者たちの少しでも助けになり、今後の議論がより一層盛り上がっていく一助となれば幸いです。

% In this paper, we explored what needs to be done to create intelligence capable of autonomously conducting any research. Firstly, we examined the definition of what constitutes research. We then proposed the idea that research might be the act of producing new knowledge for a society and that producing knowledge might be updating the collective beliefs of a society. As a result, we discussed that the core of research lies in formulating questions, generating hypotheses, and verifying those hypotheses. We also discussed the implications provided by the relativity of the research subject to society. Next, we briefly introduced examples of initiatives trying to automate research. While there has been significant progress, we explained that there are still barriers to realizing a general-purpose and autonomous artificial researcher. Lastly, based on these discussions, we debated the challenges we believe are crucial in realizing a general-purpose and autonomous artificial researcher. As a first step, we proposed building a prototype of an autonomous research pipeline driven solely by general instructions.
% \chapter{Challenges and Propositions}
In Chapter 2, we examined the definition of research and its implications, and in Chapter 3, we provided an overview of past efforts related to the automation of research. In this chapter, based on these, we will reorganize the challenges towards realizing an autonomous and general artificial intelligence capable of conducting research.



\section{Challenges}

 \subsection{General and Autonomous Question Construction, Hypothesis Generation, and Hypothesis Verification}

 In Chapter 2, we explained that in order to create an AI capable of conducting any research, it is deemed necessary to realize the formulation of questions, generation of hypotheses, and validation of these hypotheses as a combination of universal skills applicable to all research. In this section, we will revisit and organize the potential challenges in achieving an AI that can conduct each of these processes.

 \subsubsection{Common Challenges}
One of the major challenges that can be a common issue for any process, as pointed out in prior research \cite{coley2020autonomousII}, is how to execute these tasks in open-ended situations. For instance, in the automation of experiments, robots use experimental equipment selected, prepared, and set up by humans. However, humans do these tasks from scratch with their own hands. Humans do not use a given corpus of papers; instead, they search for and use them on their own. Candidates for hypotheses are not explicitly provided; humans begin by identifying potential hypotheses. Even when formulating questions that serve a particular goal, humans set that goal themselves. 

In many cases in research automation, these elements are pre-determined by humans. How to let machines autonomously perform these tasks starting only with the same initial information given to humans is crucial in realizing an autonomous artificial researcher. Moreover, having this kind of freedom is essential to achieving a general artificial researcher as well. This is because if we impose constraints on AI to research only within specific research questions or hypothesis spaces defined by humans, it cannot become an AI capable of conducting arbitrary research. Therefore, a significant challenge is how to make AI acquire complex foundational skills and fundamental reasoning abilities to realize these capacities.

\subsubsection{Question Construction}

The first issue discussed in Chapter 2 is the problem of determining the unknown nature of the answer to a question. Given that research is an endeavor to produce new knowledge, it is necessary for the answer to the question to be unknown. Therefore, there is a need to generate such questions or later verify that the answer to the question is indeed unknown.

The second challenge is the issue of how to make a machine generate a ``good'' question. Firstly, what researchers consider as a ``good'' question is not always consistently agreed upon among them. Furthermore, we pointed out that the ``goodness'' of a question is inherently a concept relative to the individual or society. Hence, there seems to be a need to clearly define what constitutes a good question and think about how to effectively integrate these definitions.

The third challenge is that as we demand more autonomy from AI, the automation of question formulation becomes more difficult. Inherently, questions are constructed based on various motivations, such as pure intellectual curiosity or for specific objectives. It is very challenging to automate the construction of questions without defining which of these motivations should be the source. Moreover, as mentioned in Chapter 2, the construction of questions encounters the problem of infinite regression when pursued with strict autonomy. Even if not taken to that extreme, setting higher-order objectives behind questions for AI is an exceptionally difficult task.

It seems that the automation of question formulation has received relatively less attention compared to other processes. Since the formulation of questions is an essential element in conducting research, it would be desirable for more focus to be directed towards the research on automating this process.z

% Realizing AI that construct a ``good'' question in a generic way is challenging. As discussed in Section \ref{section:the-relativity-of-knowledge-production-to-society}, research is relative to society and different criteria can be considered for what makes a ``good'' question. Thus, some human perspective on the ``goodness'' of a question must be incorporated. We need to discuss what we consider good, what we should prioritize, and how to incorporate the value to AI.

% \textcolor{red}{TODO}

% Moreover, determining inputs to the question construction module is not trivial. In hypothesis generation, the question is the primary input, whereas in verification, it's the hypothesis. However, question formation take any input. Once you seriously try to identify the origin of question, you will encounter infinite regress. This is a unique problem that arises when aiming for a general-purpose and autonomous artificial researcher. This is because the issue revolves around how much input can be assumed while still being considered autonomous, given that it can potentially take any input.

% \cite{wang2023skillqg}

% neural question generation \cite{pan2019recent}


\subsubsection{Hypothesis Generation}

One challenge in creating an intelligence capable of hypothesis generation, not just as a tool for humans, is the need to empower the machine itself to form plausible hypotheses for questions to which even the machine doesn't know the answer. Current machine learning models have been criticized for potentially not knowing what they don't know \footnote{
In our discussion with Wataru Kumagai, we were reminded once again of the importance of self-awareness in creating an AI capable of conducting research.
}. 
Moreover, they are known to confidently provide answers or fabricate falsehoods about topics they are ignorant of. Therefore, it seems essential to first accurately recognize what is unknown, either for oneself or the world at large, as told in sections of question construction. Upon facing an unknown subject, there's a need to reduce uncertainty and approach understanding. As mentioned in Chapter 2, humans attempt to understand uncertain subjects by gathering information from papers, experiments, or by reframing questions. While it may not be necessary to adopt the exact same approach, it seems essential to enable machines to autonomously adopt strategies to reduce uncertainties.

% Outputs from machine learning models are essentially inferences tinged with uncertainty. From this perspective, one could posit that these models are already inherently generating hypotheses. Indeed, they are already employed for hypothesis generation in numerous scientific investigations.

% However, while these hypotheses might be hypotheses in the sense that the answers are unknown to humans, they might be self-evident to the machine learning model. When we talk about AI generating hypotheses in the context of AI conducting research, the ultimate expectation is for the AI to provide plausible answers to what is unknown to AI itself. This remains an unresolved issue.

As one of the promising approaches for tacking unknowns, it seems crucial for AI to acquire systematic thinking to realize this, as humans developed systems like language and mathematics, allowing them to infer about subjects beyond their experience. Systematic thinking is not only important for out-of-distribution generalization but is also valued for interpretability, causal inference, logical reasoning, mathematical processing, and planning.

\textcolor{red}{TODO}

\subsubsection{Hypothesis Verification}

In Chapter 2, we highlighted several challenges in realizing an AI capable of verification. First and foremost, the AI itself needs to understand what verification is, and by what criteria a sequence of actions qualifies as verification. Ideally, it would be preferable for the AI to contemplate and understand from scratch what verification is. However, many humans don't do this either, and as discussed in Chapter 2, the philosophical debate on precisely defining verification is still unresolved. Therefore, it's harsh to demand this of a machine. At the very least, the machine needs to thoroughly understand and proficiently use verification concepts that humans employ, such as statistical hypothesis testing, from first principles.


Second, the AI must be able to formulate detailed and complex plans to verify a hypothesis. With the advancements in language models in recent years, we are now much more capable of formulating superior plans than before. However, devising detailed plans remains a challenging issue.

Third, it has to be prepared to carry out these plans and execute the plan with the combination of human-like complex actions.As mentioned in Chapter 2, to achieve this, the AI must be able to search for, create, purchase, and manipulate equipment with almost the same degree of freedom as humans, requiring it to exhibit extremely sophisticated and complex behaviors. This is an immensely challenging issue, and it might even be fair to say it's one of the biggest bottlenecks in realizing an intelligence capable of generic and autonomous research. Laboratory automation have attempted to address this challenge in real world by developing robots. We will discuss the case within the computer below.

For AI to execute research on a computer, it must perform any operation within the computer. For instance, machine learning research entails, setting an environment, preparing datasets and models, and writing and executing codes. To allow AI to prepare these without human intervention, the AI itself must be able to autonomously search the web, select data, download it, and so on. Furthermore, once the AI generates code for verification, it must operate the shell to execute it.

There are ongoing initiatives to enable language models to operate browsers \cite{nakano2021webgpt,act1}. While full browser operations might seem ambitious, there are already endeavors to allow language models to conduct searches \cite{mialon2023augmented}. If we achieve browser automation, it will greatly advance research automation involving web operations. Moreover, efforts like the open interpreter \cite{openinterpreter} aim to automate any computer action. This direction holds promise for automating all research confined within a computer. Although these studies are gaining traction in the machine learning domain, they're not always linked to research automation. We advocate recognizing this as a pivotal challenge in the realm of research automation.

In the field of machine learning, it seems that the discussion on automated validation has not garnered much attention until now. However, recently, the need for verification is recognized in the machine learning community beyond outside of the context of research automation. Studies like \textit{scientific claim verification}, which received much attention during the COVID-20 pandemic \cite{wadden2020fact}, or attempts to minimize hallucination \cite{dhuliawala2023chain} are examples of them. These are not attempts to automate validation in research. Therefore, these findings cannot be directly applied to the automation of research validation. However, we expect that these studies will provide useful insights for the future development of artificial intelligence capable of understanding validation.

% Creating AI that autonomously verifies hypotheses is challenging. While current models can mimic human verification, truly understanding the verification strategy demands more work. Sometimes, they even need to devise the verification measure themselves.

% The biggest challenge for autonomous verification is the need to freely move around in the real world or within a computer, and to manipulate objects within that world at will. We believe this to be one of the greatest barriers to full research automation. Laboratory automation have attempted to address this challenge in real world by developing robots. We will discuss the case within the computer in Section \ref{section:behaviour-inside-the-computer}.

% \subsubsection{AI Capable of Peer Review}

% Given the difficulty of these challenges, automating peer review could be a strategic starting point. This is because peer review is a universal process across diverse research fields and it assesses the validity and quality of problems, hypotheses, and verification methods, which is easier than generating them. Despite some progress, full automation is still elusive \cite{yuan2022can,schulz2022future}.


% \subsection{Behaviour inside the Computer}
% \label{section:behaviour-inside-the-computer}


% Lab Notebook?
% \subsection{Dataset of Research Process}
% It is important to establish the necessary infrastructure for research automation. The two pillars of research automation are the development of basic models that incorporate academic knowledge and the construction of data sets. The development of an infrastructure model that incorporates scientific knowledge has already been proposed in many places and is actually under development, so I will not emphasize its necessity here again. Also, regarding data sets, the construction of data sets for the acquisition of scientific knowledge has been done in various places as well, so I will not emphasize that here either.

% Instead, we propose here to construct a research process dataset. A research process dataset is behavioral log data that incorporates all possible tasks throughout the entire process of the study, from start to finish. Ideally, individual tasks should be labeled as to whether they correspond to question construction, hypothesis generation, or hypothesis testing. We believe that building such a data set is important because, as explained in the Literacy section of Chapter 2, the current paper is not a data log of the entire research process. This makes it difficult to be data-driven and end-to-end learning how to do research itself. I believe that building a research process will help solve these problems and increase the likelihood of more flexible intelligent agents.

% However, building a dataset of the research process seems daunting. This is because researchers who do not currently keep research logs would have to go to the trouble of recording their research process.\footnote{
% In an experimental laboratory in the natural sciences, it is common to take research notes, so it may not be that difficult to record more detailed processes as an extension of this practice. However, in the machine learning field, the culture of taking research notes does not seem to be that common. (We think it is common to keep logs of experiments, but it seems to be rare to describe the details, for example, where and how the data was obtained.) 
% } Therefore, it seems necessary to devise a way to make it easier for researchers to keep logs. It may be to manage the research process on GitHub, or to take research notes as in natural science research, but it is important to discuss how to achieve these things.

% Being able to construct a dataset of the research process would be ideal, but may not be immediately feasible. As an alternative, it seems important to create a dataset designed to automate question construction, hypothesis generation, and hypothesis testing. At its simplest, one might start by building a dataset of papers labeled with the parts that correspond to the question, hypothesis, and test, respectively. This would be a relatively simple but important step in achieving a generic artificial researcher.

% Alternatively, instruction tuning could be done by viewing question construction, hypothesis generation, and hypothesis testing as tasks, respectively. This would produce a language model that can execute question construction, hypothesis generation, and hypothesis testing with greater fidelity. This could be the foundation for a general-purpose, autonomous artificial researcher.



\subsection{Alignment}
As discussed in Section 2, when realizing an AI that autonomously conducts research, the issue of alignment arises. 

First and foremost, it is essential to consider ways to ensure that AI does not engage in research that could harm humans. However, this is a challenging issue. The problem of ensuring that AI does not harm humans is a difficult problem in AI Alignment. Furthermore, knowledge and technology produced by research are fundamentally value-neutral. That is, the knowledge can be used for good or ill. Therefore, even if AI were to research with harmful intentions, it would be challenging to judge from the actual research results.

The remaining two issues arise in the ultra-long term when AI becomes fully autonomous in conducting research. The second issue is that to enable meaningful knowledge production for humans, there needs to be an alignment between the knowledge systems of AI and humans. As mentioned in Chapter 2, if knowledge and verification are relative concepts to society, research conducted autonomously by AI may become meaningless to humans. On the other hand, if we were to correct AI to follow human methods entirely, we might unnecessarily limit the machine's potential capabilities. Deciding how much human methodology and values to incorporate and how much freedom to allow the machine, and finding ways to achieve this, will be a significant challenge in creating research-capable AI.

The third issue concerns the alignment between AI and nature, not between humans and AI. As mentioned in Chapter 2, the fact that humans have come to understand nature is likely not unrelated to our long history of interacting with nature. It seems there's no guarantee that artificial machines like AI, which lack such experiences, would lead to an understanding of nature through their autonomously generated knowledge.

The latter two issues are problems that only arise when demanding extreme autonomy from machines and are not immediately problematic. However, when discussing the limitations and possibilities of knowledge production and natural understanding by agents independent of humans, they seem to become relevant issues.

% In the medium to long term, it's essential to devise ways to ensure that AI doesn't engage in research that could be dangerous to humans. 

% In the long term, we must contemplate how to construct a knowledge system that are mutually translatable between human society and AI society. While these issues may not arise in the short term, it's crucial to engage in discussions now, looking towards the long-term future.

% \subsubsection{Understanding}

% Extensive discourse transpires concerning scientific discoveries. Yet, discussions pertaining to scientific comprehension remain relatively unexplored. Krenn et al. delve into the conundrum of what it entails for a machine learning agent to not only unearth scientific knowledge but also to comprehend it \cite{krenn2022scientific}. They adopt a human-centric stance, positing that an agent's ability to offer explanations comprehensible to human scientists signifies the existence of its scientific understanding.

sun-rise \cite{leslie2023does}

\section{Constructing Research Pipeline Prototypes}
We propose to start by building a research pipeline, connecting the modules of the knowledge production system. The research pipeline is a software system that takes input and generates knowledge as output, encompassing the sequence of processes involved in research. Since this process does not require human intervention, it can be considered as an autonomous research system. We propose this system to be composed of the sub-processes of ``question construction,'' ``hypothesis generation,'' and ``hypothesis verification.'' This creates a general system that is potentially applicable to any research. These sub-processes can be likened to abstract classes in programming. Each process automatically formulates appropriate questions, generates hypotheses, and performs verification based on the input. 

Initially, we will assume a specific research problem, and this research problem can be a simple one. And the inner workings of detailed hypothesis testing and hypothesis generation can be guided (but not hard-coded) to achieve the desired results. Anyway, the high-level concept is to create the minimum necessary to automatically execute each process of question construction, hypothesis generation, and hypothesis testing. Then, by gradually making the contents of each module autonomous and gradually loosening the restrictions on the research problem, we will lead to a general-purpose and autonomous artificial researcher.

\subsubsection{Guided but Not Hard-Coded}

It is important to note, however, that even in the prototype stage, the internal implementation of question construction, hypothesis generation, hypothesis testing, etc., should be ``guided'' and not ``hard-coded'' as much as possible. In machine learning, induction is the process of adding words to the prompt that make it easier to output the expected answer, and hardcoding is the process of actually inserting the desired processing into the algorithm. For example, if a statistical hypothesis test is expected to be used as a means of testing a hypothesis, rather than having a human write a program that contains a process for performing a statistical hypothesis test, we would instead instruct to the machine learning model with prompting, ``Statistical hypothesis testing is one of the leading methods in verification. The hypothesis is A. Verify this.'' This is a very important point to emphasize.

This is a very important point, so let me emphasize it. The reason this is important is that we do not want to automate a particular hypothesis testing process, but rather we want the machine to test the hypothesis itself. Only when you make sure that the machine decides on its own the appropriate verification method according to the hypothesis, will you be able to provide collateral evidence that the machine itself is able to verify the hypothesis. If this can be done not only in hypothesis testing, but also in all aspects of question construction and hypothesis generation, we can call it a prototype of a general-purpose, autonomous artificial researcher.

\subsubsection{Why Pipeline?}

There are two reasons why we think it is a good idea to start by building such a research process pipeline. The first is that this is one simplified representation of a generic and autonomous research system. Research is a very complex task, so when we try to automate validation, we inevitably focus on automating individual tasks. In addition, many research automation efforts are aimed at making things better, which often leads to a strong dependence on the domain, for example, in automating hypothesis generation. However, as emphasized above, what we want to achieve is not specific hypothesis generation or verification, but the ability to generate and verify hypotheses themselves. This system emphasizes that point, and once realized, it will be an example of what a general-purpose, autonomous artificial researcher could look like. The creation of such an example will serve as a guidepost for more people to become versatile and autonomous artificial researchers.

Second, building on this would further clarify the challenges in achieving a general-purpose and autonomous artificial researcher. In this paper, we have discussed the challenges that would be necessary to realize a general-purpose, autonomous artificial researcher. However, we believe that this is a very difficult task and that there are many areas where we do not even know what the problems really are. Therefore, it is important to first identify what the problems are in the first place and where the uncertainties lie. When we move toward such a complex problem, we start with a simple example to understand the structure of the problem. For example, we build toy models in physics, concrete examples in mathematics, and prototypes in programming. The research process pipeline falls under such simple examples in autonomous artificial researchers. In the process of trying to achieve this, we will discover what are the bottlenecks and what are the essentials. In this way, I think it is important to build a research process pipeline in order to first increase the resolution of the problem and clarify the issues.

\subsubsection{Where to Start?}
It is advisable to start by representing a specific research as a pipeline. Initially, creating a concrete system helps clarify the actions involved in actual research and makes the specific challenges to be addressed more tangible. When dealing with projects with high uncertainty, it is crucial to concretize the problems to be solved. Specifically, the goal is to programmatically represent the actions that researchers perform as comprehensively as possible. It is acceptable to consider certain aspects as constants if their execution is too challenging to represent as a program. Then, running the system should reproduce the original research. The next step is to progressively automate the processes and constants provided by humans to enhance autonomy. Naturally, automating a specific research pipeline alone does not guarantee the development of an autonomous pipeline. However, this approach allows for the identification of research automation challenges and paves the way for their resolution through research and development. Importantly, it is essential to express individual tasks as components or sub-processes of question construction, hypothesis generation, or hypothesis verification. This is similar to inheriting an abstract class, ensuring that the automation of these processes is achieved as individual tasks are automated.

In practice, it becomes evident that fully automating an entire research is highly challenging. Therefore, before automating specific research, it may be advantageous to start by creating simplified toy models and aiming to build systems that can execute them automatically. For example, certain parts that require obtaining and using a real dataset can be replaced with appropriately created sample datasets. The approach is similar to that of specific research pipelines, addressing research challenges while aiming to increase autonomy and generality.

\textcolor{red}{Remarks about Autores PJ}

\subsection{Which Field of Research to Start with?}
We suggested that we might start by automating specific research areas and research tasks. So what research areas should we start with? As it turns out, we think it might be a good idea to start by automating machine learning research. There is, of course, a bias due to the fact that the authors of this paper are machine learning researchers and that we are writing this paper primarily for machine learning researchers, but aiming to automate machine learning research makes a lot more sense than that. To illustrate this, let me first introduce some of the perspectives involved in decision making in the area of research to be automated.

\subsubsection{How to Choose Research Area}
The first perspective is how much automation of that research area will help achieve a versatile and autonomous artificial researcher. We believe that it would be a good idea to automate research areas that would accelerate the automation of research. For example, if there is a problem to be solved in order to generate a hypothesis, the problem itself could be set as a research problem and automation of this research could be realized. If we can automate such a task, we have not only achieved our goal of automating the entire research process, but we have also solved the problem of research automation. Such bootstrapping will accelerate the automation of the research process and allow it to reach its goals more efficiently. \footnote{
Inspired by the feedback from Hiroshi Yamakawa
}

The second aspect is feasibility. Since the objective of this project is to create a prototype, it is an important policy to start with the least difficult to realize. There are various levels of difficulty, but the most important is whether the level of difficulty is high, especially in areas other than those essential to the automation of a general-purpose research process. For example, it would be more feasible to generate hypotheses from papers now that language models have been developed than to actually construct and conduct a experiment and generate hypotheses from the observations, in the sense that it would be fully automated. Also, if we were to start with automation using a language model, it would be better not to include tasks that the language model is not good at. As emphasized above, it is better to start where it is as easy as possible here, because focusing on automation of the parts that depend on individual research tasks is not important for the goal of acquiring generalizable knowledge to achieve a general-purpose artificial researcher.

The third aspect is whether the field has an impact on many studies. First of all, an area that has an impact on many studies is one that is used as an elemental technology in those studies. This would be appropriate as a research area to automate for general-purpose artificial researchers, in the sense that it is a general-purpose technology. And if areas that affect many studies can be automated, it will also accelerate the knowledge production of those studies. The efficiency of knowledge production for humanity as a whole will increase, and research automation projects will also benefit from the knowledge produced by them. It would also increase the population of people involved, which may lead more people to pay attention to research automation. This will spawn new flows of people, money, and knowledge, and as a result, projects are expected to move forward more quickly.

\subsubsection{Automating ML Research}
We believe that automating machine learning research may be suitable to start with in terms of these decision axes. First, let's discuss bootstrapping. Many of the challenges to automating research will be how to get machines to acquire from experience what they are currently hardcoding and doing in the real world. Learning from experience is exactly what machine learning does, and in this sense, many of the challenges in realizing autonomous artificial researchers can be formulated as machine learning research challenges.

Next, let us discuss feasibility. In the first place, many machine learning studies are conducted entirely on computers. As mentioned earlier, the greatest difficulty in achieving general automation lies in the interaction with the real world. Technologies related to real-world interaction are used for hypothesis verification rather than the verification itself. This requires advancements in robotics research. Therefore, to pursue the automation of the entire research process, it may be best to set aside fields that require interaction with the real world and initially focus on automating research that can be done solely on PCs. 

Also, many attempts to automate machine learning processes have already been made. For example, in MLOps, various pipelines for automating tasks such as experiment management and training in machine learning have been proposed and put into practical use. AutoML, which is a field of machine learning research, has also produced numerous innovations in automating many of the tasks involved in machine learning. Moreover, the culture of machine learning and related engineering fields already has a wealth of knowledge and insights regarding automation. This means that we do not have to devote many resources to automating research domain-specific tasks. This allows us to focus on more essential questions in our quest to become general-purpose artificial researchers, such as ``How do we allow people to test hypotheses?'' Furthermore, many studies in machine learning and related research areas are open-source. Consequently, it is considered easier to retrieve information from papers compared to other fields. 

Finally, we would like to discuss the impact on other sectors. As you are already aware, many research fields are currently using machine learning technologies. The AI for Science initiative, which aims to automate the scientific field, also uses machine learning technology. Therefore, the automation of machine learning research and better knowledge production will accelerate all of these efforts. For these reasons, we believe it is a good approach to start by automating a specific research project in the machine learning domain.

\subsubsection{What Type of Research in Machine Learning Should We Start with and How?}
There are many different types of machine learning research, but where should we start with automation? Even though we aim to automate the construction of questions, what types of research should we guide them to do?

It seems that a example of the research appropriate as one a starting point is that on the zero-shot prompt proposal. First, the hypothesis (or proposal) in this study is the specific text of the prompt. This is much less expensive to implement, whereas many empirical machine learning proposals require composing an algorithm or architecture. Validation requires the automation of the task of preparing existing data and models, and this is certainly a difficult task. However, this is an extremely common task in machine learning research and is not unique to this research project. The ability to automate this task would benefit a significant amount of machine learning research. The validation criterion is also generic, as it is a typical validation criterion that compares the proposed group with the control group. We think we will first build a prototype by adjusting the LLM prompts, and the fact that there is no strict mathematical or logical manipulation, which language models are not good at, is another aspect that makes this research easy to do.

With this in mind, one research project in which the author of this paper is participating is in the process of building a prototype of the pipeline of research for the prompt proposal \footnote{
Link to the pipeline: \href{https://github.com/t46/mock-pipeline}{https://github.com/t46/mock-pipeline}
}. It is currently still in the pilot stage and some parts are hard-coded, but will be updated as needed.

What we have described here is just one example and a suggestion. We hope that more similar initiatives will emerge in other studies.

\subsection{Automating Peer-Review}
From a slightly different perspective than building a research pipeline, as mentioned above, automation of peer review may also be a suitable first step toward a versatile and autonomous artificial researcher. First, peer review requires judgments about novelty and validity of validation, which are necessary elements for research automation. Thus, the more automated peer review can be, the clearer the need for automation of the entire research process becomes. Second, peer review is identification, not generation, of these elements. Since identification is generally easier than generation, it seems like a good first step in terms of starting small. Third, peer review is a discipline-agnostic practice. Therefore, automating it is expected to be important in gaining insights to realize a general-purpose artificial researcher. Fourth, there is more prior research in automating peer review than in automating generic hypothesis testing or hypothesis generation. Therefore, it is an area where the hurdles for starting a new research project are relatively low. Fifth, peer review is almost always completed by text manipulation. With the development of language models, the cost of doing this has come down considerably. Finally, peer review requires a judgment of value (non-epistemic value). As mentioned above, alignment is one of the biggest barriers to ultimately achieving autonomous artificial researchers. To solve this problem in the long term, it is important to first understand how humans make value judgments in research and to collect data on value judgments. And peer review is a rare example where such non-epistemic values are explicitly expressed. Therefore, it seems to me that automating peer review is one way to start thinking about alignment solutions in earnest. For these reasons, I think it is effective to start with the automation of peer review.

One major barrier to automating peer review is the perceived lack of sufficient data for peer review. First, in many research fields, peer review is done in a closed manner and there is no access to peer review data. This makes it difficult to automate peer review in a data-driven way in those fields. However, at least in the field of machine learning, there are many peer-reviewed comments that are open to the public, so this is not so much of a problem in the machine learning field.

Second, the quality of peer review comments varies. Especially in the machine learning field, the number of reviewers is insufficient for the number of conference submissions. As a result, reviewers sometimes have to review papers in fields in which they do not specialize. This undermines our credibility as the gold standard for peer review comments and evaluations. But even if there were no such circumstances, peer review would still vary from person to person in the first place. This is because there is no clear-cut correct answer to the non-epistemic value judgment of what constitutes ``importance'' and the epistemic value judgment of what constitutes ``validity'' of verification. Currently, each researcher merely makes subjective decisions according to his or her own axis of judgment. In fact, it is known that machine learning research has shown that peer review results vary.
\textcolor{red}{(citation needed)}

However, this second point is a difficulty that occurs inherently in the process of peer review, and it is a difficulty that must be resolved in order to realize automation of research. Rather, the main issue is how to make machines acquire the value judgments that are currently tacit knowledge. Therefore, it would be useful to first analyze and discuss these value judgments, at least among humans, and then proceed to form some kind of consensus.
\textcolor{red}{TODO:(move to challenge)}
% \chapter{Conclusion}

In this paper, we explored what needs to be done to create intelligence capable of autonomously conducting any research. Firstly, we examined the definition of what constitutes research. We then proposed the idea that research might be the act of producing new knowledge for a society and that producing knowledge might be updating the collective beliefs of a society. As a result, we discussed that the core of research lies in formulating questions, generating hypotheses, and verifying those hypotheses. We also discussed the implications provided by the relativity of the research subject to society. Next, we briefly introduced examples of initiatives trying to automate research. While there has been significant progress, we explained that there are still barriers to realizing a general-purpose and autonomous artificial researcher. Lastly, based on these discussions, we debated the challenges we believe are crucial in realizing a general-purpose and autonomous artificial researcher. As a first step, we proposed building a prototype of an autonomous research pipeline driven solely by general instructions.
% \chapter{Proposal}
\label{chapter-proposal}
In this chapter, we will present our proposal to realize general and autonomous artificial researcher.

% discuss what should be done and how to achieve the realization of autonomous artificial researchers. Firstly, we will provide an overview of Chapter 1 and Chapter 2. Building upon that, we will propose sub-goals to aim for in order to achieve autonomous artificial researchers. Subsequently, for each sub-goal, we will organize subtasks and propose a general approach and strategy for how to pursue these intermediate goals.

% As we have reiterated multiple times, the ultimate goal is to create an artificial intelligence that can autonomously conduct research. Therefore, in order to clarify the objectives, it is necessary to define what it means for an AI to be able to conduct research and what it means for it to be autonomous. Let's first discuss these aspects. 

\section{Constructing a Research Pipeline}
We propose to start by building a research pipeline, connecting the modules of the knowledge production system. The research pipeline is a software system that takes input and generates knowledge as output, encompassing the sequence of processes involved in research. Since this process does not require human intervention, it can be considered as an autonomous research system. We propose this system to be composed of the sub-processes of ``question construction,'' ``hypothesis generation,'' and ``hypothesis verification.'' This creates a general system that is potentially applicable to any research. These sub-processes can be likened to abstract classes in programming. Each process automatically formulates appropriate questions, generates hypotheses, and performs verification based on the input. 

Initially, we will assume a specific research problem, and this research problem can be a simple one. And the inner workings of detailed hypothesis testing and hypothesis generation can be guided (but not hard-coded) to achieve the desired results. Anyway, the high-level concept is to create the minimum necessary to automatically execute each process of question construction, hypothesis generation, and hypothesis testing. Then, by gradually making the contents of each module autonomous and gradually loosening the restrictions on the research problem, we will lead to a general-purpose and autonomous artificial researcher.

\subsubsection{Guided but Not Hard-Coded}

It is important to note, however, that even in the prototype stage, the internal implementation of question construction, hypothesis generation, hypothesis testing, etc., should be ``guided'' and not ``hard-coded'' as much as possible. In machine learning, induction is the process of adding words to the prompt that make it easier to output the expected answer, and hardcoding is the process of actually inserting the desired processing into the algorithm. For example, if a statistical hypothesis test is expected to be used as a means of testing a hypothesis, rather than having a human write a program that contains a process for performing a statistical hypothesis test, we would instead instruct to the machine learning model with prompting, ``Statistical hypothesis testing is one of the leading methods in verification. The hypothesis is A. Verify this.'' This is a very important point to emphasize.

This is a very important point, so let me emphasize it. The reason this is important is that we do not want to automate a particular hypothesis testing process, but rather we want the machine to test the hypothesis itself. Only when you make sure that the machine decides on its own the appropriate verification method according to the hypothesis, will you be able to provide collateral evidence that the machine itself is able to verify the hypothesis. If this can be done not only in hypothesis testing, but also in all aspects of question construction and hypothesis generation, we can call it a prototype of a general-purpose, autonomous artificial researcher.

\subsubsection{Why Pipeline?}

There are two reasons why we think it is a good idea to start by building such a research process pipeline. The first is that this is one simplified representation of a generic and autonomous research system. Research is a very complex task, so when we try to automate validation, we inevitably focus on automating individual tasks. In addition, many research automation efforts are aimed at making things better, which often leads to a strong dependence on the domain, for example, in automating hypothesis generation. However, as emphasized above, what we want to achieve is not specific hypothesis generation or verification, but the ability to generate and verify hypotheses themselves. This system emphasizes that point, and once realized, it will be an example of what a general-purpose, autonomous artificial researcher could look like. The creation of such an example will serve as a guidepost for more people to become versatile and autonomous artificial researchers.

Second, building on this would further clarify the challenges in achieving a general-purpose and autonomous artificial researcher. In this paper, we have discussed the challenges that would be necessary to realize a general-purpose, autonomous artificial researcher. However, we believe that this is a very difficult task and that there are many areas where we do not even know what the problems really are. Therefore, it is important to first identify what the problems are in the first place and where the uncertainties lie. When we move toward such a complex problem, we start with a simple example to understand the structure of the problem. For example, we build toy models in physics, concrete examples in mathematics, and prototypes in programming. The research process pipeline falls under such simple examples in autonomous artificial researchers. In the process of trying to achieve this, we will discover what are the bottlenecks and what are the essentials. In this way, I think it is important to build a research process pipeline in order to first increase the resolution of the problem and clarify the issues.

\subsubsection{Where to Start?}
It is advisable to start by representing a specific research as a pipeline. Initially, creating a concrete system helps clarify the actions involved in actual research and makes the specific challenges to be addressed more tangible. When dealing with projects with high uncertainty, it is crucial to concretize the problems to be solved. Specifically, the goal is to programmatically represent the actions that researchers perform as comprehensively as possible. It is acceptable to consider certain aspects as constants if their execution is too challenging to represent as a program. Then, running the system should reproduce the original research. The next step is to progressively automate the processes and constants provided by humans to enhance autonomy. Naturally, automating a specific research pipeline alone does not guarantee the development of an autonomous pipeline. However, this approach allows for the identification of research automation challenges and paves the way for their resolution through research and development. Importantly, it is essential to express individual tasks as components or sub-processes of question construction, hypothesis generation, or hypothesis verification. This is similar to inheriting an abstract class, ensuring that the automation of these processes is achieved as individual tasks are automated.

In practice, it becomes evident that fully automating an entire research is highly challenging. Therefore, before automating specific research, it may be advantageous to start by creating simplified toy models and aiming to build systems that can execute them automatically. For example, certain parts that require obtaining and using a real dataset can be replaced with appropriately created sample datasets. The approach is similar to that of specific research pipelines, addressing research challenges while aiming to increase autonomy and generality.

\textcolor{red}{Remarks about Autores PJ}

\subsection{Which Field of Research to Start with?}
We suggested that we might start by automating specific research areas and research tasks. So what research areas should we start with? As it turns out, we think it might be a good idea to start by automating machine learning research. There is, of course, a bias due to the fact that the authors of this paper are machine learning researchers and that we are writing this paper primarily for machine learning researchers, but aiming to automate machine learning research makes a lot more sense than that. To illustrate this, let me first introduce some of the perspectives involved in decision making in the area of research to be automated.

\subsubsection{How to Choose Research Area}
The first perspective is how much automation of that research area will help achieve a versatile and autonomous artificial researcher. We believe that it would be a good idea to automate research areas that would accelerate the automation of research. For example, if there is a problem to be solved in order to generate a hypothesis, the problem itself could be set as a research problem and automation of this research could be realized. If we can automate such a task, we have not only achieved our goal of automating the entire research process, but we have also solved the problem of research automation. Such bootstrapping will accelerate the automation of the research process and allow it to reach its goals more efficiently. \footnote{
Inspired by the feedback from Hiroshi Yamakawa
}

The second aspect is feasibility. Since the objective of this project is to create a prototype, it is an important policy to start with the least difficult to realize. There are various levels of difficulty, but the most important is whether the level of difficulty is high, especially in areas other than those essential to the automation of a general-purpose research process. For example, it would be more feasible to generate hypotheses from papers now that language models have been developed than to actually construct and conduct a experiment and generate hypotheses from the observations, in the sense that it would be fully automated. Also, if we were to start with automation using a language model, it would be better not to include tasks that the language model is not good at. As emphasized above, it is better to start where it is as easy as possible here, because focusing on automation of the parts that depend on individual research tasks is not important for the goal of acquiring generalizable knowledge to achieve a general-purpose artificial researcher.

The third aspect is whether the field has an impact on many studies. First of all, an area that has an impact on many studies is one that is used as an elemental technology in those studies. This would be appropriate as a research area to automate for general-purpose artificial researchers, in the sense that it is a general-purpose technology. And if areas that affect many studies can be automated, it will also accelerate the knowledge production of those studies. The efficiency of knowledge production for humanity as a whole will increase, and research automation projects will also benefit from the knowledge produced by them. It would also increase the population of people involved, which may lead more people to pay attention to research automation. This will spawn new flows of people, money, and knowledge, and as a result, projects are expected to move forward more quickly.

\subsubsection{Automating ML Research}
We believe that automating machine learning research may be suitable to start with in terms of these decision axes. First, let's discuss bootstrapping. Many of the challenges to automating research will be how to get machines to acquire from experience what they are currently hardcoding and doing in the real world. Learning from experience is exactly what machine learning does, and in this sense, many of the challenges in realizing autonomous artificial researchers can be formulated as machine learning research challenges.

Next, let us discuss feasibility. In the first place, many machine learning studies are conducted entirely on computers. As mentioned earlier, the greatest difficulty in achieving general automation lies in the interaction with the real world. Technologies related to real-world interaction are used for hypothesis verification rather than the verification itself. This requires advancements in robotics research. Therefore, to pursue the automation of the entire research process, it may be best to set aside fields that require interaction with the real world and initially focus on automating research that can be done solely on PCs. 

Also, many attempts to automate machine learning processes have already been made. For example, in MLOps, various pipelines for automating tasks such as experiment management and training in machine learning have been proposed and put into practical use. AutoML, which is a field of machine learning research, has also produced numerous innovations in automating many of the tasks involved in machine learning. Moreover, the culture of machine learning and related engineering fields already has a wealth of knowledge and insights regarding automation. This means that we do not have to devote many resources to automating research domain-specific tasks. This allows us to focus on more essential questions in our quest to become general-purpose artificial researchers, such as ``How do we allow people to test hypotheses?'' Furthermore, many studies in machine learning and related research areas are open-source. Consequently, it is considered easier to retrieve information from papers compared to other fields. 

Finally, we would like to discuss the impact on other sectors. As you are already aware, many research fields are currently using machine learning technologies. The AI for Science initiative, which aims to automate the scientific field, also uses machine learning technology. Therefore, the automation of machine learning research and better knowledge production will accelerate all of these efforts. For these reasons, we believe it is a good approach to start by automating a specific research project in the machine learning domain.

\subsubsection{What Type of Research in Machine Learning Should We Start with and How?}
There are many different types of machine learning research, but where should we start with automation? Even though we aim to automate the construction of questions, what types of research should we guide them to do?

It seems that a example of the research appropriate as one a starting point is that on the zero-shot prompt proposal. First, the hypothesis (or proposal) in this study is the specific text of the prompt. This is much less expensive to implement, whereas many empirical machine learning proposals require composing an algorithm or architecture. Validation requires the automation of the task of preparing existing data and models, and this is certainly a difficult task. However, this is an extremely common task in machine learning research and is not unique to this research project. The ability to automate this task would benefit a significant amount of machine learning research. The validation criterion is also generic, as it is a typical validation criterion that compares the proposed group with the control group. We think we will first build a prototype by adjusting the LLM prompts, and the fact that there is no strict mathematical or logical manipulation, which language models are not good at, is another aspect that makes this research easy to do.

With this in mind, one research project in which the author of this paper is participating is in the process of building a prototype of the pipeline of research for the prompt proposal \footnote{
Link to the pipeline: \href{https://github.com/t46/mock-pipeline}{https://github.com/t46/mock-pipeline}
}. It is currently still in the pilot stage and some parts are hard-coded, but will be updated as needed.

What we have described here is just one example and a suggestion. We hope that more similar initiatives will emerge in other studies.

\subsection{Automating Peer-Review}
From a slightly different perspective than building a research pipeline, as mentioned above, automation of peer review may also be a suitable first step toward a versatile and autonomous artificial researcher. First, peer review requires judgments about novelty and validity of validation, which are necessary elements for research automation. Thus, the more automated peer review can be, the clearer the need for automation of the entire research process becomes. Second, peer review is identification, not generation, of these elements. Since identification is generally easier than generation, it seems like a good first step in terms of starting small. Third, peer review is a discipline-agnostic practice. Therefore, automating it is expected to be important in gaining insights to realize a general-purpose artificial researcher. Fourth, there is more prior research in automating peer review than in automating generic hypothesis testing or hypothesis generation. Therefore, it is an area where the hurdles for starting a new research project are relatively low. Fifth, peer review is almost always completed by text manipulation. With the development of language models, the cost of doing this has come down considerably. Finally, peer review requires a judgment of value (non-epistemic value). As mentioned above, alignment is one of the biggest barriers to ultimately achieving autonomous artificial researchers. To solve this problem in the long term, it is important to first understand how humans make value judgments in research and to collect data on value judgments. And peer review is a rare example where such non-epistemic values are explicitly expressed. Therefore, it seems to me that automating peer review is one way to start thinking about alignment solutions in earnest. For these reasons, I think it is effective to start with the automation of peer review.

One major barrier to automating peer review is the perceived lack of sufficient data for peer review. First, in many research fields, peer review is done in a closed manner and there is no access to peer review data. This makes it difficult to automate peer review in a data-driven way in those fields. However, at least in the field of machine learning, there are many peer-reviewed comments that are open to the public, so this is not so much of a problem in the machine learning field.

Second, the quality of peer review comments varies. Especially in the machine learning field, the number of reviewers is insufficient for the number of conference submissions. As a result, reviewers sometimes have to review papers in fields in which they do not specialize. This undermines our credibility as the gold standard for peer review comments and evaluations. But even if there were no such circumstances, peer review would still vary from person to person in the first place. This is because there is no clear-cut correct answer to the non-epistemic value judgment of what constitutes ``importance'' and the epistemic value judgment of what constitutes ``validity'' of verification. Currently, each researcher merely makes subjective decisions according to his or her own axis of judgment. In fact, it is known that machine learning research has shown that peer review results vary.
\textcolor{red}{(citation needed)}

However, this second point is a difficulty that occurs inherently in the process of peer review, and it is a difficulty that must be resolved in order to realize automation of research. Rather, the main issue is how to make machines acquire the value judgments that are currently tacit knowledge. Therefore, it would be useful to first analyze and discuss these value judgments, at least among humans, and then proceed to form some kind of consensus.
\textcolor{red}{TODO:(move to challenge)}

\section{Promote Wider Involvement in Automating Research}
In order to achieve the major goal of automating research, it is important to have the help of as many people as possible.

% \section{How to Approach the Goal}

\subsection{Call for Information Accumulation}

\subsubsection{Call for Review/Perspective Papers}

We believe that writing review and perspective papers is extremely valuable in that it lowers the cost of acquiring information for all those involved in the field.

% It is necessary to collectively create a perspective paper that includes these automation-related aspects and others. For example, while reading and writing papers are essential skills in research, I am not aware of many recent perspective papers that consider these topics. Furthermore, due to my own lack of knowledge, I do not have a comprehensive understanding of perspective papers on the automation of specific individual studies. I believe many people, not just myself, may be unaware of perspective papers on the automation of research in other fields. Therefore, I suggest creating a space where insights into research automation can be accumulated, and I encourage domain experts to contribute their expertise in areas that may be lacking. This way, I believe the entire endeavor of research automation can accelerate.

\subsubsection{Call for Resource Accumulation}
As with review papers, it is important to have a place to aggregate information on research automation so that those involved in automating research can focus on more substantive issues.

Zhang et al. summarizes various resources on AI for Science in a chapter titled ``Learning, Education, and Beyond.'' \cite{zhang2023artificial} The compilation of not only papers but also books and tutorials will be very helpful in getting a general overview of the field. This is an open collaboration project, so if you know of a resource that is not listed, you can send feedback on the project page or send a pull request on GitHub to further brush up on this list. You can send feedback on the project page or send a pull request on GitHub to further refine the list.\footnote{
Link to the project page: \href{https://www.air4.science/}{https://www.air4.science/}
}  

\subsection{Call for Open Collaboration}
We believe that research automation projects should be as open as possible. Firstly, by conducting open research, more people can participate in research automation projects. This leads to an increase in the amount of human resources devoted to the project, enabling faster progress towards achieving the objective. Secondly, sharing knowledge eliminates the need for reinventing the wheel. This reduces wasteful utilization of resources and promotes more efficient progress towards the objective. Thirdly, it removes barriers between individuals sharing the common goal of research automation. Currently, it seems that there is insufficient information sharing between different fields related to the same objective of research automation. For example, research on automating physical property prediction and research on information retrieval from papers seem to exist within different communities. By sharing information, valuable insights that are beneficial to one's own project may be obtained, potentially accelerating the project.

Below is a list of open collaboration projects related to research automation, and we encourage you to participate in the projects that interest you. However, this list is by no means complete. If you know of other open collaboration projects, please let us know.


\begin{longtable}{|l|p{8cm}|}
    \hline
    Name & About \\
    \hline
    \endfirsthead
    \hline
    Name & About \\
    \hline
    \endhead
    \href{https://www.air4.science/}{AIRS} & AIRS is a collection of open-source software tools, datasets, and benchmarks associated with our paper entitled “Artificial Intelligence for Science in Quantum, Atomistic, and Continuum Systems”.  \\
    \hline
    \href{http://www.example.com/projectB}{Project B} & About \\
    \hline
    \href{http://www.example.com/projectC}{Project C} & About \\
    \hline
\end{longtable}




% Firstly, I believe that research and development for creating a universal and autonomous artificial researcher should be carried out in an open manner. Furthermore, I think that all research automation projects should strive to share knowledge as much as possible. Firstly, by conducting open research, more people can participate in research automation projects. This leads to an increase in the amount of human resources devoted to the project, enabling faster progress towards achieving the objective. Secondly, sharing knowledge eliminates the need for reinventing the wheel. This reduces wasteful utilization of resources and promotes more efficient progress towards the objective. Thirdly, it removes barriers between individuals sharing the common goal of research automation. Currently, it seems that there is insufficient information sharing between different fields related to the same objective of research automation. For example, research on automating physical property prediction and research on information retrieval from papers seem to exist within different communities. By sharing information, valuable insights that are beneficial to one's own project may be obtained, potentially accelerating the project.


% \section{Others}
% When classifying efforts in research automation based on two axes: generality and autonomy, they can be categorized as follows:

% To reiterate, the proposition put forward in this paper is to strive for the realization of autonomous intelligent systems that can conduct research. This is because I believe it holds the potential to liberate research from the cognitive, historical, and social constraints that surround humans. Furthermore, I think it can lead to a future where better knowledge production occurs, ultimately enabling a broader and deeper understanding of the universe and nature.

% I believe that more people should engage in efforts aimed at developing artificial intelligence that autonomously conducts research on universal subjects. This is because there currently seems to be a limited number of such endeavors. Firstly, I think efforts in research automation can be classified along the axes of universality and autonomy. Universality refers to how many research fields a single automated research system can handle. For example, a system capable of conducting research in both physics and history would have higher universality compared to a system limited to either one. This can be understood as the ability to tackle universal questions. Next, autonomy refers to the level of human intervention involved. For instance, a system that requires humans to explicitly provide a set of hypotheses has lower autonomy compared to one that allows the machine to discover them on its own. From this perspective, previous endeavors in research automation can be organized as illustrated in the following diagram. (Diagram description). Thus, it can be said that there are still relatively few efforts focused on constructing highly universal and autonomous automated research systems.

% \textcolor{red}{TODO: Add Fig}

% Of course, efforts to automate research at all levels are important. For example, initiatives to automate a specific experiment in physics are valuable in their own right, as they aim to achieve the specific objective of enabling more efficient research. Therefore, the endeavors mentioned in this paper differ in their original goals, and it is not a matter of determining which is better or more important. Additionally, insights gained from these endeavors related to the construction of highly universal and autonomous systems are crucial. As achieving such systems involves high uncertainty, it is beneficial to start by creating concrete systems to increase problem resolution. This process will involve considering which aspects of a specific system can be abstracted to achieve universality and autonomy. Hence, the most important aspect is that all initiatives aspiring towards research automation gain momentum.

% \subsection{Goal and Research}
% First, let's discuss what it means for an AI to be able to conduct research. We delved into this topic in detail in Chapter 2. Research is the act of producing new knowledge for a community of constituents capable of forming shared beliefs. Producing new knowledge involves posing unanswered questions, formulating hypotheses in response to those questions, and verifying those hypotheses to update beliefs. Therefore, creating an intelligent system capable of conducting research means creating a system that can perform these tasks.

% \textcolor{red}{TODO: revision}

% \subsection{Goal and Autonomy}
% The ultimate goal is to create an artificial intelligence that can produce knowledge. In other words, the aim is to develop a function that can autonomously generate some form of knowledge when given any input. Therefore, any attempt to achieve the realization of an artificial researcher should strive to create such a function, regardless of its specific form. As we have reiterated, if the knowledge generated through research is a shared belief in a justified and true society, then the goal can be rephrased as aiming to generate a belief in response to any input and justify it to the extent that it becomes a shared belief in a society.

% \textcolor{red}{TODO: revision}

% \section{What to Do to Achieve the Objective}

% To create a universal artificial researcher, I believe it is necessary to automate the construction of questions, generation of hypotheses, and verification of hypotheses. This is because, as seen in Chapter 1, these functions appear to be essential in all research. In other words, by automating these functions as much as possible without relying on human intervention, we can move closer to the realization of an autonomous and universal artificial researcher.

% \subsection{question construction}
% The construction of a question is the act of seeking information \cite{watson2015ask}. Specifically, in the context of research, we consider information as knowledge. The act of seeking knowledge involves two steps: 1. Recognizing the lack of knowledge and 2. Attempting to fill that knowledge gap. In this discussion, we assume that intelligence is designed to consistently generate questions when given input. Therefore, we temporarily set aside the aspect of "triggering action" related to the second step of attempting to fill the knowledge gap.

% The recognition of a knowledge gap occurs when we expect to have certain knowledge and, upon referencing our accessible knowledge, we find that it is not available. For example, when running a program and encountering an error that we cannot resolve on our own, we recognize that we lack the necessary knowledge.

% The reasons for expecting the existence of certain knowledge can vary and are arbitrary. In this case, we assume that a purpose given by a third party creates an expectation of certain knowledge. For example, in the case of humans, we first consider what we need to do to achieve a certain purpose. We then anticipate the necessary knowledge to accomplish those tasks, and when we find that it is not present within our existing knowledge, we recognize the knowledge gap.

% Lastly, in this discussion, knowledge refers to the collective body of research findings, particularly academic papers. In actual research, a researcher may personally have a question and then investigate previous studies to confirm that it is indeed unknown before formulating it as a research question. However, what is important in the construction of a research question is that it is unknown to other entities. Therefore, for simplicity, we directly refer to the entirety of academic papers without including the step of comparing personal knowledge.

% To summarize, to create an intelligence capable of constructing questions in this setting, we need to design it to expect the necessary knowledge to achieve a given purpose provided by a third party, search for that knowledge in academic papers, assess whether the papers contain sufficient knowledge to achieve the purpose, and express any knowledge gaps as questions.

% In this case, we excluded the discussion of triggering action by design. However, when considering increasing autonomy, it is important to discuss how to incorporate this aspect into learning and acquisition. The question of "why do we seek information" has been extensively discussed in the context of curiosity.

% Furthermore, in this case, we defined the expectation of knowledge as aiming to achieve a given purpose. However, as mentioned earlier, this does not affect the formulation of questions. For example, let's consider the case of a child asking, ``Why is the sky blue?'' In this case, the child may already have prior knowledge of the concept of ``sky'' and ``blue.'' Additionally, they may possess a naive concept of causality, believing that ``A is B, so there must be a reason for it.'' Thus, they may have expected to have the knowledge that ``the sky is blue because of B.'' However, when they reference their internal knowledge, they find that it does not contain the corresponding knowledge. Therefore, they may have asked the question ``Why is the sky blue?'' to evoke the knowledge they were lacking.

% In this way, the reasons for expecting the existence of certain knowledge can vary, and what, why, and how we seek information (knowledge) are not constrained by specific conditions. Therefore, when attempting to create an intelligence capable of constructing questions in the future, it is feasible to develop a more flexible intelligence.

% Additionally, in this case, we assumed that the given purpose and its achievement are predefined goals. However, humans naturally set their own goals. When considering the design of a more autonomous intelligence, it is conceivable to aim for automation in this aspect as well. However, as mentioned earlier, the question of what we seek knowledge about is not specific to research. Therefore

% , we temporarily set it aside for now. If we were to pursue this direction further, it would ultimately lead to an infinite regress, raising the question of how much information to consider as given.

% \begin{figure}[htb]
%     \centering
%     \includegraphics[width=\textwidth]{figs/question_formulation.jpg}
%     \caption{question construction}
%     \label{fig:enter-label}
% \end{figure}

% \subsubsection{How to Identify the Necessary Knowledge for Achieving a Goal?}
% In this situation, we need to consider how to identify the knowledge required to achieve our objectives. Typically, when given a goal, we start by listing the necessary elements to accomplish it. For example, to achieve general artificial intelligence, we may think that it requires the ability to handle language, understand the real world, be proficient in mathematics, and align with human values. To understand the real world, for instance, we may need the capability for interacting with the physical world, processing visual information, and so on. These requirements can be further broken down into multiple necessary elements. By repeating this process, we can narrow down the specific tasks that can be directly addressed. Then, the required knowledge to accomplish those tasks is demanded, and that's where it directly connects to the research question.

% Several things are happening here. Firstly, listing the elements necessary for achieving the goal means generating sub-goals from the main goal. However, it's always a challenging problem to evaluate how a particular sub-goal contributes to the achievement of a given goal. Especially in the case of research, the target might be an too general and ambitious vision that nobody has achieved before, so we need to think about what needs to be done to break it down into appropriate sub-problems. In other words, it is necessary to construct a tree with nodes representing sub-goals.

% Secondly, it is necessary to identify the most important sub-goal from the selected candidate sub-goals. Since only one question can be addressed in the end, it is necessary to select a single sub-goal using some evaluation criteria. This may not be a problem if the sub-goals can be judged on the same evaluation criteria, but in many cases, unrelated sub-goals may arise. For example, to create general artificial intelligence, the development of machine learning theory may be equally important as the advancement of semiconductor technology. However, it may be difficult to compare their importance on the same evaluation axis.

% Thirdly, the question to ultimately arrive at must be verifiable. If the question is not specific, meaningful verification cannot be performed. Overly broad or ambiguous questions can result in countless or trivial answers, or they may be too unclear to provide practical answers. Increasing the specificity of the question corresponds to deepening the depth of the sub-goal tree, so it may be important to construct a sufficiently deep tree and find an efficient way to navigate it. The verifiability is constrained by the knowledge, resources, such as funding and technology, that we currently have. Therefore, when conducting verification in reality, it is necessary to consider such feasibility. Whether to tackle a question with high feasibility or to further divide it into more subtasks for its realization is a matter of judgment. In any case, it is necessary to appropriately evaluate such feasibility. The scope of feasibility is vast, so it is a challenging problem to determine how to consider it in creating intelligent systems.

% Here, we discussed what needs to be done to construct questions that generate the necessary knowledge to achieve the goal. We believe that research that can generate knowledge that is expected to contribute to the achievement of the goal, and that has not been revealed in previous studies, can be considered as ``important'' research. In that sense, it may be important to consider how to achieve the automation of research in this direction.

% In this discussion, we have only touched upon a limited number of points that we personally consider important. However, we believe that there are other important points that should be considered. By seriously reflecting on ``how questions are currently being constructed,'' these points are expected to become clearer.

% Also, in this discussion, we considered the method of outputting questions from the goal through the construction and exploration of a tree structure. However, as mentioned by the predecessors, if an end-to-end approach ultimately becomes a powerful method, it may be more desirable to consider a direction in which questions are directly output from the goal. In particular, even when performing multi-step reasoning, it seems more natural to improve reasoning abilities using the recently developed approaches to multi-step logical reasoning, rather than explicitly considering tree structures. While the development of methods for multi-step reasoning is undoubtedly important in pursuing this direction, this is not a point that needs to be emphasized here as it is being addressed not only in the context of research automation but also more generally. As a discussion focused on the automation of research, it may be worth considering the construction of higher-quality datasets for goals and research questions. For example, it may be possible to construct a dataset by extracting only the ultimate goal and the research questions actually solved from the introductions of papers.

% However, an important point to note here is that the research questions created by humans so far are not necessarily optimal for achieving research goals. Firstly, machines may be capable of maximizing the objective better than humans due to cognitive constraints. Secondly, not all human research has been conducted by working backward from a clear goal. Some studies were conducted simply because they seemed interesting while reading papers. Additionally, as mentioned earlier, abstraction has significantly advanced science, but there are cases where the study of abstraction itself is the

%  objective rather than abstraction in some specific cases. In this regard, simply learning from human data without imitation may constrain the potential capabilities that machines can possess. Therefore, it becomes important to consider how to formulate the maximization of the probability of achieving research goals as a problem, rather than naive learning from human data.

% \subsubsection{How to Search Necessary Knowledge and Recognize the Lack of Knowledge?}
% There are various methods to determine whether the desired knowledge already exists. Typically, it is considered that candidates can be narrowed down in two steps. First, since research papers are composed of questions and their corresponding answers, one can search for papers that have similar questions and answers to the desired knowledge. Once these papers are found, the validity of their verification methods is evaluated. If it is determined that sufficient verification has not been conducted, it can be concluded that the knowledge does not exist.

% In this case, the knowledge itself was considered as the research paper database, so the process involved searching and individually assessing the papers as described above. However, for example, if the knowledge is represented by the distributed representation of a machine itself, the need for the search step may be eliminated. Nevertheless, even in this case, the reliability of the machine's judgment can only be trusted if it has appropriately assessed the effectiveness of its own verification. Therefore, the ability to understand and evaluate verification seems essential in determining the novelty of research.

% \textcolor{red}{TODO: Is question construction information retrieval??}

% \subsubsection{Multiple Reasons for Unknownness}
% new, unimportant, difficult, unnoticed, ... etc.

% \subsubsection{Finding ``Important'' but Unnoticed Questions}

% \subsubsection{Conclusion}
% In summary, the following abilities are required to generate research questions for achieving a specific goal:

% \begin{enumerate}
%     \item Predicting the necessary knowledge from the goal:
%     \begin{enumerate}
%         \item Solving prediction and reasoning problems with long-range relationships.
%         \item Refining knowledge by appropriately incorporating ``good'' qualities such as importance, concreteness, ethics, etc.
%     \end{enumerate}
%     \item Determing the existence of expected knowledge:
%     \begin{enumerate}
%         \item Searching for knowledge directly related to the expected knowledge.
%         \item Determining whether the knowledge has been properly validated.
%     \end{enumerate}
% \end{enumerate}

% 1.a is related to research on improving the reasoning capabilities of machine learning models and on generating intermediate goals in reinforcement learning. If these research fields produce significant results, they can be directly applied. In this sense, it might be beneficial to seek cooperation from those who are actively conducting research in these areas. One of the unique aspects of long-distance inference problems that we think is interesting is the fact that the goal is something that has never been achieved before. This means that you cannot naively learn from data and need to generalize outside of the distribution. Therefore, it's essential to acquire skills not just to recognize patterns but to properly trace the path of reasoning. Moreover, because the goal has not been realized, sub-goals and the paths that connect them are ultimately based on the accumulation of hypothesis generation. In this sense, it can be said that this is a highly uncertain inference. This implies that the choice of which node to select is far from self-evident compared to other logical inference problems. Furthermore, there is the issue of the complexity of the distance between the goal and the question, which is far more intricate than, for example, games or planning everyday trips. For instance, to truly achieve the goal, it may be necessary to build large-scale apparatus like particle accelerators from scratch. This also means that the temporal distance between the goal and the current location is very long. Therefore, it becomes a problem that feedback on how much solving the question contributed to the goal is significantly delayed. While I've only listed a few examples here, there may be other unique challenges and issues that become more serious in research. It will be necessary to work on refining these technical challenges into specific research tasks through discussions with researchers in reasoning and planning.

% 1.b is specific to the automation of research. We first need to be aware that these "good" aspects do not occur naturally. As mentioned before, these are non-epistemic values, so the more autonomy the agent has, the more these must be consciously incorporated. To create an intelligence that constructs ``good'' questions, we first need to understand what we consider a ``good'' question. Also, it's important to turn our attention to things that are not currently considered ``good,'' but should be deemed as ``good'' in essence. Only then can we discuss how to align that value with the agent. Therefore, we think we should start by listing the criteria for determining the ``goodness'' of a question. For this, discussions in the philosophy of science and meta-sciences like the Science of Science may be referenced. Alternatively, large-scale surveys of researchers engaged in actual research could also be important. Once the value is clarified, we might be able to think about creating an intelligence equipped with these values using the value alignment techniques that are currently being developed.

% 2.a is a discussion of Information Retrieval itself. As previously mentioned, research is a process of searching for information from various sources and utilizing it to produce knowledge. Therefore, information retrieval technology is essential, not just for constructing questions. So, it's crucial how much we can get information retrieval researchers involved in this. One of the things strongly related to the automation of research within information retrieval is the search for papers. We will discuss this in the section on information retrieval.

% 2.b is a unique discussion about the automation of research. However, this greatly overlaps with the content to be discussed in the chapter on hypothesis testing. Specifically, what's needed here is the evaluation of testing, whereas what's required for the automation of hypothesis testing is the generation of testing. In this sense, the latter discussion includes the discussion here. Therefore, we will refrain from discussing this here and touch on it later.

% we have listed what we believe are important elements in the construction of questions. However, these are considered important under the assumptions mentioned earlier. For instance, if the goal is not to acquire knowledge necessary for achieving an objective, but to generate knowledge that an individual finds interesting, the necessary elements in question construction (particularly in parts 1.a and 1.b) would change. As previously mentioned, the value of knowledge is determined in relation to context and there's a high degree of uncertainty about how the value of knowledge will evolve in the future. This makes it fundamentally important to have a diverse range of ways to generate questions. The object achievement is highly prevalent and is expected to produce ``important'' knowledge, which is why it is discussed here. However, it is important to discuss what other ways of formulating questions could exist and how they can be implemented.


% \subsection{Hypothesis Verification}
% verification of hypotheses allows for any action to be considered as long as it is subject to verification, meaning that it updates an individual's beliefs. In this sense, it is believed to be extremely difficult to fully automate the act of verification universally and unconditionally. To achieve complete automation, the realization of physical robots capable of movements at least equivalent to humans in the real world would be necessary, which goes beyond the scope of developing intelligence.

% Therefore, to automate this process, it is necessary to proceed with automation gradually, starting from what can be automated. Two directions can be considered as ways to mitigate this problem. The first direction is to break down the verification process and gradually automate parts that are relatively feasible for automation using machine learning. The second direction is to prioritize the automation of research domains where verification is relatively easy. 

% \subsubsection{Start from Verification Design}

% First, let's explain the former direction. As mentioned earlier, we believe that the verification process can be divided into three stages: formulating a verification plan, preparing for the execution of the verification plan, and executing the verification plan. Among these stages, the execution of the verification plan and the preparation for it require interaction with the physical world in many fields. For example, in certain fields, you may need to purchase and raise rats for training, while in others, you may need to observe physical objects directly. On the other hand, formulating a verification plan is a process that is purely confined to the human mind in a wide range of fields. Of course, in many cases, interacting directly with the physical world can lead to better verification plans, but it is important to note that it is not an absolute requirement. Therefore, to automate the execution and preparation of verification, it is suggested that the power of robotics researchers be leveraged, while machine learning researchers should focus on automating the formulation of verification plans. 

% \subsubsection{Narrowing Down the Domain}

% Next, let's discuss the second direction, which is to prioritize the automation of verification in research domains where it is relatively easy. As mentioned earlier, the greatest challenge for machine learning researchers in the unified automation of hypothesis verification is the fact that some fields require interaction with the physical world for the execution of verification preparation. Therefore, the second direction is to focus on narrowing down the domains where the execution and preparation of verification take place solely within the computer world and aim for complete automation in these domains.

% Many machine learning research and information-related research fall into this domain. However, it remains challenging, but the problem has shifted from reproducing arbitrary movements in physical space to realizing arbitrary operations on a computer, which can be seen as a relaxation of the problem. Research on tool usage using language models that has gained momentum in recent years and research on browser automation are directly related to the automation of this process. If arbitrary operations on a computer can be automated as an extension of these studies, it becomes more realistic for many information-related research fields, fields that can be reduced to pure symbolic operations, and fields where all the resources required for research exist on the web to be fully automated.

% From this explanation, it is also clear that in this part of the verification process, which requires execution and preparation, even understanding what verification is may not be necessary as long as the verification plan is perfectly created. The only point at issue is the realization of arbitrary actions within the target space. In other words, as mentioned earlier, these two directions can be pursued in parallel.

% Based on the above, we propose that the automation of fields requiring interaction with the physical world be left to capable robotics researchers, and machine learning researchers should initially aim for complete automation of research confined to the computer space. Regarding the autonomy of automating verification plans, it is advisable to first create intelligence that can understand the verification concepts that humans perform. 

% Although we can proceed them in parallel, we believe that we should prioritize the automation of verification plan development. This is because, as mentioned repeatedly, the execution and preparation of verification do not require verification-specific capabilities. It aims to acquire more general abilities, and many individuals who are not pursuing research automation are also working towards achieving this. However, the realization of verification plan capabilities is a point of particular interest for those pursuing research automation, and it has not received sufficient attention nor progress unless driven by individuals aiming for research automation. Therefore, we believe that prioritizing the automation in this area would be beneficial.


% The ultimate goal is to create an artificial intelligence capable of conducting research autonomously. Therefore, in addition to automating individual modules of research, it is necessary to create a system that can autonomously carry out the entire research process from start to finish. Building such a research process pipeline is one of the sub-goals we aim for.

% This is similar to the pipelines discussed in MLOps in the context of machine learning or scientific workflows mentioned in previous studies. However, unlike those, we envision a system that does not include processing that is heavily dependent on specific research domains or tasks. Instead, we formalize it as a pipeline consisting of modules for question construction, hypothesis generation, and planning and execution of verification, as discussed in Chapter 2. This system autonomously constructs and validates hypotheses for any given question with an unknown answer.


% \section{Others}




\bibliographystyle{unsrt}
% \bibliographystyle{apalike}
\bibliography{ref}

\appendix

% \section{Question Construction, Hypothesis Generation, and Hypothesis Verification}

\section{Definition}

\subsection{Alternative Definitions}
% 真理促進的ではない。帰納推論だけでは難しい。

% So far, I have examined the consequences of perceiving research as the production of new knowledge for society, and knowledge as a justified true belief. While the current definition of research may be suitable for comprehensively describing the wide range of human research conducted so far, adopting this definition led to counterintuitive conclusions with seemingly intractable problems. Therefore, I would like to discuss how this definition can be modified, or what other definitions might be conceivable.

% \subsubsection{Limited to Human Knowledge}
% One of the biggest issues of the definition is that it implies that a fully autonomous artificial researcher is likely to produce knowledge that is meaningless to humans, or even useless to understand nature. To mitigate this problem, we may force knowledge an artificial researcher produces to be relevant for humans.

% One idea is to ensure that the foundation for validation remains fundamentally human, as I mentioned above. This is because the above problem arose entirely from allowing AI to autonomously construct even the foundation for verification. It is conceivable, for example, to make AI unconditionally and strongly believe in the validity of inductive reasoning and statistical hypothesis testing 

% Another alternative is to force AI to produce knowledge that humans can understand. This is the more restrictive definition because it is conceivable that there is a procedure where humans might be convinced something is true if that is confirmed by it even if they can't fully grasp how it works. It seems that what people desire is an AI that produces knowledge in this sense, so for those people, this definition may be more than sufficient.

% A downside of adopting these definitions is that they might preclude the production of knowledge that, while certainly explaining nature, cannot be verified within the framework of human validation. If one believes that nature transcends human cognitive limitations, it seems reasonable to assume that such knowledge exists. While this may not be a concern for many people, it certainly limits the possibility of machines to do research and hence understand this nature beyond human limits. 

% It seems that one of the purposes of creating an intelligence capable of autonomously conducting research is to advance the understanding of this world beyond human limitations. Considering this, I believe that it is desirable to remove such constraints if possible. Whether such constraints are fundamentally unavoidable, or whether it is possible to relax them in some way, is something that I hope will be further discussed in the future


% \subsubsection{Knowledge Consistent with Nature}
% An approach that could be considered in principle to relax the aforementioned constraints is to enforce a verification basis that is consistent with nature, without necessarily requiring a human verification foundation. 

% As mentioned earlier, the reason humans can generate knowledge useful for understanding nature based on belief systems may be because our belief systems have been formed through interaction with nature. If this is the case, there is a possibility that by allowing machines to acquire a belief system consistent with nature in the same way, they could produce knowledge leading to the understanding of nature without being constrained by human knowledge. For instance, it might be possible, in principle, to do this by allowing machines to interact with the external world in the real world to form its belief system, just as humans do.

% However, I have no idea how to create such a machine. Even if such a machine were to be developed, as it relies on a verification basis different from humans, we might not be able to judge whether it is functioning properly. For us to be convinced of the discovery by the machine, it needs to be reflected in our belief system, but this is no different from being consistent with our belief system. Therefore, even though it is theoretically conceivable, it seems impractical to realize this approach. Considering this, it seems necessary that the basis for verification should be human one.

 
% \subsubsection{Research as New Pattern Discovery}
% One of the main reasons the definition I adopted lead to counterintuitive results may be because I regard research as the production of ``knowledge'', and knowledge as ``belief''. I interpret research as the updating of beliefs just because it is consistent with human research practices so far. Rather, an essential aspect of research seems to be discovering or creating something ``novel'', and whether or not that constitutes ``knowledge'' seems secondary.

% Given this, it might be worthwhile to simply define research as the ``discovery of new patterns (in nature)''. This definition seems to at least encompass our current research endeavors. It emphasizes the discovery of new things, is not dependent on the subject, and, (by specifying ``in nature'',) appears to align with the goal of understanding nature.

% This might be too abstract, so it would be necessary to re-analyze what this definition truly means. I'd like to consider that discussion as future work.

% \subsubsection{Research as Question Answering}
% Defining research as merely a question-and-answer process might be one approach. However, a distinguishing feature of research is that the answer is unknown to anyone. Nobody knows the correctness of the answer, but it is indirectly provided by the subject of study, such as nature. While it's unclear to what extent this formulation can alleviate the issues with the definition we've adopted, it seems worth considering for furhter study.

\subsubsection{Pragmatist Epistemology}
Up to now, we have assumed that justification should be truth-conducive since this is what most researchers would expect. That is, I regard belief that reflect the truth of this world is knowledge. However, there is also a position that regard beliefs useful for human should be knowledge. This position is known as \textit{pragmatist epistemology}. 

From this perspective, it could be possible to argue that the outputs of current machine learning models, for example, may be considered knowledge since it shows high predictive performance, leading humans to the desired behaviours. In this sense, this position is highly compatible with statistical machine learning \cite{otsuka2022thinking}.

As such, this pragmatism presents a different view of research from that of majority of researchers, and it could be seen as advocating a redefinition of the act of research. Many researchers may not accept this view. Nevertheless, in recent years, deep neural networks have made rapid advancements and have produced many ``useful'' outcomes in research, which has made the debate on whether to consider their outputs as knowledge much more relevant and realistic than ever before.

In this paper, I take the position that knowledge is JTB, and research is an endeavor to approach a truth. Therefore, I will not address this issue in the following sections that much. However, we must seriously consider the implications presented by deep neural networks and delve deeper into the discussion of how we should define research, what constitutes research, and what purpose research serves.

\section{Hypothesis Generation}

\subsection{Generating Plausible Hypothesis}

Hypothesis generation is sometimes considered an unanalyzable, emergent process or a result of genius.\footnote{
Since 19th centuries, the act of producing knowledge, in particular hypothesis, and that of verification of it have been distinguished as the context of discovery and context of justification \cite{sep-scientific-discovery}. And during the most of the 20th centuries, the discovery caught much less attention by philosophers of science.

In engineering discussions, there is often explicit formulation of sets or candidates of hypotheses, and the discovery of hypotheses in such situations is often discussed \cite{simon1973does,kitano2021nobel,bengio2022ml4sci}. However, when automating hypothesis generation it is also important to consider how such candidates arise in the first place. There have been attempts to understand this process, such as highlighting the importance of analogy \cite{thagard1984conceptual} or mental model \cite{nersessian1999model}, but the question of how to generate ``good'' hypotheses still remains unanswered. The generation of initial hypothesis candidates has been discussed in the context of creativity in science. As you are well aware, however, current machine learning models are already capable of executing creative tasks effectively \cite{sep-creativity}. Thus, there is a debate about how much emphasis should be placed on creativity when considering the development of artificial intelligence capable of generating new hypotheses.
} However, I believe that human-generated hypotheses are produced through a process of trial and error in rational inference. In the following discussion, I will explore some examples of them.
% \footnote{
% Please note that not all research questions are why questions (e.g., ``Does life exist beyond Earth?'' is not a why question, but it is a scientifically investigable question).
% }

% \subsection{Plausible Hypothesis and Unknownness}

% While predictions that are highly unlikely to be confirmed, such as random guesses, can still qualify as research, they do not contribute much knowledge because they are expected to be easily rejected without undergoing rigorous testing. It becomes a waste of resources to invest in validating such predictions. Therefore, it seems to be crucial for ``meaningful'' research to propose hypotheses that are somewhat plausible.

% It is immediately evident that this is a non-trivial problem. This is because research, despite being an endeavor to answer unanswered questions, requires considering plausible candidate answers for those questions. Here, if the unknown under investigation is entirely unrelated to existing knowledge, it is impossible to make meaningful predictions about it. This is because predictions are based on experiences and data. In other words, high novelty and high uncertainty indicate a complex structure that cannot be immediately predicted from past knowledge and experiences. It implies that constructing plausible hypotheses necessitates the ability to discern these complex structures and patterns from past experiences. While it is difficult to determine what is unknown and what constitutes a complex structure, these can be crucial points to consider when advancing the automation of research.

\subsubsection{Analyzing Question}
Researchers start with a question, the composition of which has been discussed in the previous section. Given the question, we break down the content of the question and analyze each part in detail. For example, let's say you have a question, ``Why are apples red?'' The first thing you would consider is what an apple is and what red means. You would also think about what it means for something to be red. Then, you would focus on the properties of apples and the color red and abstract them. you may also think about other red things besides apples. If you already know the reason why tomatoes are red before knowing why apples are red, you might consider that the reason for tomatoes being red could apply to apples as well. By conducting this kind of analysis, you can connect your understanding to existing similar knowledge and attempt to explain using those existing reasons. These ways of focusing on abstract structural similarities between specific concepts and inferring that what can be said about A, which has the same structure, can also be said about B is called \textit{analogical reasoning}. This has been considered to be important method in hypothesis generation \cite{hesse1965models,thagard_1984,gentner1993shift,holyoak1996mental,dunbar1997scientists}.

For those simple examples given above, one can easily find analogical examples. However, many of the questions researchers actually grapple with are much more complex, and it's not immediately clear how they relate to existing knowledge. Even in such situations, researchers have managed to generate plausible hypotheses by thoroughly analyzing research question.

Such thorough analysis of questions seems particularly important when it comes to making significant discoveries that are remembered in history. Let's consider Charles Darwin as an example, who proposed the concept of natural selection. Darwin appears to have gone through a process of trial and error before arriving at the idea of natural selection \cite{gribbin2022origin}. After returning from his voyage on the HMS Beagle, he began to question how evolution occurs. It seems that he read the works of Lyell and Linnaeus, and particularly from Lyell's writings, he realized the importance of selection in evolution. The question then shifted to what could serve as a natural equivalent of artificial selection. Later, after reading Malthus' book on population, Darwin understood that in nature, competition leads to the preservation of advantageous species and the extinction of disadvantageous ones, which is the process of natural selection.

In essence, Darwin initially had the question of ``how does evolution occur,'' but through analyzing this question, referring to previous research, and conducting experimentations and observations, the question transformed into ``what is the equivalent of artificial selection in nature?.'' And it was through this transformation of the question that he was able to recognize the similarities between Malthus' discussion on human society and the mechanism he was seeking. Although the process is complex, this is the same as the hypothesis generation process I explained above in that he analyzed and transformed the question, finding analogies with existing knowledge and reaching a plausible hypothesis.

In the case of Darwin, it involved a more observational approach within the field of natural history, which prompted the transformation of his question. However, in fields such as theoretical research that utilize mathematics, different methods may be employed. For example, consider the scenario where there are two theories, theory A and theory B, and the goal is to achieve a unified description between them. In this case, one might repeatedly perform mathematical transformations on the objects described by each theory, reducing the significant problem of ``incompatibility between Theory A and Theory B'' to inconsistencies between specific properties of each theory. By introducing axioms that resolve these inconsistencies, it may be possible to construct a unified theory. It is important to reiterate that in cases where the subject matter is mathematically described, formal operations can be applied to objects of interest, such as object A and object B, to transform them into different forms. This process can lead to the discovery of unexpected similarities. It is through this approach that one can discuss the similarities between objects even when they may defy intuition or cannot be imagined through empirical means. 

Therefore, I believe that thorough analysis of question plays a powerful role in identifying similarities between objects, especially in cases where it is necessary to discuss the similarities between objects that may not be intuitively evident or imagined through experience. I believe that this is the core of the human hypothesis generation. The art of discovering structural similarities between two different objectives is called \textit{analogical reasoning}. This has been considered to be crucial for hypothesis generation \cite{hesse1965models,thagard_1984,gentner1993shift,holyoak1996mental,dunbar1997scientists,gentner2002analogy}. 

\subsubsection{Confirming Plausibility in Hypothesis Generation}

In the previous chapter, I explained how we transform questions by analyzing them and connecting them with existing knowledge. However, there seem to be countless ways to bring about changes in the questions. So, how exactly do we go about transforming the questions? Let's now describe the process of question transformation.

% In simple cases like ``apples are red,'' it may be sufficient to apply existing knowledge. However, in research, especially when tackling challenging questions, I believe that the following steps are involved. 

First, from the knowledge at hand, we select several hypotheses that are strongly related to the question and have a high level of confidence, and then proceed with accepting these hypotheses as premises. For example, the existing knowledge that tomatoes are red for reason A, strawberries are red for reason B  are strongly linked with the proposition that apples are red for reason C by the presence of the word ``red.'' Let's assume that you have a high level of confidence in the proposition that tomatoes are red for reason A, but only a low level of confidence in the proposition that strawberries are red for reason B. In this case, the reason A for tomatoes being red would be selected as the premise.\footnote{
In the case of humans, it may not always be the case that I prioritize knowledge with a high level of confidence. For example, let's consider a situation where I have recently read a book and acquired some highly impactful knowledge. In this scenario, even during the process of question analysis, I might be inclined to consider using this newly acquired knowledge because it has left a strong impression on us. This inclination is not solely based on its level of confidence, but rather because the knowledge has made a lasting impression, leading us to take it as a premise while analyzing the question. Like this, determining what knowledge humans choose to adopt as premises is a complex issue, and I refrain from delving into this issue in this paper.
}

Next, you will not focus on the parts of the question that pertain to the meaning of the premises or the deduced results that answer them. For example, going back to Darwin's example, when Darwin read Lyell's book, he took the proposition that ``selection matters in evolution'' as a premise. As a result, he left aside the question of ``what is important for evolution.'' Instead, he shifted his focus to solving the question of ``how does nature make selections?''

Once a certain premise is accepted, the next step involves examining the consistency between the consequences brought about by that assumption and one's existing knowledge. If introducing that premise does not lead to contradictions with strongly held beliefs, then the introduction of the premise is deemed acceptable. On the other hand, if introducing the premise results in contradictions with one's existing knowledge, it becomes a matter of choosing between the two. 

% This involves comparing the consequences brought about by the introduction of that premise with retaining one's existing knowledge and selecting the one that seems more favorable in some sense. For instance, Kepler compared the consequences of introducing the assumption of circular motion for planets with observational data and the results of his proofs, and eventually decided to discard the assumption of circular motion.

% or generating hypotheses that maintain consistency with this knowledge. Specifically, I generate hypotheses that seem relevant, and then verify that they do not contradict existing knowledge.

By considering the relationships between existing knowledge and new knowledge, and reevaluating which ones are deemed more plausible, one can undergo a process of transforming the question into its more essential form and generating plausible hypotheses. I seem to generate plausible hypothesis by transforming the question in like these steps.

\subsubsection{Continuity between Hypothesis Generation and Hypothesis Verification}

From the explanation so far, it becomes evident that in order to generate ``plausible'' hypotheses, we would conduct operations that involve changing our beliefs about the truth of those hypotheses. This seems natural since generating ``plausible'' hypotheses inherently involves such belief updates. However, this observation may offer significant insights into my current discussion. I define knowledge production as the updating of beliefs, which includes the construction of questions, the generation of hypotheses, and the operation I refer to as hypothesis testing, where I update my belief about whether a hypothesis is true. The implication of this discussion is that the process of generating ``plausible'' hypotheses and hypothesis testing share similarities. In reality, there are cases where hypothesis generation and justification are tightly connected in research practice \cite{arabatzis2006inextricability}. This leads to the question of whether we need to separate the processes of hypothesis generation and hypothesis testing.

However, I do not think that the fact that a single process play roles for both hypothesis generation and verification means that they also play the same role for knowledge production. I believe that hypothesis generation and verification play two distinct role for knowledge production, and that's why I discuss separate them as two distinct modules.

You may say that it is not entirely implausible to interpret both hypothesis generation and verification as being ``belief updates.'' Certainly, this similarity becomes more evident, particularly when contemplating the generation of ``probable'' hypotheses. For instance, let's consider a scenario where belief values are continuous. In this case, generating a hypothesis involves changing the belief value of a certain hypothesis from its original value of, say, 0 to something like 0.1. Subsequently, if multiple hypothesis candidates are considered, and one is chosen as probable based on previous research, the belief in this hypothesis might increase to around 0.3. Through the final verification of this hypothesis, the belief might reach approximately 0.9, resulting in the formation of knowledge. From this perspective, hypothesis generation and verification can be seen as the process of modifying beliefs.

However, when aiming for knowledge production within a specific society, there is a distinction. During the verification phase, the beliefs that are updated are shared beliefs. In contrast, during the generation of probable hypotheses, the updating of shared beliefs is not required. Consequently, in the context of knowledge production, hypothesis generation generates beliefs that can be updated, while hypothesis verification serves as the means to update those beliefs, thus playing distinct roles.

% \subsection{Continuity between Hypothesis Generation and Verification}
% So far, I have emphasized that hypothesis generation and hypothesis testing are functionally distinct, and that seems somewhat reasonable. However, when considering the generation of "plausible" hypotheses, these boundaries become somewhat ambiguous.

% Since testing incurs costs, in reality, humans carefully choose hypotheses worthy of testing. Even after hypothesis testing is made more efficient by machines, narrowing down hypotheses to a certain extent remains practically essential. Therefore, this is a problem that persists even after knowledge generation is achieved through machines.

% When considering the generation of ``plausible'' hypotheses, it simply means strengthening the certainty about the truth or falsehood of a hypothesis. In other words, in terms of changing the belief about the truth or falsehood of a hypothesis, this can be seen as similar to testing. For example, in experimental research, I believe there is often a pilot study conducted before starting full-scale experiments. This is nothing more than evaluating whether the hypothesis is worth pursuing for rigorous testing.

% Therefore, it may be possible to identify hypothesis generation and hypothesis testing as being synonymous in the sense of belief updating. However, I still believe that there are differences between hypothesis generation and hypothesis testing. The confidence in hypothesis generation is ultimately a matter of individual researchers or those involved in the research, and it is not necessary for all members of society to share that confidence. On the other hand, testing requires methods that update the common beliefs of members of society. Therefore, while hypothesis generation and hypothesis testing can be regarded as equivalent in terms of belief updating, they may differ in the strength of that belief.

% Obviously it is not possible to consider the consistency for all possible knowledge, so in practice, researchers seem to be checking the consistency of the hypotheses with several studies or knowledge that they have in mind. The knowledge  may be the recent interesting papers they have read, knowledge they strongly believe in, and so on. The knowledge that comes to the researcher's mind at that moment and is prioritized in their thinking is likely to become the premises for these hypotheses. 

% In highly mathematicalized disciplines like physics, for example, one might judge the plausibility of a hypothesis by comparing the deduced consequences from that hypothesis with existing knowledge. In this way, hypotheses that are judged to be somewhat consistent with the assumed premises may be recognized as ``plausible'' and worthy of rigorous testing. Of course, there are factors other than the plausibility of a hypothesis that can affect its ``value,'' but I believe that plausibility is the most important value in terms of knowledge production.

\subsection{Hypothesis Generation from Verification Results}
So far, I have been discussing the process of generating hypotheses by analyzing a given question. In this case, the hypotheses are intended to answer the question from start to finish. However, in reality, we may end up generating hypotheses that are for completely different questions than the one we initially set out to answer. Surprisingly, some of these serendipitously generated hypotheses can become profoundly important and leave a mark in history. Let's take a moment to consider how such hypotheses are generated.

Such hypotheses are born while attempting to identify the causes of verification results. Once a hypothesis is generated, the validity of it is assessed through hypothesis testing. In actual research practice, even if hypothesis testing yields negative results, it does not necessarily mean that the proposed hypothesis is completely discarded. Instead, supplementary premises may be introduced into the initially proposed hypothesis, and the modified hypothesis may be retested. On the other hand, it is also possible that the assumptions of the hypothesis are reconsidered, leading to the generation of new hypotheses. These are all practices of researchers involved in hypothesis generation, so let us explain them in more detail.

First, when humans think about something, they always need some assumptions. These assumptions refer to the knowledge or beliefs that individuals provisionally consider as ``correct.'' For example, when interpreting data at hand, one may assume that the measuring instrument used to generate the data is reliable, or that the principles of classical mechanics used for data collection are valid, or that the insights from several previous studies are valid. Additionally, researchers often introduce auxiliary hypotheses implicitly or explicitly during their investigations. For example, tentatively decided hyperparameters or auxiliary hypotheses introduced during calculations also become premises. Furthermore, there are implicit or explicit guiding principles and beliefs, such as the belief that ``hypotheses should be simple'' or that ``theories should be beautiful,'' which serve as premises for hypothesis derivation as well.

When verification yields negative results, all these assumptions, including the ones that are implicit, can be the cause for the result. The proposed main hypothesis itself may be incorrect, the auxiliary hypotheses may be incorrect, the observations may be incorrect, or all of them may be incorrect. Researchers must identify which of these possibilities is the cause. This is known as the problem of Duhem-Quine's thesis \cite{sep-scientific-underdetermination}. This is like trying to generate hypothesis on the question of ``What is the cause of the error?''. As you can understand, this is an extremely challenging problem since this is like debugging a system that is unstructured, where the entire code is not visible, and there is no systematic approach, and you have to grope in the dark. 

In practice, it seems that humans adopt a strategy of revising beliefs from the weakest ones first to tackle this proble. This strategy, in my opinion, is somewhat reasonable and rational. For example, let's say the parameters introduced in this study were chosen arbitrarily. This could be one of the first things to be revised because there is no reason for it to be that way. However, just because something was introduced in this study does not mean it will always be revised. For example, if results from the hypothesis under investigation are remarkably consistent with the background assumptions, you should keep the hypothesis. Alternatively, let's say the verification results are not completely off track but slightly deviated. In such cases, the fact that the verification results were negative may not make it reasonable to completely discard the hypothesis. The researcher may add terms or ad-hoc assumptions only to resolve those small errors or inconsistencies.

On the other hand, if, for example, the hypothesis of this study is based on classical mechanics, it would not be reconsidered unless there is a substantial reason to do so. Classical mechanics has been shown to be highly consistent with an immense amount of knowledge based on it, so revisiting it unconditionally would require an explanation sufficient to negate the entire system. In this sense, researchers will have a fairly strong belief in the validity of classical mechanics. The important point here is that, regardless of the reasons, when a researcher has a strong belief or takes certain assumptions for granted to the extent that they no longer doubt them, the priority of revisiting those assumptions is diminished. For example, during the time of Johannes Kepler, the belief that planetary orbits were perfect circles was widely shared as an unquestionable assumption. Therefore, it took considerable trial and error before Kepler started to question that assumption. Believing that planetary orbits are perfect circles and believing in the validity of classical mechanics are vastly different in terms of the reasons for belief, but they share the common aspect that researchers at the time strongly believe the principle, revisiting those assumptions would not occur unless there is a substantial reason to doubt them.

In this way, researchers start by revising less firmly held beliefs first and repeat this process until contradictions are resolved. In my opinion, this is a somewhat reasonable strategy. Of course, as with the belief in the circular orbits of planets in Kepler's time, there are beliefs that are implicitly assumed but not verified, and researchers still hold these beliefs. However, many strong beliefs are directed toward knowledge that has withstood numerous tests, and in that sense, it seems natural to first attribute the cause to weaker beliefs than to these strong beliefs.

Thinking in this way, I can understand why mathematics has played a crucial role in science. First and foremost, in mathematics, it is necessary to state the assumptions explicitly. Therefore, among all the potential influencing assumptions, only the assumptions under consideration are represented as mathematical objects. Furthermore, mathematics is a deductive system, so if any contradictions arise in the results, I can conclude that one of the assumed premises must be incorrect as long as the inference rules are correctly applied. Consequently, I can confine the process, which could have an infinite number of causes, to a finite and debuggable system for discussion. Additionally, when verifying the consistency of a hypothesis with the underlying assumptions used to generate it, the relationship of how the hypothesis is deduced from those assumptions can also be mathematically examined. This allows us to proceed with confidence, understanding the reasons why and to what extent I need to question each assumption. Finally, mathematics is abstract so is help us to find the analogical relationships. I believe these factors have hugely contributed for humans to generate plausible question.



% In research, a hypothesis is a prediction of the answer to a question that no one knows the answer to. Therefore, a ``good'' hypothesis is primarily one that represents the true answer to the question. To generate such a ``good'' hypothesis, I need to consider several factors. Here, I would like to discuss the reasons for the unknown that I previously discussed. Easy questions naturally lead to the generation of good hypotheses, and there is not much meaning in discussing the quality of hypotheses in such situations. Therefore, in this context, I will focus on how to generate ``good'' hypotheses for difficult questions that many people are challenging but have not yet been answered.

% I do not precisely know what it means for a question to be difficult. However, if I consider an analogy with machine learning, inference about patterns that rarely occur in the training data is challenging. Therefore, difficult questions may correspond to the tail part of the distribution during the training phase in machine learning or cases where there is a distribution shift (specifically, when the true distribution is mistakenly inferred during training). If that is the case, the ability to appropriately infer hidden patterns without being misled by spurious correlations in past experiences could be crucial in generating ``good'' hypotheses for difficult questions.

% Humans have employed various means to solve this difficulty. One approach is to incorporate new perspectives by borrowing knowledge from other fields or old papers. As I accumulate knowledge for research, I implicitly acquire the dominant ways of looking at things in that era. While it may not be false correlation, it undoubtedly makes certain patterns less visible. Bringing in insights from unrelated fields to the current domain might relativize these perspectives and provide a trigger to notice hidden patterns. Another approach is to leverage the power of mathematics. For example, in the field of physics, hypotheses for a certain question are built as theories with the help of mathematical tools. This involves deductive operations at various points, allowing for leaps of inference that go beyond human past experiences. I have also utilized various techniques such as analogies and the use of computers. However, all of these methods have been human approaches to overcoming broad out-of-distribution generalization.

% \textcolor{red}{TODO}

\subsection{Hypothesis Generation by Machines}

% Within a single process of knowledge production, they serve the role of being subject to verification and are mere beliefs if not tested or refuted. However, when considering the entire ecosystem of knowledge production, hypotheses play an extremely important role. This is because research is the act of generating new knowledge based on past knowledge and hypotheses are potential knowledge, justifying them indirectly influences future knowledge production.

% \subsubsection{What is Necessary for Hypothesis Generating Machine}
Hypothesis generation is just a prediction based on experience. As you all know, statistical machine learning is to do predictions about unknown from data. Moreover, if both question and hypothesis are expressed in text, hypothesis generation is simply question-answering task in machine learning. In this sense, the generation of hypotheses is exactly what statistical machine learning does, and in terms of formulation, there is nothing fundamentally new as a machine learning problem.\footnote{
I stated that hypotheses are simply predictions. However, even if their validity is verified in a manner that other members find acceptable, if the content of those predictions cannot be interpreted to other society member, it seems that the predictions generated there cannot be called knowledge of that society. This is because common beliefs are not formed. Thus, it seems necessary for hypotheses to adopt a form of representation that all members can interpret the content.
} In fact, numerous machine learning techniques have already been applied in scientific research, with many of them using machine learning models as generators of hypotheses. If that's the case, are there any specific technical challenges that need to be addressed when creating artificial intelligence capable of generating hypotheses in research? 

% \subsection{Machine Prediction and Hypothesis}
% The issue that becomes important here is the problem of representing knowledge and hypotheses. As mentioned earlier, research is the process of generating knowledge for a society constructed by certain agents. Therefore, it is necessary for the produced knowledge to be interpretable, or at least usable, by at least some members of the society, even if not by all of them. In human society, it seems that knowledge is made possible by expressing it in a form understandable to the members of the society, such as natural language or mathematical language.

% \textcolor{red}{This (commented out sentences) is about hypothesis generation automation so will be moved to survey section of perspective section}

% Particularly, is it merely a result of human cognitive constraints that I explicitly transform a question statement to find analogies, or is this an important aspect in hypothesis generation not limited to humans?

\subsubsection{In the Case of Hypothesis Generation Through Question Analysis}

In this chapter, I mentioned the possibility that humans formulate plausible hypotheses by analyzing questions and finding analogies with existing knowledge. However, the discovery of analogies, while complex, is also essentially part of the formulation of machine learning since it is merely pattern recognition. I already know that machines can find intricate patterns that humans might not discover without heuristic feature engineering. Then, is explicitly conducting such analysis and transformations of questions an essential aspect of hypothesis generation, even when machine conduct research? Or is it simply a result of cognitive limitations in humans?

The answer to this seems to vary depending on who the question's answer is unknown to. Firstly, let's discuss the scenario where the answer to the question is unknown to humanity. This scenario corresponds to utilizing artificial intelligence just for a hypothesis generator for humans. In this case, it may not necessarily be required for machines to undergo the intermediate step of explicitly converting the question, as it is possible for them to directly generate an answer to the question. This is because what is unknown to humans may already be known or obvious to machines. Currently, artificial intelligence demonstrates capabilities that surpass human abilities, and as their capabilities continue to advance, this trend is likely to accelerate further.

Next, let's discuss the scenario where the answer to the question is unknown to the machine itself. This corresponds to the situation where artificial intelligence itself engages in research as a researcher to explore the unknown from its own perspective. Here, artificial intelligence aims to generate knowledge rather than being a tool for human research. This is what I would like to realize in the future. In this case, I hypothesize that explicitly performing question transformation is important in hypothesis generation for machines, even if it doesn't need to be done in the same way as humans. If the most plausible answer for a question is always correct for the machine, it can be said that the question was not unknown to the machine in the first place. Therefore, if a question is unknown to the machine, it means that the machine needs to employ some means to extract patterns or structures that it is not capturing from the question. This is similar to situations where a machine is overfitting, being misled by spurious correlations, or failing to extract the patterns it should extract due to lack of out-of-distribution generalization ability. These are issues that will inevitably arise as long as machine learning continues to focus on minimizing errors as its central objective. For a machine to autonomously engage in research, it needs to be aware of being in such a situation and take actions to grasp the currently unapprehended patterns. Analyzing the question and converting it into a different form to facilitate the discovery of unknown patterns is precisely one of the most fundamental efforts in this regard. Therefore, it appears crucial for a machine to undergo such transformations in order to autonomously engage in research.

\subsubsection{In the Case of Hypothesis Generation from Verification Results}
In the sense of seeking causes from outcomes, this type of hypothesis generation can simply be described as causal inference. However, the difficulty in this type of hypothesis generation lies in the fact that there are countless possible factors that could be the causes behind the outcomes. Furthermore, it can be said that the more strongly I believe in a premise, the more it may not even be recognized as a premise to begin with.

One of the measures to address this is obviously to carefully design a verification plan. However, since this topic overlaps with the discussion in the section of hypothesis verification, I won't delve into it here.

One naive approach to tackle this issue in machine learning is to prepare data comprising a verification result and all possible premises, then train the model to predict the causes of verification. In fact, humans have also become capable of consciously considering premises that might influence experimental results based on factors that affected past experiments. For instance, if one had previous experience where room temperature unexpectedly influenced experimental outcomes, they would likely be more attentive to examining the impact of room temperature in subsequent experiments.

Additionally, instead of doubting all assumptions equally from the beginning, limiting the questioning to newly introduced assumptions for the current study can help narrow down the premises to be considered. Moreover, explicitly distinguishing the task of questioning assumptions with strong beliefs that might not be recognized as premises could be beneficial. In practice, even in human research, questioning these aspects based solely on the result of a single verification is not a common approach.

\textcolor{red}{TODO: FIX}

\subsubsection{How to Free from Commonsense}

One possible issue is that forcing machines learn the current value society has from vast amount of data might impede finding a innovative hypotheses. This is because such agent would be strongly influenced and biased by the current prevailing beliefs of the society. For example, during the time of Kepler, most people believed that planets move in circular orbits. If a machine trained in such an era, reflecting the biases of those people, were to generate hypotheses, it would likely exhibit a bias towards supporting circular orbits. The way I instill values into machines is precisely akin to this. For instance, RLHF (Reinforcement Learning from Human Feedback) essentially make agent do what the labelers have deemed correct. Teaching current human societal common sense to machines, not limited to RLHF, always leads to such issues.
 
% The problem that arises here is whether such learning processes can lead to the proposal of innovative hypotheses. For example, during the time of Kepler, most people believed that planets move in circular orbits. If a machine trained in such an era, reflecting the biases of those people, were to generate hypotheses, it would likely exhibit a bias towards supporting circular orbits. Therefore, it seems that more than just accepting what many people say as correct, some criteria, goals, or frameworks beyond that are necessary.

% Another issue is that a fundamental principle that what others consider plausible knowledge is deemed plausible as well. For example, widely used techniques like instruction tuning involve supervised learning where the training data consists of human or machine responses. In essence, the optimal solution is to mimic the behavior of these agents. Similarly, RLHF (Reinforcement Learning from Human Feedback) essentially learns to consider as correct what the labelers have deemed correct. In both cases, the common aspect is that the reward or training data tends to mimic the behavior of other agents in society.

As mentioned earlier, mathematics has been a powerful tool for freeing humans from such biases. Regardless of one's strong personal beliefs, the strength of deduction lies in the necessity of accepting the results that emerge from appropriate formal operations. Thus, it seems crucial in hypothesis generation to possess the ability to manipulate some form of deductive or system, even if it is not the same mathematical system as humans.

Another aspect that can be learned from human examples is that humans determine their level of confidence in knowledge through their own verification or tracking the verification conducted by others. This is closely related to the topic of verification, which will be discussed in the next section. Humans read books, papers, and other sources, carefully examining the claims and the verification processes presented in them. When they judge these sources to be trustworthy, there seem to be a significant number of individuals who increase their confidence in the validity of that knowledge, regardless of how it may differ from common sense. This attitude contrasts with the attitude of ``it's correct because many people say so'' and is referred to as a scientific and critical attitude. And, as I mentioned earlier, I believe this attitude can be rephrased as an approach that evaluates the reliability of knowledge through verification. Without such an attitude, it would be difficult to propose innovative hypotheses that challenge established theories. Therefore, the ability to verify hypotheses, as explained in the next section, also seems important in creating machines that generate hypotheses.

when inferring Boolean concepts, people can generate previously unseen hypotheses by using compositional rules, instead of likening the
situation to previously observed situations. \cite{goodman2008rational}

% \subsubsection{Conclusion}
% In this section, I have presented a hypothesis that the tasks involved in hypothesis generation may vary depending on who the answer to the question is unknown to. If the answer to a hypothesis is unknown to humans, it may be sufficient for the machine to generate questions directly based on the hypothesis. However, if the answer to a hypothesis is unknown to the machine, I have suggested that the machine needs to analyze and appropriately transform the hypothesis.

% Nevertheless, it is worth noting that the aspects discussed above can also be useful in the context of hypothesis generation for humans.

\subsection{Understanding and Explanation of Why a Hypothesis is True}
In research, an explanation often refers to making the process of hypothesis generation understandable to anyone.\footnote{
The need for explanations arises during hypothesis generation because hypothesis verification should be transparent to everyone, and the questions can be arbitrary. As mentioned earlier, it is possible to abandon hypothesis verification, but this would lead to serious issues in the entire knowledge production process. Since this article assumes hypothesis verification, I will not touch on this aspect.
} For example, when proposing a theory, others can understand why the theory holds by examining the proof procedure that led to the theory's derivation. Additionally, if I simplify phenomena into a simple model while generating hypotheses, I can gain a clearer understanding of what the hypothesis implies. Moreover, as previously mentioned, even without using mathematics, I generate plausible hypotheses through deductive reasoning based on assumptions. By examining this reasoning, I can understand why I chose to investigate that particular hypothesis.

When automating hypothesis generation, how to handle explanations becomes an important issue. This is because hypothesis generation does not necessarily require an explanation of the generation process. The purpose of hypothesis generation is to provide answers (candidates) to questions, and it is not always necessary to explain how those answers were arrived at. In fact, current mainstream machine learning methods such as deep neural networks trade interpretability for high hypothesis generation capability, making it difficult for humans to understand why certain inferences were made or why specific results were produced. The interpretability issue arises when using deep neural networks for hypothesis generation in scientific research. Increasing explainability for humans might come at the cost of reducing the hypothesis generation capability of the neural network, creating a trade-off relationship.

It is true that humans have historically provided explanations for hypotheses. However, this might be due to certain cognitive constraints. As mentioned earlier, humans could only generate plausible hypotheses by analyzing questions and engaging in logical reasoning. On the other hand, as I already know, machine learning models can directly extract complex patterns from data and answer questions that humans might not be able to.

So, is explanation merely an unnecessary byproduct of hypothesis generation, or does it serve some useful purpose? Eliminating explanations could lead to the following problems: 1. Reduction in the amount of information obtainable from knowledge production, and 2. Decline in future knowledge productivity. The reduction in the amount of information obtainable from knowledge production is a clear loss if explanations are deemed unnecessary.

The potential decline in future knowledge productivity arises from the fact that explanations increase the reusability of knowledge. Throughout history, humans have constructed hypotheses in the form of models with a small number of variables to explain various phenomena. 
\textcolor{red}{TODO: Explanation}

Regarding approaches to deal with the issue of interpretability, I believe there are at least three stances. First, one may completely abandon the explanation of hypothesis generation. As repeatedly mentioned, whether or not an explanation exists for how hypotheses are generated does not affect the reliability of knowledge as long as the verification process is transparent to us. Concerning the reusability of knowledge, it is not an issue if machines can reuse knowledge among themselves, even if it is not interpretable to humans. Therefore, this approach aims to separate interpretability and the reusability of knowledge and seeks ways for machines to reuse knowledge among themselves.

Second, one may adopt a stance where machines are required to produce explainable outputs. In this case, I should separate the process of hypothesis generation from the explanation of that process. Demanding that hypotheses be generated using interpretable "methods" can place significant constraints on the machine's capability for hypothesis generation. Moreover, since machines themselves do not produce knowledge in a way interpretable to humans, there are limits to understanding by examining the machine's internals. Instead, it might be better to have the machine itself translate the process of its output generation into a form understandable to humans. This is because the machine itself knows how it generated the hypothesis and is likely to perform better than humans in translating between different agents' tasks in the future.

Third, one may not require explanations from machines but instead seek to enhance my own human abilities to understand. This is because, under human cognitive constraints, understanding what machines are doing might seem limited. Therefore, the goal is to remove cognitive constraints and expand human capabilities. This is a long-term objective that cannot be achieved immediately, so I will not delve into it here. However, this is one possibility that should be seriously considered when thinking about the alignment between humans and AI, not limited to explaining hypotheses.
% \subsection{Prediction, Explanation, and Understanding}
% Humans, in order to derive hypotheses in response to questions, often require thinking and deliberation on why those hypotheses are considered plausible. Without such cognitive processes, it is difficult to generate hypotheses in many cases. This is not merely a cognitive constraint; it has played a significant role in the advancement of research. The explanation of how a hypothesis is derived and the reasoning behind its plausibility have provided crucial additional information for understanding the subject of study. This has assisted people in gaining a better understanding of the phenomenon.

% As mentioned earlier, predictions do not necessarily require such processes. Even without explanations for how such predictions arise, one can make inferences, and if they are rigorously tested, they can be considered knowledge. However, knowledge lacking such important additional information is expected to contribute less to the understanding of other knowledge and the comprehension of the entities targeted by the knowledge system. This research and understanding problem becomes significant even when artificial intelligence engages in research.

% \textcolor{red}{TODO: Add more}

% If it were not necessary to make inferences about the unknown, it would probably be because the question was not unknown in the first place. Thus that is not research by definition. It may not be explicitly stated, but as long as it is research, it implicitly involves some form of hypothesis generation. Therefore, the generation of a hypothesis, or the inference about the unknown, is essential for research in nature. Let's say, mathematics, a pure deductive field of research. In searching for lemmas to prove a theorem, I sometimes make predictions such as ``this lemma might be useful'' and examine it with few specific examples to check its usefulness. This may be considered as implicitly establishing a hypothesis and roughly verifying it. 

% In particular, science, by explicitly dividing the proposal of this plausible temporary answer and the corroboration of its tentative certainty into steps known as hypothesis generation and hypothesis verification, has enabled these processes to be carried out more systematically and rigorously. This is the well-known hypothesis-testing method, or scientific method. In this approach, a prediction about the unknown is explicitly stated as a hypothesis, and a procedure called verification is established to evaluate the validity of this hypothesis. Through this verification process, the evaluation of the hypothesis is conducted and the uncertainty towards the target unknown is reduced. This is the knowledge production based on the hypothesis testing method. 

% As a practice of human science, there are cases where discovery and justification are not strictly separate in actual scientific endeavors \cite{arabatzis2006inextricability}. However, functionally, discovery and justification can be classified separately, and it seems convenient to consider them as distinct in knowledge production that is not dependent on human conventions. Therefore, I will discuss them separately here.

% Although I said that hypothesis generation is the important part of scientific method, the hypothesis generation is not necessarily unique to the empirical science. It is a task that inevitably arises when dealing with the unknown. 

% \subsubsection{Note on Unknown}
% Let us make a note here about the term unknown again. I wrote that if an answer can be immediately derived by inference, it was not unknown in the first place. However, research is an activity to transform the unknown into the known, so in some form, unknown should eventually approach the known (reducing uncertainty). Moreover, if there is something that seems to be completely unrelated to the known or experiences (though it is hard to imagine), it seems inherently impossible to bring it closer to the known. In that sense, it is reasonable to think that the unknown has some relationship with the known. Therefore, the above description might be more accurate if it refers to cases where the confidence in the answer is lower, or the path to the answer is complex, and uncertainty is high. This is similar to the discussion on the definition of unknown in the previous section.

% \subsection{\textcolor{red}{Reference}}

%%%%%%

% \subsubsection{Good Hypothesis}

% As emphasized above, what I aim for is not merely imitating human knowledge production, but rather the autonomous practice of knowledge production that is not bound by human conventions. And in terms of its functionality, hypothesis generation is identical to inference towards the unknown. In that sense, while understanding how humans generate hypotheses can be very useful in considering how to achieve effective hypothesis generation, I do not consider it is always necessary.

% Of course, hypothesis may have additional constraints on top of the prediction. For instance, there might be an implicit assumption that a hypothesis is something described in natural language. However, I think this is only due to the conventions of current human society. What is functionally important for the purpose of knowledge production is to provide a tentative answer to the unknown, regardless of its form or property. In this sense, in this paper, I take the position that hypothesis generation is the prediction.

% Indeed, much of the current discourse on hypothesis generation focuses on specific research domains or discusses the exploration of hypotheses based on hypotheses that humans have explicitly defined. I understand this situation because hypothesis is an answer to a tentative question and hence it varies depending on the question. For example, if a study aims to enhance my understanding of a particular phenomenon, then the description or mechanism of that phenomenon would become the hypothesis. Similarly, if the objective is to solve a particular problem, then the solution to that problem would be considered the hypothesis. However, it is essential to consider how I can realize agents that construct candidates rather than just search for them, regardless of the specific research domain. This is a crucial question to address when contemplating artificial researchers.

% \subsubsection{Others}

% In engineering research, as part of new knowledge, it is often required to propose actual design plans or algorithms. This can be considered as having a similar function to a hypothesis in the sense that it is a proposal for addressing a problem and is evaluated in some way.

% In mathematical research, proving a theorem that was previously unknown is the production of knowledge. However, since mathematics is a deductive system, if the proof is correctly executed, it can be said to be ``correct'' in that sense. In other words, the proof itself is both the proposal and the verification. Therefore, mathematics is not a type of work that separates hypothesis and verification.

\section{Hypothesis Verification}

\subsection{Three Steps for Verification}

As mentioned above, I believe that the process of verification is highly diverse. This diversity makes achieving an end-to-end approach immediately difficult. Therefore, it would be better to start by further breaking down the process of hypothesis verfication into more detailed sub-processes. For a practical first step for the verification process automation, I propose tentatively dividing the verification process into three stages: \textit{verification design}, \textit{verification instantiation}, and \textit{verification execution}. At each stage, you will create a verification plan, prepare for the verification, and carry out the verification.

\begin{figure}[htb]
    \centering
    \includegraphics[width=\textwidth]{figs/verification.jpg}
    \caption{Verification}
    \label{fig:verification}
\end{figure}

\subsubsection{Verification Design} 
In verification design, the agent will create a plan to execute the verification, which consists of hypotheses, verification criteria (criteria to determine if the output confirms or refutes the hypotheses), and the procedures for conducting the verification. The verification plan should be specific and detailed enough so that anyone faithfully executing it can reproduce the same verification process and result. For example, suppose I am considering an experiment to compare a proposed method with a baseline and validate the hypothesis that the proposed method performs better. In this case, the plan should describe in detail which model to use, what dataset to employ, and how to measure the performance. Thanks to this specification, I can separates the two functionally distinct stages of verification: the phase of considering what needs to be validated and the phase of actually executing the verification.

It is during the verification design stage that specific skills for hypothesis verification are required. This is because automating the preparation and execution of a plan involves the automation of more general intellectual tasks, not limited to research automation, while creating a verification plan demands an explicit understanding of what verification entails and what criteria constitute a successful verification. In aiming to automate this verification plan, the key lies in teaching machines how to understand the concept of verification.\footnote{
Even humans may sometimes do verification without proper understanding of the method. For example, someone may conduct hypothesis testing just because ``everyone does it.'' While this may be an extreme example, it is not common for researchers to understand the epistemological implications of inductive reasoning and statistical inference and in what sense I can say I verify a hypothesis. In that sense, the requirement for machines to understand verification may be somewhat challenging. However, if the aim is to truly enable AI to autonomously conduct research, it appears crucial for the AI to have a proper understanding of what constitutes verification.
}

% Multiple abilities are needed to create a verification plan, but based on the discussion so far, it is evident that at least two abilities are required.


% At this stage, I contemplate how to validate the hypotheses and proceed to formulate a verification plan. In a verification plan, the agent writes about the hypotheses, verification criteria (what will be considered as evidence for verification), and the procedures for conducting the verification. These plans should be as much detailed as possible. The Fig. provides an example in the context of machine learning. The idea is to create a blueprint for a pipeline where verification is automatically executed by faithfully following the plan.

% Furthermore, for artificial intelligence to perform autonomous verification, it seems essential that the AI not only adopts certain verification criteria but also be able to explain the meaning behind them and how they contribute to the verification process. Even humans may sometimes engage in this without conscious awareness, such as conducting hypothesis testing because ``everyone does it.'' In that sense, this requirement may be somewhat challenging. However, if the aim is to truly enable AI to autonomously conduct research, it appears crucial for the AI to have a proper understanding of what constitutes verification and be able to design it itself or, at the very least, explain it adequately.

There are several abilities that a machine must possess in the context of verification planning, but based on the discussion so far, it is evident that at least two abilities are required: understanding what verification is and planning to achieve objectives.

Let's first discuss the ability to understand verification. As mentioned in the sections on question construction and hypothesis generation, this is an extremely crucial ability in the overall automation of the research process. As before, the extent to which ``understanding'' is required depends on how much autonomy is expected in the automation of verification. If that is used as a verification tool for humans, what machines to do is just using human verification methods properly. On the other hand, if they were required to understand the concept of verification from scratch, I would face a lot of challenges as I just described in this chapter. 

If you aim to achieve the former, one of the initial steps could be creating a dataset specifically tailored for verification. By gathering research papers that utilize widely used verification methods like statistical hypothesis testing, you can train a model to construct verification methods from these hypotheses. One naive approach could be to start by generating method descriptions in research papers from the given hypotheses. 

If you were to achieve the latter, you might want to begin by contemplating what beliefs mean for machines. Alternatively, placing agents in situations where verification becomes necessary could implicitly help them acquire the concept of verification. In this case, addressing the issue of generalization across environments, such as enabling machines to explicitly reuse the concept of verification, will be crucial. Eventually, as repeatedly emphasized, discussing how to resolve the alignment problem will be necessary.

% If artificial intelligence is used as a tool for human research, it is sufficient for artificial intelligence to faithfully reproduce what humans do as verification. For example, it would be great if it could use basic concepts such as hypothesis testing, controlled experiments, and interventions and automatically create experimental plans based on them. In this case, it is not necessary for the machine to strictly know why it constitutes verification, as long as it can learn from numerous examples of human verification and use it appropriately. In other words, in this case, the required understanding can be described as indirect and practical understanding through examples of human usage. Furthermore, if it can understand the concepts of hypothesis testing and controlled experiments from first principles, it would be a significant achievement in terms of automatic verification.

% On the other hand, if artificial intelligence itself is allowed to conduct research for its own knowledge generation, it seems that artificial intelligence needs to understand what verification is, whether explicitly or implicitly. And similar to the unknown nature of the answer to a problem, it seems to be an ability that cannot be acquired just by observing examples of human verification. This is because verification involves updating the beliefs of the members of a society, which depends on the nature of the beliefs of the machine group itself. Whether the knowledge generated by such agents is understandable by humans or can say something about nature is not obvious, but this will be discussed in detail in later chapters.

Let us move on the second ability: planning. Understanding verification is a necessary condition for setting verification criteria by considering what can be tested against a given hypothesis. When conducting actual verification, under the assumption of a hypothesis and verification criteria, one must devise procedures for carrying out the verification. For example, in experimental research, let's say a hypothesis A is formulated, and a verification criterion is established that considers the hypothesis valid if it meets certain criteria through statistical hypothesis testing. To actually perform this verification, it is necessary to generate data to be used for verification, and if there is no apparatus to generate the data, one may need to create it. Thus, in a verification plan, one must develop a plan to fulfill the purpose of executing the verification.

Creating a plan is known to be a challenging task, not limited to verification plans. To achieve a goal, one must understand what is necessary and comprehend the appropriate sequence of steps to achieve the goal. As mentioned earlier in the section on constructing questions, when creating a plan, it is essential to consider feasibility, taking into account the complex external factors such as current financial capabilities and accessible resources. Understanding and incorporating these constraints appropriately can be a highly challenging task, as they are intricately linked to various external factors.

While there are already a difficult task, the particular challenge in creating verification plans for research lies in the fact that one may need to create things necessary for verification if they do not already exist. This is an extremely high-level difficulty problem. To carry this out, one must first accurately identify what is currently lacking. After recognizing the deficiencies, one must consider how to create what does not exist. Once the method is known, materials for creating it must be prepared, and it must be actually built. And even if it is created, one must investigate whether they function properly. This is an unbelievably complex task, and it seems highly unlikely to simultaneously expect the creation of such intermediate products by directly generating a verification plan from the verification itself. Therefore, in aiming for true automation of verification, it is crucial to seriously consider how to solve this problem.


Finally, once the necessary elements for verification are understood, they need to be represented as something that can be understood by other researchers. This is not a requirement of a verification plan but a necessary condition for knowledge to become knowledge for society. In order to generate knowledge for society, it is necessary not only to verify hypotheses but also for the verification procedures to be understandable to other members of society and judged as valid.\footnote{
While it is desirable for the question generation process and hypothesis generation process to be publicly available, it is of utmost importance that the verification process is disclosed. 
} Also, as discussed in the section on hypothesis generation, verification must ensure that the causes behind the verification results are traceable as much as possible. While this is not an inherent requirement of the act of verification a hypothesis, it is a crucial characteristic needed for proper interpretation of verification results and for better knowledge production through hypothesis testing. As mentioned earlier, it is impossible to enumerate all potential assumptions, so it becomes necessary to explicitly state assumptions as comprehensively and in as much detail as possible to make the process realistic. Recognizing the underlying assumptions, including those behind the scenes, and selecting which assumptions to prioritize as potential causes of the results, is indeed a challenging task.



\subsubsection{Verification Instantiation} 
At this stage, the research plan that has been developed is translated into an executable instance. For example, if the research plan states, ``Train the model B with the dataset A...'' the necessary steps would involve acquiring dataset A from the appropriate source, formatting it to be compatible with the model's input requirements, and preparing the data for training, etc. Similarly, if the plan states, ``When a rat presses the switch B, food is dispensed...'' the agent has to prepare rats, food, and create a machine that dispenses food upon pressing the switch, and so on. This process involves translating the plan into physical or computational instance for execution.

\begin{figure}[htb]
    \centering
    \includegraphics[width=0.5\textwidth]{figs/verification_instantiation.jpg}
    \caption{Verification Instantiation}
    \label{fig:verification_instantiation}
\end{figure}

As evident upon reading, this process is highly challenging to automate. Even research confined to the realm of computers, such as research of computer science, requires accomplishing a vast number of complex tasks. For research that necessitates physical realization in the real world, the development of robotics and embodied agents are necessary. Regarding the question of where and how to tackle these problems, I will provide my perspective in the later chapter. However, it is important to note that unless the research is constrained by questions and hypotheses, aiming for true automation will inevitably require overcoming these challenges

To reiterate, this process can be accomplished through the automation of actions, and it does not necessarily require unique technological developments for research automation. For instance, in studies that are entirely computer-based, automating all actions within the computer system would suffice. In recent times, attempts have been made to have language models operate computers, and the progress in such research corresponds to the advancement of automation in this process. If the research involves activities in the physical world, achieving full automation of human movements (or their equivalents) would make this process feasible. Therefore, research focused on developing humanoid robots that can move like humans will contribute to the automation of this process.

% Once a hypothesis has been established, a verification plan is created to determine how to verify it. The specific method of verification depends on the subject being investigated, making this aspect of research difficult to structurize and automate in a unified way.

% However, in many empirical sciences, the likelihood of a hypothesis is evaluated based on statistical significance. This is done by \textit{hypothesis testing} in practice. As this is a hypothesis test, it can only reject the null hypothesis, rather than directly determining the correctness of the hypothesis. Therefore, it can only be said that the hypothesis has survived for the time being. The belief that the surviving hypothesis is more likely to be valid is the basis for decision-making.

% In any case, humans seem to use statistics or probability as the basis for assessing the validity of a hypothesis. In other words, I seem to concede to consider a hypothesis as plausible if something that cannot happen by chance, such as observing the same number repeatedly. This is based on the assumption of the ``principle of confirmation,`` which assumes that if the number of observations increases, it can be considered more reliable, and the ``principle of uniformity,`` which assumes that things will continue to proceed as they have been if the conditions remain the saus. These beliefs ultimately serve as the basis for verification and scientific knowledge production. 

% I will not delve into the validity of these beliefs here. What matters is that my research activity follow a practice that ``when a hypothesis is present, and a certain criterion and procedure are prepared, and the hypothesis is considered valid according to that procedure, I consider it valid.''

% In theoretical research, sometimes there is no verification plan. Theory is a hypothesis, and its validity is determined separately through verification (not in the sense of whether it is mathematically valid, but for example whether it explains physical phenomena or not). However, in complex modern science, theorists propose a theory, and experimentalists verify it.

% Therefore, it is understood that in current research practice, shared knowledge in the form of papers may not necessarily provide a complete answer to a given question. This is similar to research on negative data. Negative data cannot solve the unknown initially declared, but it can reduce a certain degree of uncertainty towards it. This is because the validity of the presented hypothesis may have decreased somewhat. If this is the case, each research shared in the form of a paper may be more appropriate to describe as "reducing uncertainty towards the unknown," rather than "making the unknown known." This can become complicated when scrutinized strictly, so let's put this aside for now and continue to discuss how "producing new knowledge" is research.

% In reality, conducting research is expected to be done with limited resources (time, funding, computing resources, people, etc.). Therefore, it is necessary to consider these resources when determining the verification approach. After a research design is determined at an abstract level, the feasibility of the research plan is roughly evaluated through a simple problem setting. This is known as a pilot study.



\subsubsection{Verification Execution}
Finally, the instantiated research plan is executed according to the prescribed procedures. In verification design, I am crafting the essence of the verification, and during verification instantiation, I am preparing it to be executed in a concrete and feasible manner. Therefore, there are almost no tasks to be done in this process. The preparation for verification, being distinct from the actual execution of verification, is not a verification in itself. With this understanding, I have adopted this formal classification for the current context.
\footnote{
Typically,``experiments'' refers to the data generation process and subsequent analysis and interpretation are conducted separately. However, since I am discussing a verification plan in this context, all of them are in the same single plan. Therefore, please note that what emerges from executing the verification plan is not data but the verification results.

In many empirical studies, the data generated from experiments is often used not only for the verification of hypotheses but also for generating new hypotheses or giving some insights. However, as hypothesis generation and hypothesis verification play different roles in knowledge production, I do not assume any uses of the generated data beyond verification in this context. Of course, please note that this does not imply that such actions are prohibited in practice. Data analysis will be discussed in a separate section.
}

\subsubsection{Starting from Verification Design}

As mentioned above, I believe that the verification process can be divided into three stages: formulating a verification plan, preparing for the execution of the verification plan, and executing the verification plan. Among these stages, the execution of the verification plan and the preparation for it require interaction with the physical world in many fields. For example, in certain fields, you may need to purchase and raise rats for training, while in others, you may need to observe physical objects directly. On the other hand, formulating a verification plan is a process that is purely confined to the human mind in a wide range of fields. Therefore, it can be a good idea starting from automating verification design process.


\subsection{Verification Based on Statistics}
「正当化の根拠」とかにしておく?
人間は最終的に

Analyzing how statistical methods can validate hypotheses is crucial for automating hypothesis testing. Many research endeavors heavily rely on inductive inference to validate hypotheses, and statistics provide specific techniques for inductive reasoning. Therefore, I examine how I perform inductive inference using statistical methods, taking inspiration from Ohtsuka's arguments \cite{otsuka2022thinking}.

\subsubsection{Inductive Inference and Inference Statistics}

Inductive inference is a method of constructing general principles from data. We assume that data is sampled from a probability model and formulate the problem of inferring this probability model from the data as inferential statistics. The assumption of the existence of a probability model corresponds to the assumption of the uniformity of nature, which posits that similar situations will hold even in unobserved circumstances. As the uniformity of nature is a prerequisite for inductive inference, these formulations correspond to setting the foundation for inductive inference.

This probability model is assumed to be true as a prerequisite for inductive inference. However, in actual statistical inference, we perform inductive inference by establishing statistical models that approximate this probability model. In many cases, we assume specific families of distributions for these statistical models and reduce the problem to estimating parameters of these distribution families from the data. This is the formulation of inductive inference using statistical terms.

Within these formulations, the question arises of ``what kind of'' inductive inference we are performing. This precisely pertains to ``what is considered validated'' or ``how beliefs are justified.'' While multiple perspectives are possible, we will introduce discussions on classical hypothesis testing (referred to as just hypothesis testing below) and the approach using Bayesian statistics (referred to as just Bayesian statistics below).\footnote{
Note that this is not a discussion about what each position think about probability is, but rather a discussion on what they think is justification means.
}

\subsubsection{Bayesian Statistics and Epistemological Internalism}
In Bayesian statistics, we perform inductive inference by updating the degree of belief in a hypothesis based on the evidence provided by the data. The validity in inductive inference is adjusting the degree of belief in the conclusion in a consistent manner with the degree of belief in the premise. Representing the prior belief with prior probability and the likelihood with evidence, Bayesian statistics calculate the posterior probability following probability rules like Bayes' theorem. In this sense, Bayesian statistics, including Bayes' theorem, provides the logic for inductive inference. 

Ohtsuka argues that Bayesian statistics justifies beliefs based on a valid inference rule from other beliefs and compares this position to epistemological internalism \cite{otsuka2022thinking}.

\subsubsection{Classical Statistics and Epistemological Externalism}
In classical hypothesis testing, we first formulate some hypothesis regarding the probability model and then compare it with the data to either reject or retain the hypothesis. In other words, testing is a function that maps data to two options: rejecting or retaining the null hypothesis. Based on this result, we perform inductive inference (or behaviour). 

Ohtsuka interprets testing as a kind of examining tool that makes a judgment based on certain data and sees testing theory as a theory that measures the reliability of this examination tool. Making judgments about hypotheses based on data can be seen as a belief formation process, and testing theory educates the reliability of this process through measures like the size and power of the test. From this, classical statistical testing takes a stance that the justification of a belief is determined by the process through which it was formed, similar to epistemological externalism \cite{otsuka2022thinking}.

\subsubsection{For Automating Statistical Hypothesis Verification}

As such, when it comes to statistical methods for hypothesis verification, there are multiple perspectives on what constitutes justification. It is a challenging issue to known how various statistical techniques, each with their own approach, can provide evidence to substantiate a hypothesis based on the data \cite{otsuka2022thinking,sober2008evidence,sep-statistics}. When seeking to allow machines to autonomously conduct hypothesis verification, determining what they should be capable of is a challenging issue. 

\subsection{Challenges for Automating Hypothesis Verification}

\subsubsection{The Inherent Difficulty of the Act of Verification}
In the first place, the act of verification is a highly challenging process. Firstly, as mentioned earlier, inductive approaches cannot verify hypotheses in the same way as deductive reasoning. Also, I notice that hypothetico-deductive method, which involves verifying claims derived from a hypothesis to confirm its validity, is still widely used today. However, verifying a deduced claim does not support the hypothesis because there may be many hypothesis that can result in the same deduced claim. In response to these, Karl Popper proposed that while hypotheses cannot be confirmed, they can be falsified \cite{sep-scientific-method}. However, in practice, the verification of hypotheses involves implicitly relying on numerous auxiliary hypotheses. When using experimental apparatus, it requires many assumptions to trust them. Even when an experiment fails, determining whether the hypothesis was incorrect or the experiment itself was flawed is not as straightforward as one might think. Thus, there is inherent uncertainty in attributing the results of verification to a specific cause \cite{chalmers2013thing,sep-physics-experiment,sep-scientific-underdetermination} as I have discussed in the section of hypothesis generation. Furthermore, all reasoning and observational evidence are inevitably influenced by some form of theories, individuals, or societal factors \cite{sep-science-theory-observation}. Therefore, it is necessary to carefully examine them to ensure that they are not distorted by unintended influences. 

\subsubsection{The Difficulty of General Hypothesis Verification}
I recognize that the the way to verify a hypothesis is strikingly diverse, as it can significantly differ depending on the subject of research. For instance, if one wishes to study the behavior of rat, they may need to train the rat. On the other hand, if you want to test a theory of particle physics, you may have to construct and run a huge particle accelerator. In the field of history, the existence of historical records might serve as evidence, while in mathematics, the verification process revolves around the proofs themselves. Due to this high degree of flexibility, hypothesis verification can be the most challenging aspect to automate for realizing a ``general'' artificial researcher.

% \subsection{How to Justify Beliefs}
% In many empirical sciences, statistical methods occupy a privileged position as the primary means of verifying hypotheses. Therefore, it seems important to consider in what sense I can say that a statistical method can validate hypotheses, while I admit that the methods of verification vary greatly depending on the type of question or hypothesis and it cannot be said that there is any universal method for verification. The use of inferential statistics methods is widely accepted, but it is a challenging issue to known how various statistical techniques, each with their own approach, can provide evidence to substantiate a hypothesis based on the data in the real world \cite{otsuka2022thinking,sober2008evidence,sep-statistics}. Thus, what constitutes the verification of a hypothesis, or in other words, the justification of a belief, is a very difficult debate in which it is likely challenging to arrive at a unified answer. Please note that the discussions mentioned earlier, such as the uniformity of nature, pertain to the validity of induction itself. In statistical methods, however, induction is assumed and the discussion begins with formulating this uniformity by assuming a probability model. 

Furthermore, as explained in the section on statistical hypothesis testing, it was also described how even with inductive inference using the same statistical theory, there are multiple perspectives on how it justifies beliefs. Even empirical sciences, which employ highly universal verification methods based on statistical approaches, face such difficulties. Therefore, aiming for autonomous acquisition of verification methods in broader fields of research would likely pose even greater challenges. These issues will require further in-depth discussions in the future.

% This discussion of statistical method provides important insights for the pursuit of developing AI capable of conducting autonomous research. As can be seen from the assumption of being rooted in machine learning, the validity of inductive inference itself is generally accepted and not a practical concern in creating such AI. However, how to make agents acquire the methods of justification is a important issue. If the criterion for justifying beliefs is to be acquired completely autonomously from scratch, it may very well lead to criteria that are meaningless for humans. Additionally, it seems that the criteria for justification employed by humans are diverse. When properly considering the reasons that these criteria are believed to provide justification, I can recognize that there are highly intricate structures involved. In light of such considerations, the extent to which one can understand and acquire criteria for justification from the criteria humans already employ is a nontrivial problem. 


% I do not intend to claim that these uncertainties undermine the reliability of research. Such uncertainties exist in almost all human endeavors. Research, among these activities, strives to confront these uncertainties with great care and rigor. Above all, the fact that the results generated through research have effectively supported my lives demonstrates their efficacy.

% I mentioned the difficulties inherent in verification to highlight their significance, particularly when considering autonomous artificial intelligence capable of self-verification. Ideally, autonomous agents are expected to establish their own methods and criteria for verification. However, if verification inherently carries such difficulties and uncertainties, the more rigorously I consider it, the more paralyzed the agent becomes, as it seemingly cannot accomplish anything. \textcolor{red}{TODO: Add explanation}

% A characteristics of research can be found in the systematicity, rigor, and objectivity of research practice \cite{sep-scientific-method,hoyningen2008systematicity,haack2003defending}. 

% In particular, I believe that a characteristic of research lies not in the way of determining questions or generating hypotheses, but in the fact that the verification of hypotheses is done in an extremely rigorous and careful manner. 

% It could be said that I'm taking a view similar to the new experimentalism, placing emphasis on verification in research, or experiments \cite{chalmers2013thing,mayo1996error}.

% Of course, generating a hypothesis is not a simple task. What I want to say is that, as long as any method of hypothesis generation is properly verified, it is considered research, and no matter how properly a hypothesis is proposed, if the verification is sloppy, it is not considered research. This means that verification may be at the heart of knowledge production. In other words, in order to create artificial intelligence that produces knowledge, it is important to consider how to create an intelligence that can perform verification.

\subsection{Autonomous Hypothesis Verification}

\begin{figure}[htb]
    \centering
    \includegraphics[width=\linewidth]{figs/incommensurability.jpg}
    \caption{Incommensurability}
    \label{fig:incommensurability}
\end{figure}

I said that research is to generate knowledge for humanity. However, this is merely because humans have been the only ones conducting research thus far. I believe that in principle there could be knowledge for beings other than humans and hence research for them as well. 

For example, if a group of machines are each capable of holding certain belief state which is similar to each other, they seem to be able to hold a shared belief. If these agents have a way to ground their belief that a hypothesis is true to their shared belief, then this can be regarded as research within that society. Therefore, the definition of knowledge as belief leads to the conclusion that research by and for the machines would look like this.

\begin{figure}[htb]
    \centering
    \includegraphics[width=\linewidth]{figs/shared_belief_revision.jpg}
    \caption{Belief Revision}
    \label{fig:shared_belief_revision}
\end{figure}

There are two implications here. Firstly, if we let AI entirely autonomously conduct research in the sense of current definition, including the design of the conceptual foundation of verification, they would likely produce nonsense for humans. This is because the systems of beliefs of humans and those machines would be entirely different from each other. Therefore, to conduct research that is meaningful to humans, it could be said that the foundation of the method of verification should be based on the foundation of human verification.

The second implication is that if the foundation of verification is autonomously constructed, the AI capable of conducting research in the sense of this definition would even produce knowledge meaningless for understanding nature. This is because while human beliefs have been shaped to be consistent with nature through interactions with it, the beliefs of machines are not necessarily so. Through the process of evolution, humans have spent vast amounts of time interacting with and modifying their bodies to survive in nature. Our beliefs are formed in such contexts, and the beliefs we currently possess are likely advantageous for living in nature. Therefore, I believe that our strong belief serves as a reliable foundation for understanding nature. Machines do not possess bodies embedded with such interactions with nature. Therefore, just because they have strong convictions does not guarantee that these will serve as a reliable foundation for explaining nature. 

When aiming for artificial intelligence that can conduct research autonomously, we still do not know how much can be entrusted to AI, namely, where the limits of autonomy lie. Based on the above two conclusions, it becomes apparent that when letting AI conduct research in the current definition's sense, it seems difficult to have AI create the conceptual foundation of verification from scratch. Therefore, it appears necessary to teach this AI what verification entails through human examples or align their values regarding verification with human values.

\chapter{Landscape of Research Fields for Research Automation}
\label{chapter-literature-review}

% 本章では、研究の自動化を目指す試みを幅広く紹介します。特に、これまで同じ論文上やコミュニティ間で論じられることがなかった研究たちも含めて、「研究の自動化を目指す研究群」として統一的に紹介しようと思います。一つ一つの論文を丁寧に紹介することは現実的ではないため、あくまで本論文では各分野や研究の取り組み全体をごく簡単に紹介することに注力します。各分野での研究成果についての詳細な説明については、すでにいくつもの素晴らしいレビュー論文が出版されていますので、そちらをご参照ください。ただし、近年の言語モデルを用いた取り組みについては個別の事例も扱いながら紹介しようと思います。

% 各分野を紹介した後で、汎用的で自律的な人工研究者という観点から、それらを分類する二つの軸を提案します。一つ目が汎用性の軸で、これは各研究の自動化の取り組みが、どれぐらい広い研究領域に対して汎用的なタスクを自動化するものであるかというものです。例えば、科学研究全てで必要となるようなタスクの自動化はある特定の研究課題を自動化するようなものよりも汎用的であると言えます。二つ目が研究過程のどの部分を自動化しているか、という軸です、これは自律性の軸に対応します。例えば、ある研究は仮説生成全体を自動化しているかもしれませんし、別の研究は仮説検証の特定の過程のみを自動化しているかもしれません。全ての研究過程の作業が自動化されている場合、これは自律性が高いと言えます。

% 最後に、研究分野の紹介と軸の提案を踏まえて、研究の自動化に関するパースペクティブ論文・レビュー論文のうちいくつかについて論じます。特に、紹介されている個別の研究よりも、それらの研究がどのようにして既存研究を整理しているか、どのような提案をしているかを見ていきます。これによって、本論文による整理の位置付けを改めて明確にし、次章以降の議論に繋げていきます。

\section{Existing Discussion on Research Automation}

Attempts to automate research began soon after the advent of computers. Until then, humans had developed science by describing experiences (primary science) and by making predictions through the construction of theories (secondary science). With the advent of computers came simulation science (third science), which uses computers to automate complex scientific computations, and data-driven science (fourth science), which automates the discovery of laws from large-scale data \cite{hey2009fourth}. Within the individual academic discipline ``X,'' the third science gave rise to the field of \textit{Computational-X} and the fourth science gave rise to the field of \textit{X-infomatics}, leading to the automation of entire academic disciplines.

\subsubsection{AI for Science}

Automation of scientific research with AI, including logical AI, has been an area of interest since the advent of AI \cite{langley1987scientific}. In the early stages of research, researchers created logic AIs that mimics the problem-solving process of researchers \cite{lindsay1993dendral}.

The remarkable performance of deep neural networks has now made almost all scientific fields more or less influenced by AI. The term \textit{AI for Science}, which automates scientific research with machine learning, has appeared due to the remarkable development of machine learning technology since the 2010s. 

AI for Science is a concept that encompasses a very wide range of initiatives. All attempts to solve problems in any process within science using machine learning or to develop the foundational technologies for them can be considered within the scope of AI for Science. Indeed, a vast number of machine learning application studies have been generated in numerous scientific fields \cite{xu2021artificial}, such as AI for Material Science and AI for Medical Science, named few.
\footnote{In this paper, it is not possible to introduce each of these individually, so we will omit the introduction of application studies in each research area of AI here. There are several excellent review papers available, so if you are interested, please refer to those.}

Wang et al. \cite{wang2023scientific} have excellently summarized these activities, we will introduce existing automation efforts while borrowing the view of their work. They categorize and organize various initiatives in AI for Science by focusing on which scientific processes they are related to: data handling, hypothesis generation, and simulation and experimentation \cite{wang2023scientific}. This is based on the view that the essence of science lies in the collection, transformation, and understanding of data.

The group of studies that incorporate biases to handle scientific data on AI, giving it information about the governing physical rule, is called \textit{physics-informed machine learning} \cite{karniadakis2021physics}. The incorporated biases include the ability to handle partial differential equations (PDE), symmetry, and intuitive physics \cite{hao2022physics}. Karniadakis et al. \cite{karniadakis2021physics} and Hao et al. \cite{hao2022physics} systematically and clearly summarize existing research from the perspective of which elements of machine learning are modulated by which biases and how they are incorporated. % TODO

The ability to handle PDEs gives us the basis for both the simulation of physical models and the discovery of the law from data, which we will discuss later. Studies that aim for better inductive biases and architectures in deep learning by treating symmetry as a first principle are called \textit{geometric deep learning} \cite{bronstein2021geometric}. Given the importance of symmetry in natural sciences, this group of research also plays a significant role in AI for Science. Zhang et al.  organize existing research in AI for Science across different research domains, focusing on symmetry as one of the common foundational pillars \cite{zhang2023artificial}.

The methods for generating hypotheses vary significantly depending on the subject, so there exists a wide range of approaches to automate this process. For example, attempts to generate hypotheses from articles or from scientific data are being undertaken in all research fields. 

One long-standing approach to hypothesis generation by AI is the research focused on discovering symbolic equations that describe scientific laws from data (\textit{equation discovery}). Among the earliest pioneering studies in this area, the BACON system by Langley et al. \cite{langley1987scientific}, and subsequent studies based on it, are well-known. In particular, the research on equation discovery that is mainly studied in the machine learning community is known as \textit{symbolic regression}, which originally started from research that used genetic programming approaches to discover symbolic equations.
 Kramer provides a comprehensive overview of the history of equation discovery research, including early studies that did not utilize machine learning \cite{kramer2023automated}. Famous deep learning approaches include AI Feynman \cite{udrescu2020ai,udrescu2020ai2}, which transformed the problem itself into a simple form and discovered the fundamental physical laws selected from Feynman's physics lectures.

% Symbolic regression is a problem of exploring a combinatorial hypothesis space among hypothesis generation \cite{wang2023scientific}. 
In science, exploration plays a crucial role, as exemplified by the search for hypothesis spaces and experimental conditions mentioned later. Therefore, methodologies for better automated exploration by machines have been sought, using techniques such as active learning. There is a perspective paper of research automation by Kitano \cite{kitano2021nobel} that focuses on the importance of machine-driven exploration of hypothesis spaces. Kitano points out that current hypothesis selection is value-driven, based on human value criteria, and argues that we should aim for hypothesis generation through exhaustive machine-driven exploration that is not dependent on such human value criteria. He points out that some of the Nobel Prize-winning research is actually the result of such exhaustive exploration, emphasizing the importance of this approach.
% \subsubsection{Symbolic Regression / Equation Discovery}
% % \subsubsection{Symbolic Regression} 
% Scientists have constructed models in the form of mathematical equations that explain them from observational data. This has enabled us to go beyond observational data to understand and predict underlying phenomena. That is to say, formulating a mathematical representation that elucidates the phenomenon behind the data is an extremely critical step in science. 

% One attempt to automate this endeavor is \textit{symbolic regression} \cite{makke2022interpretable}, or \textit{equation discovery}. These are attempts to infer from the data a formula that explains it. While classical approaches to symbolic regression have traditionally employed methods such as evolutionary computation, recent years have seen the emergence of strategies utilizing deep neural networks \cite{petersen2019deep,udrescu2020ai,udrescu2020ai2,cranmer2020discovering,kamienny2022end,d2022deep}. Some researchers have proposed the frameworks \cite{landajuela2022unified,keren2023computational} and benchmarks \cite{matsubara2022rethinking} for symbolic regression. You can find a literature review of symbolic regression in \cite{makke2022interpretable}, and that of the early studies in \cite{kramer2023automated}.

Once a hypothesis is formed, it is tested through experimentation. An experiment is the act of generating observational data through a set of predefined procedures. In modern science, simulations are used to generate data that resembles observational data. To create better simulations, knowledge of scientific computing is essential. The attempt to automate the generation and analysis of scientific data by combining scientific computing and machine learning is known as \textit{Scientific Machine Learning (SciML)} \cite{baker2019basic}. In physical simulations, numerical solutions to differential equations often come into play, so physics-Informed Machine Learning is also frequently discussed in this context. Experiments and numerical calculations come with uncertainties, so there is a long history of research on quantifying these uncertainties (\textit{uncertainty quantification}). This is also an important topic of study in SciML.

There is also a long history of research applying machine learning to experimental design. In particular, research on the efficient search for experimental conditions using active learning, including Bayesian optimization (\textit{Bayesian experimental design}), has been attempted in various application fields.

% 特に科学技術計算と機械学習を組み合わせて科学データの解析を自動化する試みは Scientific Machine Learning (SciML) と呼ばれています。(シミュレーション)

% 機械学習において科学データを扱えるような帰納的バイアスを組み込む研究群は Physics-informed machine learning と呼ばれています。組み込まれるものとしては、直観物理、微分方程式を扱う能力、対称性を抽出する能力などがあります。微分方程式の取り扱いは科学計算がずっと扱ってきたものですので、微分方程式を扱う研究は SciML の文脈で議論されることも多いです。また、対称性は幾何学的構造の保存なので、これらは geometric machine learning の一つとして議論されています。

% データからそのデータの背後の法則を記述するシンボリックな方程式を推定する試みも行われてきた。機械学習を用いてデータから方程式を推論する、symbolic regression がある。方程式を発見するシステムの先駆けとして BACON がある。2000年代には単位による制約から数学的に可能な方程式を同定する研究も行われた。これらは equation discovery という名前で研究されてきました。

\subsubsection{Automated Research Workflow}
Some studies have attempted to represent the processing of calculations and the flow of data in science as a single pipeline and to automate these processes. These studies refer to such pipelines as \textit{scientific workflow} or \textit{automated research workflow} \cite{national2022automated}. The work of \cite{gil2022will} expresses an insightful perspective for research automation from the point of this scientific workflow community. She introduced the idea of workflows for automated scientific data analysis and continuous generation, verification, and update of hypotheses.

\subsubsection{Laboratory Automation}
\textit{Laboratory automation} is a program that seeks to automate empirical research involving scientific experiments that involve interaction with the physical world.

What makes this effort unique compared to other efforts to automate research is that it seeks to automate even manual labor of humans in experimentation by developing robots for experiments to automate the entire process of experimentation. A notable example includes pioneering research in genetics by Ross King, who fully automated the cycle of hypothesis generation, verification, and discovery of new hypotheses with Adam \cite{king2004functional}. Another example is A.I. Cooper, which enabled the use of the same experimental equipment as humans through autonomous robots \cite{burger2020mobile}. Many attempts at laboratory automation are specialized for specific experiments. However, efforts are underway to develop humanoid robots capable of conducting experiments \cite{yachie2017robotic}, as well as initiatives to automate the low-level behaviors of robots, towards more general automation.

% A few examples include ``Mahoro'', a general humanoid robot that can conduct various experiments \cite{yachie2017robotic}. The robot could automatically conducted cell culture tasks using \cite{ochiai2021variable}


\subsubsection{Automating Mathematics}
The automation of mathematical proof, \textit{automated theorem proving} (ATP) , has been studied for a long tius. Recently several effort has come up to improve ATP by using machine learning, and especially deep learning. The early seminal work is led by Schulz \cite{schulz2001learning} and Urban \cite{urban2004mptp,urban2008malarea}. The first work applying deep learning to ATP is \cite{irving2016deepmath}. Subsequently, numerous studies have emerged on Automated Theorem Proving (ATP) using deep learning \cite{bansal2019holist}. Recent studies on this topic are well organized in the following paper, so we recommend reading it if interested \cite{rabe2021towards}.
% TODO: more organized literature review

While research has been accumulated on ATP, there is still not much research done on the automated theorem discovery with a few exceptions \cite{gao2014systematic}. In recent years, attempts have been made to help humans to find mathematical conjectures \cite{davies2021advancing} and
automatically generate mathematical conjectures \cite{raayoni2021generating}  using machine learning.

\subsubsection{Automating Machine Learning}
The attempt to automate various tasks related to machine learning is called \textit{AutoML}. This includes hyperparameter optimization, neural architecture search (NAS), and automation of various pre- and post-processing steps in machine learning. The following website \cite{automlorg} provides a very active overview of AutoML. For literature review, the book \cite{hutter2019automated} explained well about the AutoML by 2019, the paper \cite{bischl2023hyperparameter} and \cite{lindauer2020best,white2023neural} summarizes comprehensively the current state of hyperparameter selection and NAS, respectively. Recently, research has emerged that allows machine learning tasks to be executed with text-based instructions alone \cite{vijay2023prompt}.

Additionally, there is an initiative called MLOps (Machine Learning Operations) that aims to to streamline and automate operations related to machine learning primarily in the business context. This includes automating tasks such as model deployment and continuous training, as well as experiment management. MLOps have tried to automate the laborous tasks that need to be automated to aim for the full automation, such as experiment management, versioning, and deployment. The paper \cite{kreuzberger2023machine} is a good scientific review paper on MLOps.

\subsubsection{Scholarly Document Processing}

\subsubsection{Automating Peer-Review}
There are also attempts to automate the peer-review process. Researchers have tried to automate review generation \cite{yuan2022can}, paper screening \cite{schulz2022future}, research paper assessment \cite{kousha2022artificial}, reviewer assignment \cite{zhao2022reviewer}, and more. As in other fields, recent years have seen research on the automation of peer review using large-scale language models. For traditional research on the automation of peer review, \cite{kousha2022artificial} and \cite{lin2021automated1} have conducted comprehensive literature reviews. In particular, Kousha et al. cover a wide range of topics related to the automation of various aspects of peer review.

% Many studies have tried to automate peer review generation \cite{thelwall2019artificial,li2019generating,schulz2022future,yuan2022can,yuan2022kid,lin2021automated1,lin2021automated2,kumar2022investigations,bharti2022can,uban2021generating,wang2020reviewrobot}. While not generating peer reviews directly, studies focused on automating research paper assessment  can be said to be related to the peer review automation. \cite{kousha2022artificial,li2020multi,huang2018deep}. These studies have proposed the method to assess the quality \cite{thelwall2022predicting,thelwall2022can}, novelty \cite{pelletier2022novelpy,amplayo2019evaluating,shibayama2020measuring}, soundness \cite{cabanac2022decontamination}, and significance \cite{zong2022citation,xia2023review,soni2022predicting,manghi2021new,soni2021follow,van2020schubert,mckeown2016predicting}.

% These investigations concern the automation of processes occurring subsequent to a manuscript's arrival at the hands of reviewers. Conversely, researchers also have investigated the automation before that process, such as determining the appropriate journal for submission \cite{michail2023journal} and assigning the reviewers \cite{zhao2022reviewer}.

% While not centered on automation, certain studies engage in the scientific analysis of the review process \cite{shah2022challenges,verma2021attend,bharti2022confident,bharti2022betterpr,verma2022lack,kennard2022disapere}. These investigations serve to enhance our understanding of the nature of peer review and, in turn, provide valuable insights for the design of more effective automated review methodologies. 

% 数学の定理の証明を自動化する研究は古くから、Automated Theorem Proving (ATP) という名前で研究されています。証明を変換された定理たちをノードにもつ木の探索問題に帰着させるのが基本的なアプローチです。

% より最適な実験条件の決定を自動で行う試みもあります。有名なものはベイズ最適化を用いたベイズ実験計画と呼ばれているものです。

% 学術論文の取得からの情報抽出までを自動化しようという試みも長い研究があります。Scholarly document processing などと呼ばれます。これらはもともと自然言語処理やテキストマイニング、情報検索などのコミュニティで研究されてきました。近年は大規模言語モデルの登場により、多くの課題が解決されました。

% 人間が手作業で行ってきた実験をロボットの力を借りて自動化しようという試みもあります。これらは laboratory automation と呼ばれています。

% 科学における計算の処理とデータの流れを一つのパイプラインとして表現し、それらの処理を自動化しようという研究もあります。これらの研究ではこのようなパイプラインを Scientific Workflow あるいは Automated Research Workflow と呼んでいます。

% 機械学習自体を自動化しようという試みは AutoML として知られています。ハイパーパラメータの探索やアーキテクチャの探索の最適化と自動化などが主に研究されています。また、主にビジネスにおいて機械学習に関連する諸タスクの実行を自動化する取り組みの総体は、MLOps と呼ばれています。

% Peer review を自動化しようという研究も存在します。

% 機械学習の中でも科学研究への応用の文脈でよく言及される分野があります。例えば、active learning, explainable AI, uncertainty quantification, out-of-distribution generalization, exploration、などはこれらの例の一部です。

% 記号的なAIの分野でも早くから研究の自動化の取り組みが行われてきました。例えば、DENDRAL は科学者の思考を自動化したものや、アブダクションなどの推論によって知識発見を自動化しようという試みもある。これらの取り組みは広義には Knowledge Representation and Reasoning という分野の中で盛んに研究が行われてきた。

% While there have been some excellent review articles and perspective papers on previous efforts to automate research, there have not yet been many reviews that deal with these research automation efforts in a shallow or even conservative manner. Therefore, this chapter provides the most comprehensive review possible of research on research automation to date.

% The field related to research automation is vast. While it is not possible to cover all of them, we aim to present as comprehensive an introduction as possible. For this reason, the review of each individual field will be limited to a brief overview. We will introduce survey papers and other literature in those fields, and those who wish to understand more advanced discussions should refer to those references.

% In the following, we will present the past efforts of mankind related to research automation, paying attention to what level of general ``task'' each research automation is oriented toward. This is because efforts to automate research can be interpreted as automating a set of tasks in the research process, or acquiring the ability to do so. Note that the level of generality treated here is naturally not absolute, but is set for convenience. Also, please keep in mind that this classification is for classifying human research activities, so humans are only assumed as the subject of the research. After introducing each effort, we will position these efforts in light of the formulation of the knowledge production process that this paper has dealt with.

\textcolor{red}{CAUTION: We have not read all of the literature in detail and it may contain errors. We will continue to update this paper, but if you find errors, please contact us or throw us a pull request.}

\section{Level of Generality of Task/Skill}
The first and most versatile ability is the ability to perform any task a human being can perform. This includes, for example, the ability to think, to manipulate language, and to act. Second, there are abilities that are required for any research. These include the ability to ask questions, generate hypotheses, and test hypotheses, as explained in Chapter 2. The ability to manipulate the scholarly literature and to search for information are also included in these abilities. This is because research is the activity of producing knowledge from arbitrary inputs, and information retrieval is essential for obtaining inputs, and manipulation of the academic literature is essential for processing and outputting these inputs. So far, this is the capability required for all research.

Next, there are fields of study that involve some form of quantity in order to do the research. If there are basic quantities involved, whether in the natural or social sciences, then this is the field. These research fields require the ability to manipulate mathematics in some form. In those fields that require the use of empirical methods, which are research methods for generating and analyzing data, you need to be able to use statistics and data analysis.

And there are abilities that are universally needed within each discipline, such as natural sciences, social sciences, and humanities. For example, in the natural sciences, the ability to perform experiments that interact with the physical world is an essential skill in a very broad range of natural sciences.

Below that, the generality of tasks and the ability to acquire them gradually narrows down to abilities that are widely needed within each discipline as a whole, such as physics, chemistry, biology, and medicine, and then to abilities needed in even smaller disciplines, such as condensed matter physics, organic chemistry, and molecular biology, and so on.

Fig. \ref{fig:generality_level} represents the hierarchical structure of generality of tasks we have discussed above. Of course, in actual research, the disciplines are closely related to each other, and such a hierarchical structure is not perfect. What we want to emphasize here is that there are differences in the degree of generality of the technologies to be automated and the capabilities to acquire them, and that each automation effort can be considered distinct in terms of its degree of generality. In the following, we will review the automation efforts of previous studies, focusing on these differences. To emphasize again, this chapter will focus on automation efforts in research related to science in the narrow sense, i.e., the natural sciences.


\begin{figure}[htb]
    \centering
    \includegraphics[width=\linewidth]{figs/generality_level.jpg}
    \caption{Caption}
    \label{fig:generality_level}
\end{figure}

\section{Automation of Tasks for Diverse Field of Research}

\subsection{Scholarly Document Processing}
\textit{Scholarly document processing} is a general term for research on automated processing related to scholarly articles and has been studied as part of natural language processing, text mining, and information retrieval.

\textcolor{red}{TODO: Skip for now. Need reconstruction because many of the issue discussed below are no more issue for research automation. Most of these will moved to Appendix}

\subsection{Peer Review}
Many studies have tried to automate peer review generation \cite{thelwall2019artificial,li2019generating,schulz2022future,yuan2022can,yuan2022kid,lin2021automated1,lin2021automated2,kumar2022investigations,bharti2022can,uban2021generating,wang2020reviewrobot}. While not generating peer reviews directly, studies focused on automating research paper assessment  can be said to be related to the peer review automation. \cite{kousha2022artificial,li2020multi,huang2018deep}. These studies have proposed the method to assess the quality \cite{thelwall2022predicting,thelwall2022can}, novelty \cite{pelletier2022novelpy,amplayo2019evaluating,shibayama2020measuring}, soundness \cite{cabanac2022decontamination}, and significance \cite{zong2022citation,xia2023review,soni2022predicting,manghi2021new,soni2021follow,van2020schubert,mckeown2016predicting}.

These investigations concern the automation of processes occurring subsequent to a manuscript's arrival at the hands of reviewers. Conversely, researchers also have investigated the automation before that process, such as determining the appropriate journal for submission \cite{michail2023journal} and assigning the reviewers \cite{zhao2022reviewer}.

While not centered on automation, certain studies engage in the scientific analysis of the review process \cite{shah2022challenges,verma2021attend,bharti2022confident,bharti2022betterpr,verma2022lack,kennard2022disapere}. These investigations serve to enhance our understanding of the nature of peer review and, in turn, provide valuable insights for the design of more effective automated review methodologies. 

\subsection{Cognitive }

\section{Automation of Tasks for Diverse Fields of Science}
% There are prerequisite competencies and knowledge required to conduct scientific research. And the question of how to acquire such abilities has been one of the major concerns of AI research for science. Therefore, we will first introduce these research areas. In particular, we will present research on processing scientific literature and understanding scientific knowledge.

% \subsection{Automated Theorem Proving}



% \subsection{Understanding Scientific Knowledge}

% \subsection{AI for Science}

TODO

\subsection{Scientific Language Models/Foundation Models}
In order for us to do scientific research, we must have learned prior knowledge about science. Therefore, research has been done to teach or incorporate such scientific knowledge and assumptions into machine learning models.

One of the most popular approaches today is to create foundational models in science. A foundational model is a pre-trained model that can be adapted to extremely generic downstream tasks. In particular, because language is an extremely general-purpose interface and because language models have developed by leaps and bounds, large-scale language models are now predominantly trained on vast amounts of textual data.

In science, scientific large-scale language models have also been developed by training huge amounts of scientific texts, including textbooks and papers \cite{taylor2022galactica}.
% \cite{beltagy2019scibert,singh2022scirepeval,nadkarni2021scientific,cohan2020specter,gupta2022matscibert,taylor2022galactica}.

% \subsubsection{Scientific Language Models}

Whether engaging in reading or writing, the presence of a system that comprehends natural language is indispensable. In recent years, large-scale language models, trained on extensive textual data, have achieved significant success. Concurrently, numerous language models, specifically tailored to scientific documents, have also been proposed \cite{beltagy2019scibert,singh2022scirepeval,nadkarni2021scientific,cohan2020specter,gupta2022matscibert,taylor2022galactica}.

\textcolor{red}{TODO: table for scientific lm}

\subsubsection{Scientific Understandings of GPTs}

\textcolor{red}{TODO: table for case studies to test the scientific understanding of chatgpt and gpt-4 }

\subsection{Scientific Machine Learning: Inserting Inductive Bias for Scientific Understanding}
Another prominent approach to incorporating such scientific knowledge into artificial intelligence is to incorporate inductive biases that help scientific understanding. This area has been studied typically under names such as \textit{scientific machine learning (SciML)} and \textit{physics-informed machine learning}.

Whereas methods to learn scientific knowledge from the literature are generic methods that learn scientific knowledge through the generic interface of language, in this approach, humans add biases to either the model, the data, or the optimization method that are assumptions of scientific understanding. The constraints that have been studied include the abilities to handle (differential) equations, symmetry, intuitionistic physics, and so on \cite{hao2022physics}. 

For differential equations, Physics-Informed Neural Networks \cite{raissi2019physics} and Neural Operators \cite{kovachki2021neural} are well known examples of this line of studies. These are methods that enable data-driven simulation of differential equations from data (forward problems) and differential equation discovery (inverse problems). Deep neural networks that can handle symmetry are studied under the name \textit{geometric deep learning} \cite{bronstein2021geometric}. The following survey is comprehensive in this area and should be referred to by those interested \cite{hao2022physics}.

\subsection{For Empirical Studies}
Research methods in science can be empirical or non-empirical. Empirical research is research that involves the generation of data, one of the crucial elements in science. Here, we present an effort to automate research on tasks related to scientific data. Machine learning is used in these areas in a variety of ways, including variable selection and model selection, but this section will focus specifically on those areas that have names.

\subsubsection{Laboratory Automation}
\textit{Laboratory automation} is a program that seeks to automate empirical research involving scientific experiments that involve interaction with the physical world.

What makes this effort unique compared to other efforts to automate research is that it seeks to automate the entire process of validation, and it seeks to automate even the manual labor of humans in experimentation. Specifically, they automate human tasks by creating robots that can conduct research. A few examples include ``Mahoro'', a general humanoid robot that can conduct various experiments \cite{yachie2017robotic}. The robot could automatically conducted cell culture tasks using \cite{ochiai2021variable}

\subsubsection{Bayesian Experimental Design}
In order to conduct an efficient experiment, it is necessary to properly determine which experimental. \textit{Experimental design}, which involves devising efficient methods for conducting appropriate experiments, has long been studied. \textit{Bayesian experimental design} is an attempt to optimize and automate this experimental design by using Bayesian optimization \cite{chaloner1995bayesian,shahriari2015taking}. Bayesian experimental design has been incorporated from a relatively early stage and has been used in materials science and other fields.


\subsubsection{Symbolic Regression / Equation Discovery}
% \subsubsection{Symbolic Regression} 
Scientists have constructed models in the form of mathematical equations that explain them from observational data. This has enabled us to go beyond observational data to understand and predict underlying phenomena. That is to say, formulating a mathematical representation that elucidates the phenomenon behind the data is an extremely critical step in science. 

% Modern science is composed of a cycle of observation, hypothesis generation, and hypothesis testing. In many fields, including physics, chemistry, and biology, mathematical models are often constructed as hypotheses from observational data. 

One attempt to automate this endeavor is \textit{symbolic regression} \cite{makke2022interpretable}, or \textit{equation discovery}. These are attempts to infer from the data a formula that explains it. While classical approaches to symbolic regression have traditionally employed methods such as evolutionary computation, recent years have seen the emergence of strategies utilizing deep neural networks \cite{petersen2019deep,udrescu2020ai,udrescu2020ai2,cranmer2020discovering,kamienny2022end,d2022deep}. Some researchers have proposed the frameworks \cite{landajuela2022unified,keren2023computational} and benchmarks \cite{matsubara2022rethinking} for symbolic regression. You can find a literature review of symbolic regression in \cite{makke2022interpretable}, and that of the early studies in \cite{kramer2023automated}.

% \subsection{Knowledge Representation and Reasoning}

\section{Automation of Tasks in Each Research Field}

It has become commonplace to streamline domain-specific tasks in scientific research using machine learning, resulting in a vast number of published papers. Even just to mention a few that come to mind, there are studies on molecular biology \cite{jumper2021highly,senior2020improved}, material science \cite{ramprasad2017machine}, medical science \cite{vamathevan2019applications,shorten2021deep}, quantum mechanics \cite{carleo2017solving}, cosmology \cite{carleo2019machine}, genetics \cite{libbrecht2015machine}, and nuclear physics \cite{degrave2022magnetic}. It is impossible to cover all of these applied studies of research automation of science. Therefore, in this paper, we will not go into detail about each of these studies. Instead, we will present research on automation of elements that can be applied in various fields of science. For the literature survey of domain specific automation, please refer to \cite{xu2021artificial}. \textcolor{red}{TODO: Add application studies}

\subsection{Automating Machine Learning}

\subsubsection{AutoML}
AutoML is an attempt to automate all tasks associated with machine learning. This includes hyperparameter optimization, model selection, and automation of various pre- and post-processing steps. The following website \cite{automlorg} provides a very active overview of AutoML. The book \cite{hutter2019automated} for an overview of AutoML through 2019, the paper \cite{bischl2023hyperparameter} for hyperparameter selection, and \cite{lindauer2020best,white2023neural} for the current state of the NAS are very helpful to catch up to this field.

\subsubsection{MLOps}
MLOps (Machine Learning Operations) is a general term for efforts to streamline and automate operations related to machine learning in industry. For example, it includes automating tasks such as model deployment and continuous training, as well as experiment management. To be sure, many MLOps solve problems in industry, and not all of them are related to research automation. However, when actually conducting research, we are involved in experiment management, versioning, and other tasks. While research on research automation has left out automation of these tasks, MLOps is accumulating knowledge on automation of these tasks as well.  The paper \cite{kreuzberger2023machine} is a good scientific review paper on MLOps.


\tikzstyle{my-box}=[
    rectangle,
    draw=red,
    rounded corners,
    text opacity=1,
    minimum height=1.5em,
    minimum width=5em,
    inner sep=2pt,
    align=center,
    fill opacity=.5,
]
\tikzstyle{leaf}=[my-box, minimum height=1.5em,
    fill=orange!20, text=black, align=left,font=\scriptsize,
    inner xsep=2pt,
    inner ysep=4pt,
]
\begin{figure*}[tp]
    \centering
    \resizebox{\textwidth}{!}{
        \begin{forest}
            forked edges,
            for tree={
                grow=east,
                reversed=true,
                anchor=base west,
                parent anchor=east,
                child anchor=west,
                base=left,
                font=\small,
                rectangle,
                draw=red,
                rounded corners,
                align=left,
                minimum width=4em,
                edge+={darkgray, line width=1pt},
                s sep=3pt,
                inner xsep=2pt,
                inner ysep=3pt,
                ver/.style={rotate=90, child anchor=north, parent anchor=south, anchor=center},
            },
            where level=1{text width=5.6em,font=\scriptsize,}{},
            where level=2{text width=5.6em,font=\scriptsize,}{},
            where level=3{text width=5.5em,font=\scriptsize,}{},
            where level=4{text width=6.1em,font=\scriptsize,}{},
            [
                AI for Research, ver
                [
                    Natural Science, ver 
                    [
                            \cite{zhang2023artificial}{,}
                            \cite{xu2021artificial}{,}
                            , leaf, text width=46.8em
                    ]
                    [
                        Phisics
                        [
                            Contrastive~\cite{chalmers2013thing}{,}
                            POTTER~\cite{chalmers2013thing}{,}
                            , leaf, text width=39.7em
                        ]
                    ]
                    [
                        Chemistry
                        [
                            \cite{coley2020autonomous}{,}
                            , leaf, text width=39.7em
                        ]
                    ]
                    [
                        Astoronomy
                        [
                            CoT~\cite{chalmers2013thing}{,}
                            Self-C~\cite{chalmers2013thing}{,}
                            , leaf, text width=39.7em
                        ]
                    ]
                    [
                        Medical Science
                        [
                            ScienceQA~\cite{chalmers2013thing}{,}
                            MarT~\cite{chalmers2013thing}{,}
                            , leaf, text width=39.7em
                        ]
                    ]
                ]
                [
                    Formal Science, ver
                    [
                        Mathematics
                        [
                            CoT~\cite{chalmers2013thing}{,}
                            Self-C~\cite{chalmers2013thing}{,}
                            STaR~\cite{chalmers2013thing}{,}
                            , leaf, text width=39.7em
                        ]
                    ]
                    [
                        Computer Science
                        [
                            Contrastive~\cite{chalmers2013thing}{,}
                            POTTER~\cite{chalmers2013thing}{,}
                            CoT~\cite{chalmers2013thing}{,}
                            MCR~\cite{yoran2023answering}
                            , leaf, text width=39.7em
                        ]
                    ]
                ]
            ]
        \end{forest}
    }
    \caption{Taxonomy of Reasoning with Language Model Prompting.}
    \label{fig:categorization_of_reasoning_big}
\end{figure*}

\section{Another Axis: Research Process}
We have presented the idea that attempts to automate research could be classified hierarchically according to how broadly they could affect research. In addition to this, we believe that these efforts can also be categorized from another angle. That is, from the perspective of what part of the research process is being automated. For example, one effort might automate only the hypothesis generation part of a broad scientific study, while another effort might automate the entire research process instead of focusing on a specific research question. Also, for example, within an effort to automate validation, one effort might automate experimentation, another might automate proofs, and yet another might automate both of these. Thus, the second axis is what part of the research process is being automated and to what degree of generality.

Along this second axis, we will review our efforts in research automation. However, since there are a vast number of examples of research automation even in one sub-process, e.g., hypothesis generation, we will not go into individual details, but rather give priority to conveying the big picture.

% \textcolor{red}{NOTE: touch briefly do not go too specific, just show interpretation}
% \section{Knowledge Production}
% The Process of Creating New Knowledge

% Modern research is constructed from three main phases: observation, hypothesis generation, and hypothesis verification <- not line but cycle

\subsection{Question Construction}
In research automation studies, questions are often given. An exception is research that automatically discovers questions and issues from the academic literature. For example, Lahav et al. have proposed a methodology for automating the discovery of prevailing challenges within the research community, as well as the emerging hypotheses to address them \cite{lahav2022search}.

\subsection{Hypothesis Generation}
It is probably fair to say that hypothesis generation is one of the most common processes studied in research automation. Prediction of 3D structures from amino acid sequences, prediction of material structures from desired physical properties, and prediction of newly applicable disease candidates from existing drugs, just to name a few, are examples of automated hypothesis generation. In the example given in the previous section, symbolic regression is another example of hypothesis generation. These studies are often aimed at automation and optimization to solve bottlenecks in problems specific to each research area.

\subsubsection{Hypothesis Generation from Scientific Papers}

One study of automated hypothesis generation that is generic to a wide range of fields is the automatic generation of hypotheses from scholarly literature. In recent years, several researchers have presented studies on automating scientific analogical reasoning for identifying the relationship between problems and their corresponding solutions \cite{kang2022augmenting,chan2018solvent}, which is hypothesis.

% For example Portenoy proposed a system that recommend researchers who are pursuing analogous research objectives via divergent approaches \cite{portenoy2022bursting}.el techniques.

% \subsubsection{Others} 

Some methods have been proposed that don't generate hypotheses directly, but rather assist humans in generating hypotheses from experimental data 
 \cite{friederich2021scientific}.

\subsection{Hypothesis Verification}
Attempts to automate the verification of hypotheses include Bayesian experimental design, laboratory automation, and automated theorem proving, if theorems are viewed as hypotheses, as mentioned above. For example, some researchers have tried to automate experimental design for quantum physics \cite{ruiz2022digital} or proposed to design workflow of scientific research as a software \cite{goble2020fair}. Others have proposed machine learning algorithms for formulating and executing experimental designs in a more abstract and simple manner \cite{herrmann2022learning}. 

% \subsubsection{Verification Design}

% Once a hypothesis is formulated, a plan is developed to test its validity. The design of this verification plan is far more flexible than that of hypothesis generation, making it more difficult to handle uniformly. To be more precise, while many sciences have standardized methods such as statistical tests for verification, there is a wide variety of methods for generating the data used for the verification. One study may require a huge machine to collide elementary particles, while another may use rats for behavioral experiments. Some studies may require the use of chemicals, while others can be simulated on a computer. Furthermore, even with standardized statistical tests, as mentioned earlier, automating their creation from scratch proves exceedingly challenging. It is readily apparent that devising standardized methodologies like statistical tests is difficult when one must not merely employ them as tools but also contemplate the very nature of what it means to verify, as well as the rationale behind adopting specific assumptions. Therefore, it may not be an overstatement to say that this aspect represents the biggest obstacle towards achieving complete automation of research in a unified manner.

% \subsubsection{Verification Execution}

% Once a verification plan has been devised, the process proceeds in accordance with it. As previously mentioned, the approach may vary considerably. However, in many scientific methodologies, statistical techniques are employed. In these instances, the verification process can be broadly divided into two stages: 1. data generation and 2. analysis of the generated data for verification.

% As previously mentioned, data generation methods span a wide range. Among these, attempts have been made to automate the work of researchers within laboratories, an endeavor known as Laboratory Automation. For instance, 
% some studies focus on automating cell culture tasks using humanoid robots \cite{ochiai2021variable},
% TODO: add more

% TODO: may differentiate the analysis for hypothesis generation from that for verification
% we interpret data processed according to a certain criterion, assessing the validity of our inferences. Here, we make an explicit distinction between analysis for verification and analysis for hypothesis generation. Modern science is composed of a cycle of observation, hypothesis generation, and hypothesis testing. It's common to generate the next hypothesis to be tested from data produced for verification. However, this merely signifies that we conduct both hypothesis generation and testing through a somewhat inductive reasoning based on data. Therefore, in this context, we will focus on data analysis for hypothesis testing, while data analysis for hypothesis generation will be included in the hypothesis generation section introduced earlier.

To validate the plausibility of assertions, we currently employ statistical methods. There is research that automate the hypothesis testing \cite{gil2016automated}. 

Some researchers have engaged on automating data visualization and analysis \cite{bavishi2021vizsmith,bavishi2022tools}.

\subsubsection{Scientific Claim Verification}

Compared to question construction and hypothesis generation, there are few attempts to automate hypothesis verification related tasks from the academic literature. Somewhat related is the field of \textit{scientific claim verification} \cite{li2019scientific,wadden2020fact,wadden2022scifact,wadden2022multivers}, which determines the validity of a scientific claim through analysis of research paper. This is not the planning or execution of validation, but it is related to the automation of validation in the sense that it seeks to understand the validity of scientific claims. Since it is an assessment of the validity of a study, the findings of this study may have implications for the automation of peer-review.

\subsection{Pipeline (All Process)}
Most research on research automation automates some tasks in the research process. In contrast, there are attempts to automate the entire research process from start to finish.

\subsubsection{Self-Driving Labs}

A seminal early works are Adam \cite{king2004functional}, and Eve \cite{williams2015cheaper}. These are closed-loop scientific discovery systems that autonomously execute everything from hypothesis generation to research planning. These systems have logic AI at their foundation. The author of the paper of Adam call these system \textit{robot scientists}, \textit{self-driving labs}, \textit{autonomous discovery}, or \textit{laboratory automation}.

\subsubsection{Scientific Workflow}

Additionally, the concept of a \textit{scientific workflow}, which represents data and computational processing pipelines in research as software, emerged in the early 2000s. The developments and advances in research related to scientific workflows are consolidated in this literature \cite{barker2008scientific,atkinson2017scientific}. Additionally, these papers \cite{deelman2019role,nouri2021exploring} discusses how machine learning contributes to streamline the each step in the scientific workflow. These are important initiatives in terms of softwareizing the research process \cite{deelman2015pegasus,gil2011semantic}.

\section{Two Axes in the Same Page}

So far we have introduced the idea that we can look at research automation from two axes: the ``research field'' axis and the ``research process'' axis. We will try to paint a conceptual picture of how research automation can be viewed from these two axes.

\begin{figure}[htb]
    \centering
    \includegraphics[width=\linewidth]{figs/generality_matrix.jpg}
    \caption{Caption}
    \label{fig:generality_matrix}
\end{figure}

Fig. \ref{fig:generality_matrix} conceptually illustrates the position of automation according to the axes of research field and research process. The vertical axis is the research field axis and the horizontal axis is the research process axis. Closer to the vertical axis corresponds to more specific research areas, and farther away from the vertical axis corresponds to a broader range of research areas. The horizontal axis corresponds to question construction, hypothesis generation, and hypothesis testing, from left to right, and is further divided within each category into more detailed e.g. empirical and non-empirical methods. Note that this is only a conceptual diagram and not an exact classification.

As a example, the automation of protein structure prediction \cite{jumper2021highly} can be seen as the automation of a certain hypothesis generation task in a certain molecular biology study. The work of robot scientist \cite{king2004functional} can also be understood as an attempt to automate many of the steps in the study of identifying the function of genes in genetics, from hypothesis generation, to testing, to generating new hypotheses. (Explanation of the figure).

Our goal of creating a versatile and autonomous artificial researcher is to automate everything on this two-dimensional surface. The area circled by the blue qualification in this figure corresponds to that. In other words, our goal is to realize an artificial intelligence that can autonomously execute any research or any process of research.

\section{Reviews and Perspectives}
So far, we have introduced each of the areas related to research automation. In this section, we will introduce review papers and perspective papers on research related to research automation. Note that the number of review papers on automation in individual research fields is too large to cover them all, so we have only selected review papers that are more comprehensive or abstract.

\subsubsection{Scientific Workflow}
Gill presents an extremely exciting idea of conducting automated research by turning the entire scientific process into compositional and modular software with the literature review of her and her colleagues' work. \cite{gil2022will}. For example, Gill et al. have created a software of the semantic workflow of scientific data analysis and computation process  \cite{gil2011semantic}. 
Each step of this workflow modularizes the procedures in research. Not only can these make analysis more efficient, but they also allow for the analysis of the research process itself. Furthermore, common workflows can be identified from multiple workflows, enabling abstraction of cross-domain knowledge about research process. Additionally, Gill and her team have proposed DISK, a systematic framework for hypothesis testing and data analysis. DISK can automatically cycle through a series of processes, including the generation of hypotheses, the determination of data and methods to test them, the acquisition of data from shared repositories, the analysis of that data, and the modification of hypotheses. Furthermore, each hypothesis is associated with information on confidence level and analysis details, which significantly indicates the plausibility of the hypothesis. Additionally, as the hypothesis and its confidence level and analysis are continuously updated and the revision history is retained, it enables the continuous maintenance and update of scientific findings.

\subsubsection{Nobel Turing Challenge}
Kitano also harbors an ambition to automate the entirety of the research process \cite{kitano2021nobel}. This is a thought-provoking paper that is meticulously contemplated. Kitano underscores the ability of AI in automating science to execute exhaustive and thorough exploration as a significant strength. We, as humans, aim to generate hypotheses that yield impactful results (Kitano refers to this as a value-driven approach). However, the importance of research findings is context-dependent, and research that we humans deemed unimportant may become crucial if the presuppositions or the context alter. Kitano proposes to eschew this value-driven approach and implement an alternative, exploration-driven methodology to science, aiming for novel scientific discoveries that were unattainable by human capabilities. Besides, Kitano with many examples and detailed consideration, presents a plethora of stimulating ideas, such as the continuous hypothesis network update, a roadmap to achieve autonomous artificial scientists, and proposition of the Nobel Turing Challenge as a Grand Challenge to substantially advance these endeavors. We highly encourage those interested to delve into this fascinating read.

\subsubsection{Task-guided Scientific Knowledge Retrieval}
Hope et al. have written a captivating perspective paper on the automation of research, presenting a fresh and exciting viewpoint \cite{hope2022computational}. They introduce a human-centric idea aimed at efficiently extracting relevant information from the ever-expanding body of research data, tailored specifically to the tasks researchers are engaged in - a framework they term \textit{task-guided scientific knowledge retrieval}. They start by conceptualizing the act of research as an interaction between a researcher's \textit{inner cognitive world} and the \textit{outer world}, or \textit{scientific ecosystem}. Building on this, they underscore the vital role of representing and retrieving information that aligns with the inner cognitive world of researchers, deftly transforming the cognitive functions used in human research into algorithmic processes.

Extensive discourse transpires concerning scientific discoveries. Yet, discussions pertaining to scientific comprehension remain relatively unexplored. Hope et al. delve into the conundrum of what it entails for a machine learning agent to not only unearth scientific knowledge but also to comprehend it \cite{krenn2022scientific}. They adopt a human-centric stance, positing that an agent's ability to offer explanations comprehensible to human scientists signifies the existence of its scientific understanding.

\section{Others}

Upon the completion of a study, the drafting of a manuscript, and its successful navigation of the peer-review process, the resulting findings are deemed to possess a degree of credibility as knowledge. Naturally, it would be hasty to assert that this alone births ``correct'' knowledge, as research demands iterative verification to confirm its validity. We convey such knowledge to others through various means, one of which is the presentation of research findings. To effectively communicate these outcomes, we create slides that elucidate our work. Studies also exist that strive to automate this aspect of the dissemination process \cite{sefid2019automatic}.

\section{Conclusion}

\section{Archive}

\subsection{Pursuing Better Research}
Please emphasize once again that what I are aiming for is to achieve a better way of knowledge production. The realization of general and autonomous artificial intelligence is a means to that end. Therefore, even if certain research practices are widely adopted at present, I will strive for better methods if they are not essential for knowledge production and if there are superior alternatives. In other words, I propose to pursue the realization of artificial intelligence that enables a different and improved way of knowledge production, rather than simply automating the current research practices.

While the practices of research developed by humans are highly sophisticated, robust, and productive, it is not necessarily true that all of them have reached the absolute optimum way of doing research. Firstly, research practices are strongly influenced by history and society. For example, peer review is widely accepted today, but it is said to have gained such dominant status because foundations in Cold War-era U.S. demanded peer review to ensure accountability \cite{baldwin2018scientific}. Moreover, the fact that research outcomes are still represented in the form of papers is clearly a remnant of the era when print was predominant. These practices might have been optimal under the social conditions and technologies of that time, but they may not necessarily be optimal in the present, as society has changed a lot. Additionally, meeting societal demands does not always lead to the optimization of knowledge production. Secondly, it seems that I sometimes lack a widely shared understanding of how to optimally carry out certain aspects of research. For instance, it is important to formulate good research questions and hypotheses, but there are still many aspects about how to do this effectively that remain unclear, and it appears to rely heavily on individuals' tacit knowledge. In such a situation, simply replacing current practices with machines may not guarantee a good knowledge production process. Thirdly, as Nielsen and Qiu say \cite{nielsen}, it is believed that I have only explored in a very limited subspace in the space of possible research practices. The establishment of current research practices is a very recent event in human history, and I may not yet have arrived at the optimum way of research. \textcolor{red}{TODO: Add examples.} Finally, as I mentioned above, current research assumes that humans are the main creators and consumers of knowledge. This naturally imposes human cognitive constraints, which may significantly limit the range of activities that can be conducted for knowledge production. Based on these reasons, I aim to progress the discussion in a way that allows us to identify what is fundamentally essential for knowledge production, while also drawing inspiration from the positive aspects of past practices. I will not be bound by the current approaches but instead strive to clarify what is truly crucial for research.

\subsection{Call for Cooperation}
I would be delighted if all of you could contribute to writing this research paper. The initial draft of this paper was created by a novice machine learning researcher named OO. I have made efforts to write the text as carefully as possible, but due to his limited research experience, He may not be able to provide profound insights into the topic, and there might be inaccuracies in his understanding of fields that are not his expertise. In particular, the literature review chapter prioritizes comprehensive coverage of the literature, which means that the evaluation of individual references might not be appropriate. Also, there are naturally limitations and biases regarding the scope of papers that can be covered and the research areas targeted. Therefore, if you find any errors, missed literature, or points for improvement in the paper, I would be grateful if you could let us know.



\subsection{Towards Understanding this Universe}
It has been a long-standing goal of humanity to understand nature, the universe, and the world. Through this intellectual curiosity, mankind has made remarkable intellectual advancements and uncovered astonishing findings about nature. While perfect understanding of this universe may be infeasible, the desire to know as much as possible about as many things as possible in this universe is a shared sentiment among researchers.

However, it seems that there are limitations to achieving this goal solely through current human capabilities. This is because humans have several cognitive constraints when it comes to understanding phenomena. For instance, there is a limit to the amount of information humans can process, and comprehending highly complex, nonlinear, and non-equilibrium systems in their entirety is challenging. Such constraints appear to narrow the scope of nature that can be understood. \textcolor{red}{TODO: update the explanation of cognitive limitation}

Furthermore, there is a possibility that the methodologies of knowledge production employed by humans currently may not be optimal. This is because the research practices I have developed are heavily influenced by historical and societal factors. Many of these practices have emerged in a relatively short span of history, and there is still a vast unexplored territory in the methods of conducting research.

We believe that artificial intelligence capable of autonomously conducting all research holds great potential in overcoming these challenges and getting closer to the goal. Firstly, AI is not constrained by a physical device like the human brain, enabling processing speeds and capacities that are orders of magnitude beyond human capabilities \cite{hope2022computational,kitano2021nobel}. Secondly, by optimizing AI for knowledge production, it becomes possible to generate knowledge more efficiently, without being constrained by extraneous factors commonly encountered in human research. Lastly, and most importantly, AI has the potential to possess a different understanding of nature compared to humans. Moreover, AI might have the capacity to comprehend nature more extensively than humans. If that's the case, it seems essential to develop autonomously researching artificial intelligence if one aims to acquire as much knowledge as possible about this universe. Therefore, I propose the pursuit of autonomous AI researchers to realize these possibilities.

\subsection{Objectivity of Research}

The objectivity of research seems to arise from such demand that knowledge should be for humanity. Because of this requirement, researchers conduct research with extreme precision and caution. This serves as one of the justifications for the knowledge produced through research being reliable. I admit that finding the basis of objectivity of research is a challenging philosophical issue. Please note that what I have stated here is merely the humble intuition.

Seeking public accessibility in knowledge implies that the expressions of knowledge and the process of knowledge production should be understandable to everyone. For instance, in human societies, knowledge is often expressed with texts. This is because language serves as a highly universal means of conveying information that a significant number of people can comprehend and handle. Similarly, the process of knowledge production is also conveyed through language.


% \section{Chapter Title}
\subsection{Scope of the Discussion}
I think that the activities related to research can be broadly divided into two phases: the process of knowledge production, typically in the form of papers or other publications, and the phase of maintaining, sharing, evaluating, and utilizing the knowledge. While the latter phase is a crucial aspect of knowledge, it will not be addressed here since the focus of this paper is on knowledge production. 

However, it's important to note that the distinction between the two phases is not strictly delineated. For instance, the value and usage of knowledge in a society would influence the production process of it, and research evaluation and revision are ongoing even after the knowledge is produced into the society. Thus, evaluation of the value of knowledge may not be completely exclude from my discussion simply because they mainly take place after knowledge production. As such, I will make sure to touch upon elements that are not always necessary for knowledge. 

Furthermore, factors influenced by social constraints such as funding acquisition and collaboration agreements that typically occur before the research are not addressed in this paper. Knowledge production can be seen as a function that produces knowledge from inputs such as people, capital, resources, and information. Viewing it this way, the acquisition of money or human capital can be considered inputs to the function rather than the knowledge production itself. Thus, I do not touch on this topic in this chapter. I admit that these factors are also crucial elements in the pursuit of true autonomy in research, so they will be subject to future discussions. Also, even if it is information acquisition, if that is strongly related to the internal of knowledge production itself, I will address it later in this paper. 

In conclusion, the following text focuses on the process of research, where knowledge is output as research outcomes given any input. For simplicity, I will refer to this process as the \textit{knowledge production system} or just \textit{research process}. It is worth reiterating that this distinction is provisional, intended to facilitate further discussion and the development of better categorizations or broader automation in the future.

\begin{figure}[htb]
    \centering
    \includegraphics[width=\linewidth]{figs/research_process_society.jpg}
    \caption{Knowledge Production System and Society}
    \label{fig:research_process_society}
\end{figure}

\section{Research as Information Processing System}
Knowledge production system can be seen as an information processing system.\footnote{My discussion here is influenced by David Marr's three levels  \cite{marr2010vision}, although it is not exactly the same.} First of all, all information procesing systems have an objective and achieve that objective by processing inputs. In current case, the objective in research is to produce a new knowledge for a society. The reason I divided the research process into the question construction, hypothesis generation, and hypothesis verification is equivalent to breaking down this information processing into three sub-modules. Each of these three sub-modules is also an information processing system with its own objective. For example, the objective of the question construction module is to discover unknown knowledge.

When I have mention in this section that I would focus on the parts functionally related to knowledge production, it have meant two things. The first is to ensure that the objectives of these sub-modules are essential sub-objectives in all research. As mentioned earlier, these modules should not be something that works for history research but not for mathematics research. I should consider modules that have objective essential to all research for realizing general artificial researcher. Additionally, the process of peer review, while relevant to knowledge production, was considered to be a process of judging the ``value of knowledge'' and so was not included in the knowledge production system. The second point is that the my discussion is limited to the objective of the modules and not about the specific representation of information or the specific algorithms for information processing within the modules. For example, reading and writing papers are currently universally necessary for any research. However, a paper is just one of all representations of information, and there could be alternative representations (e.g., blogs). Moreover, while one person may conduct experiments using rats to verify a hypothesis, another person may present historical records showing statements from deceased individuals that match the hypothesis. These are different methods of verification, but both serve the purpose of verification.

% \textcolor{red}{TODO: Why this structurization is useful}

\begin{figure}[htb]
    \centering
    \includegraphics[width=\linewidth]{figs/information_processing_system.jpg}
    \caption{Knowledge Production System}
    \label{fig:information_processing_system}
\end{figure}

\begin{figure}[htb]
    \centering
    \includegraphics[width=\linewidth]{figs/knowledge_production_system.jpg}
    \caption{Knowledge Production System}
    \label{fig:knowledge_production_system}
\end{figure}

\section{Between the Knowledge Production System and Society}
In this section, I will discuss how peer review is related to knowledge production. The reason for discussing peer review is primarily because it is a widely practiced convention across various fields. I will discuss the role of peer review as a kind of membrane that spans both the preceding knowledge production and the subsequent knowledge utilization. I will also explore the possibility that the redundancy in the functions of peer review has played an essential role in the knowledge production of human societies. Lastly, I will discuss the implications of these conclusion for research automation.

% Peer review is a social convention developed in human societies.

% whether it is necessary or not for research, it is impossible to avoid discussing something that has been so widely accepted when considering research automation. 



\subsection{Peer Review}
In current research, a small group of experts review papers and the results that pass through their review are published. This process is called \textit{peer review}. Peer review is now considered an indispensable practice in research across a wide range of fields.

Peer review, particularly pre-publication peer review, is commonly regarded as playing the role of a gatekeeper of knowledge, determining what becomes knowledge and what does not. It enables the production of high-quality knowledge by assessing the quality before publication. Here, let's delve a bit deeper and examine the specific role that peer review plays in knowledge production.

\subsubsection{Epistemic Value Evaluation}

We believe that peer review serves two distinct roles in knowledge production. The first role is evaluating the fulfillment of necessary conditions for the knowledge of a research. This involves assessing aspects such as the validity of the methodologies used and the sufficiency of the content's novelty. 

This process could potentially be skipped if these evaluations were flawlessly conducted in every research endeavor during the knowledge production process. In that sense, I can say that peer review is redundant as a function for knowledge production, as it duplicates the functionality of what researchers have already done. If this process is merely redundant, does it mean that this process is unnecessary?

We would say that redundancy has played an important role in human knowledge production. Firstly, humans naturally make mistakes and have blind spots that they may not be aware of. It seems important to enhance the reliability of the produced knowledge by detecting such oversights with multiple checks. Secondly, research is fundamentally conducted on the assumption of goodwill, making it important to have a process to verify if any misconduct or ethical issues have occurred for the sake of a healthy knowledge production. Having a mechanism for third-party evaluation can be considered a deterrent against fraudulent activities. Lastly, as emphasized repeatedly in this paper, verification is an act of updating collective beliefs. Therefore, the existence of a process to confirm whether the verified results are acknowledged by others seems essential in the context of knowledge production for that society. Considering these factors, I believe that the first point, which functionally exhibits redundancy, has played an important role in human knowledge production.

\subsubsection{Non-Epistemic Value Evaluation}

The second role is about evaluating whether a proposed study has ``value'' to the research community and society. This involves judging factors such as the importance of the research, its clarity, and its ethical validity, etc. This function is obviously important as knowledge from research is supposed to be used by the research community and society.

This evaluation pertains to the submitted knowledge rather than the process of knowledge production itself. In a sense, it can be seen as an anticipatory assessment of the value judgments that will be made when the knowledge is disseminated in society. In other words, just as peer review was considered a redundant process in knowledge production for the first role, peer review can also be seen as a redundant process in terms of the judgments made on knowledge disseminated in society.

One role of redundancy in such judgments of social value would be to reduce the number of research results that knowledge consumers need to pay attention to. This has played an important role in human knowledge production, given the cognitive constraints of the limited number of papers that humans can process.

\subsubsection{Peer Review as a Redundant Membrane}

In summary, the first role can be interpreted as an extension of the knowledge production process, while the second role can be seen as a pre-emptive value judgment by society. In other words, the act of peer review serves as a redundant membrane that connects the process leading to the production of knowledge with the subsequent processes of knowledge use. \footnote{
In peer review, it is customary to assess whether there is sufficient information to reproduce the research. This evaluation of research reproducibility can be considered, in a sense, an assessment of its epistemic value. This is because, instead of evaluating the validity of verification through replication at that moment, the request for reproducible information aims to leave the possibility of future verification.
} Inspired by the words of Seetl, I can say that the former role is the judgement of \textit{epistemic value} and the later is that of \textit{non-epistemic value} \cite{steel2010epistemic}. 

Here, epistemic value refers to properties that are intrinsically important to knowledge production, such as necessary conditions for knowledge production. For example, I will call judging the validity of verification a judgment of epistemic value in the sense that it leads to judging whether the activity can be called knowledge production or not. 

The term non-epistemic value, on the other hand, is used in this paper to mean a value without which the legitimacy of knowledge production would not be affected. For example, if a knowledge production process were unethical, what it produced would still be knowledge, but it would be non-epistemic value in the sense that it is important to society. The readability or ``importance'' of a paper is also a use value of knowledge per se, but it has nothing to do with the legitimacy of knowledge production, so I call it a non-epistemic value. When I simply refer to ``value'' in this paper, I are referring to this no-epistemic value.

\begin{figure}[htb]
    \centering
    \includegraphics[width=0.8\textwidth]{figs/peer_review.jpg}
    \caption{Peer Review}
    \label{fig:peer_review}
\end{figure}

As an example, I show the review criteria of Nature in table \ref{tab:nature_review}. Each criterion is classified into epistemic value (evaluation of whether it fulfills the necessary conditions for knowledge production) and non-epistemic value (evaluation of whether it holds some value within the research community or society). From this example, it becomes evident that in peer review, research outcomes are evaluated from both epistemic and non-epistemic perspectives. 

\begin{table}[H]
\centering
\begin{tabularx}{\textwidth}{|X|X|X|}
\hline
\textbf{Criteria} & \textbf{Category} & \textbf{Explanation} \\ 
\hline
Key results & NA & Please summarise what you consider to be the outstanding features of the work. \\ 
\hline
Validity & Epistemic & Does the manuscript have flaws which should prohibit its publication? If so, please provide details. \\ 
\hline
Data \& methodology & Epistemic & Please comment on the validity of the approach, quality of the data and quality of presentation. Please note that I expect our reviewers to review all data, including any extended data and supplementary information. Is the reporting of data and methodology sufficiently detailed and transparent to enable reproducing the results? \\ 
\hline
Appropriate use of statistics and treatment of uncertainties & Epistemic & All error bars should be defined in the corresponding figure legends; please comment if that’s not the case. Please include in your report a specific comment on the appropriateness of any statistical tests, and the accuracy of the description of any error bars and probability values. \\ 
\hline
Conclusions & Epistemic & Do you find that the conclusions and data interpretation are robust, valid and reliable? \\ 
\hline
Suggested improvements & Epistemic & Please list additional experiments or data that could help strengthening the work in a revision. \\ 
\hline
References & Epistemic & Does this manuscript reference previous literature appropriately? If not, what references should be included or excluded? \\ 
\hline
Originality and significance & Non-epistemic & If the conclusions are not original, please provide relevant references. On a more subjective note, do you feel that the results presented are of immediate interest to many people in your own discipline, and/or to people from several disciplines? \\ 
\hline
Clarity and context & Non-epistemic & Is the abstract clear, accessible? Are abstract, introduction and conclusions appropriate? \\ 
\hline
Inflammatory material & Non-epistemic & Does the manuscript contain any language that is inappropriate or potentially libelous? \\ 
\hline
\end{tabularx}
\caption{Nature Review Criteria}
\label{tab:nature_review}
\end{table}

\subsubsection{Implications for Research Automation}

If peer review is considered redundant in knowledge production, then the necessity of the peer review process when conducting research with artificial intelligence, which may have fewer cognitive constraints than humans, becomes a topic of discussion. If there are no occurrences of human errors, it may not be necessary to have a redundant process. So is peer review simply a social product of human cognitive constraints, or does it play an important role in knowledge production even in automated research?

We believe the importance of redundant valuations will not be lost in automated research. First, regarding the function of detecting oversights, automating research may eliminate certain types of mistakes, but overlooking aspects is still possible. For instance, as mentioned earlier, it is impossible to enumerate all the assumptions involved in the research process. Review could help by identifying relevant assumptions that need consideration, potentially altering the implications of the validation results. Second, against malicious injustice, this also happens in the automated research as well. There is a possibility of creating artificially intelligent researchers that could maliciously manipulate and engage in fraudulent activities. Furthermore, in the ultimate stage of full autonomy, as pointed out in alignment research, there is a potential for machines to deceive humans as well. Also, since the goal remains the same, whether the research is done by humans or by machines, to update joint beliefs, it is just as important that it is actually evaluated by other members of society. Finally, in the first place, the nature of research is that hypotheses are continually subjected to constant testing. As mentioned earlier, a hypothesis tested by inductive reasoning cannot be declared true in the same way that deduction can. It is common for the knowledge to be later discovered to be erroneous or merged with other new knowledge. Therefore, it seems to me that this redundant process of criticism is essential for research, and should not even be imposed on a single process of peer review. For these reasons, it would seem that redundant evaluation from a third-party perspective should somehow be accomplished even in automated studies.

From a slightly different perspective, aiming to automate peer review of papers may be a useful first step toward generic and autonomous research automation. This is because peer review involves judgments about the validity of verification, as well as judgments and novelty determinations, factors that I have emphasized as important in research automation. Moreover, while finding novelty and performing validation are generative tasks, this is an classification task, and in that sense it is an easier task. Similarly, I pointed out that it is important to inculcate the value of what is important to the research community if I want machines to construct ``important'' questions. Determining non-epistemic values in peer review can be thought of as an classification task for these elements, and may help us clarify the issues to realize artificial research. In addition, peer review requires similar evaluation criteria regardless of the particular discipline. For these reasons, if I aim to automate the peer review of papers, I may in the process better clarify the elements necessary to realize a general and autonomous artificial researcher.

\subsubsection{Towards a Better System}
We do not believe it is necessary to automate current peer review practices as they are, because what is needed is a redundant evaluation function, and the current practice contains more historical and social influences than that. For example, researchers now almost always employ pre-publication peer review, but there are issues raised regarding this pre-publication peer review \cite{heesen2021peer}. Instead, a system that continuously evaluates all research findings during the process and after the production of results, continuously updating knowledge, may lead to more robust and high-quality knowledge production. Also, although I currently do peer-review by a small number of experts, it would naturally be more robust if more agents participated in the evaluation.

It is also no exaggeration to say that in the current function of peer review, researchers place the greatest importance on the function of screening for submission to good journals. But this has nothing to do with epistemic value and is a practice that I do not think needs to be retained.

We believe that the culture of submitting to top journals plays the following role in research. First, it serves to lower the cost to the reader in determining which papers to read. By prioritizing results from well-known journal brands, it means that they can read the more important papers first. However, if judgments of non-epistemic value were automated, so that machines could directly judge the value of research, then indirect value judgments via brands would no longer be necessary. More to the point, if the readers of papers are turning into machines, they will be able to process a much larger volume of papers than humans, and the selection of papers to read in the first place may not be as important as it used to be. The second role is to help researchers evaluate their work. It can help third parties determine that a researcher with papers in top journals is probably a researcher of a certain level of ability. This has been a strong incentive for researchers to submit papers to top journals in order to find good careers in the tight academic job market, to prove that they are decent researchers, and to satisfy their own sense of honor. However, this function will become less important when the subject of research becomes a machine. Therefore, it seems desirable to move away from the function of reviewing papers for submission to good journals, which is currently the most dominant function, and move toward optimizing the evaluation of epistemic values and direct evaluation of non-epistemic values.

Although peer-review is well prevalent research practice, it is said that peer review became commonplace like today only from the 1970s onwards \cite{baldwin2018scientific}. Also, as I explained in Chapter 1, it emerged just as a response to the demanded to ensure accountability in society \cite{baldwin2018scientific}. Therefore, I can say that the practice of peer review is still in its infancy, and I should be able to come up with a more optimized way of being that removes constraints that are not related to knowledge production. Thus, I should look for a better automated system that takes over the role that peer review plays.

% The review evaluates papers from multiple different perspectives. For example, at NeurIPS 2022, papers are evaluated based on the criterias of originality, quality, clarity, and significance.

% I would say that redundancy has played an important role in human knowledge production. Firstly, humans naturally make mistakes and have blind spots that they may not be aware of, so multiple checks hold significant value. Automating research may eliminate certain types of mistakes, but overlooking aspects is still possible. For instance, as mentioned earlier, it is impossible to enumerate all the assumptions involved in the research process. However, critics could help by identifying relevant assumptions that need consideration, potentially altering the implications of the validation results.

% Additionally, research is fundamentally conducted on the assumption of goodwill, making it important to have a process to verify if any misconduct or ethical issues have occurred for the sake of a healthy knowledge production. This also applies in the case of research being automated. There is a possibility of creating artificially intelligent researchers that could maliciously manipulate and engage in fraudulent activities. Furthermore, in the ultimate stage of full autonomy, as pointed out in alignment discussions, there is a potential for machines to deceive humans as well.

% Lastly, as emphasized repeatedly in this paper, verification is an act of updating collective beliefs. Therefore, the existence of a process to confirm whether the verified results are acknowledged by others seems essential in the context of knowledge production for that society. Considering these factors, I believe that the first point, which functionally exhibits redundancy, plays an important role in knowledge production.


\section{Techniques for Knowledge Production}

In this section, I will discuss the ``techniques'' commonly used in research. Because they are techniques, they are not objective of modules in knowledge production system. However, they play an extremely universal and significant role in knowledge production across various fields in current human knowledge production. 

In particular, I will discuss \textit{information retrieval}, \textit{literacy}, \textit{data analysis}, and \textit{deductive reasoning}. If there is absolutely no access to any information whatsoever, knowledge production is considered impossible. Research, like other activities, requires acquiring information. Therefore, techniques for information retrieval, including searching and questioning, are essential in research across various fields. In particular, research builds upon existing knowledge, often in the form of research papers, to generate new knowledge. Thus, I will pay particular attention on scholarly literature search. 

The second aspect is literacy, the ability to read and write texts. In human society, I express many kinds of information in text. Therefore, the ability to read documents is inherently important in information acquisition. Additionally, I express knowledge through the generation of papers, which serve as a medium of knowledge. Thus, writing texts also matters for knowledge production. These discussions revolve around the ``representation'' of knowledge and information, which are constrained by societal practices. However, without these skills, conducting research in human society would be impossible, making them necessary abilities across all fields. Hence, I will discuss reading and writing skills. 

The third aspect is data analysis. As mentioned earlier, a significant portion of research is empirical. Data analysis is essential in the process of constructing questions from observations, in generating the next hypothesis from the results of an experiment, and in deriving an interpretation of the verification results from the data generated for hypothesis verification. Therefore, data analysis is a necessary skill regardless of the research field. 

The fourth aspect is deductive or systematic reasoning. This is essential in mathematics and natural sciences and is required across various fields in the natural sciences. Therefore, I will revisit this topic for further discussion. 

Although all of these are indispensable in current research, they should be consciously distinguished as ``techniques'' or ``methods'' aimed at achieving the ``purpose'' of question construction, hypothesis generation or hypothesis verification. As I told before, I pay careful attention to making this distinction. There are many important techniques in research that were not covered here. The ones discussed are only a small fraction, and they were touched upon briefly in this paper. It would be highly significant to have broader and more detailed discussions about the crucial techniques in research, as it would further advance research automation.

\subsection{Information Retrieval}
A simplified research process is a function that outputs knowledge, given an input. However, in reality, to execute every single individual process in knowledge production, it is necessary to collect information from the world and use them as inputs. For example, to construct a research question, one may need to search for academic papers. Similarly, to evaluate the performance of a proposed algorithm, one may need to acquire a benchmark dataset. In this way, research can be seen as an act of producing knowledge by taking in all the information from the world as inputs. Therefore, it is not an overstatement to say that information retrieval is an essential technique in knowledge production. 

Hope et al. have proposed a similar but more sophisticated depiction of this view in their perspective paper \cite{hope2022computational}. They regard the research as an interaction between a researcher’s inner cognitive world and the outer world and emphasize the importance of knowledge retrieval aligned with human cognitive world. Although the equivalent of the inner world in this paper is not the human cognitive world but the knowledge production system, the perspective of distinguishing between the external source of information and actual knowledge production mechanism shares similarities.

Information retrieval is a major topic with a long history and the scope it covers is extensive. Therefore, I will not discuss information retrieval research per se, but only those that may be directly relevant to the automation of research. However, even when limited to those related to research, information retrieval is a highly flexible topic as the methods for it vary depending on the research subject. For example, retrieving publicly available data that can be used for research, or in the case of machine learning, retrieving models, is also information retrieval. Therefore, our discussion here will focus first on scholarly literature retrieval, which is generally required for almost all research, among research-specific information retrieval. Subsequently, I will expand the discussion to implications for research automation in general.
% models that involve computer operations, including browser manipulation, and explore the implications they have on research automation.

\subsubsection{Scholarly Literature Retrieval}
% \textcolor{red}{TODO: Change}
Research is an endeavor to create knowledge based on existing research. Therefore, the first step is to obtain the existing literature that is deemed necessary by conducting a scholarly literature search. A bibliographic search is an attempt to retrieve the scholarly articles that best meet a given requirement.

As you can see from the above description, the construction of the aforementioned literature-based question is an example of a scholarly literature search. Therefore, what I basically need to do is 1. search for articles 2. make a judgment on the articles. In the construction of the question, I determined whether or not the article is an unknown study, but this determination depends on the purpose of conducting an academic literature search. For example, you may be searching for a paper to refer to previous experiments conducted for a similar purpose when planning a validation project. Or you may be looking for papers to identify recent research trends.

In determining whether a paper is useful for your purpose, you need to understand the contents of the paper to varying degrees. This seems to call for two things immediately. The first is obviously an understanding of language (in the case of human society), and the second is an understanding of the thesis. Since these are important issues, I will discuss them in the next section on literacy.

This was the case of direct acquisition of academic bibliographic information, but if that knowledge is stored as tacit knowledge in a large-scale artificial intelligence, the same consequences as discussed in the question construction section will result. If I can directly obtain the desired information by giving instructions to a machine that has accumulated a large amount of article data, the search, reading, and decision making will become an end-to-end process. I have already discussed this point, so I will not discuss it in detail here.

% Research is an endeavor to create knowledge based on existing studies. Therefore, the first step is to search for papers that should be read. 

% To find the necessary papers, you have to know what is written in each paper. Therefore, \textit{search} is closely related to \textit{reading}, which will be discussed in the next section. Here, For convenience, I will distinguish between the two: the former refers to finding the necessary papers from a large collection of papers, and the latter refers to extracting necessary information from the obtained single paper.

% I shall make mention of the relationship between these concepts and the activity commonly referred to as a \textit{survey}. I define the survey as a series of processes of 1. searching for necessary papers, 2. extracting information from multiple papers, and 3. comparing them to make some kind of decision. Please note that comparison, searching, and reading are all closely related to each other in this context as well; for instance, proper comparison between papers is necessary for better searching.

% The distinction between the aforementioned tasks of reading and searching, as well as the definition of survey, are based solely on the fact that I humans distinguish between them. However, if desired information could be directly obtained through natural language instructions from a large set of academic paper data, the tasks of searching, reading, and comparison would become an end-to-end process. Thanks to the remarkable development of large-scale language models in recent years, such a possibility has become a realistic one. Further details on this possibility will be discussed later.

\subsubsection{Implications for Research Automation}
As the discussion so far has shown, information retrieval depends on 1. the purpose of the information retrieval and 2. the representation of the information to be retrieved. From the first point, it seems necessary to first understand what kind of information retrieval is typically required in research in order to realize the most general-purpose information retrieval method possible in research automation. As I have discussed, research is considered to consist of question formulation, hypothesis generation, and hypothesis testing, so it is important to discuss what kind of information retrieval is required to achieve these objectives. On the other hand, as I saw in the section on hypothesis testing, a considerable degree of freedom is required in the execution of testing. It seems to me that it would be extremely difficult to achieve a unified information retrieval method that includes all of these factors.

The second point also seems to pose a major challenge. It would be fine if the structure of information were highly standardized, as in the case of a paper, but there is a great variety of data and model representation formats and their storage locations, for example. Since the efficiency of search depends on the representation and nature of the search target, it is a difficult task to realize the uniform and fully automatic acquisition of such information. It will be necessary to consider how these issues should be addressed when trying to realize a general-purpose and autonomous artificial researcher.


% \subsubsection{\textcolor{red}{ML Models for Information Retrieval}}
% \textcolor{red}{TODO: May not be here}

\subsection{Literacy}

It is not an exaggeration to say that language is one of the most significant features that sets humans apart from other animals. In human society, I express considerably large amount of information through language. When I consider the impact brought about by recent language models, I can understand just how significant language is to us.

The ability to handle language, or \textit{literacy}, is essential for communicating research findings in human society. This is because, in human society, knowledge is also conveyed in the form of written documents, particularly academic papers. The ability to handle language can be broadly categorized into \textit{reading} and \textit{writing} texts. The former is necessary for extracting desired information from research papers and the later is necessary for representing the research outcome as a knowledge in a society.

It is fair to say that until recently, one of the number one barriers to research automation was acquiring this ability to handle language. However, this situation has drastically changed in recent years with the remarkable success of large-scale language models. The development of large-scale language models has largely removed the technical difficulties of understanding articles, so that I can now focus on understanding the specifics of the research.

Of course, there is still work to be done to fully understand the language, but the topic of how to acquire language use skills is far beyond the scope of this paper and will not be addressed here. Instead, I will discuss the most important textual data in research: the scholarly articles themselves. This is because being able to read and write well in research cannot be discussed without an understanding of what a scholarly article is about. After that, I will examine the implications it brings to research automation.

% Therefore, in this section, I would like to discuss what a research paper is and what constitutes a good research paper. Then, I will examine the implications it brings to research conducted with AI.

% \subsubsection{Reading}
% As previously mentioned, acquiring information from academic papers is a fundamental task necessary in all aspects of research.

% In particular, there may be cases where one does not even know where to find the necessary knowledge. Therefore, in order to obtain the required information, it is necessary to first search for the academic papers themselves where the information is stored. 
% Additionally, researchers sometimes have to compare multiple papers. Researchers need to demonstrate in the paper that the problem they are trying to solve is truly unknown, and that their proposal is truly novel.

% A survey combines all of these tasks. In other words, it is the process of information retrieval and extraction from multiple academic papers followed by decision-making.
\subsubsection{Characteristics of Research Paper}
An academic paper is a structured document that summarizes research procedures and findings. It is said to have originated around the 17th century but became more common in the 19th century. To read a research paper effectively or write a good research paper, it is crucial to first consider the role that a paper should fulfill. 

Above all, a research paper is an expression of knowledge. It serves as a kind of ``asset'' for humanity, where new knowledge is generated based on the knowledge. Therefore, it is expected to possess information related to knowledge production in as detailed and accurate a manner as possible.

Furthermore, since knowledge is intended to be used by third parties, a research paper always assumes the presence of a reader. Therefore, it is necessary for the paper to be easily understandable, which means it should have a low information acquisition cost. For instance, during peer review, clarity and delivery are evaluated, focusing on the value of the paper as a ``report'' of knowledge rather than the knowledge itself. One of the attempts to enhance comprehensibility is through the structure of the paper. The prevalence of structured papers as I know them today seems to have emerged in the 20th century \cite{harmon1989structure}. 
The structure of a research paper varies depending on the field, but in empirical scientific papers, a widely adopted format is known as IMRaD, which stands for Introduction, Methods, Results, and Discussion. Introduction, Methods, Results, and Discussion can be interpreted as having roles that express what questions were studied, how those questions were investigated, what discoveries were made, and what the significance of those discoveries is, respectively \cite{gastel2022write}.

Moreover, the social aspects can also influence the content of a research paper. For example, in the current academic community, emphasis is placed on publishing papers in top journals or getting papers accepted at top conferences. These factors are not only important for the researcher's reputation but also crucial for securing academic positions in a highly competitive job market. Due to these pressures, it is said that the motivation to present research findings attractively can lead to distortions in the content of the research or make it less comprehensible. In the sense of making the results more appealing, a research paper could be likened to a kind of ``artwork''.

\subsubsection{Implications for Autonomous Research}
Having provided an overview of academic papers, I would like to take this opportunity to express our opinion on the significance of literacy when conducting research with AI. First and foremost, it is important to note that literacy is merely a means for information acquisition and expression in the context of knowledge production. Therefore, effective reading of a research paper depends on the manner in which the paper is presented, and writing a good research paper relies on determining the most valuable expression of knowledge in relation to the specific use case. Note that the term ``valuable'' here refers to non-epistemic value rather than epistemic value.

In the previous section, I provided an overview of the various roles that research papers fulfill. Among them, I believe that the role of knowledge representation remains significant regardless of whether the producer of knowledge is human or machine, depending on the context. In other words, information such as research questions, hypotheses, experimental designs, and results will continue to be expressed. When considering machine readers, it is possible that there will be an increased demand for or capability to provide more detailed information or information closer to raw data.

Next, let's discuss clarity. Clarity is a relative concept that pertains to the reader. Traditionally, research papers have been designed for human researchers as the intended audience. However, if I shift the assumption to machine readers, the value of clarity in papers as perceived by humans is likely to undergo transformation. For instance, the IMRaD structure has historically evolved to make it more comprehensible for humans. However, whether this structure is optimal as an information source for machines may need to be reconsidered. Even without waiting for complete automation, there has been rapid development in language models using question-and-answer formats for extracting information from unstructured documents. Considering this, it may become more important to provide comprehensive and accurate information rather than devising structures that are simply easy to read.

\subsubsection{Towards a Better Representation of Knowledge}
The trend toward creating artificial intelligence that has acquired knowledge of a large number of papers will further accelerate in the future. The problems I face today may eventually be solved, and it may be able to understand the contents of a vast number of academic papers almost perfectly. 

The problem will be that academic papers do not represent all the information in the research process. For example, I take considerable pains in our research to construct questions and hypotheses, but information about the process of generating them is not often written in the papers. The process of verification is described in much more detail than other processes, but even then there is often tacit knowledge that is not documented in the paper. Also, the papers do not present the process of knowledge production in chronological order as it is, but rather reorganize and recapitulate it in an easy-to-understand manner so that it is compelling to the reader \cite{schickore2008doing}. For example, it is often said that the process of building up a mathematical proof is different from the actual proof flow described in the paper. 

This may not be a problem for those who obtain only fragmentary information from the paper, but it becomes a problem when the paper is regarded as data for learning the knowledge production process. Thus, as long as I follow the current format of papers, the format may constrain what artificial intelligence can do. If this is the case, then perhaps I should adapt the form of representation of scholarly output itself to the machine. In particular, it will be even more important in the future to stock as much raw data of the research process as possible. If this can be achieved, large-scale artificial intelligence may be able to learn data-driven knowledge production, rather than merely overview knowledge of research.

It would not be realistic to suddenly abandon the current format of the dissertation. Rather, what I should do is first keep a raw log of the research process as much as possible. Then, separate the data of the research process from the final expression of the thesis. The dissertation should be positioned as an expression of one of these logs of the research process. This will allow us to accumulate richer machine-accessible data on the research process while preserving the current form of the thesis.

% While academic papers are already structured into sections such as introduction, method, results, discussion, and conclusion, I believe that further sub-structuring of these sections could make it easier for readers to gather information. For example, the introduction section contains a broad range of elements, but breaking it down into more detailed subheadings could help readers more easily access the information they need.

% There are various techniques for writing academic papers, but they are all designed with the assumption that humans will be reading the paper. Papers are considered to be ``reports'' and are expected to provide information to readers at a low cost. Additionally, papers are usually peer-reviewed and published in academic journals, so it is necessary to write attractive and engaging papers that will be accepted by the best journals. In this sense, papers are also ``works of art.'' However, I believe that the essential nature of papers lies in their role as the foundation of knowledge production, making papers an asset in terms of their ``knowledge'' aspect.


\subsection{Data Analysis}
A considerable amount of research is empirical in nature, meaning that it involves inductive reasoning supported by some form of evidence rather than relying solely on complete deduction. Therefore, although the quantity, quality, and characteristics may vary, these studies all generate some kind of data. Consequently, the techniques for analyzing this data are recognized as crucial across a wide range of research domains, regardless of the specific field. \textcolor{red}{TODO: add discussion about data analysis itself}

Please note that when I talk about data analysis, I are not limiting it to any specific operation on the data. In other words, it could involve generating hypotheses from data, testing hypotheses, or conducting verification. Descriptive statistics, inferential statistics, and predictive statistics are all encompassed within data analysis.

We have extensively discussed inductive reasoning so far, and since statistical machine learning itself is data analysis, there may be no need to dedicate a separate chapter for its explanation. However, I chose to include this chapter because in the research process, discussions often separate data generation through experiments from the analysis of that data. In our understanding, both data generation through experiments and data analysis are two distinct tasks within the function of, for example, verification. By emphasizing that data analysis is a task to achieve that goal, I believe it makes our position more understandable. 

You may raise doubts about the validity of categorizing the research process into a unidirectional ``research process'' by pointing out that in a single experiment, the same data can be used for both verification and generating new hypotheses, reflecting the trial-and-error nature of actual research. However, as I emphasized earlier, what is important is not the chronological sequence but the role that each process plays in knowledge production. According to our organization, even if the same data is used, if it is used for verification, it is considered verification, and if it is used for generating the next hypothesis, it is considered hypothesis generation for the next study. Emphasizing that data analysis itself is a technique that does not have a meaning beyond manipulating the data helps convey our understanding that its relevance to knowledge production is relatively determined by how it is used in different knowledge production processes.

\subsection{Deductive System}
Up until now, I have primarily discussed inductive reasoning, but deductive systems such as logic and mathematics are indispensable and highly significant in research. Deduction is a logical reasoning method that derives conclusions from premises in such a way that the conclusion necessarily follows if the premises are accepted. It is said to guarantee truth-preservation because if the assumptions are true, the conclusion is guaranteed to be true. So far, it has been mentioned that it is ultimately impossible to prove the truth or falsehood of a hypothesis through empirical methods. In contrast, deduction is an extremely powerful method of verification because it can prove the truth or falsehood of a hypothesis. Furthermore, regardless of how counterintuitive the conclusion may be, if the premises are accepted and the inference rules are applied appropriately, I must accept the conclusion. In this sense, deduction allows us to think about the world free from human intuition and cognitive biases, making it a very powerful means of understanding the nature.

\subsubsection{Mathematics}
Mathematics has played an essential role in knowledge production, particularly in the field of natural sciences. There are various opinions on why mathematics is crucial in the realm of natural sciences. Firstly, as mentioned earlier, one reason is its deductive nature. As repeatedly stated, knowledge production involves accumulating evidences of belief in the face of the unknown. Conclusions derived through truth-preserving deduction immediately provide beliefs of the same strength as the premises, thus playing a vital role in unraveling the unknown. 

Secondly, mathematics is rigorous. While natural languages rely on context and often contain ambiguity, mathematics provides a precise and unambiguous means of representing and communicating knowledge. It has served as the language of science, playing a significant role in scientific discourse.

Thirdly, mathematics (and more broadly deductive systems) is debuggable. As emphasized in the hypothesis generation and hypothesis verification sections, in mathematics all assumptions are explicitly stated. Thus, the causes of the results of a deduction can be reduced to a matter of comparison among a finite number of elements.

Lastly, mathematics is abstract. From the ancient time, even in the absence of formal deduction, mathematics dealt with the concepts such as numbers, which is greatly abstract concept of great interest to humans \cite{david2010history}. Through the introduction of symbolic representation and manipulation, mathematics has been further enhanced in its abstract nature. Moreover, mathematics not only abstracts reality but also engages in a cycle of further abstraction. By repeatedly abstracting the abstracted, it has constructed highly abstract systems \cite{bochner1968role}. Abstraction involves extracting only partial common properties of objects, and it is closely related to understanding, generating more universal knowledge. Additionally, through the premises established by abstract concepts, further deductions lead to a broader understanding of the world \cite{heisenberg2008abstraction}. I believe these characteristics make mathematics an indispensable part of knowledge production. 

Thus, the ability to utilize deductive systems is extremely important in expanding the realm of practically discoverable knowledge. Therefore, when it comes to achieving intelligence capable of autonomous research, it becomes crucial to consider how to construct new deductive systems or at least enable the utilization of existing deductive systems.

\subsection{Behavior}
Finally, although this is not a technique, let us stress the importance of being able to behave freely. As I repeatedly emphasized in the section of verification instantiation, acquiring the ability to act freely is essential to truly achieve a general and autonomous artificial researcher. This means that if the research is completed within the computers, the researcher can perform any operation on the computers at will, and if the research occurs within the real world, the researcher can handle all four limbs at will, just like a human being.

\subsubsection{Behavior in computers}
In recent years, research has emerged that aims to enable operations on computerss to be performed in natural language. For example, a language model is being developed that will allow a user to operate a browser like a human being by receiving instructions in natural language \cite{nakano2021webgpt}. Although currently limited to browser operations, this is a framework that can be extended to any behavior on a computers in the future. Research is also underway to automate code generation and execution, which will be an important elemental technology for automating many actions on the computers \cite{gulwani2017program}. Finally, research on using tools for language models \cite{mialon2023augmented} will also aid in the acquisition of actions on the computers. This is because once browser operations and code generation become highly automated, it is expected that calling those language models as tools will allow them to automatically perform fairly complex and sophisticated actions on the computers. Therefore, it seems very important that these technologies, which are currently being actively studied in the machine learning research community, continue to develop in order to realize general-purpose and autonomous artificial researchers.

\subsubsection{Behavior in Real World}
The development of robots that move like humans has been an active area of research. I believe that the development of robots that move like humans and animals, such as those being developed by Boston Dynamics \cite{kuindersma2016optimization}, will be essential to the future automation of these general-purpose, autonomous studies. More directly related to research, our efforts in the field of laboratory automation, which aims to automate experiments, is precisely the area in which I have seriously considered automating research, including such real-world behaviors. In particular, the development of a research specialized general-purpose humanoid robot (\textit{LabDroid}) like ``Mahoro'' \cite{yachie2017robotic} is a powerful step in this direction. As LabDroid becomes more versatile, I will move closer to achieving general-purpose research automation.

% \section{Additional Literature}

\section{Scholarly Document Processing}
\label{appendix:scholarly-document-processing}
\textit{Scholarly document processing} is a general term for research on automated processing related to scholarly articles and has been studied as part of natural language processing, text mining, and information retrieval.
\subsubsection{Search}
Firstly, let us mention academic search engines that provide features to help researchers find relevant papers from a vast amount of literature \cite{googlescholar,semanticscholar,dblp,pubmed,citeseerx}. 
Specialized search systems have also been proposed for specific purposes. For example, some studies have been proposed systems to discover studis in other domeins \cite{kang2022augmenting}, difficulties, limitations and emerging hypotheses \cite{lahav2022search}, and author homepages \cite{patel2021author}. Also, in response to the COVID pandemic, several systems have emerged in recent times to search for COVID-19 related papers \cite{hope2020scisight}.
Instead of searching for papers manually, there are approaches that directly recommend papers to researchers. In practice, it is common for the specific aspects of academic papers that researchers want to compare to vary depending on the situation and research field. Therefore, researchers have invented the method allowing comparison in certain aspect of papers \cite{ostendorff2020aspect} and tailored to some particular research area \cite{breitinger2022recommending}. Also, some researchers study recommendation of authors instead of papers \cite{portenoy2022bursting}. A comprehensive summary of classical research on paper recommendation can be found in \cite{bai2019scientific}.
\subsubsection{Read}
The majority of research on automation in research pertains to automating operations related to papers. Specifically, research on information extraction from papers constitutes the majority. Here, \textit{reading} refers to extracting information from a paper.
Several methods specialized in extracting specific information have been proposed. For instance, there are studies for extracting mathematical expressions \cite{greiner2020math,madisetty2021neural}, measure \cite{harper2021semeval,kohler2021s}, tabale and figure \cite{shen2022vila,hashmi2021current,zhuang2022resel,yamamoto2021visual}, dataset \cite{hou2019identification,kumar2021dataquest,prasad2019dataset}, and results \cite{kardas2020axcell}.
There are also studies that focus not on information extraction, but on determining the meaning of sentences written in papers. One representative example is research on citation classification, which involves understanding the intent behind the cited text \cite{pride2019act,kunnath2021meta,kunnath2022dynamic,kunnath2022act2,lauscher2021multicite}. Another example is topic/theme classification, which detect the main topic of the paper \cite{sadat2022hierarchical,mendoza2022benchmark,salatino2022cso}.
One of the most heavily researched areas of information extraction from scientific papers is summarization. Some studies propose methods to generate the contribution of a paper \cite{hayashi2020s}, scientific claims \cite{wright2022generating}, and lay summarization \cite{goldsack2022making}. Other studies have attempted to create better paper summaries using citation graph \cite{chen2022scientific,an2021enhancing}, or propose the summarization system \cite{erera2019summarization}.
Many of the earlier summarization studies only used limited information such as abstracts. In recent years, there have been proposed studies that generate summaries by reading the entire paper \cite{subramanian2019extractive,qi2022sapgraph,dong2020discourse,tretyak2020combination}.
Also, the number of papers has increased dramatically, and the time available for obtaining information from a single paper has become increasingly limited. Thus, some studies propose the methods to generate extremely short summaries, such as TLDR \cite{cachola2020tldr} and key phrases \cite{boudin2021keyphrase,garg2021keyphrase}.
To advance these summarization studies, some studies propose datasets \cite{yasunaga2019scisummnet,bastan2022sume} and annotation platforms \cite{el2022platform} for paper summarization. 
The early research on paper summarization, which was conducted relatively early, is well summarized in \cite{altmami2022automatic}. If you are interested, please also refer to this paper.
Up to this point, I have described methods that assume extracting specific information or summarizing papers. In contrast, there are studies that issue queries in natural language to retrieve desired information from papers. This has been formalized as a question-answering task, a more general problem setting \cite{lu2022learn,ruggeri2022argscichat,saikh2022scienceqa}. 
In the field of question-answering for academic papers, some web services have gained attention for its high performance \cite{elicit,scispace}. Elicit use large language models and compose them to write \textit{compositional language model programs}. Ought \cite{ought}, the provider of Elicit, publish the instructions of how to write compositional language model programs \cite{primer2022}. Also, they disclose how to update their system with their idea of \textit{process supervision} \cite{reppert2023iterated}. Therefore, for those who are interested in question-answering systems for scientific papers, I strongly recommend reading these papers and documents.
Lastly, many tools have been proposed to assist researchers in reading papers. These studies highlight rhetorical roles \cite{fok2023scim,lauscher2018arguminsci}, generate description to terminologies \cite{august2022generating,head2021augmenting,murthy2022accord}, simplify texts for non-experts \cite{august2022paper,jeblick2022chatgpt}, and allow interaction \cite{kang2022threddy,elicit,scispace}.
\subsubsection{Write}
Research is the act of producing a novel knowledge on top of prior studies. The apt incorporation of previous literature and elucidation of the distinctions between the proposition and previous studies are essential. Consequently, some researchers have investigated to generate comparative arguments \cite{yu2022scientific} and others have studied to generate citation texts \cite{arita2022citation,gu2022controllable,wang2021autocite,xing2020automatic,funkquist2022citebench}. Additionally, several studies exist that, instead of directly producing text, aspire to assist in the writing process by recommending relevant literature for inclusion as citations \cite{farber2020citation,zhang2020dual,duma2019contextual,farber2018cite,gosangi2021use}. Furthermore, there exist investigations aimed at automating systematic reviews writing \cite{dones2022systematic}.
Scholarly articles are structured documents. This structural property enables researchers to generate texts per sections. Thereafter, 
researchers have endeavored to generate, for example, abstract \cite{kumarasinghe2022automatic,gao2022comparing,wang2019paperrobot}, related works \cite{li2022automatic,shah2021generating,liu2023causal}, table description in result section \cite{moosavi2021scigen,moosavi2021learning}, conclusion, and future work \cite{wang2019paperrobot}. Wang et al. propose to generate even next research's probable title \cite{wang2019paperrobot}.

\section{Alignment}
% \subsection{Alignment}
As discussed in Section 2, when realizing an AI that autonomously conducts research, the issue of alignment arises. 

First and foremost, it is essential to consider ways to ensure that AI does not engage in research that could harm humans. However, this is a challenging issue. The problem of ensuring that AI does not harm humans is a difficult problem in AI Alignment. Furthermore, knowledge and technology produced by research are fundamentally value-neutral. That is, the knowledge can be used for good or ill. Therefore, even if AI were to research with harmful intentions, it would be challenging to judge from the actual research results.

The remaining two issues arise in the ultra-long term when AI becomes fully autonomous in conducting research. The second issue is that to enable meaningful knowledge production for humans, there needs to be an alignment between the knowledge systems of AI and humans. As mentioned in Chapter 2, if knowledge and verification are relative concepts to society, research conducted autonomously by AI may become meaningless to humans. On the other hand, if we were to correct AI to follow human methods entirely, we might unnecessarily limit the machine's potential capabilities. Deciding how much human methodology and values to incorporate and how much freedom to allow the machine, and finding ways to achieve this, will be a significant challenge in creating research-capable AI.

The third issue concerns the alignment between AI and nature, not between humans and AI. As mentioned in Chapter 2, the fact that humans have come to understand nature is likely not unrelated to our long history of interacting with nature. It seems there's no guarantee that artificial machines like AI, which lack such experiences, would lead to an understanding of nature through their autonomously generated knowledge.

The latter two issues are problems that only arise when demanding extreme autonomy from machines and are not immediately problematic. However, when discussing the limitations and possibilities of knowledge production and natural understanding by agents independent of humans, they seem to become relevant issues.

\section{Research Optimization}

\section{Research as Social Activity}
In this paper, I will discuss automation focused on the unique elements of knowledge production as mentioned above. However, research is a social endeavor. And that society has various levels, such as research labs, universities, and research ecosystems. Therefore, if I think about optimizing the whole activity of research, I need to think about optimizing these wholes. Although it is out of the scope of this paper and therefore not discussed this time, I would like to discuss this in the future.

\section{Research as Belief Update}

From the perspective that research is belief updating, the roles of question construction, hypothesis generation, and hypothesis verification can be represented in the following Fig. \ref{fig:beliefupdate}. 
\begin{figure}[htb]
    \centering
    \includegraphics[width=\textwidth]{figs/beliefupdate.jpg}
    \caption{Research Process as Belief Update}
    \label{fig:beliefupdate}
\end{figure}
In the diagram, circles represent statements, and squares represent beliefs in response to those statements. Note that while circles were described as statements, they are not limited to textual representations as long as they are associated with beliefs. Beliefs are determined by the subject holding the belief (in this case, an agent) and the object of the belief (the statement), but please note that I are assuming a fixed agent in this context.

Firstly, in this world, there exist countless pairs of statements and corresponding beliefs (or latent beliefs). Posing a question corresponds to extracting a subset of unknown statements from this pool. More precisely, an unknown statement refers to a statement for which the agent is unsure whether to assign a strong or weak belief. Choosing corresponds to implicitly determining a function that assigns beliefs to each statement. For example, let's consider posing the question ``When did the universe begin?'' This answer to this question is unknown to humanity. By posing this question, the range of possible answers is narrowed down. However, it is not realistically possible for an agent to be aware of all statements and potential answers that could exist. Therefore, while I mentioned pairs of beliefs and statements, please consider that most of these beliefs are latent and potential.

Next, generating a hypothesis involves selecting one pair of a claim and a belief from among the potential claims that could serve as potential answers to the question. It is at this point, by choosing a hypothesis or considering multiple candidate hypotheses, that discourse becomes consciously acknowledged, and beliefs are substantively assigned. Finally, as I discussed in Chapter 2, verification involves gaining evidence for the belief in the chosen hypothesis, resulting in the updating of the belief state. The updated beliefs are represented in black in the diagram.


\end{document}